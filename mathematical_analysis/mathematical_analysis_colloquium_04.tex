\documentclass[a4paper]{article}
\usepackage{header}
\usepackage{float}
\usepackage{cmap}

\newcommand\enumtocitem[3]{\item\textbf{#1}\addtocounter{#2}{1}\addcontentsline{toc}{#2}{\protect{\numberline{#3}} #1}}
\newcommand\defitem[1]{\enumtocitem{#1}{subsection}{\thesubsection}}
\newcommand\proofitem[1]{\enumtocitem{#1}{subsection}{\thesubsection}}

\newtheorem{theorem1*}{Theorem}

\newtheoremstyle{named}{}{}{}{}{\bfseries}{}{.5em}{Теорема \thmnote{#3}}
\theoremstyle{named}
\newtheorem*{namedtheorem}{Theorem}

\newcommand{\italicbold}[1]{\emph{\textbf{#1}}}

\newlist{colloq}{enumerate}{1}
\setlist[colloq]{label=\textbf{\arabic*.}}

\everymath{\displaystyle}

\title{\HugeМатематический анализ, Коллоквиум 4}
\author{
	Балюк Игорь \\
	\href{https://teleg.run/lodthe}{@lodthe},
    \href{https://github.com/LoDThe/hse-tex}{GitHub} \\
}

\usepackage[yyyymmdd,hhmmss]{datetime}
\settimeformat{xxivtime}
\renewcommand{\dateseparator}{.}
\date{Дата изменения: \today \ в \currenttime}

\begin{document}
    \maketitle

    \tableofcontents

    \newpage

    \href{https://www.youtube.com/watch?v=dQw4w9WgXcQ}{Оригинальный список вопросов}

    Огромное спасибо Егору Косову: документ состоит из его конспектов и моих вставок.

    \section{Метрические и нормированные пространства.}

        \href{https://www.dropbox.com/s/donysz87em9jfhs/%D0%9B%D0%B5%D0%BA%D1%86%D0%B8%D1%8F%208.pdf?dl=0}{Оригинальный конспект.}

        \begin{definition*}
            Пусть $X$ --- множество. Функция $d: X \times X \to [0; +\infty)$ называется метрикой, если
            \begin{enumerate}
            \item $d(x, y) = 0 \iff x = y$;
            \item $d(x, y) = d(y, x) \forall x, y \in X$;
            \item $d(x, z) \leq d(x, y) + d(y, z) \forall x, y, z \in X$.
            \end{enumerate}

            Пара $(X, d)$ называется метрическим пространством.
        \end{definition*}

        Говоря простым языком, метрика --- это расстояние между двумя объектами. Мы будем часто работать с Евклидовой метрикой: пусть $x, y \in \RR^n$, тогда $d(x, y) = \sqrt{(x_1 - y_1)^2 + \dots + (x_n - y_n)^2}$.

        \begin{definition*}
            Пусть $X$ --- линейное пространство. Функция $\norm{\cdot}: X \to [0; +\infty)$ называется нормой, если
            \begin{enumerate}
            \item $\norm{x} = 0 \iff x = 0$;
            \item $\norm{\lambda x} = |\lambda| \norm{x}, \forall x \in X$;
            \item $\norm{x + y} \leq \norm{x} + \norm{y} \forall x, y \in X$.
            \end{enumerate}

            Пара $(X, \norm{\cdot})$ называется нормированным пространством.
        \end{definition*}

        Нормой является привычнам нам длина вектора. Аналогично матрике, мы будем часто работать с Евклидовой нормой: пусть $x \in \RR^n$, тогда $\norm{x} = \sqrt{x_1^2 + \dots + x_n^2}$.

        Всякое нормированное пространство является метрическим с метрикой $d(x, y) = \norm{x - y}$.

        \begin{definition*}
            Пусть $X$ --- линейное пространство. Функция $\langle \cdot, \cdot \rangle: X \times X \to \RR$ называется скалярным произведением, если для всех $x, y, z \in X$ и всех $a, b \in \RR$ выполнены следующие условия:
            \begin{enumerate}
            \item $\langle x, x \rangle \geq 0$ и $\langle x, x \rangle = 0 \iff x = 0$;
            \item $\langle x, y \rangle = \langle y, x \rangle$;
            \item $\langle ax + by, z \rangle = a \langle x, z \rangle + b \langle y, z \rangle$.
            \end{enumerate}

            Линейное пространство $X$ со скалярным произведением называется Евклидовым.
        \end{definition*}

        Мы будем часто работать со следующим скалярным произведением: пусть $x, y \in \RR^n$, тогда $\langle x, y \rangle = x_1 \cdot y_1 \dots x_n \cdot y_n$.

        \begin{lemma*} (Неравенство Коши-Буняковского) 
            Пусть $\langle \cdot, \cdot \rangle$ скалярное произведение на линейном пространстве $X$, тогда $\forall x, y \in X$
            \begin{equation*}
                |\langle x, y \rangle| \leq \sqrt{\langle x, x \rangle} \cdot \sqrt{\langle y, y \rangle}.
            \end{equation*}
        \end{lemma*}
        
\end{document}