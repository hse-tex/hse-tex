\documentclass[a4paper]{article}
\usepackage{header}


\newcommand\enumtocitem[3]{\item\textbf{#1}\addtocounter{#2}{1}\addcontentsline{toc}{#2}{\protect{\numberline{#3}} #1}}
\newcommand\defitem[1]{\enumtocitem{#1}{subsection}{\thesubsection}}
\newcommand\proofitem[1]{\enumtocitem{#1}{subsection}{\thesubsection}}

\newcommand{\italicbold}[1]{\emph{\textbf{#1}}}

\newlist{colloq}{enumerate}{1}
\setlist[colloq]{label=\textbf{\arabic*.}}

\graphicspath{
    {img/}
}


\title{\HugeМатематический анализ, Коллоквиум 2}
\author{
	Балюк Игорь \\
	\href{https://teleg.run/lodthe}{@lodthe},
    \href{https://github.com/LoDThe/hse-tex}{GitHub} \\
}
\date{2019 --- 2020}

\begin{document}
    \maketitle

    \tableofcontents

    \newpage

    \section{Вопросы предварительной части коллоквиума}

	Список вопросов предварительной части коллоквиума, ответ на которые	необходим для подготовки к основной части.

    \begin{colloq}
    
    \defitem{Определение непрерывности функции в точке.}

    	Пусть $f(x)$ --- функция, определенная на промежутке $I$ ($I$ --- это её область определения) и пусть $c$ --- произвольная точка из $I$. Предположим, что для любого $\eps > 0$ существует $\delta > 0$:

    	\begin{equation*}
    		\forall x \in I: \; |x - c| < \delta \implies |f(x) - f(c)| < \eps
    	\end{equation*}

    	Тогда функция $f(x)$ \italicbold{непрерывна} в точке $c$. 

    	Заметьте, если $c$ --- это левая граница $I$, то условие имеет вид (функция непрерывна в точке $c$ справа, аналогично для непрерывности слева).
    	\begin{equation*}
    		\forall x \in I: \; c < x < c + \delta \implies |f(x) - f(c)| < \eps
    	\end{equation*}

    	\begin{theorem*}
    		Также, функция $f(x)$ непрерывна в точке $a$. Тогда найдётся такое $\delta > 0$, что функция $f(x)$ ограничена окрестностью $U_{\delta}(a)$ точки $a$. 
    	\end{theorem*}

	\defitem{Точки разрыва, их классификация.}
	\defitem{Теорема о непрерывности сложной функции.}
	\defitem{Формулировки первой и второй теорем Вейерштрасса.}
	\defitem{Понятие производной функции в точке.}
	\defitem{Геометрический и физический смысл производной.}
	\defitem{Уравнение касательной к графику функции в точке.}
	\defitem{Понятие дифференцируемости функции в точке.}
	\defitem{Правила дифференцирования (производная суммы, произведения, частного).}
	\defitem{Формула вычисления производной сложной функции.}
	\defitem{Таблица производных основных элементарных функций.}
	\defitem{Понятие дифференциала (первого) функции в точке.}
	\defitem{Геометрический смысл дифференциала.}
	\defitem{Определение локального экстремума. Необходимое условие для внутреннего локального экстремума (теорема Ферма).}
	\defitem{Формулы Лагранжа и Коши.}
	\defitem{Многочлен Тейлора и формула Тейлора для функций одной переменной.}
	\defitem{Формулы Маклорена для основных элементарных функций.}
	\defitem{Правило Лопиталя.}

    \end{colloq}

    \section{Вопросы на знание доказательств}
    \begin{colloq}
    \setlength\parindent{20pt}

    \proofitem{Определения непрерывности функции в точке, их эквивалентность. Точки разрыва, их классификация.}
	\proofitem{Непрерывность основных элементарных функций.}
	\proofitem{Арифметические свойства непрерывных функций.}
	\proofitem{Теорема о непрерывности сложной функции.}
	\proofitem{Свойства функций, непрерывных на отрезке (первая и вторая теоремы Вейерштрасса).}
	\proofitem{Теорема Коши о прохождении непрерывной функции через промежуточные значения.}
	\proofitem{Понятие производной функции в точке.}
	\proofitem{Геометрический и физический смысл производной.}
	\proofitem{Уравнение касательной к графику функции в точке.}
	\proofitem{Понятие дифференцируемости функции в точке.}
	\proofitem{Необходимое условие дифференцируемости.}
	\proofitem{Правила дифференцирования.}
	\proofitem{Теорема о дифференцируемости и производной сложной функции.}
	\proofitem{Теорема о дифференцируемости обратной функции.}
	\proofitem{Таблица производных основных элементарных функций.}
	\proofitem{Производные функций, графики которых заданы параметрически.}
	\proofitem{Понятие дифференциала (первого) функции в точке.}
	\proofitem{Геометрический смысл дифференциала.}
	\proofitem{Инвариантность формы первого дифференциала.}
	\proofitem{Производные и дифференциалы высших порядков функции одной переменной в точке.}
	\proofitem{Понятие об экстремумах функции одной переменной.}
	\proofitem{Локальный экстремум. Необходимое условие для внутреннего локального экстремума (теорема Ферма).}
	\proofitem{Основные теоремы о дифференцируемых функций на отрезке (теорема Ролля, формулы Лагранжа и Коши).}
	\proofitem{Многочлен Тейлора и формула Тейлора для функций одной переменной с остаточным членом в форме Пеано и Лагранжа.}
	\proofitem{Формулы Маклорена для основных элементарных функций.}
	\proofitem{Правило Лопиталя.}
	\proofitem{Достаточное условие строгого возрастания (убывания) функции на промежутке.}
	\proofitem{Достаточные условия локального экстремума для функции одной переменной.}
	\proofitem{Выпуклые (вогнутые) функции одной переменной.}
	\proofitem{Достаточные условия выпуклости (вогнутости).}
	\proofitem{Точки перегиба.}
	\proofitem{Необходимые и достаточные условия для точки перегиба.}
	\proofitem{Асимптоты графика функции одной переменной.}

    \end{colloq}

\end{document}