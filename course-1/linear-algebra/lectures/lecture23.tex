\section{Лекция 23}


\subsection{Описание всех ортонормированных базисов в терминах одного ортонормированного базиса и матриц перехода}

Пусть $\E = (e_1, \dots, e_n)$ --- ортонормированный базис в $E$.

Пусть $\E' = (e'_1, \dots, e'_n)$ --- какой-то другой базис.

$(e'_1, \dots, e'_n) = (e_1, \dots, e_n) \cdot C$, $C \in M_n^{0}(\RR)$.

\begin{proposal}
    $\E'$ --- ортонормированный базис $\iff C^{T} \cdot C = E$.
\end{proposal}

\begin{proof}
    $G(e'_1, \dots, e'_n) = C^{T} \underbracket{G(e_1, \dots, e_n)}_E C = C^{T} C$.

    $\E'$ ортонормированный $\iff G(e'_1, \dots, e'_n) = E \iff C^{T} C = E$.
\end{proof}


\subsection{Ортогональные матрицы и их свойства}

\begin{definition}
    Матрица $C \in M_n(\RR)$ называется \textit{ортогональной} если $C^{T} C = E$.
\end{definition}

\begin{comment}
    $C^{T} C = E \iff C C^{T} = E \iff C^{-1} = C^{T}$.
\end{comment}

\begin{properties}~
    \begin{enumerate}
    \item $C^{T} C = E \implies $ система столбцов $C^{(1)}, \dots, C^{(n)}$ --- это ортонормированный базис в $\RR^n$,
    \item $C C^{T} = E \implies $ система строк $C_{(1)}, \dots, C_{(n)}$ --- это тоже ортонормированный базис в $\RR^n$,
    \end{enumerate}
    В частности, $|c_{ij}| \leq 1$.
    \begin{enumerate}[resume]
    \item $\det C = \pm 1$.
    \end{enumerate}
\end{properties}

\begin{example}
    $n = 2$.
    Ортогональный матрицы:
    \begin{equation}
        \begin{gathered}
            \begin{pmatrix} 
                \cos \phi & -\sin \phi \\
                \sin \phi & \cos \phi
            \end{pmatrix} \\
            \det = 1
        \end{gathered}
        \hspace{1cm}
        \begin{gathered}
            \begin{pmatrix} 
                \cos \phi & \sin \phi \\
                \sin \phi & -\cos \phi
            \end{pmatrix} \\
            \det = -1
        \end{gathered}
    .\end{equation}
\end{example}


\subsection{Ортогональное дополнение подмножества евклидова пространства}

\begin{definition}
    \textit{Ортогональное дополнение} множества $S \subseteq \EE$ --- это множество $S^{\perp} := \{x \in \EE \mid (x, y) = 0 \ \forall y \in S\}$.
\end{definition}

\begin{exercise}~
    \begin{enumerate}
    \item $S^{\perp}$ --- подпространство в $\EE$.
    \item $S^{\perp} = \left< S \right>^{\perp}$.
    \end{enumerate}
\end{exercise}


\subsection{Размерность ортогонального дополнения подпространства, ортогональное дополнение к ортогональному дополнению подпространства}
\subsection{Разложение евклидова пространства в прямую сумму подпространства и его ортогонального дополнения}

Далее считаем, что $\dim \EE = n < \infty$.

\begin{proposal}
    Пусть $S \subseteq \EE$ --- подпространство.
    Тогда:
    \begin{enumerate}
    \item $\dim S^{\perp} = n - \dim S$.
    \item $\EE = S \oplus S^{\perp}$.
    \item $(S^{\perp})^{\perp} = S$.
    \end{enumerate}
\end{proposal}

\begin{proof}~
    \begin{enumerate}
    \item 
        Пусть $\dim S = k$ и $e_1, \dots, e_k$ --- базис $S$.
        
        Дополним $e_1, \dots, e_k$ до базиса $e_1, \dots, e_n$ всего $\EE$.

        Тогда, $\forall x = x_1 e_1 + \dots + x_n e_n \in \EE$.

        \begin{align*}
            x \in S^{\perp} &\iff (x, e_i) = 0 \ \forall i = 1, \dots, k \\
                            &\iff \begin{cases}
                                (e_1, e_1) x_1 + \dots + (e_n, e_1) x_n = 0 \\
                                (e_1, e_2) x_1 + \dots + (e_n, e_2) x_n = 0 \\
                                \dots \\
                                (e_1, e_k) x_1 + \dots + (e_n, e_k) x_n = 0
                            \end{cases}
        .\end{align*}

        Это ОСЛУ с матрицей $G \in \text{Mat}_{k \times n}(\RR)$, причём левый $k \times k$ блок в $G$ --- это $\underbrace{G(e_1, \dots, e_k)}_{\det \neq 0}$.

        Это означает, что $\rk G = k$.

        Следовательно, пространство решений этой ОСЛУ имеет размерность $n - k$.

        Отсюда, $\dim S^{\perp} = n - k = n - \dim S$.

    \item
        \begin{enumerate}
        \item $\dim S + \dim S^{\perp} = k + (n - k) = n = \dim E$.
        \item $v \in S \cap S^{\perp} \implies (v, v) = 0 \implies v = 0 \implies S \cap S^{\perp} = \{0\}$.
        \end{enumerate}

        А значит, $E = S \oplus S^{\perp}$.

    \item
        Заметим, что $S \subseteq (S^{\perp})^{\perp}$ (по определению).

        $\dim (S^{\perp})^{\perp} = n - \dim S^{\perp} = n - (n - \dim S) = \dim S$.

        Следовательно, $S = (S^{\perp})^{\perp}$.
        \qedhere
    \end{enumerate}
\end{proof}


\subsection{Ортогональная проекция вектора на подпространство, ортогональная составляющая вектора относительно подпространства}

$S$ --- подпространство $ \implies \EE = S \oplus S^{\perp}$

$\forall v \in \EE \ \exists! \ x \in S, y \in S^{\perp}$, такие что $x + y = v$.

\begin{definition}~
    \begin{enumerate}
    \item 
        $x$ называется \textit{ортогональной проекцией} вектора $v$ на подпространство $S$.

        Обозначение: $x = \pr_S v$.

    \item
        $y$ называется \textit{ортогональной составляющей} вектора $v$ относительно подпространства $S$.

        Обозначение: $y = \ort_S v$.
    \end{enumerate}
\end{definition}


\subsection{Формула для ортогональной проекции вектора на подпространство в терминах его ортогонального (ортонормированного) базиса}


Пусть $S \subseteq \EE$ --- подпространство.

$e_1, \dots, e_k$ --- ортогональный базис в $S$.

\begin{proposal}
    $\forall v \in \EE \quad \pr_S v = \sum_{i = 1}^{k} \dfrac{(v, e_i)}{(e_i, e_i)} e_i$.

    В частности, если $e_1, \dots, e_k$ ортонормирован, то $\pr_S v = \sum_{i = 1}^{k} (v, e_i) e_i$.
\end{proposal}

\begin{proof}
    Пусть $e_{k + 1}, \dots, e_n$ --- ортогональный базис в $S^{\perp}$. Тогда $e_1, \dots, e_n$ --- ортогональный базис в $\EE$.

    \begin{equation*}
        v = \underbrace{\sum_{i = 1}^{k} \dfrac{(v, e_i)}{(e_i, e_i)} e_i}_{\in S} + \underbrace{\sum_{i = k + 1}^{n} \dfrac{(v, e_i)}{(e_i, e_i)} e_i}_{\in S^{\perp}}
    .\end{equation*}

    Отсюда,
    \begin{equation*}
        \pr_S v = \sum_{i = 1}^{k} \dfrac{(v, e_i)}{(e_i, e_i)} e_{i}
    .\qedhere\end{equation*}
\end{proof}


\begin{comment}
    \hyperref[lec22:gram-schmidt_second_property]{Свойство 2} из метода Грама-Шмидта говорит, что 
    \begin{equation*}
        f_i = e_i - \pr_{\left< f_1, \dots, f_{i - 1}\right>} e_i = \ort_{\left< f_1, \dots, f_{i - 1} \right>} e_i
    .\end{equation*}
\end{comment}


\subsection{Явная формула для ортогональной проекции вектора на подпространство в $\RR^n$, заданное своим базисом}

Пусть $\EE = \RR^n$ со стандартным скалярным произведением.

$S \subseteq \EE$ --- подпространство, $a_1, \dots, a_k$ --- базис $S$.

Пусть $A := (a_1, \dots, a_k) \in \text{Mat}_{n \times k}(\RR)$, $A^{(i)} = a_i$.

\begin{proposal}
    $\forall v \in \RR^n \quad \pr_S v = A (A^{T} A)^{-1} A^{T} v$.
\end{proposal}

\begin{proof}
    Корректность: $A^{T} A = G(a_1, \dots, a_k) \in M_k^{0}(\RR)$.

    Положим $x := \pr_S v$, $y := \ort_S v$.

    Так как $x \in S$, $x = A \cdot \begin{pmatrix} \alpha_1 \\ \dots \\ \alpha_k \end{pmatrix}$, $\alpha_i \in \RR$.

    $y \in S^{\perp} \implies A^T y = 0$.

    \begin{align*}
    A (A^{T} A)^{-1} A^{T} v
    &= A(A^{T}A)^{-1}A^{T} (x + y) \\
    &= A\lefteqn{\underbracket{\phantom{(A^{T}A)^{-1}A^{T} A}}_E} (A^{T} A)^{-1} A^{T} \overbracket{A \begin{pmatrix} \alpha_1 \\ \dots \\ \alpha_k \end{pmatrix}}^{x} + A(A^{T} A)^{-1} \underbracket{A^{T} y}_{0} \\
    &= A \begin{pmatrix} \alpha_1 \\ \dots \\ \alpha_k \end{pmatrix} = x = \pr_S v
    .\end{align*}
\end{proof}


\subsection{Теорема Пифагора в евклидовом пространстве}

\begin{theorem}
    Пусть $x, y \in \EE$, $(x, y) = 0$. Тогда $|x + y|^2 = |x|^2 + |y|^2$.
\end{theorem}

\begin{proof}
    \begin{equation*}
        |x + y|^2 = (x + y, x + y) = \underbrace{(x, x)}_{|x|^2} + \underbrace{(x, y)}_{0} + \underbrace{(y, x)}_{0} + \underbrace{(y, y)}_{|y|^2} = |x|^2 + |y|^2
    .\qedhere\end{equation*}
\end{proof}


\subsection{Расстояние между векторами евклидова пространства}

\begin{definition}
    \textit{Расстояние} между векторами $x, y \in \EE$ --- это $\rho(x, y) = |x - y|$.
\end{definition}


\subsection{Неравенство треугольника}

\begin{proposal}
    $\forall a, b, c \in \EE \implies \rho(a, b) + \rho(b, c) \geq \rho(a, c)$.
\end{proposal}

\begin{proof}
    Пусть $x = a - b$, $y = b - c$. Тогда, $a - c = x + y$.
    Достаточно доказать, что $|x| + |y| \geq |x + y|$.

    \begin{equation*}
        |x + y|^2 = |x|^2 + \underbrace{2(x, y)}_{\leq |x||y|} + |y|^2 \leq |x|^2 + 2|x||y| + |y|^2 = (|x| + |y|)^2
    .\qedhere\end{equation*}
\end{proof}


\subsection{Расстояние между двумя подмножествами евклидова пространства}

Пусть $P, Q \subseteq \EE$ --- два подмножества.

\begin{definition}
    \textit{Расстояние} между $P$ и $Q$ --- это
    \begin{equation*}
        \rho(P, Q) := \inf_{x \in P, y \in Q} \rho(x, y)
    .\end{equation*}
\end{definition}


\subsection{Теорема о расстоянии от вектора до подпространства}

\begin{theorem}
    Пусть $x \in \EE$, $S \subseteq \EE$ --- подпространство. Тогда, $\rho(x, S) = \left|\ort_S x\right|$, причем $\pr_S x$ --- это ближайший к $x$ вектор из $S$.
\end{theorem}

\begin{proof}
    Положим $y = \pr_S x$, $z = \ort_S x$. Тогда, $x = y + z$.
    Для любого $y' \in S$, $y' \neq 0$ имеем
    \begin{equation*}
        \rho(x, y + y')^2 = |x - y - y'|^2 = |z - y'|^2 = |z|^2 + |y'|^2 > |z|^2 = |x - y|^2 = \rho(x, y)^2
    .\qedhere\end{equation*}
\end{proof}


\subsection{Псевдорешение несовместной системы линейных уравнений (метод наименьших квадратов)}

СЛУ $Ax = b$, $A \in \text{Mat}_{m \times n}(\RR)$, $x \in \RR^n$, $b \in \RR^m$.
\begin{equation*}
    x_0 \text{ --- решение системы} \iff Ax_0 = b \iff Ax_0 - b = 0 \iff |Ax_0 - b| = 0 \iff \rho(Ax_0, b) = 0
.\end{equation*}

Если СЛУ несовместна, то $x_0$ называется \textit{псевдорешением}, если $\rho(Ax_0, b)$ минимально.

\begin{equation*}
    \rho(Ax_0, b) = \min_{x \in R^n} \rho(Ax, b)
.\end{equation*}

$x_0$ --- решение задачи оптимизации $\rho(Ax, b) \xrightarrow[x \in \RR^n]{} \min$.
