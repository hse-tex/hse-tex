\section{Лекция 12.12.2019}

\subsection{Миноры}

Пусть $A \in \text{Mat}_{m \times n}(F)$.

\begin{definition}
    \textit{Минором} матрицы $A$ называется определитель всякой квадратной подматрицы в $A$.
\end{definition}

\begin{example}
    \begin{equation*}
        A = \begin{pmatrix} 1 & 2 & 3 \\ 4 & 5 & 6 \end{pmatrix}
    .\end{equation*}

    \begin{itemize}
    \item 6 миноров порядка 1,
    \item 3 минора порядка 2

        \begin{math}
            \begin{vmatrix} 1 & 2 \\ 4 & 5 \end{vmatrix}, \quad
            \begin{vmatrix} 1 & 3 \\ 4 & 6 \end{vmatrix}, \quad
            \begin{vmatrix} 2 & 3 \\ 5 & 6 \end{vmatrix}
        .\end{math}
    \end{itemize}
\end{example}


\subsection{Теорема о ранге матрицы}

\begin{theorem}
    Для любой $A \in \text{Mat}_{m \times n}(F)$ следующие 3 числа равны:
    \begin{enumerate}[label=(\arabic*),nosep]
    \item \label{lec14:eq_thm:1} $\rk A$ (столбцовый ранг),
    \item \label{lec14:eq_thm:2} $\rk A$ (строковый ранг),
    \item \label{lec14:eq_thm:3} наибольший порядок ненулевого минора в $A$.
    \end{enumerate}
\end{theorem}

\begin{proof}
    \ref{lec14:eq_thm:1} $=$ \ref{lec14:eq_thm:2} -- уже знаем.

    Пусть $S$ -- квадратная подматрица в $A$ порядка $r$ и $\det S \neq 0$.
    Тогда $r = \rk S \leq \rk A$.
    Отсюда, \ref{lec14:eq_thm:3} $\leq$ \ref{lec14:eq_thm:1}.

    \bigskip
    Пусть теперь $\rk A = r$. Найдем в $A$ ненулевой минор порядка $r$.

    Так как $\rk A = r$, в $A$ есть $r$ линейно независимых столбцов $A^{(i_1)}, \dots, A^{(i_r)}$. 

    Составим из них матрицу $B$. Тогда $\rk B = r$.

    Так как \ref{lec14:eq_thm:1} $=$ \ref{lec14:eq_thm:2} для $B$, то в $B$ можно найти $r$ линейно независимых строк.

    Пусть $S$ -- подматрица в $B$, составленная из этих строк. 

    $S$ -- квадратная подматрица порядка $r$ и $\rk S = r \implies \det S \neq 0 \implies $ нашли. Значит, \ref{lec14:eq_thm:3} $\geq$ \ref{lec14:eq_thm:1}.

    \bigskip
    Итог: \ref{lec14:eq_thm:3} $=$ \ref{lec14:eq_thm:1}.
\end{proof}


\subsection{Приложения ранга матрицы к исследованию СЛУ}

Пусть $A \in \text{Mat}_{m \times n}(F)$, $b \in F^m$, $x \in F^n$ -- столбец неизвестных.
\begin{equation*}
    \tag{$\star$}
    \label{lec14:axb}
    Ax = b
.\end{equation*}

$(A \mid b)$ -- расширенная матрица.

\subsubsection{Теорема Кронекера-Капелли}

\begin{theorem}[Кронекера-Капелли]
    СЛУ \eqref{lec14:axb} совместна $\iff \rk (A \mid b) = \rk A$.
\end{theorem}

\begin{proof}
    При элементарных преобразованиях строк
    \begin{itemize}[nosep]
        \item[--] сохраняется множество решений,
        \item[--] сохраняются числа $\rk (A \mid b)$ и $\rk A$.
    \end{itemize}

    \bigskip
    Следовательно, вопрос сводится к ситуации когда $A$ имеет ступенчатый вид.
    
    В ступенчатом виде СЛУ совместна тогда и только тогда, когда нет строк вида $(0, \dots, 0 \mid \underbrace{\star}_{\neq 0})$.

    То есть матрицы $(A \mid b)$ и $A$ имеют одно и то же число ненулевых строк.

    Значит, $\rk (A \mid b) = \rk A$.
\end{proof}


\subsubsection{Критерий существования единственного решения у совместной системы линейных уравнений в терминах ранга её матрицы коэффициентов}

\begin{theorem}
    Пусть СЛУ \eqref{lec14:axb} совместна. Тогда, она имеет единственное решение тогда и только тогда, когда~$\rk A = n$, где $n$ -- число неизвестных.
\end{theorem}

\begin{proof}
    Снова все сводится к ситуации, когда $(A \mid b)$ имеет ступенчатый вид.

    Тогда, единственное решение $\iff$ нет свободных неизвестных $\iff$ ступенек ровно $n$ $\iff$ $\rk A = n$.
\end{proof}


\subsubsection{Критерий существования единственного решения у системы линейных уравнений с квадратной матрицей коэффициентов в терминах её определителя}

\begin{corollary}
    Пусть $A$ квадратна (то есть $m = n$). Тогда СЛУ \eqref{lec14:axb} имеет единственное решение $\iff$ $\det A \neq 0$.
\end{corollary}

\begin{proof}
    Единственное решение $\iff$ $\rk A = n$ $\iff$ $\det A \neq 0$.
\end{proof}

\begin{comment}
    Это единственное решение равно $x = A^{-1}b$.
\end{comment}


\subsubsection{Размерность пространства решений однородной системы линейных уравнений в терминах ранга её матрицы коэффициентов}

Пусть теперь СЛУ однородна, то есть $b = 0$.
\begin{equation*}
    \tag{$\star$}
    \label{lec14:ax0}
    Ax = 0
.\end{equation*}
Пусть $S \subseteq F^n$ -- множество её решений. Знаем, что $S$ -- подпространство в $F^n$.

\begin{proposal}
    $\dim S = n - \rk A$.
\end{proposal}

\begin{proof}
    Пусть $r$ -- число ненулевых строк в ступенчатом виде матрицы $A$.
    Тогда $r = \rk A$.

    Мы уже строили ФСР для \eqref{lec14:ax0} из $n - r$ векторов.

    Значит, $\dim S = n - r = n - \rk A$.
\end{proof}


\subsubsection{Реализация подпространства в $F^n$ в качестве множества решений однородной системы линейных уравнений}

\bigskip
Пусть $b_1, \dots, b_p \in F^{n}$,
\begin{equation*}
    B := (b_1, \dots, b_p) \in \text{Mat}_{n \times p}(F)
.\end{equation*}

Пусть $a_1, \dots, a_q \in F^{n}$ --- ФСР для ОСЛУ $B^{T}x = 0$.
\begin{equation*}
    A := (a_1, \dots, a_q) \in \text{Mat}_{n \times q}(F)
.\end{equation*}

\begin{proposal}
    $\langle b_1, \dots, b_p \rangle$ есть множество решений ОСЛУ $A^T x = 0$.
\end{proposal}

\begin{proof}
    Пусть $S = \{x \in F^n \mid A^T x = 0\}$.
    \begin{align*}
        \forall i = 1, \dots, q \quad B^T a_i = 0 &\implies B^T A = 0 \\
        &\implies A^T B = 0 \implies A^T b_j = 0 \quad \forall j = 1, \dots, p
    .\end{align*}

    Значит, $b_j \in S \quad \forall j = 1, \dots, p$.

    Но тогда, $\langle b_1, \dots, b_p \rangle \subseteq S$.

    Пусть $r = \rk \{b_1, \dots, b_p\} = \dim \langle b_1, \dots, b_p \rangle = \rk B$.

    При этом, $\rk A = q = n - r$.

    Тогда, $\dim S = n - \rk A = n - (n - r) = r$.

    Следовательно, $\langle b_1, \dots, b_p \rangle = S$.
\end{proof}

\begin{corollary}
    Всякое подпространство в $F^n$ является решением некоторой ОСЛУ.
\end{corollary}


\subsection{Координаты вектора по отношению к фиксированному базису векторного пространства}

Пусть $V$ --- векторное пространство, $\dim V = n$, \ $e_1, \dots, e_n$ --- базис.

Знаем, что $\forall v \in V \ \exists! \alpha_1, \dots, \alpha_n \in F$, такие что, $v = \alpha_1 e_1 + \dots + \alpha_n e_n$.

\begin{definition}
    Скаляры $\alpha_1, \dots, \alpha_n$ называются координатами вектора $v$ в базисе $e_1, \dots, e_n$.
\end{definition}

\begin{example}
    $V = F^n$.

    $v = \begin{pmatrix} x_1 \\ \dots \\ x_n \end{pmatrix}$.

    Тогда, $x_1, \dots, x_n$ -- координаты вектора $v$ в стандартном базисе пространства $F^n$.
\end{example}


\subsection{Описание всех базисов конечномерного векторного пространства в терминах одного базиса и матриц координат}

Пусть теперь $e'_1, \dots, e'_n$ -- какой то другой набор векторов в $V$. Тогда,
\begin{align*}
    e'_1 &= c_{11} e_1 + c_{21} e_2 + \dots + c_{n1} e_n \\
    e'_2 &= c_{12} e_1 + c_{22} e_2 + \dots + c_{n2} e_n \\
    &\dots \\
    e'_n &= c_{1n} e_1 + c_{2n} e_2 + \dots + c_{nn} e_n
.\end{align*}
\begin{equation*}
    (e'_1, \dots, e'_n) = (e_1, \dots, e_n) \cdot C, \text{ где } C = (c_{ij})
.\end{equation*}

в $j$-м столбце матрицы $C$ стоят координаты вектора $e'_j$ в базисе $e_1, \dots, e_n$.

\begin{proposal}
    $(e'_1, \dots, e'_n)$ -- базис в $V$ $\iff$ $\det C \neq 0$.
\end{proposal}

\begin{proof}~
    \begin{description}
        \item[$\implies$] 
            $e'_1, \dots, e'_n$ -- базис, значит $(e_1, \dots, e_n) = (e'_1, \dots, e'_n) \cdot C' = (e_1, \dots, e_n) \cdot C \cdot C'$.

            Так как $e_1, \dots, e_n$ линейно независимы, то $C \cdot C' = E \implies \det C \neq 0$.

        \item[$\impliedby$]
            $\det C \neq 0 \implies \exists C^{-1}$.

            Достаточно доказать, что $e'_1, \dots, e'_n$ линейно независимы.

            Пусть
            \begin{equation*}
                \alpha_1 e'_1 + \dots + \alpha_n e'_n = 0
            .\end{equation*}

            Тогда,
            \begin{equation*}
                (e'_1, \dots, e'_n) \begin{pmatrix} \alpha_1 \\ \dots \\ \alpha_n \end{pmatrix} = 0
                \implies (e_1, \dots, e_n) \cdot C \cdot \begin{pmatrix} \alpha_1 \\ \dots \\ \alpha_n \end{pmatrix} = 0
            .\end{equation*}

            Так как $e_1, \dots, e_n$ линейно независимы, то
            \begin{equation*}
                C \begin{pmatrix} \alpha_1 \\ \dots \\ \alpha_n \end{pmatrix} = 0
            .\end{equation*}

            Домножаем слева на $C^{-1}$, получаем
            \begin{equation*}
                \begin{pmatrix} \alpha_1 \\ \dots \\ \alpha_n \end{pmatrix} = \begin{pmatrix} 0 \\ \dots \\ 0 \end{pmatrix}
            .\qedhere\end{equation*}
    \end{description}
\end{proof}


\subsection{Матрица перехода от одного базиса конечномерного векторного пространства к другому}

Пусть $(e_1, \dots, e_n)$ и $(e'_1, \dots, e'_n)$ --- два базиса в $V$,
\begin{equation*}
    (e'_1, \dots, e'_n) = (e_1, \dots, e_n) \cdot C
,\end{equation*}
при этом $\det C \neq 0$.

\begin{definition}
    Матрица $C$ называется \textit{матрицей перехода} от базиса $(e_1, \dots, e_n)$ к базису $(e'_1, \dots, e'_n)$.
\end{definition}

\begin{comment}
    Матрица перехода от $(e'_1, \dots, e'_n)$ к $(e_1, \dots, e_n)$ --- это $C^{-1}$.
\end{comment}


\subsection{Формула преобразования координат вектора при замене базиса}

Пусть $v \in V$, тогда
\begin{align*}
    v &= x_1 e_1 + \dots + x_n e_n \\
    v &= x'_1 e'_1 + \dots + x'_n e'_n
.\end{align*}

\begin{proposal}
    \begin{equation*}
        \begin{pmatrix} x_1 \\ \dots \\ x_n \end{pmatrix} 
        = C \cdot \begin{pmatrix} x'_1 \\ \dots \\ x'_n \end{pmatrix}
    .\end{equation*}
\end{proposal}

\begin{proof}
    Имеем
    \begin{equation*}
        v = (e_1, \dots, e_n) \begin{pmatrix} x_1 \\ \dots \\ x_n \end{pmatrix}
    \end{equation*} 

    С другой стороны,
    \begin{equation*}
        v = (e'_1, \dots, e'_n) \begin{pmatrix} x'_1 \\ \dots \\ x'_n \end{pmatrix} = (e_1, \dots, e_n) \cdot C \begin{pmatrix} x'_1 \\ \dots \\ x'_n \end{pmatrix}
    \end{equation*}

    Так как $e_1, \dots, e_n$ линейно независимы, то
    \begin{equation*}
        \begin{pmatrix} x_1 \\ \dots \\ x_n \end{pmatrix}
        = C \begin{pmatrix} x'_1 \\ \dots \\ x'_n \end{pmatrix}
    .\end{equation*}
\end{proof}
