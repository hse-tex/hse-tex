\section{Лекция 22}


\begin{comment}
    Всякое подпространство $U \subseteq E$ тоже является евклидовым пространством со скалярным произведением $(\bigcdot, \bigcdot) \big|_U \leftarrow$ ограничение на $U$.
\end{comment}


\subsection{Длина вектора евклидова пространства}

\begin{definition}
    \textit{Длина} вектора $x \in \EE$ --- это $|x| := \sqrt{(x, x)}$.
\end{definition}

\textbf{Свойства}:
    \begin{enumerate}[nosep] 
    \item $|x| \geq 0$, причем $|x| = 0 \iff x = 0$.
    \item $\lambda \in \RR \implies |\lambda \cdot x| = \underbrace{|\lambda|}_{\text{модуль}} \cdot \underbrace{|x|}_{\text{длина}} $
\end{enumerate}

\begin{example}
    Если $\EE = \RR^n$ со стандартным скалярным произведением, то $|x| = \sqrt{x_1^2 + \dots + x_n^2}$.
\end{example}

\begin{comment}
    Если $\EE = \text{Mat}_{m \times n}(\RR)$, $(A, B) = \tr(A^{T} B)$

    Тогда, $|A| = \sqrt{\sum_{i = 1}^{m} \sum_{j = 1}^{n} a_{ij}^2} \leftarrow$ это обозначается как $\norm{A}_{F}$ и называется \textit{нормой Фробениуса}, \textit{фробениусовой нормой}.
\end{comment}


\subsection{Неравенство Коши–Буняковского}

\begin{proposal}[неравенство Коши-Буняковского]
    $\forall x, y \in \EE$ верно $|(x, y)| \leq |x| \cdot |y|$, причём равенство $\iff$ $x$, $y$ пропорциональны.
\end{proposal}

\begin{proof}
    Случаи:
    \begin{enumerate}
    \item $x, y$ пропорциональны. Тогда, можно считать, что $y = \lambda x$, $\lambda \in \RR$.

        $|(x, y)| = |(x, \lambda x)| = |\lambda| |(x, x)| = |\lambda| |x|^2 = |x| \cdot |\lambda x| = |x| \cdot |y|$.

    \item $x, y$ не пропорциональны. Тогда $x, y$ линейно независимы.

        Значит они образуют базис в $\left< x, y \right>$.

        Получаем
        \begin{equation*}
            \begin{vmatrix} 
                (x, x) & (x, y) \\
                (y, x) & (y, y)
            \end{vmatrix} > 0 \quad \text{(критерий Сильвестра)}
        .\end{equation*}

        Отсюда, $(x, x) \cdot (y, y) - (x, y)^2 > 0 \implies (x, y)^2 < |x|^2 \cdot |y|^2$.
    \end{enumerate}
\end{proof}

\begin{example}
    Пусть $\EE = \RR^n$ со стандартным скалярным произведением, тогда
    \begin{equation*}
        |x_1 y_1 + \dots + x_n y_n| \leq \sqrt{x_1^2 + \dots + x_n^2} \cdot \sqrt{y_1^2 + \dots + y_n^2}
    .\end{equation*}
\end{example}


\subsection{Угол между ненулевыми векторами евклидова пространства}

Пусть $x, y \in \EE \setminus \{0\}$, тогда $-1 \leq \frac{(x, y)}{|x| \cdot |y|} \leq 1$.

\begin{definition}
    Угол между ненулевыми векторами $x, y \in \EE$, это такой $\alpha \in [0, \pi]$, что $\cos \alpha = \frac{(x, y)}{|x| \cdot |y|}$.

    Тогда $(x, y) = |x| |y| \cos \alpha$.
\end{definition}


\subsection{Матрица Грама системы векторов евклидова пространства}

Пусть $v_1, \dots, v_k$ --- произвольная система векторов.

\begin{definition}
    \textit{Матрица Грама} этой системы --- это
    \begin{equation*}
        G(v_1, \dots, v_k) = \begin{pmatrix}
            (v_1, v_1) & (v_1, v_2) & \dots & (v_1, v_k) \\
            (v_2, v_1) & (v_2, v_2) & \dots & (v_2, v_k) \\
            \vdots & \vdots & \ddots & \vdots \\
            (v_k, v_1) & (v_k, v_2) & \dots & (v_k, v_k)
        \end{pmatrix}
    .\end{equation*}
\end{definition}

\begin{example}
    $\EE = \RR^n$ со стандартным скалярным произведением.

    $a_1, \dots, a_k \in \RR^n \leadsto A := (a_1, \dots, a_k) \in \text{Mat}_{n \times k}(\RR)$.

    Тогда, $G(a_1, \dots, a_k) = A^T \cdot A$.
\end{example}


\subsection{Определитель матрицы Грама: неотрицательность, критерий положительности}

\begin{proposal}
    $\forall v_1, \dots, v_k \in \EE \implies \det G(v_1, \dots, v_k) \geq 0$.

    Более того, $\det G(v_1, \dots, v_k) > 0 \iff v_1, \dots, v_k$ линейно независимы. 
\end{proposal}

\begin{proof}
    Пусть $G := G(v_1, \dots, v_k)$.
    Случаи:
    \begin{enumerate}
    \item $v_1, \dots, v_k$ линейно независимы. Тогда, $G$ --- матрица билинейной формы $(\bigcdot, \bigcdot) \Big|_{\left< v_1, \dots, v_k \right>}$ в базисе $v_1, \dots, v_k$ подпространства $\left< v_1, \dots, v_k \right>$, а значит $\det G > 0$ по критерию Сильвестра.
    
    \item $v_1, \dots, v_k$ линейно зависимы. Тогда, $\exists (\alpha_1, \dots, \alpha_k) \in \RR^k \setminus \{0\}$, такие что $\alpha_1 v_1 + \dots + \alpha_k v_k = 0$.

        А значит,  $\forall i = 1, \dots, k \implies \alpha_1 (v_1, v_i) + \dots + \alpha_k (v_k, v_i) = 0$.
        
        Отсюда, $a_1 G_{(1)} + \dots + \alpha_k G_{(k)} = 0 \implies$ строки в $G$ линейно зависимы $\implies \det G = 0$.
        \qedhere
    \end{enumerate}
\end{proof}


\subsection{Ортогональные векторы}

\begin{definition}
    Векторы $x, y \in \EE$ называются \textit{ортогональными}, если $(x, y) = 0$.
\end{definition}


\subsection{Ортогональные и ортонормированные системы векторов}

\begin{definition}
    Система ненулевых векторов $v_1, \dots, v_k$ называется
    \begin{enumerate}[nosep]
        \item \textit{ортогональной}, если $(v_i, v_j) = 0 \ \forall i \neq j$ (то есть $G(v_1, \dots, v_k)$ диагональна),
        \item \textit{ортонормированной}, если $(v_i, v_j) = 0 \ \forall i \neq j$ и $(v_i, v_i) = 1$ ($\iff |v_i| = 1$).
            То есть $G(v_1, \dots, v_k) = E$.
    \end{enumerate}
\end{definition}

\begin{comment}
    Всякая ортогональная (и в частности ортонормированная) система векторов автоматически линейно независима.
    \begin{equation*}
        \det G(v_1, \dots, v_k) = |v_1|^2 \cdot |v_2|^2 \dotsm |v_k|^2 \neq 0
    .\end{equation*}
\end{comment}


\subsection{Ортогональный и ортонормированный базис}

\begin{definition}
    Базис пространства называется \textit{ортогональным} (соответственно \textit{ортонормированным}), если он является ортогональной (ортонормированной) системой векторов.
\end{definition}


\begin{example}
    $\EE = \RR^n$ со стандартным скалярным произведением.

    Тогда, стандартный базис является ортонормированным.
    \begin{equation*}
        \begin{pmatrix} 1 \\ 0 \\ \dots \\ 0 \end{pmatrix},
        \begin{pmatrix} 0 \\ 1 \\ \dots \\ 0 \end{pmatrix},
        \dots,
        \begin{pmatrix} 0 \\ 0 \\ \dots \\ 1 \end{pmatrix}
    .\end{equation*}
\end{example}


\subsection{Координаты вектора в ортогональном (ортонормированном) базисе}

Пусть $\EE$ --- евклидово пространство, $(e_1, \dots, e_n)$ --- ортогональный базис.

$v \in \EE$.

\begin{proposal}
    $v = \dfrac{(v, e_1)}{(e_1, e_1)}e_1 + \dfrac{(v, e_2)}{(e_2, e_2)}e_2 + \dots + \dfrac{(v, e_n)}{(e_n, e_n)}e_n$.

    В частности, если $e_1, \dots, e_n$ ортонормирован, то $v = (v, e_1)e_1 + \dots + (v, e_n) e_n$.
\end{proposal}

\begin{proof}
    $v = \lambda_1 e_1 + \lambda_2 e_2 + \dots \lambda_n e_n$.

    $\forall i = 1, \dots, n \quad (v, e_i) = \lambda_1 (e_1, e_i) + \dots + \lambda_n (e_n, e_i)$.

    Так как базис ортогонален, то $(v, e_i) = \lambda_i (e_i, e_i) \implies \lambda_i = \dfrac{(v, e_i)}{(e_i, e_i)}$.
\end{proof}


\subsection{Теорема о существовании ортонормированного базиса}

\begin{theorem}
    Во всяком конечномерном евклидовом пространстве существует ортонормированный базис.
\end{theorem}

\begin{proof}
    Следует из теоремы о приведении квадратичной формы $(x, x)$ к нормальному виду (который будет $E$ в силу положительной определённости).
\end{proof}

\begin{corollary}
    Всякую ортогональную (ортонормированную) систему векторов можно дополнить до ортогонального (ортонормированного) базиса.
\end{corollary}

\begin{proof}
    Пусть $e_1, \dots, e_k$ --- данная система. 

    Пусть $e_{k + 1}, \dots, e_{n}$ --- это ортогональный (ортонормированный) базис в $\left< e_1, .., e_k \right>^{\perp}$.

    Тогда $e_1, \dots, e_n$ --- искомый базис.
\end{proof}



\subsection{Метод ортогонализации Грама–Шмидта}

Как построить ортогональный (ортонормированный) базис в $\EE$?

Если $f_1, \dots, f_n$ --- ортогональный базис, то $\left(\frac{f_1}{|f_1|}, \dots, \frac{f_n}{|f_n|}\right)$ --- ортонормированный базис.

Тогда, достаточно построить ортогональный базис.

Пусть $e_1, \dots, e_k$ --- линейно независимая система векторов.

$i$-й угловой минор в $G(e_1, \dots, e_k)$ --- это $\det G(e_1, \dots, e_i) > 0$.

\bigskip
Следовательно, применим метод Якоби:

$\exists!$ система векторов $f_1, \dots, f_k$, такая что 
\begin{align*}
&f_1 = e_1, \\
&f_2 \in e_2 + \left< e_1 \right>, \\
&f_3 \in e_3 + \left< e_1, e_2 \right>, \\
&\dots, \\
&f_k \in e_k + \left< e_1, \dots, e_{k - 1} \right>
\end{align*}


И выполнены следующие свойства $\forall i = 1, \dots, k$:

\begin{enumerate}[start=0,nosep]
\item $f_1, \dots, f_k$ ортогональны.
\item $\left< e_1, \dots, e_i \right> = \left< f_1, \dots, f_i \right>$.
    \label{lec22:gram-schmidt_second_property}
\item $f_i = e_i - \sum_{j = 1}^{i - 1} \dfrac{(e_i, f_j)}{(f_j, f_j)} f_j$ \customlabel{lec22:star}{($\star$)}.
\item $\det G(e_1, \dots, e_i) = \det G(f_1, \dots, f_i)$ \customlabel{lec22:heart}{$(\heartsuit)$}.
\end{enumerate}

\bigskip
Построение базиса $f_1, \dots, f_k$ по формулам \ref{lec22:star} называется методом (процессом) ортогонализации Грамма-Шмидта.
