\section{Лекция 20}


\subsection{Угловые миноры матрицы квадратичной формы}

$G \in M_{n}, k \in \{1, \dots, n\} \leadsto G_{k}$ := левый верхний $k \times k$ блок матрицы $G$

\begin{definition}
    Величина $\delta_k (G)$ := $\det G_{k}$ называется $k$-м угловым минором матрицы $G$    

\begin{equation*}
G = \begin{pmatrix}
        g_{11} & \vline & g_{12} & \vline & g_{13} & \vline & \dots \\ 
        \cline{1-1}
        g_{21} &        & g_{22} & \vline & g_{23} & \vline & \dots \\ 
        \cline{1-3}
        g_{31} &        & g_{32} &        & g_{33} & \vline & \dots \\
        \cline{1-5}
        \vdots &        & \vdots &        & \vdots &        & \ddots
    \end{pmatrix}
\end{equation*}


\end{definition}

\subsection{Метод Якоби для симметричных билинейных форм}

Пусть $\beta \colon V \times V \to F$ --- симметричная билинейная форма, $\E$ --- базис $V$, $B = B(\beta, \E), \delta_{k} = \delta_k(B)$

\begin{theorem}
    Предположим, что $\delta_{k} \neq 0  \ \forall k = 1, \dots, n-1$, тогда $\exists C \in M^{0}_{n}$ вида
    \begin{equation*}
    C = \begin{pmatrix} 
        1 & \star & \dots & \star & \star \\
        0 & 1 & \ddots & \star & \star \\
        \vdots & \vdots & \ddots & \ddots & \vdots \\
        0 & 0 & \dots & 1 & \star \\
        0 & 0 & \dots & 0 & 1
    \end{pmatrix}
    \end{equation*}
    Такая что матрица $B' = C^{T} B C$ имеет вид $B' = \diag\left(\delta_1, \frac{\delta_2}{\delta_1}, \dots, \frac{\delta_n}{\delta_{n - 1}}\right)$.
\end{theorem}

\begin{proof}
    Анализ симметричного алгоритма Гаусса.

    На каждой итерации в случае 1 все симметричные элементарные преобразования имеют вид $\text{Э}_{1}(i, j, \lambda)$, причём всегда при $i > j$. 
    При этом все угловые миноры сохраняются. Благодаря условию $i > j$ это элементарные преобразования 1 типа, не меняющие определителя для каждой $G_{k}$

    Если на какой-то итерации возник не случай 1, то получаем 
    \begin{equation*}
        \begin{pmatrix}
            b_{11} & 0       & \dots  & 0       & 0       & \vline & \dots & 0 \\
            0       & b_{22} & \dots  & 0       & 0       & \vline & \dots & 0 \\
            \vdots  & \vdots  & \ddots & \vdots  & \vdots  & \vline & \dots & \vdots \\
            0       & 0       & \dots  & b_{k-1, k-1} & 0   & \vline & \dots & 0 \\
            0       & 0       & \dots  & 0       & 0       & \vline & \dots & 0 \\
            \cline{1-5}
            \vdots  & \vdots  & \dots  & \vdots  & \vdots  &        & \ddots & \vdots \\
            0       & 0       & \dots  & 0       & 0       &        & \dots & \star
            \end{pmatrix}
            k \leq n-1
        \end{equation*}

    Но тогда $\delta_{k} = 0 \implies$ противоречие.

    Итог: на всех итерациях возникает случай 1 $\implies$ алгоритм приводит к диагональному виду $B' = \diag(d_{1}, \dots, d_{n})$.
    \begin{equation*}
        \delta_{k}(B') = d_{1} \cdot \dots \cdot d_{k} = \delta_{k}(B) \implies d_{k} = \frac{\delta_{k}}{\delta_{k - 1}} \implies B' = \diag(\delta_{1}, \frac{\delta_{2}}{\delta_{1}}, \dots, \frac{\delta_{n}}{\delta_{n - 1}})
    \end{equation*}

    Причём при $i > j, \ \widehat{\text{Э}}_{1} (i, j, \lambda) \colon B \mapsto U B U^{T}$, где U --- единичная матрица с $\lambda$ на $i$-ой строке в $j$-ом столбце.
    \begin{equation*}
        C = U_{1}^{T} U_{2}^{T} \dots U_{S}^{T}
    \end{equation*}

    Очевидно, что перемножение матриц такого типа будет давать верхнюю унитреугольную матрицу $C$.
\end{proof}

\begin{comment}
    Матрица вида 
    $\begin{pmatrix}
        1 & \dots & \star \\
        \vdots & \ddots & \vdots \\
        0 & \dots & 1
    \end{pmatrix}$
    называется верхней унитреугольной
\end{comment}

\begin{comment}
    В доказательстве не использовалось свойство $1 + 1 \neq 0$, то есть свойство работает для любого поля.
\end{comment}


\subsection{Квадратичные формы на векторном пространстве}

Пусть $\beta \colon V \times V \to F$ --- билинейная форма на $V$.


\begin{definition}
    Отображение $Q_\beta \colon V \to F$, $Q_\beta(x) = \beta(x, x)$, называется \textit{квадратичной формой}, ассоциированной с билинейной формой $\beta$.
\end{definition}

Пусть $\E$ --- базис $V$, $x = x_1 e_1 + \dots x_n e_n$, $B = B(\beta, \E)$.

Тогда,
\begin{equation*}
    Q_\beta(x) = (x_1 \dots x_n) B \begin{pmatrix} x_1 \\ \dots \\ x_n \end{pmatrix} = \sum_{i = 1}^{n} \sum_{j = 1}^{n} b_{ij} x_i x_j = \sum_{i = 1}^{n} b_{ii} x_i^2 + \sum_{1 \leq i < j \leq n}^{n} (b_{ij} + b_{ji}) x_i x_j
.\end{equation*}


\subsection{Примеры}

\subsubsection{}

$V = F^n$, $\beta(x, y) = x_1 y_1 + \dots + x_n y_n \implies Q_\beta(x) = x_1^2 + \dots + x_n^2$.

\subsubsection{}

$V = F^2$, $\beta(x, y) = 2x_1 y_2 \implies Q_\beta(x) = 2x_1 x_2$.

Если $\E$ --- стандартный базис, то $B(\beta, \E) = \begin{pmatrix} 0 & 2 \\ 0 & 0 \end{pmatrix}$.

\subsubsection{}

$V = F^2$, $\beta(x, y) = x_1 y_2 + x_2 y_1 \implies Q_\beta(x) = 2x_1 x_2$.

Если $\E$ --- стандартный базис, то $B(\beta, \E) = \begin{pmatrix} 0 & 1 \\ 1 & 0 \end{pmatrix}$.


\subsection{Соответствие между симметричными билинейными формами и квадратичными формами}

\begin{proposal}
    Пусть в поле $F$ выполнено условие $1 + 1 \neq 0$. Тогда отображение $\beta \mapsto Q_\beta$ является биекцией между симметричными билинейными формами на $V$ и квадратичным формами на $V$.
\end{proposal}


\begin{proof}~
    \begin{description}
    \item[Сюръективность] $Q$ --- квадратичная форма $ \implies Q = Q_\beta$ для некоторой билинейной формы на $V$.

        То есть $Q(x) = \beta(x, x) \ \forall x \in V$.

        Положим $\sigma(x, y) = \frac{1}{2} \left[\beta(x, y) + \beta(y, x)\right]$, тогда $\sigma$ --- симметричная билинейная форма.
        \begin{equation*}
            Q_{\sigma} (x) =  \sigma(x, x) = \frac{1}{2}\left[\beta(x, x) + \beta(x, x)\right] = \beta(x, x) = Q_{\beta} (x) \implies Q_{\sigma} = Q_{\beta}
        .\end{equation*}

    \item[Инъективность] Пусть $Q_{\beta}$ --- квадратичная форма, соответствующая симметричной билинейной форме $\beta$.

        $Q_\beta(x + y) = \beta(x + y, x + y) = \underbrace{\beta(x, x)}_{Q_\beta(x)} + \underbrace{\beta(x, y) + \beta(y, x)}_{\text{равны между собой}} + \underbrace{\beta(y, y)}_{Q_\beta(y)} \implies \beta(x, y) = \frac{1}{2} \left[Q_\beta(x + y) - Q_\beta(x) - Q_\beta(y)\right]$.
    \end{description}
\end{proof}


\subsection{Симметризация билинейной формы и поляризация квадратичной формы}

\begin{comment}~
    \begin{enumerate}
        \item Билинейная форма $\sigma(x, y) = \frac{1}{2} \left(\beta(x, y) + \beta(y, x)\right)$ называется \textit{симметризацией} билинейной формы $\beta$.

            Если $B$ и $S$ --- матрицы билинейных форм $\beta$ и $\sigma$ в некотором базисе, то $S = \frac{1}{2} (B + B^{T})$.

        \item Симметричная билинейная форма $\beta(x, y) = \frac{1}{2} \left[Q(x + y) - Q(x) - Q(y)\right]$ называется \textit{поляризацией} квадратичной формы $Q$.
    \end{enumerate}
\end{comment}

Далее считаем, что $1 + 1 \neq 0$ в $F$ если не оговорено обратное.

\begin{definition}
    Матрицей квадратичной формы $Q$ в базисе $\E$ называется матрица соответствующей симметричной билинейной формы (поляризации) в базисе $\E$.

    Обозначение: $B(Q, \E)$.
\end{definition}

\begin{example}
    Пусть $Q(x_1, x_2) = x_1^2 + x_1 x_2 + x_2^2$.

    Если $\E$ --- стандартный базис, то $B(Q, \E) = \begin{pmatrix} 1 & \frac{1}{2} \\ \frac{1}{2} & 1 \end{pmatrix}$.
\end{example}


\subsection{Канонический вид квадратичной формы}

\begin{definition}
    Квадратичная форма $Q$ имеет в базисе $\E$ \textit{канонический вид}, если $B(Q, \E)$ диагональна.

    Если $B(Q, \E) = \diag(b_1, b_2, \dots, b_n)$, то $Q(x_1, \dots, x_n) = b_1 x_1^2 + b_2 x_2^2 + \dots + b_n x_n^2$.
\end{definition}


\subsection{Теорема о приведении квадратичной формы к каноническому виду.}

\begin{theorem}
    $\forall $ квадратичной формы $Q(x) \ \exists$ базис, в котором она принимает канонический вид.
\end{theorem}

\begin{proof}
    $\forall$ квадратичной формы $\exists$ ассоциированная с ней симметричная билинейная форма. Для симметричной билинейной формы по \hyperref[lec19:symmetric_bilinear_form_diagonalization]{теореме} $\exists$ базис, где она имеет диагональный вид. Соответственно, ассоциированная с ней квадратичная форма в этом базисе будет иметь канонический вид.
\end{proof}


\subsection{Нормальный вид квадратичной формы над полем $\RR$}

\begin{definition}
    Квадратичная форма над $\RR$ имеет \textit{нормальный вид} в базисе $\E$, если в этом базисе
    \begin{equation*}
        Q(x_1, \dots, x_n) = \epsilon_1 x_1^2 + \dots + \epsilon_n x_n^2
    ,\end{equation*}
    где $\epsilon_i \in \{-1, 0, 1\}$.
\end{definition}


\subsection{Приведение квадратичной формы над $\RR$ к нормальному виду}

\begin{corollary}
    Для всякой квадратичной формы $Q$ над $\RR$ существует базис, в котором $Q$ имеет нормальный вид.
\end{corollary}

\begin{proof}
    Знаем, что $\exists$ базис $\E$, такой что $B(Q, \E) = \diag(b_{1}, \dots, b_{n})$, то есть в координатах это $Q(x) = b_{1} x_{1}^{2} + \dots + b_{n} x_{n}^{2}$.

    Положим $C = \diag(c_{1}, \dots, c_{n})$, где
    \begin{equation*}
        c_i = \begin{cases}
            \frac{1}{\sqrt{|b_i|}}, &b_i \neq 0 \\
            1, &b_i = 0
        \end{cases}
    .\end{equation*}
    Тогда, взяв $B(Q, \E') = C^{T} B C = \diag(c_{1}^{2} b_{1}, \dots, c_{n}^{2} b_{n}) = \diag(\epsilon_{1}, \dots, \epsilon_{n})$, где
    \begin{equation*}
        \epsilon_i = \sgn b_i = \begin{cases}
            1, &b_i > 0 \\
            0, &b_i = 0 \\
            -1, &b_i < 0
        \end{cases}
    .\end{equation*}
    Это соответствует замене координат $x_i = c_i \cdot x_{i}'$
\end{proof}

\begin{comment}
    Для $F = \CC$ аналогичные рассуждения показывают, что $\exists$ базис, в котором $Q$ имеет вид \eqref{lec20:1}, где $k = \rk Q$.
    \begin{equation}
        \label{lec20:1}
        Q(x_{1}, \dots, x_{n}) = x_{1}^{2}, \dots, x_{k}^{2}
    .\end{equation}
\end{comment}

