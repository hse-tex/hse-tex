\section{Прямое произведение групп. Разложение конечной циклической группы. Теорема о строении конечных абелевых групп}

\begin{definition}
    \textit{Прямым произведением} групп $G_1, \dots, G_m$ называется множество
    \begin{equation*}
        G_1 \times \dots \times G_m = \{(g_1, \dots, g_m) \mid g_1 \in G_1 , \dots, g_m \in G_m\}
    \end{equation*}
    с операцией $(g_1, \dots, g_m)(g_1', \dots, g_m') = (g_1g_1', \dots, g_m g_m')$.
\end{definition}

Ясно, что эта операция ассоциативна, обладает нейтральным элементом $(e_{G_1}, \dots, e_{G_m})$ и для каждого элемента $(g_1, \dots, g_m)$ есть обратный элемент $(g_1^{-1}, \dots, g_m^{-1})$.

\begin{comment}
    Группа $G_1 \times \dots \times G_m$ коммутативна в точности тогда, когда коммутативна каждая из групп $G_1, \dots, G_m$.
\end{comment}

\begin{comment}
    Если все группы $G_1, \dots, G_m$ конечны, то $\left|G_1 \times \dots \times G_m\right| = \left|G_1\right| \cdots \left|G_m\right|$.
\end{comment}

\begin{definition}
    Группа $G$ \textit{раскладывается в прямое произведение} своих подгрупп $H_1, \dots, H_m$ если отображение $H_1 \times \dots \times H_m \to G$, $(h_1, \dots, h_m) \mapsto h_1 \cdots h_m$, является изоморфизмом.
\end{definition}

\begin{theorem}
    Пусть $n = ml$ --- разложение натурального числа $n$ на два взаимно простых сомножителя. Тогда имеет место изоморфизм групп
    \begin{equation*}
        \ZZ_n \simeq \ZZ_m \times \ZZ_l
    .\end{equation*}
\end{theorem}

\begin{proof}
    Рассмотрим отображение
    \begin{equation*}
        \phi \colon \ZZ_n \to \ZZ_m \times \ZZ_l, \quad \phi(a \mathop{\mathrm{mod}} {n}) = (a \mathop{\mathrm{mod}} m, a \mathop{\mathrm{mod}} l)
    .\end{equation*}

    \begin{enumerate}
    \item Корректность следует из того, что $n \divby m$, $n \divby l$;
    \item $\phi$ --- гомоморфизм:
        \begin{equation*}
            \phi((a + b) \bmod n) = \phi(a \bmod n) + \phi(b \bmod n)
        .\end{equation*}
    \item
        $\phi$ инъективен:

        Если $\phi(a \bmod n) = (0, 0)$, то $a$ делится на $m$ и на $l$.
        Но так как $\gcd(m, l) = 1$, получаем что $a$ кратно $n$.

        А значит $a \equiv 0 \pmod n$.
        То есть $\ker \phi = \{0\}$.

    \item $\phi$ сюръективно, так как $|\ZZ_n| = n = m \cdot l = |\ZZ_m \times \ZZ_l|$.
        \qedhere
    \end{enumerate}
\end{proof}

\begin{corollary}
    Пусть $n \geq 2$ --- натуральное число и $n = p_1^{k_1} \cdots p_s^{k_s}$ --- его разложение в произведение простых множителей (где $p_i \neq p_j$ при $i \neq j$). Тогда имеет место изоморфизм групп
    \begin{equation*}
        \ZZ_n \simeq \ZZ_{p_1^{k_1}} \times \dots \times \ZZ_{p_s^{k_s}} 
    .\end{equation*}
\end{corollary}

\begin{definition}
    Конечная абелева группа $A$ называется \textit{примарной}, если $\left|A\right| = p^{k}$, где $p$ --- простое и $k \in \NN$.
\end{definition}

\begin{theorem}
    Пусть $A$ --- конечная абелева группа. Тогда $A \simeq \ZZ_{p_1^{k_1}} \times \dots \times \ZZ_{p_t^{k_t}}$, где $p_1, \dots, p_t$ --- простые числа (не обязательно различные!) и $k_1, \dots, k_t \in \NN$. Более того, набор примарных циклических множителей $\ZZ_{p_1^{k_11}}, \dots, \ZZ_{p_t^{k_t}}$ определен однозначно с точностью до перестановки (в частности, число этих множителей определено однозначно). 
\end{theorem}
