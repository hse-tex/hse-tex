\section{Критерий того, что факторкольцо $\KK[x]/(h)$ является полем. Базис и размерность факторкольца $\KK[x]/(h)$ как векторного пространства над полем $\KK$}

$h=a_nx^n+...+a_1x+a_0\in K[x], \deg{h}>n>0$.\\
$F:=K[x]/(h)\quad f\in K[x]\leadsto \overline{f}:=f+(h)\in F$\\
$\overline{f}=\overline{0}\Leftrightarrow f\vdots h$\\
\textbf{Предложение.} $F$ -- поле $\Leftrightarrow$ h неприводим.
\begin{proof}
    $\Rightarrow$ Если $h=h_1\cdot h_2$, $\deg{h_i}<n\Rightarrow \overline{h}=\overline{h_1}\cdot \overline{h_2}$. Так как $\overline{h}=0$, то $\overline{h_1}\cdot\overline{h_2}=0\Rightarrow $ в F есть делители нуля $\Rightarrow $ F не поле.\\
    $\Leftarrow$ $f\in K[x], \overline{f}\neq\overline{0}\Rightarrow f\!\!\not\vdots\,h\Rightarrow \gcd(f,h)=1\Rightarrow \exists \ u,v\in K[x]: \ 1=uf+vh\Rightarrow \overline{1}=\overline{u}\overline{f}+\overline{v}\overline{h}=\overline{u}\overline{f}\Rightarrow$ $\overline{f}$ обратим, в силу произвольности выбора f все элементы обратимы $\Rightarrow$ F -- поле.
\end{proof}
\noindent Рассмотрим отображение $K\rightarrow F, \alpha\rightarrow \overline{\alpha}=\alpha+(h)$, оно инъективно $\Rightarrow$ K отождествляется с подполем в F $\Rightarrow$ F становится векторным пространством над K.\\
\textbf{Предложение.} Элементы $\overline{1},\overline{x},...,\overline{x}^{n-1}$ образуют базис в F над К. В частности, $\dim_KF=n$
\begin{proof}
    $\overline{f}\in F, f\in K[x]$. Поделим f на h с остатком:\\
    $f=q\cdot h+r,\begin{cases} r=0\\\deg{r}<n \end{cases}\Rightarrow \overline{f}=\overline{q}\cdot \overline{h}+\overline{r}=\overline{r}\in\langle \overline{1},\overline{x},...,\overline{x}^{n-1}\rangle$\\
    Если $b_0\overline{1}+b_1\overline{x}+...+b_{n-1}\overline{x}^{n-1}=0$ для некоторых $b_i\in K$, то $b_0x+...+b_{n-1}x^{n-1}\vdots h\\\Rightarrow b_0=...=b_{n-1}=0$
\end{proof}