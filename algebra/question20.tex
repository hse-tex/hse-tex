\section{Остаток многочлена относительно заданной системы многочленов. Системы Грёбнера. Характеризация систем Грёбнера в терминах цепочек элементарных редукций}

\begin{definition}
    Если $g \overset{F}{\rightsquigarrow} r$ и $r$ нередуцируем, то $r$ называется \textit{остатком} многочлена $g$ относительно $F$.
\end{definition}

\begin{comment}
    Вообще говоря, остаток определен неоднозначно.
\end{comment}

\begin{definition}
    Множество $F$ называется \textit{системой Грёбнера}, если $\forall g \in R$ остаток $g$ относительно $F$ определен однозначно, то есть не зависит от цепочки приводящих к нему элементарных преобразований.
\end{definition}

\begin{proposal}
    Следующие условия эквивалентны:
    \begin{enumerate}
    \item $F$ --- система Грёбнера;
    \item $\forall g \in R$ обладает свойством:

        Если $g \xrightarrow[]{f_1} g_1$ и $g \xrightarrow[]{f_2} g_2$ --- две элементарных редукции, то $\exists g' \in R$, такой что 
        \begin{math}
            g_1 \overset{F}{\rightsquigarrow} g' \text{ и } g_2 \overset{F}{\rightsquigarrow} g'.
        \end{math}
    \end{enumerate}
\end{proposal}

\begin{proof}~
    \begin{description}
        \item[$(1) \implies (2)$] В качестве $g'$ можно взять остаток $g$ относительно $F$.
        \item[$(2) \implies (1)$] Пусть

            $B(F) := $ <<все многочлены из $R$, для которых остаток относительно $F$ определен неоднозначно>>.

            $E_F(g) $ --- множество всех элементарных редукций многочлена $g$ относительно $F$.

            Пусть $B(F) \neq \varnothing$ и $g \in B(F)$.

            Если $E_F(g) \cap B(F) \neq \varnothing$, то возьмём $g_1 \in E_F(g) \cap B(F)$.

            Если $E_F(g_1) \cap B(F) \neq \varnothing$, то возьмём $g_2 \in E_F(g_1) \cap B(F)$.

            И так далее.

            По лемме о конечности цепочек элементарных редукций, $\exists i \in \NN$, такой что $E_F(g_i) \cap B(F) = \varnothing$.

            Тогда $\exists$ две такие цепочки элементарных редукций

            \begin{equation*}
                \left.
                \begin{array}{c}
                    g_i \to h_1 \to \dots \to r_1 \\
                    g_i \to h_2 \to \dots \to r_2
                \end{array}
                \right\} \text{ остатки } r_1 \neq r_2
            .\end{equation*}

            По условию $\exists r \in R$ --- нередуцируемый многочлен, такой что $h_1 \rightsquigarrow r$, $h_2 \rightsquigarrow r$.

            Так как $h_1, h_2 \not\in B(F)$, то $r_1 = r = r_2$ --- противоречие.
            \qedhere
    \end{description}
\end{proof}

\begin{comment}~
    \begin{enumerate}
        \item $g \overset{F}{\rightsquigarrow} g_1 \implies \forall m \in M \quad mg \overset{F}{\rightsquigarrow} mg_1$;
        \item $g_1 - g_2 \overset{F}{\rightsquigarrow} 0 \implies \exists g' : g_1 \rightsquigarrow g'$ и $g_2 \rightsquigarrow g'$.
    \end{enumerate}
\end{comment}
