\section{Теорема о том, что кольцо многочленов от одной переменной над полем является кольцом главных идеалов}

\begin{definition}
    Коммутативное кольцо $R$ без делителей нуля называется \textit{кольцом главных идеалов} (КГИ), если всякий идеал в $R$ является главным.
\end{definition}

\begin{example}
    $\ZZ$ --- все идеалы это $k\ZZ = (k)$ ($k \geq 0$) --- главные.
\end{example}

\begin{proposal}
    $K[x]$ --- КГИ.
\end{proposal}

\begin{proof}
    Пусть $I \lhd K[x]$. Если $I = \{0\}$, то $I = (0)$ --- главный.

    Если $I \neq \{0\}$, то выберем в $I$ многочлен наименьшей степени $g \neq 0$.

    Тогда $(g) \subseteq I$. Пусть $f \in I$, разделим $f$ на $g$ с остатком:

    $f = q \cdot g + r$, где либо $r = 0$, либо $\deg r < \deg g$. Но тогда $r = f - q \cdot g \in I$.

    Так как $\deg g$ минимально, то $r = 0 \implies f \in (g) \implies I \subseteq (g)$.

    Итог: $I = (g)$.
\end{proof}
