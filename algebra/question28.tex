\section{Алгебраические и трансцендентные элементы. Минимальный многочлен алгебраического элемента и его свойства}

$K\subseteq F$\\
\textbf{Определение.} Элемент $\alpha\in F$ называется \textbf{алгебраическим} над K, если $\exists\ f\in K[x], \deg{f}\geqslant 1$, такой что $f(\alpha)=0$ и \textbf{трансцендентным} иначе.\\\\
\textbf{Определение.} \textbf{Минимальным многочленом} элемента $\alpha\in F$, алгебраического над $K$, называется такой $h\in K[x],\deg{h}\geqslant 1$, что $h(\alpha)=0$ и $h$ имеет минимальную степень.\\\\
\textbf{Свойства минимального многочлена}\\
Пусть $K\subseteq F$ -- расширение полей, $\alpha \in F$ -- элемент, алгебраический над K, и $h\in K[x]$ -- его минимальный многочлен. Тогда:\\
\indent 1) h определён однозначно с точностью до пропорциональности.\\
\indent 2) Для всякого $f\in K[x]$ имеем $f(\alpha)=0\Leftrightarrow f\vdots h$\\
\indent 3) h неприводим над K
\begin{proof}
    Положим $I=\{ f\in K[x] \ | \ f(\alpha)=0\}$. Тогда $I$ -- идеал в $K[x]$. \\
    Так как $K[x]$ -- КГИ, то $\exists \ g\in I: I=(g)$.\\
    $h(\alpha)=0\Rightarrow h\in I\Rightarrow h\vdots g\Rightarrow$ $h$ пропорционален $g$ в силу минимальности $\Rightarrow (1)$ и $(2)$\\
    (3) Если $h=h_1h_2$, $\deg{h_i}<\deg{h}, \ i=1,2$. Тогда либо $h_1(\alpha)=0$ либо $h_2(\alpha)=0$, ну а это противоречие, так как мы выбирали минимальный h. 
\end{proof}