\documentclass[a4paper]{article}
\usepackage{../source/header}


\newcommand\enumtocitem[3]{\item\textbf{#1}\addtocounter{#2}{1}\addcontentsline{toc}{#2}{\protect{\numberline{#3}} #1}}
\newcommand\defitem[1]{\enumtocitem{#1}{subsection}{\thesubsection}}
\newcommand\proofitem[1]{\enumtocitem{#1}{subsubsection}{\thesubsubsection}}

\newlist{colloq}{enumerate}{1}
\setlist[colloq]{label=\textbf{\arabic*.}}


\title{\HugeЛинейная алгебра, Коллоквиум}
\author{
	Бобень Вячеслав \\
	\href{https://teleg.run/darkkeks}{@darkkeks},
    \href{https://github.com/LoDThe/hse-tex}{GitHub} \\
    Большую часть исходного кода предоставила Левина Александра. \\
    Благодарность выражается Левину Александру за видеозаписи лекций.
}
\date{2019 --- 2020}

\begin{document}
    \maketitle

    \epigraph{
        ``К коллоку можете даже не готовиться''.
    }{\rightline{{\rm --- Роман Сергеевич Авдеев}}}

    \tableofcontents

    \newpage

    \section{Определения и формулировки}

    \begin{colloq}
    % Лекция 1
    \defitem{Сумма двух матриц, произведение матрицы на скаляр}

        Для любых $A, B \in \text{Mat}_{m \times n}$
        \begin{itemize}
            \item \emph{Сложение} $A + B := (a_{ij} + b_{ij})$
            \item \emph{Умножение на скаляр} $\alpha \in \RR \implies \lambda A := (\lambda a_{ij})$
        \end{itemize}

    % Лекция 1
    \defitem{Транспонированная матрица} 

        $A \in \text{Mat}_{m \times n} = \begin{pmatrix}
            a_{11} & a_{12} & \dots & a_{1n} \\
            a_{21} & a_{22} & \dots & a_{2n} \\
            \vdots & \vdots & \ddots & \vdots \\
            a_{m1} & a_{m2} & \dots & a_{mn}
        \end{pmatrix} \leadsto A^T \in \text{Mat}_{n \times m} := \begin{pmatrix}
            a_{11} & a_{21} & \dots & a_{m1} \\
            a_{12} & a_{22} & \dots & a_{m2} \\
            \vdots & \vdots & \ddots & \vdots \\
            a_{1n} & a_{2n} & \dots & a_{mn}
        \end{pmatrix}$

        $A^T$ ---  \textit{транспонированная матрица}

    % Лекция 1
    \defitem{Произведение двух матриц}

        Пусть $A = (a_{ij}) \in \text{Mat}_{m \times n}$

        \bigskip

        $A_{(i)} = \begin{pmatrix} a_{i1}, a_{i2}, \dots, a_{in} \end{pmatrix} $ --- $i$-я строка матрицы $A$

        $A^{(j)} = \begin{pmatrix} a_{1j} \\ a_{2j} \\ \vdots \\ a_{mn} \end{pmatrix} $ --- $j$-й  столбец матрицы $A$

        \begin{enumerate}[label=\arabic*)]
            \item 
                Частный случай: умножение строки на столбец той же длинны
            
                $\underbrace{(x_1, \dots, x_n)}_{1 \times n} 
                \underbrace{\begin{pmatrix}
                    y_1 \\ \vdots \\ y_n
                \end{pmatrix}}_{n \times 1} 
                = x_1 \cdot y_1 + \dots + x_n \cdot y_n$
                
            \item
                Общий случай:

                $A$ - матрица размера $m \times \underline{n}$
                
                $B$ - матрица размера $\underline{n} \times p$
                
                Количество строк матрицы $A$ равно количеству столбцов матрицы $B$ --- условие согласованности матриц
                
                $AB := C \in \text{Mat}_{m \times p}$, где $C_{ij} = A_{(i)} B^{(j)}$
        \end{enumerate}

    % Лекция 2
    \defitem{Диагональная матрица, умножение на диагональную матрицу слева и справа}

        \begin{definition}
            Матрица $A \in M_n$ называется \textit{диагональной} если все ее элементы вне главной диагонали равны нулю ($a_{ij} = 0$ при $i \neq j$)
        \end{definition}

        \begin{lemma}
            $A = diag(a_1, \dots, a_n) \in M_n \implies$
            \begin{enumerate}
            \item $\forall B \in \text{Mat}_{n \times p} \implies AB = \begin{pmatrix}
                    a_1 B_{(1)} \\
                    a_2 B_{(2)} \\
                    \vdots \\
                    a_n B_{(n)} 
                \end{pmatrix}$
            \item $\forall B \in \text{Mat}_{m \times n} \implies BA = \begin{pmatrix} 
                    a_1 B^{(1)} & a_2 B^{(2)} & \dots & a_n B^{(n)}
                \end{pmatrix}$ 
            \end{enumerate}
        \end{lemma}

    % Лекция 2
    \defitem{Единичная матрица, её свойства}

        \begin{definition}
            Матрица $E = E_n = diag(1, 1, \dots, 1)$ называется \textit{единичной матрицей} порядка $n$. 

            \begin{equation*}
                E = \begin{pmatrix} 
                    1 & 0 & \dots & 0 \\
                    0 & 1 & \dots & 0 \\
                    \vdots & \vdots & \ddots & \vdots \\
                    0 & 0 & \dots & 1
                \end{pmatrix} 
            \end{equation*}
        \end{definition}

        \textbf{Свойства}:
        \begin{enumerate}
            \item $EA = A \quad \forall A \in \text{Mat}_{n \times p}$
            \item $AE = A \quad \forall A \in \text{Mat}_{p \times n}$
            \item $AE = EA = A \quad \forall A \in M_n$
        \end{enumerate}

    % Лекция 2
    \defitem{След квадратной матрицы и его поведение при сложении матриц, умножении матрицы на скаляр и транспонировании}

        \begin{definition}
            \textit{Следом} матрицы $A \in M_n$ называется число $trA = a_{11} + a_{22} + \dots + a_{nn} = \sum_{i=1}^n a_{ii}$ 
        \end{definition}

        \textbf{Свойства}:
        \begin{enumerate}
        \item $\tr(A + B) = \tr A + \tr B$
        \item $\tr \lambda A = \lambda \tr A$
        \item $\tr A^T = \tr A$
        \end{enumerate}

    % Лекция 2
    \defitem{След произведения двух матриц}

        $\tr(AB) = \tr(BA)$

        $\forall A \in \text{Mat}_{m \times n}, B \in \text{Mat}_{n \times m}$

    % Лекция 2
    \defitem{Совместные и несовместные системы линейных уравнений}

        \begin{definition}
            СЛУ называется 

            -- \textit{совместной}, если у нее есть хотя бы одно решение

            -- \textit{несовместной}, если решений нет
        \end{definition}

    % Лекция 3
    \defitem{Эквивалентные системы линейных уравнений}

        \begin{definition}
            Две системы уравнений от одних и тех же неизвестных называются \textit{эквивалентными}, если они имеют одинаковые множества решений.
        \end{definition}

    % Лекция 3
    \defitem{Расширенная матрица системы линейных уравнений}

        $Ax = b$, $A \in \text{Mat}_{m \times n}$, $b \in \RR^m$

        Полная информация о СЛУ содержится в её \textit{расширенной матрице}.
        \begin{equation*}
            \begin{pmatrix} A \mid b \end{pmatrix} = \begin{amatrix}{4}{1}
            a_{11} & a_{12} & \dots & a_{1n} & b_1 \\
            a_{21} & a_{22} & \dots & a_{2n} & b_2 \\
            \vdots & \vdots & \ddots & \vdots \\
            a_{m1} & a_{m2} & \dots & a_{mn} & b_m
            \end{amatrix}
        \end{equation*}

    % Лекция 3
    \defitem{Элементарные преобразования строк матрицы}

        \begin{tabular}{c|c|c}
            тип & СЛУ & расширенная матрица \\
            \hline
            1. & K $i$-му уравнению прибавить $j$-ое, умноженное на $\lambda \in \RR \ (i \neq j)$ & $\text{Э}_1(i, j, \lambda)$ \\
            2. & Переставить $i$-е и $j$-е уравнения $(i \neq j)$ & $\text{Э}_2(i, j)$ \\
            3. & Умножить $i$-ое уравнение на $\lambda \neq 0$ & $\text{Э}_3(i, \lambda)$
        \end{tabular}

        \begin{enumerate}
        \item
            $\text{Э}_1(i, j, \lambda)$: к $i$-ой строке прибавить $j$-ую, умноженную на $\lambda$ (покомпонентно), 

            $a_{ik} \mapsto a_{ik} + \lambda a_{jk} \ \forall k = 1, \dots, n$,
            
            $b_i \mapsto b_i + \lambda b_j$.
            
        \item
            $\text{Э}_2(i, j)$: переставить i-ую и j-ую строки.

        \item
            $\text{Э}_3(i, \lambda)$: умножить i-ю строку на $\lambda$ (покомпонентно).
        \end{enumerate}

        $\text{Э}_1, \text{Э}_2, \text{Э}_3$ называются \textit{элементарными преобразованиями строк расширенной матрицы}.

    % Лекция 3
    \defitem{Ступенчатый вид матрицы}

        \begin{definition}
            Матрица $M \in \text{Mat}_{m \times n}$ называется \textit{ступенчатой}, или имеет ступенчатый вид, если:
            \begin{enumerate}
            \item Номера ведущих элементов её ненулевых строк строго возрастают.
            \item Все нулевые строки стоят в конце.
            \end{enumerate}
        \end{definition}
        \begin{equation*}
            M = \begin{pmatrix}
                0 & \dots & 0 & \diamond & * & * & * & * & * & * \\
                0 & \dots & 0 & 0 & \dots & \diamond & * & * & * & * \\
                0 & \dots & 0 & 0 & \dots & 0 & 0 & \diamond & * & * \\
                \vdots & \ddots & \vdots & \vdots & \ddots & \vdots & \vdots & \vdots & \vdots & \vdots \\
                0 & \dots & 0 & 0 & \dots & 0 & 0 & 0 & \diamond & * \\
                0 & \dots & 0 & 0 & \dots & 0 & 0 & 0 & 0 & \diamond \\
                0 & \dots & 0 & 0 & \dots & 0 & 0 & 0 & 0 & 0
            \end{pmatrix}
        ,\end{equation*}
        где $\diamond \neq 0$, $*$ -- что угодно. 

    % Лекция 3
    \defitem{Улучшенный ступенчатый вид матрицы}

        \begin{definition}
            M имеет \textit{улучшенный ступенчатый вид}, если:

            \begin{enumerate}[nosep]
            \item M имеет обычный ступенчатый вид.
            \item Все ведущие элементы равны 1.
            \item В одном столбце с любым ведущим элементом стоят только нули.
            \end{enumerate}
        \end{definition}

        \begin{equation*}
            M = \begin{pmatrix}
                0 & \dots & 0 & 1 & * & 0 & * & 0 & 0 & * \\
                0 & \dots & 0 & 0 & \dots & 1 & * & 0 & 0 & * \\
                0 & \dots & 0 & 0 & \dots & 0 & 0 & 1 & 0 & * \\
                \vdots & \vdots & \vdots & \vdots & \vdots & \vdots & \vdots & \vdots & \vdots & \vdots \\
                0 & 0 & 0 & \dots & \dots & \dots & \dots & 0 & 1 & * \\
                0 & 0 & 0 & \dots & 0 & 0 & 0 & 0 & 0 & 0
            \end{pmatrix}
        .\end{equation*}

    % Лекция 3
    \defitem{Теорема о виде, к которому можно привести матрицу при помощи элементарных преобразований строк}

        \begin{theorem}~
            \begin{enumerate}[label=\arabic*),nosep]
            \item Всякую матрицу элементарными преобразованиями можно привести к ступенчатому виду. 
            \item Всякую ступенчатую матрицу элементарными преобразованиями строк можно привести к улучшенному ступенчатому виду.
            \end{enumerate}
        \end{theorem}

        \begin{corollary}
            Всякую матрицу элементарными преобразованиями строк можно привести к \textbf{улучшенному} ступенчатому виду.
        \end{corollary}

    % Лекция 4
    \defitem{Общее решение совместной системы линейных уравнений} 
        % TODO

    % Лекция 4
    \defitem{Сколько может быть решений у системы линейных уравнений с действительными коэффициентами?} 
        % TODO

    % Лекция 4
    \defitem{Однородная система линейных уравнений. Что можно сказать про её множество решений?}
        
        \begin{fact}
            Всякая ОСЛУ имеет нулевое решение $(x_1 = x_2 = \dots = x_n = 0)$.
        \end{fact}

        \begin{corollary}
            Всякая ОСЛУ либо имеет ровно 1 решение (нулевое), либо бесконечно много решений.
        \end{corollary}

    % Лекция 4
    \defitem{Свойство однородной системы линейных уравнений, у которой число неизвестных больше числа уравнений}
        
        \begin{corollary}
            Всякая ОСЛУ, у которой число неизвестных больше числа уравнений, имеет ненулевое решение
        \end{corollary}

    % Лекция 4
    \defitem{Связь между множеством решений совместной системы линейных уравнений и множеством решений соответствующей ей однородной системы}

        Пусть дана совместная СЛУ $Ax = b$

        Частное решение СЛУ --- это какое-то одно её решение.

        \begin{proposition}
            Пусть $Ax = b$ -- совместная СЛУ.

            $x_0$ -- частное решение $Ax + b$

            $S \subset \RR^n$ -- множество решений ОСЛУ $Ax = 0$

            $L \subset \RR^n $ -- множество решений $Ax = b$.

            Тогда, $L = x_0 + S$, где $x_0 + S = \{x_0 + v \mid v \in S\}$
        \end{proposition}

    % Лекция 4
    \defitem{Обратная матрица}

        \begin{definition}
            Матрица $B \in M_n$ называется \textit{обратной}, к $A$, если $AB = BA = E$.

            Обозначение: $B = A^{-1}$
        \end{definition}

    % Лекция 4
    \defitem{Перестановки множества $\{1, 2, . . . , n\}$}

        \begin{definition}
            \textit{Перестановкой множества} $\{1, 2, \dots, n\}$ называется упорядоченный набор $(i_1, i_2, \dots, i_n)$, в котором каждое число от 1 до $n$ встречается ровно один раз. 
        \end{definition}

        Обозначение: $P_n$ -- множество всех перестановок множества $\{1, 2, \dots, n\}$.

        Например, $(4, 2, 1, 3) \in P_4$.

        \begin{definition}
            \textit{Подстановкой} на множестве $\{1, 2, \dots, n\}$ называется всякое биективное (взаимно однозначное) отображение множества $\{1, 2, \dots, n\}$ в себя.

            \begin{equation*}
                \sigma : \{1, 2, \dots, n\} \to \{1, 2, \dots, n\}
            .\end{equation*}

            \begin{equation*}
                \begin{pmatrix}
                    1 & 2 & 3 & \dots & n \\
                    i_1 & i_2 & i_3 & \dots i_n
                \end{pmatrix}
            \end{equation*}
        \end{definition}

    % Лекция 5
    \defitem{Инверсия в перестановке. Знак перестановки. Чётные и нечётные перестановки}

        Пусть $\sigma \in S_n$, $i, j \in \{1, 2, \dots, n\}$, $i \neq j$

        \begin{definition}
            Пара $\{i, j\}$ (неупорядоченная) образует \textit{инверсию} в $\sigma$, если числа $i - j$ и $\sigma(i) - \sigma(j)$ имеют разный знак (то есть либо $i < j$ и $\sigma(i) > \sigma(j)$, либо $i > j$ и $\sigma(i) < \sigma(j)$).
        \end{definition}

        \begin{definition}
            \textit{Знак} подстановки $\sigma$ -- это  число $\sgn(\sigma) = (-1)^{<\text{число инверсий в }\sigma>}$.
        \end{definition}

        \begin{definition}
            $\sigma$ называется \textit{четной}, если $\sgn(\sigma)$ = 1 (четное количество инверсий), и \textit{нечетной} если $\sgn(\sigma) = -1$ (нечетное количество инверсий).
        \end{definition}

    % Лекция 5
    \defitem{Произведение двух перестановок}

        \begin{definition}
            \textit{Произведением} (или \textit{композицией}) двух подстановок $\sigma, \rho \in S_n$ называется такая постановка $\sigma \rho \in S_n$, что $(\sigma \rho)(x) := \sigma (\rho (x))$ $\forall x \in \{1, \dots, n\}$.
        \end{definition}

    % Лекция 5
    \defitem{Тождественная перестановка и её свойства. Обратная перестановка и её свойства}

        \begin{definition}
            Подстановка $id = \begin{pmatrix}
                1 & 2 & \dots & n \\
                1 & 2 & \dots & n
            \end{pmatrix} \in S_n$ называется \textit{тождественной} перестановкой.
        \end{definition}

        \textbf{Свойства:}

        $\forall \sigma \in S_n \quad id \cdot \sigma = \sigma \cdot id = \sigma$.

        $\sgn(id) = 1$.


        \begin{definition}
            $\sigma \in S_n$, $\sigma = \begin{pmatrix}
                1 & 2 & \dots & n \\
                \sigma(1) & \sigma(2) & \dots & \sigma(n)
            \end{pmatrix} \implies$ подстановка $\sigma^{-1} := \begin{pmatrix}
                \sigma(1) & \sigma(2) & \dots & \sigma(n) \\
                1 & 2 & \dots & n
            \end{pmatrix}$ называется \textit{обратной} к $\sigma$ перестановкой.
        \end{definition}

        \textbf{Свойства:}
        $\sigma \cdot \sigma^{-1} = id = \sigma^{-1} \cdot \sigma$

    % Лекция 5
    \defitem{Теорема о знаке произведения двух перестановок}

        \begin{theorem}
            $\sigma, \rho \in S_n \implies \sgn(\sigma \rho) = \sgn \sigma \cdot \sgn \rho$.
        \end{theorem}

    \defitem{Транспозиция. Знак транспозиции}

        Пусть $i, j \in \{1, 2, \dots, n\}$, $i \neq j$.

        Рассмотрим перестановку $\tau_{ij} \in S_n$, такую что

        $\tau_{ij}(i) = j$.

        $\tau_{ij}(j) = i$.

        $\tau_{ij}(k) = k \ \forall k \neq i, j$.

        \begin{definition}
            Подстановки вида $\tau_{ij}$ называются \textit{транспозициями}.
        \end{definition}

        \begin{comment}
            $\tau$ -- траспозиция $\implies \tau^2 = id, \tau^{-1} = \tau$.
        \end{comment}

        \begin{lemma}
            $\tau \in S_n$ -- транспозиция $\implies \sgn(\tau) = -1$.
        \end{lemma}

    \defitem{Общая формула для определителя квадратной матрицы произвольного порядка}

        \begin{definition}
            Определителем матрицы $A \in M_n$ называется число
            \begin{equation*}
                \det A = \sum_{\sigma \in S_n} \sgn(\sigma) a_{1\sigma(1)} a_{2 \sigma(2)} \dots a_{n\sigma(n)}
            .\end{equation*}

            ($\sum_{\sigma \in S_n}$ -- сумма по всем перестановкам)
        \end{definition}

    % Лекция 5
    \defitem{Определители 2-го и 3-го порядка}

        \begin{itemize}
        \item
            $n = 2$

            $S_2 = \left\{ \begin{pmatrix} 1 & 2 \\ 1 & 2 \end{pmatrix}, \begin{pmatrix} 1 & 2 \\ 2 & 1 \end{pmatrix} \right\}$

            $A = \begin{pmatrix} a_{11} & a_{12} \\ a_{21} & a_{22} \end{pmatrix} \implies \det A = (+1) a_{11} a_{22} + (-1) a_{12} a_{21} = a_{11} a_{22} - a_{12} a_{21}$

        \item
            $n = 3$

            $S_3 = \left\{
            \begin{pmatrix} 1 & 2 & 3 \\ 1 & 2 & 3 \end{pmatrix},
            \begin{pmatrix} 1 & 2 & 3 \\ 2 & 3 & 1 \end{pmatrix},
            \begin{pmatrix} 1 & 2 & 3 \\ 3 & 1 & 2 \end{pmatrix},
            \begin{pmatrix} 1 & 2 & 3 \\ 3 & 2 & 1 \end{pmatrix},
            \begin{pmatrix} 1 & 2 & 3 \\ 2 & 1 & 3 \end{pmatrix},
            \begin{pmatrix} 1 & 2 & 3 \\ 1 & 3 & 2 \end{pmatrix} \right\}$

            $\det A = \begin{vmatrix} a_{11} & a_{12} & a_{13} \\ a_{21} & a_{22} & a_{23} \\ a_{31} & a_{32} & a_{33} \end{vmatrix} = a_{11} a_{22} a_{33} + a_{12} a_{23} a_{31} + a_{13} a_{21} a_{32} - a_{13} a_{22} a_{31} - a_{12} a_{21} a_{33} - a_{11} a_{23} a_{32}$.
        \end{itemize}

    % Лекция 6
    \defitem{Поведение определителя при разложении строки (столбца) в сумму двух}

        Если $A_{(i)} = A_{(i)}^1 + A_{(i)}^2$, то $\det A = \det \begin{pmatrix}
            A_{(1)} \\ \vdots \\ A_{(i)}^1 \\ \vdots \\ A_{(n)}
        \end{pmatrix} + \det \begin{pmatrix}
            A_{(1)} \\ \vdots \\ A_{(i)}^2 \\ \vdots \\ A_{(n)}
        \end{pmatrix}$.        

        \bigskip
        Пример:
        \begin{equation*}
            \begin{vmatrix}
                a_1 & a_2 & a_3 \\
                b_1 + c_1 & b_2 + c_2 & b_3 + c_3 \\
                d_1 & d_2 & d_3
            \end{vmatrix} = \begin{vmatrix}
                a_1 & a_2 & a_3 \\
                b_1 & b_2 & b_3 \\
                d_1 & d_2 & d_3
            \end{vmatrix} + \begin{vmatrix}
                a_1 & a_2 & a_3 \\
                c_1 & c_2 & c_3 \\
                d_1 & d_2 & d_3
            \end{vmatrix}
        \end{equation*}

        Аналогично, если $A^{(j)} = A^{(j)}_1 + A^{(j)}_2$, то $\det A = \det (A^{(1)} \cdots A^{(j)}_1 \cdots A^{(n)}) + \det (A^{(1)} \cdots A^{(j)}_2 \cdots A^{(n)})$.

    % Лекция 6
    \defitem{Поведение определителя при перестановке двух строк (столбцов)}

        Если в $A$ поменять местами две строки или два столбца, то $\det A$ поменяет знак.

    % Лекция 6
    \defitem{Поведение определителя при прибавлении к строке (столбцу) другой, умноженной на скаляр}

        Если к строке (столбцу) прибавить другую строку (столбец), умноженный на скаляр, то $\det A$ не изменится.

    % Лекция 6
    \defitem{Верхнетреугольные и нижнетреугольные матрицы}

        \begin{definition}
            Матрица называется \textit{верхнетреугольной}, если $a_{ij} = 0$ при $i > j$, \textit{нижнетреугольной}, если $a_{ij} = 0$ $i < j$.
        \end{definition}
        \begin{equation*}
            \begin{pmatrix}
                a_{11} & a_{12} & a_{13} & \dots & a_{1n} \\
                0 & a_{22} & a_{23} & \dots & a_{2n} \\
                0 & 0 & a_{33} & \dots & a_{3n} \\
                \vdots & \vdots & \vdots & \ddots & \vdots \\
                0 & 0 & 0 & \dots & a_{mn}
            \end{pmatrix} \text{ -- верхнетреугольная}
        \end{equation*}

        \begin{equation*}
            \begin{pmatrix}
                a_{11} & 0 & 0 & \dots & 0 \\
                a_{21} & a_{22} & 0 & \cdots & 0 \\
                a_{31} & a_{32} & a_{33} & \cdots & 0 \\
                \vdots & \vdots & \vdots & \ddots & \vdots \\
                a_{m1} & a_{m2} & a_{m3} & \cdots & a_{mn}
            \end{pmatrix} \text{ -- нижнетреугольная}
        \end{equation*}

    % Лекция 6
    \defitem{Определитель верхнетреугольной (нижнетреугольной) матрицы}

        Если $A$ верхнетреугольная или нижнетреугольная, то $\det A = a_{11} a_{22} \dots a_{nn}$.

    % Лекция 6
    \defitem{Определитель диагональной матрицы. Определитель единичной матрицы}

        Так как матрица диагональна, она верхнетреугольна. Тогда, её определитель равен произведению элементов на диагонали:

        $\det A = a_{11} \cdot a_{22} \cdot \dots \cdot a_{nn}$.

        Значит, определитель единичной матрицы -- 1.

        $\det E = 1 \cdot 1 \cdot \dots \cdot 1 = 1$.

    % Лекция 7
    \defitem{Матрица с углом нулей и её определитель}

        \begin{proposal}
            \begin{equation*}
                A = \left(
                    \begin{array}{c|c}
                        P & Q \\
                        \hline
                        0 & R
                    \end{array}
                \right) \text{ или } A =
                \left(
                    \begin{array}{c|c}
                        P & 0 \\
                        \hline
                        Q & R
                    \end{array}
                \right), \ P \in M_k, \ R \in M_{n-k} \implies \det A = \det P \det R
            .\end{equation*}
        \end{proposal}

        Матрица с углом нулей:
        \begin{equation*}
        \left(
        \begin{array}{c|ccc}
          * & * & * & * \\
          \hline
          0 & * & * & * \\
          0 & * & * & * \\
          0 & * & * & *
        \end{array}
        \right)
        \end{equation*}

        НЕ матрица с углом нулей:
        \begin{equation*}
        \left(
        \begin{array}{c|ccc}
          * & * & * & * \\
          * & * & * & * \\
          \hline
          0 & * & * & * \\
          0 & * & * & *
        \end{array}
        \right)
        \end{equation*}

    % Лекция 7
    \defitem{Определитель произведения двух матриц}

        \begin{theorem}
            $A, B \in M_n \implies \det(AB) = \det A \det B$.
        \end{theorem}

    % Лекция 7
    \defitem{Дополнительный минор к элементу квадратной матрицы}

        \begin{definition}
            \textit{Дополнительным минором} к элементу $a_{ij}$ называется определитель $(n-1) \times (n-1)$ матрицы, получающейся из $А$ вычеркиванием $i$-ой строки и $j$-го столбца.

            Обозначение: $\overline{M}_{ij}$.
        \end{definition}        

    % Лекция 7
    \defitem{Алгебраическое дополнение к элементу квадратной матрицы}

        \begin{definition}
            \textit{Алгебраическим дополнением к} элементу $a_{ij}$ называется число $A_{ij} = (-1)^{i+j} \overline{M}_{ij}$.
        \end{definition}

    % Лекция 7
    \defitem{Формула разложения определителя по строке (столбцу)}

        \begin{theorem}
            При любом фиксированном $i \in \{1, 2, \dots, n\}$,
            \begin{equation*}
                \det A = a_{i1} A_{i1} + a_{i2} A_{i2} + \dots + a_{in} A_{in} = \sum_{j = 1}^n a_{ij} A_{ij} \text{ -- разложение по i-й строке}
            .\end{equation*}

            Аналогично, для любого фиксированного $j \in \{1, 2, \dots, n\}$,
            \begin{equation*}
                \det A = a_{1j} A_{1j} + a_{2j} A_{2j} + \dots + a_{nj} A_{nj} = \sum_{i = 1}^{n} a_{ij} A_{ij} \text{ -- разложение по j-у столбцу}
            .\end{equation*}
        \end{theorem}

    % Лекция 7
    \defitem{Лемма о фальшивом разложении определителя}

        \begin{lemma}~
            \begin{enumerate}
            \item
                При любых $i, k \in \{1, 2, \dots, n\} : i \neq k \implies \sum_{j = 1}^n a_{ij} A_{kj} = 0$.
            \item
                При любых $j, k \in \{1, 2, \dots, n\} : j \neq k \implies \sum_{i = 1}^n a_{ij} A_{ik} = 0$
            \end{enumerate}
        \end{lemma}

    % Лекция 7
    \defitem{Невырожденная матрица}

        \begin{definition}
            Матрица $A \in M_n$ называется \textit{невырожденной}, если $\det A \neq 0$, и \textit{вырожденной} иначе (то есть $\det A = 0$).
        \end{definition}

    % Лекция 7
    \defitem{Присоединённая матрица}

        \begin{definition}
            \textit{Присоединенной} к А матрицей называется матрица $\widehat{A} = (A_{ij})^T$.
        \end{definition}

    % Лекция 7
    \defitem{Критерий обратимости квадратной матрицы}

        \begin{theorem}
            $A$ обратима (то есть $\exists A^{-1}$) $\iff$ $A$ невырождена ($\det A \neq 0$).
        \end{theorem}

    % Лекция 7
    \defitem{Явная формула для обратной матрицы}

        $A^{-1} = \frac{1}{\det A} \widehat{A}$

    \defitem{Критерий обратимости произведения двух матриц. Матрица, обратная к произведению двух матриц}
        % TODO

    % Лекция 8
    \defitem{Формулы Крамера}

        Пусть есть СЛУ $Ax = b (\star)$, $A \in M_n$, $x = \begin{pmatrix} x_1 \\ \dots \\ x_n \end{pmatrix} \in \RR^n$, $b = \begin{pmatrix} b_1 \\ \dots \\ b_n \end{pmatrix} \in \RR^n$.

        Также, $\forall i \in \{1, 2, \dots, n\}$, $A_i = (A^{(1)}, \dots, A^{(i - 1)}, b, A^{(i + 1)}, \dots, A^{(n)})$.

        \begin{theorem}
            Если $\det A \neq 0$, то СЛУ ($\star$) имеет единственное решение и его можно найти по формулам:
            \begin{equation*}
                x_i = \frac{\det A_i}{\det A}
            .\end{equation*}
        \end{theorem}

    % Лекция 8
    \defitem{Что такое поле?}

        \begin{definition}
            \textit{Полем} называется множество $F$, на котором заданы две операции ``сложение'' ($(a, b) \to a + b$) и ``умножение'' ($(a, b) \to a \cdot b$), причем $\forall a, b, c \in F$ выполнены следующие условия:

            \begin{enumerate}[nosep]
                \item $a + b = b + a$ (коммутативность сложения)
                \item $(a + b) + c = a + (b + c)$ (ассоциативность сложения)
                \item $\exists 0 \in F : 0 + a = a + 0 = a$ (нулевой элемент)
                \item $\exists (-a) \in F: a+(-a)=(-a)+a=0$ (противоположный элемент)

                    $\uparrow$ абелева группа $\uparrow$
                \item $a(b+c) = ab + ac$ (дистрибутивность)
                \item $ab=ba$ (коммутативность умножения)
                \item $(ab)c=a(bc)$ (ассоциативность умножения)
                \item $\exists 1 \in F \setminus \{0\} : 1 a = a 1 = a$ (единица)
                \item Если $a \neq 0$, $\exists a^{-1} \in F : a a^{-1} = a^{-1} a = 1$ (обратный элемент)
            \end{enumerate}
        \end{definition}

    % Лекция 8
    \defitem{Алгебраическая форма комплексного числа. Сложение, умножение и деление комплексных чисел в алгебраической форме}

        \begin{definition}
            Представление числа $z \in \CC$ в виде $a + bi$, где $a, b \in \RR$ называется его \textit{алгебраической формой}.
            Число $i$ называется \textit{мнимой единицей}.

            $a =: Re(z)$ -- \textit{действительная} часть числа $z$.
            $b =: Im(z)$ -- \textit{мнимая} часть числа $z$.
        \end{definition}

    % Лекция 8
    \defitem{Комплексное сопряжение и его свойства: сопряжение суммы и произведения двух комплексных чисел}

        \begin{definition}
            Число $\overline{z} := a - bi$ называется \textit{комплексно сопряженным} к числу $z = a + bi$.

            Операция $z \to \overline{z}$ называется \textit{комплексным сопряжением}.
        \end{definition}

        \subsubsection{Свойства комплексного сопряжения}

        \begin{itemize}[nosep]
        \item $\overline{\overline{z}} = z$.
        \item $\overline{z + w} = \overline{z} + \overline{w}$.
        \item $\overline{zw} = \overline{z} \cdot \overline{w}$.
        \end{itemize}        

    % Лекция 8
    \defitem{Геометрическая модель комплексных чисел, интерпретация в ней сложения и сопряжения}

        Числу $z = a + bi$ соответствует точка (или вектор) на плоскости $\RR^2$ с координатами $(a, b)$.
        Сумме $z + w$ соответствует сумма соответствующих векторов.
        Сопряжение $z \to \overline{z}$ -- это отражение $z$ относительно действительной оси.

    % Лекция 9
    \defitem{Модуль комплексного числа и его свойства: неотрицательность, неравенство треугольника, модуль произведения двух комплексных чисел}

        \begin{definition}
            Число $|z| = \sqrt{a^2 + b^2}$ называется \textit{модулем числа} $z = a + bi \in \CC$ (то есть длина соответствующего вектора).
        \end{definition}

        \textbf{Свойства}
        \begin{enumerate}
        \item $|z| \geq 0$, причем $|z| = 0 \iff z = 0$.
        \item $|z + w| \leq |z| + |w|$ (неравенство треугольника).
        \item $z \overline{z} = |z|^2$.

            $z \overline{z} = (a + bi)(a - bi) = a^2 - b^2i = a^2 + b^2 = |z|^2$
        \item $|zw| = |z||w|$

            $|zw|^2 = (zw) \cdot (\overline{zw}) = z \cdot w \cdot \overline{z} \cdot \overline{w} = |z|^2 |w|^2$
        \end{enumerate}

    % Лекция 9
    \defitem{Аргумент комплексного числа}

        \begin{definition}
            \textit{Аргументом числа} $z = a + bi \in \CC \setminus \{0\}$ называется число $\phi \in \RR$, такое что
            \begin{equation*}
                \cos \phi = \frac{a}{|z|} = \frac{a} {\sqrt{a^2 + b^2}}
            .\end{equation*}

            \begin{equation*}
                \sin \phi = \frac{b}{|z|} = \frac{b}{\sqrt{a^2 + b^2}}
            .\end{equation*}

            В геометрических терминах, $\phi$ есть угол между осью $Ox$ и соответствующим вектором.
        \end{definition}

    % Лекция 9
    \defitem{Тригонометрическая форма комплексного числа. Умножение и деление комплексных чисел в тригонометрической форме}

        \begin{definition}
            Представление числа $z \in \CC$ в виде $z = |z|(\cos \phi + i \sin \phi)$ называется его \textit{тригонометрической формой}.
        \end{definition}

        \begin{proposal}
            Пусть $z_1 = |z_1| (\cos \phi_1 + i \sin \phi_1)$ и $z_2 = |z_2| (\cos \phi_2 + i \sin \phi_2)$, тогда
            \begin{equation*}
                z_1 z_2 = |z_1| |z_2| (\cos (\phi_1 + \phi_2) + i \sin(\phi_1 + \phi_2))
            .\end{equation*}
        \end{proposal}

        \begin{corollary}
            В условиях предложения, предположим, что $z_2 \neq 0$.

            Тогда $\frac{z_1}{z_2} = \frac{|z_1|}{|z_2|} (\cos (\phi_1 - \phi_2) + i \sin(\phi_1 - \phi_2))$.

            В частности, $\frac{1}{|z_2|}(\cos(- \phi_2) + i\sin(- \phi_2)) = \frac{1}{|z_2|}(\cos \phi_2 - i \sin \phi_2) = \frac{\overline{z}_2}{|z_2|^2}$.
        \end{corollary}

    % Лекция 9
    \defitem{Формула Муавра}

        \begin{corollary}
            Пусть $z = |z|(\cos \phi + i \sin \phi)$. Тогда $\forall n \in \ZZ$,

            \begin{equation*}
                z^n = |z|^n (\cos(n \phi) + i \sin(n \phi)) \text{ -- формула Муавра}
            .\end{equation*}

        \end{corollary}

    % Лекция 9
    \defitem{Извлечение корней из комплексных чисел}

        Пусть $z \in \CC$, $n \in \NN$, $n \geq 2$.

        \begin{definition}
            \textit{Корнем степени n} (или \textit{корнем n-й степени)} из числа $z$ называется всякое число $w \in \CC$, что $w^n = z$.
        \end{definition}

        Положим $\sqrt[n]{z} := \{w \in \CC \ | \ w^n = z \}$.

        \bigskip
        Опишем множество $\sqrt[n]{z}$.

        $w = \sqrt[n]{z} \implies w^n = z \implies |w|^n = |z|$.

        Если $z = 0$, то $|z| = 0 \implies |w| = 0 \implies w = 0 \implies \sqrt[n]{0} = \{0\}$.

        \bigskip
        Далее считаем, что $z \neq 0$.

        $z = |z|(\cos \phi + i \sin \phi)$

        $w = |w|(\cos \psi + i \sin \psi)$

        $z = w^n = |w|^n (\cos (n \psi) + i \sin (n \psi))$

        Отсюда,
        \begin{equation*}
            z = w^n \iff
            \begin{cases}
                |z| = |w|^n  \\
                n \psi = \phi + 2 \pi k \text{, для некоторого } k \in \ZZ
            \end{cases}
            \iff
            \begin{cases}
                |w| = \sqrt[n]{|z|}  \\
                \psi = \frac{\phi + 2 \pi k}{n} \text{, для некоторого } k \in \ZZ
            \end{cases}
        \end{equation*}

        С точностью до $2 \pi l$, $l \in \ZZ$, получается ровно $n$ различных значений для $\psi$, при $k = 0, 1, \dots, n-1$.

        В результате $\sqrt[n]{z} = \{w_0, w_1, \dots, w_{n-1} \}$, где $w_m = \sqrt[n]{|z|}\left(\cos \frac{\phi + 2 \pi k}{n} + i \sin \frac{\phi + 2 \pi k}{n}\right)$

        \begin{comment}
            Числа $w_0, w_1, \dots, w_{n-1}$ лежат в вершинах правильного n-угольника с центром в начале координат.
        \end{comment}

    % Лекция 9
    \defitem{Основная теорема алгебры комплексных чисел}

        \begin{theorem}
            Всякий многочлен степени $\geq 1$ с комплексными коэффициентами имеет комплексный корень.
        \end{theorem}

    % Лекция 9
    \defitem{Теорема Безу и её следствие}

        Частный случай деления многочлена $f(x)$ на многочлен $g(x)$ с остатком: $g(x) = x - c$, $\deg g(x) = 1$:

        $f(x) = q(x) (x - c) + r(x)$, где либо $r(x) = 0$, либо $\deg r(x) < g(x) = 1$

        Значит, $r(x) \equiv r = const \in F$.

        \begin{theorem}
            $r = f(c)$.
        \end{theorem}

        \begin{corollary}
            Элемент $c \in F$ является корнем многочлена $f(x) \in F[x]$ тогда и только тогда, когда $f(x)$ делится на $(x - c)$.
        \end{corollary}

    % Лекция 9
    \defitem{Кратность корня многочлена}

        \begin{definition}
            \textit{Кратностью} корня $c \in F$ многочлена $f(x)$ называется наибольшее целое $k$ такое что, $f(x)$ делится на $(x - c)^k$.
        \end{definition}

    \end{colloq}

    \section{Вопросы на доказательство}

    \subsection{Операции над матрицами}
    \begin{colloq}
        \proofitem{Дистрибутивность произведения матриц по отношению к сложению}
        \proofitem{Ассоциативность произведения матриц}
        \proofitem{Некоммутативность произведения матриц}
        \proofitem{Транспонирование произведения двух матриц}
        \proofitem{Умножение на диагональную матрицу слева и справа}
        \proofitem{След произведения двух матриц}
    \end{colloq}

    \subsection{Системы линейных уравнений}
    \begin{colloq}
        \proofitem{Эквивалентность систем линейных уравнений, получаемых друг из друга пут м элементарных преобразований строк расширенной матрицы}
        \proofitem{Теорема о приведении матрицы к ступенчатому и улучшенному ступенчатому виду при помощи элементарных преобразований строк}
        \proofitem{Реализация элементарных преобразований строк матрицы при помощи умножения на подходящую матрицу}
        \proofitem{Метод Гаусса решения систем линейных уравнений}
        \proofitem{Связь между множеством решений совместной системы линейных уравнений и множеством решений соответствующей ей однородной системы}
        \proofitem{Общий метод решения матричных уравнений вида AX = B и XA = B}
        \proofitem{Вычисление обратной матрицы при помощи элементарных преобразований}
    \end{colloq}
    
    \subsection{Перестановки}
    \begin{colloq}
        \proofitem{Ассоциативность произведения перестановок}
        \proofitem{Некоммутативность произведения перестановок}
        \proofitem{Теорема о знаке произведения двух перестановок}
        \proofitem{Знак обратной перестановки}
        \proofitem{Знак транспозиции}   
    \end{colloq}
 
    \subsection{Определители}
    \begin{colloq}
        \proofitem{Определитель транспонированной матрицы}
        \proofitem{Поведение определителя при умножении строки (столбца) на скаляр}
        \proofitem{Поведение определителя при разложении строки (столбца) в сумму двух}
        \proofitem{Определитель матрицы с двумя одинаковыми строками (столбцами)}
        \proofitem{Поведение определителя при прибавлении к строке (столбцу) другой, умноженной на скаляр}
        \proofitem{Поведение определителя при перестановке двух строк (столбцов)}
        \proofitem{Определитель верхнетреугольной (нижнетреугольной) матрицы}
        \proofitem{Определитель с углом нулей}
        \proofitem{Определитель произведения двух матриц}
        \proofitem{Разложение определителя по строке (столбцу)}
        \proofitem{Лемма о фальшивом разложении определителя}
        \proofitem{Единственность обратной матрицы}
        \proofitem{Определитель обратной матрицы}
        \proofitem{Критерий обратимости квадратной матрицы и явная формула для обратной матрицы}
        \proofitem{Матрица, обратная к произведению двух матриц}
        \proofitem{Формулы Крамера}
    \end{colloq}
 
    \subsection{Комплексные числа}
    \begin{colloq}
        \proofitem{Построение поля комплексных чисел}
        \proofitem{Свойства комплексного сопряжения (для суммы и произведения)}
        \proofitem{Свойства модуля комплексного числа: неотрицательность, неравенство треугольника (алгебраическое доказательство), модуль произведения двух комплексных чисел}
        \proofitem{Умножение, деление и возведение в степень комплексных чисел в тригонометрической форме, формула Муавра}
        \proofitem{Извлечение корней из комплексных чисел}
    \end{colloq}
\end{document}
