\section{Теорема Леви-Деспланка и первая теорема Гершгорина.}
\subsection{Теорема Леви-Деспланка.}
\begin{definition*}
    Матрица $A$ обладает строгим строчным диагональным преобладанием, если
    $\displaystyle \forall i \ |a_{ii}| > \sum_{j = 1, i \neq j}^{n} |a_{ij}|$.
\end{definition*}

\begin{definition*}
    Матрица $A$ обладает строгим столбцовым диагональным преобладанием, если
    $\displaystyle \forall j \ |a_{jj}| > \sum_{i = 1, i \neq j}^{n} |a_{ij}|$.
\end{definition*}

\begin{theorem*}[Леви-Деспланка]
    Матрица, обладающая строгим строчным (столбцовым) диагональным
    преобладанием является невырожденной.
\end{theorem*}
\begin{proof} \ \\
    Докажем для строгого строчного, для столбцового аналогично. \\

    Представим $A$ в следующем виде: $A = diag(A)(I + diag(A)^{-1}(A - diag(A)))$,
    где матрица $diag(A)$ -- это диагональная матрица, у которой на диагонали
    стоят диагональные элементы матрицы $A$. Раскрыв скобки, можно проверить,
    что равенство действительно выполняется. \\

    Вспомним немного из курса линала:
    \begin{itemize}
        \item Матрица $A$ -- обратима $\Leftrightarrow$ матрица $A$ -- невырожденна
        \item Если $A = BC$, то $det(A) = det(B) \cdot det(C)$
    \end{itemize}

    Таким образом, нам необходимо и достаточно доказать обратимость матриц
    $diag(A)$ и $(I + diag(A)^{-1}(A - diag(A)))$. \\
    
    Обратимость первой почти очевидна: если бы на диагонали могли стоять
    нулевые элементы, то матрица $A$ не обладала бы строгим строчным диагональным
    пребладанием. \\ 
    
    Теперь заметим, что $(I + diag(A)^{-1}(A - diag(A)))$ обратима $\Leftrightarrow$
    $\exists \sum_{k = 0}^{\infty}(-diag(A)^{-1}(A - diag(A)))^k$ (вспоминаем про ряды Неймана). \\ 

    Осталось доказать, что такой ряд Неймана сходится. Не будем использовать критерий,
    а используем признак: докажем, что $||diag(A)^{-1}(A - diag(A))|| < 1$ для некоторой нормы. \\
    
    Рассмотрим бесконечную норму: $||diag(A)^{-1}(A - diag(A))||_{\infty} < 1$. Пусть это неравенство не выполняется, тогда:
    $\displaystyle max_i \dfrac{\sum_{j = 1, j \neq i}^{n}|a_{ij}|}{|a_{ii}|} \geqslant 1$ (это я просто руками записала
    бесконечную норму для матрицы). Но тогда выходит, что $A$ не обладает строгим строчным
    диагональным преобладанием -- противоречие $\Rightarrow ||diag(A)^{-1}(A - diag(A))||_{\infty} < 1 \Rightarrow
    \sum_{k = 0}^{\infty}(-diag(A)^{-1}(A - diag(A)))^k$ -- сходится $\Rightarrow$
    $(I + diag(A)^{-1}(A - diag(A)))$ обратима. \\

    Таким образом, $A$ представляет собой произведение обратимых матриц $\Rightarrow$ $A$
    и сама обратима, то есть, невырожденна.
\end{proof}
\subsection{Теорема Гершгорина.}
\begin{theorem*}[1-я теорема Гершгорина]
    Пусть $A \in \mathbb{C}$, тогда собственные значения матрицы $A$ находятся внутри\\
    \[D = D_1 \cup D_2 \cup \ldots \cup D_n\] где
    \[ \displaystyle D_k = \{z \in \mathbb{C} : |a_{kk} - z| \leqslant \sum_{i \neq k}|a_{ki}|\}\]  
\end{theorem*}
\begin{proof} \ \\
    Пусть $\lambda \not \in D \Rightarrow A - I\lambda$ обладает строгим строчным
    диагональным преобладанием (просто посмотрите на то как мы определяем $D_k$, на то, что условие
    в $D_k$ для данной $\lambda$ не выполняются, и на строение
    матрицы $A$) $\Rightarrow$ (по предыдущей теореме) $A - \lambda I$ -- невырожденна, но
    тогда $\lambda$ не может являться собственным значением $A$ (вспоминаем курс линала: характеристический
    многочлен и его корни). Получили противоречние.
\end{proof}

\begin{definition*}
    Множества $D_k$ -- круги Гершгорина.
\end{definition*}


\begin{theorem*}[2-я теорема Гершгорина без доказательства]
    Если $m$ кругов Гершгорина, образуют область $G$, изолированнных от других кругов, то в $G$ находится ровно
    $m$ собственных значений.
\end{theorem*}
