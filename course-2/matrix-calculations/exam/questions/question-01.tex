\section{Связь прямой и обратной ошибок через число обусловленности.}

Вспомним кто есть ошибки:

$\dfrac{\norm{\widetilde{f}(x) - f(x)}}{\norm{f(x)}}$ --- прямая ошибка,
где $f$ --- трушное решение, а $\widetilde{f}$ --- алгоритм

$\dfrac{\norm{\widetilde{x} - x}}{\norm{x}}$ --- обратная ошибка,
где $\widetilde{x}: f(\widetilde{x}) = \widetilde{f}(x)$

Также вспомним, что такое число обусловленности:

$\text{cond}(f, x) = \dfrac{\norm{f'(x)}}{\norm{f(x)}}\cdot\norm{x}$

Предположим, что наша задача обратно устойчива, а именно: $f(\widetilde{x}) =
    \widetilde{f}(x), \frac{\|\widetilde{x} - x\|}{\|x\|} = \text{O}(\varepsilon_{machine})$

\begin{flalign}
    \text{forward\_err}
     & = \frac{\| \widetilde{f}(x) - f(x)\|}{\|f(x)\|}                                    \\
     & = \frac{\|f(\widetilde{x}) - f(x)\|}{\|f(x)\|}                                     \\
     & \leq (\text{cond}(f, x) + \text{O}(\|\Delta x\|)) \cdot \text{backward\_err}       \\
     & = (\text{cond}(f, x) + \text{O}(\varepsilon_{machine})) \cdot \text{backward\_err} \\
     & \approx \text{cond}(f, x) \cdot \text{backward\_err}
\end{flalign}
