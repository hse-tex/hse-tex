% Здесь НЕ НУЖНО делать begin document, включать какие-то пакеты..
% Все уже подрубается в головном файле
% Хедер обыкновенный хсе-теха, все его команды будут здесь работать
% Пожалуйста, проверяйте корректность теха перед пушем

% Здесь формулировка билета
\subsection{Докажите, что интегрируемая на полуинтервале функция ограничена.}

\begin{theorem}
    Рассмотрим произвольный полуинтервал $D$. Можем взять, например, $(0;1]^n$, но подойдет любой. Тогда, если $f$ интегрируема на $D$, она ограничена.
\end{theorem}

\begin{proof}
    Пусть $f$ не ограничена на $D$ и последовательность $\{x_n\} \subset D$ такова, что $f(x_n) \to \infty$.
    Поскольку $\{x_n\}$ ограничена, то из нее можно выделить подпоследовательность, сходящуюся к некоторой предельной точке $a$ множества $D$.
    Далее будем считать, что $\{x_n\}$ это и есть выделенная подпоследовательность и $x_n \to a$.
    Точка $a$ является либо внутренней, либо граничной точкой множества $D$.
    
    Пусть последовательность разбиений $\tau_n$ с $\Delta (\tau_n) \to 0$ организована таким образом, что точка $a$ лежит внутри или на границе множества $D_{n1}$ и $\mu(D_{n1}) > 0$ (это можно сделать нарезав на кубы нужного размера, а куб, куда попала $a$ при необходимости порезать по $a$).
    Тогда точка $\xi_{n1} \in D_{n1}$ может быть выбрана так, чтобы
    \begin{equation*}
        |f(\xi_{n1}) \mu(D_1)| > n + \bigg| \sum\limits_{i > 1} f(\xi_{ni}) \mu(D_{ni}) \bigg|
    \end{equation*}
    
    Тогда интегральная сумма $|I_D(f, \tau_n, p_n)| > n$ и, следовательно, не может иметь конечный предел.
\end{proof}

