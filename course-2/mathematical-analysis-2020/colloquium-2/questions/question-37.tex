% Здесь НЕ НУЖНО делать begin document, включать какие-то пакеты..
% Все уже подрубается в головном файле
% Хедер обыкновенный хсе-теха, все его команды будут здесь работать
% Пожалуйста, проверяйте корректность теха перед пушем

% Здесь формулировка билета
\subsection{Сформулируйте теорему о выражении меры через её плотность.}
  \begin{theorem}
      Пусть мера $\nu$ такова, что $\exists$ плотность $\rho(x)$ и функция $\rho$ непрерывна на $D$.\\ 
      Пусть также $\mu(X) = 0 \Rightarrow \nu(X) = 0$.
      
      Тогда для любого жорданова множества $A \subseteq D$ верно:
      \[ \nu(A) = \int\limits_A \rho(x)\, dx \]
  \end{theorem}
