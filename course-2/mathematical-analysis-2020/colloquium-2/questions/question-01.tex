% Здесь НЕ НУЖНО делать begin document, включать какие-то пакеты..
% Все уже подрубается в головном файле
% Хедер обыкновенный хсе-теха, все его команды будут здесь работать
% Пожалуйста, проверяйте корректность теха перед пушем

% Здесь формулировка билета
\subsection{Что такое кольцо множеств? Дайте определение меры на кольце.}
    
    \begin{definition}[Маевский]
        Множество $\mathcal{F}$ называется кольцом, если:\\[-25 pt]
        \begin{enumerate}
            \item $\varnothing \in \mathcal{F}$;
            \item $\forall A,  B \in \mathcal{F} : A \cup B, A \cap B, A \setminus B \in \mathcal{F}$.
        \end{enumerate}
    \end{definition}

    \begin{definition}
        Пусть $\mathcal{F}$ --- некоторое семейство подмножеств множества $X$, 
        т.е $\mathcal{F} \subseteq 2^X$.
        
        Функция $\mu : \mathcal{F} \to [0; +\infty)$ называется мерой на $\mathcal{F}$, 
        если она обладает свойством аддитивности:
        \[ \mu(A \sqcup B) = \mu(A) + \mu(B), \]
        где $\sqcup$ --- операция объединения непересекающихся (дизъюнктных) множеств.
    \end{definition}
        
    \begin{properties}
        Если $\mu$ --- мера на кольце множеств $\mathcal{F}$ и $A, B \in \mathcal{F}$, то:\\[-25 pt]
        \begin{enumerate}
            \item $\mu(\varnothing) = 0$;
            \item $A \subseteq B \implies \mu(A) \le \mu(B)$ (монотонность);
            \item $\mu(A \cup B) = \mu(A) + \mu(B) - \mu(A \cap B)$ (ФВИ).
        \end{enumerate}
    \end{properties}
