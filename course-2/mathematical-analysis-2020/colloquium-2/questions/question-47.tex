% Здесь НЕ НУЖНО делать begin document, включать какие-то пакеты..
% Все уже подрубается в головном файле
% Хедер обыкновенный хсе-теха, все его команды будут здесь работать
% Пожалуйста, проверяйте корректность теха перед пушем

% Здесь формулировка билета
\subsection{Что можно утверждать о внутренних, предельных, граничных точках множества при диффеоморфном отображении?}

Пусть $\varphi : U \to X$~--- диффеоморфизм. 

Если $B = \varphi(A)$, $A$~--- множество в $U$, то:
\begin{enumerate}
    \item внутренняя точка множества $A$ переходит во внутреннюю точку множества $B$.
    \item предельная точка множества $A$ переходит во предельную точку множества $B$.
    \item граничная точка множества $A$ переходит во граничную точку множества $B$.
\end{enumerate}

