% Здесь НЕ НУЖНО делать begin document, включать какие-то пакеты..
% Все уже подрубается в головном файле
% Хедер обыкновенный хсе-теха, все его команды будут здесь работать
% Пожалуйста, проверяйте корректность теха перед пушем

% Здесь формулировка билета
\subsection{Докажите, что площадь гладкой $k$-мерной поверхности не зависит от ее параметризации}

Пусть задана альтернативная параметризация поверхности $S$ $\psi \colon H \to S$, $H \subset \mathbb{R}^{k}$, $v = (v_{1}, \ldots, v_{k}) \in H$, $x = \psi(v)$.
При этом $\psi$ биективно, непрерывно дифференцируемо, и соответствующая ему матрица Якоби имеет ранг $k$.
Рассмотрим отображение $\chi = \varphi^{-1} \circ \psi \colon H \to G$; $u = \chi(v)$.
Заметим, во-первых, что $\chi$ биективно как композиция двух биекций.
Во-вторых, ранги матриц Якоби для $\varphi$ и $\psi$ равны $k$, поэтому в матрице Якоби для $\varphi$ мы можем вычленить соответствующий базисный $k \times k$ минор такой, что его определитель не равен нулю.
Рассматривая соответствующие координаты вектора $x$, мы можем выразить координаты $u = \chi(v)$, и на этом пути у нас получится, что $\chi$ имеет непрерывные частные производные, потому что мы сможем представить
\[
    \pdv{u}{v} = \pqty{\pdv{\pqty{x_{i1}, \ldots, x_{ik}}}{\pqty{u_{1}, \ldots, u_{k}}}}^{-1} \times \pqty{\pdv{\pqty{x_{i1}, \ldots, x_{ik}}}{\pqty{v_{1}, \ldots, v_{k}}}}.
\]
В силу невырожденности обратной матрицы, и того факта, что ранг матрицы Якоби равен $k$ мы получим, что матрица Якоби для $\chi$ невырождена.
Следовательно, $\chi$ --- диффеоморфизм.
Это означает, что мы можем сделать замену переменной, а именно
\[
    \mu(S) = \int_{G} \sqrt{\det\pqty{\pqty{\pdv{x}{u}}^{T} \times \pqty{\pdv{x}{u}}}} \cdot \dd u = \int_{H} \sqrt{\det\pqty{\pqty{\pdv{x}{u}}^{T} \times \pqty{\pdv{x}{u}}}} \cdot \abs{\det\pqty{\pdv{u}{v}}} \dd v.
\]
Сделаем замену в матрице Якоби:
\[
    \pqty{\pdv{x}{u}}^{T} \times \pqty{\pdv{x}{u}} = \pqty{\pdv{x}{v} \cdot \pdv{v}{u}}^{T} \times \pqty{\pdv{x}{v} \cdot \pdv{v}{u}} = \pqty{\pdv{v}{u}}^{T} \times \pqty{\pdv{x}{v}}^{T} \times \pqty{\pdv{x}{v}} \times \pqty{\pdv{v}{u}}.
\]
Найдем определитель:
\[
    \det \pqty{ \pqty{\pdv{x}{u}}^{T} \times \pqty{\pdv{x}{u}} } = \det \pqty{ \pqty{\pdv{v}{u}}^{T} \times \pqty{\pdv{x}{v}}^{T} \times \pqty{\pdv{x}{v}} \times \pqty{\pdv{v}{u}} } = \pqty{\det \pqty{\pdv{v}{u}}}^{2} \cdot \det \pqty{\pqty{\pdv{x}{v}}^{T} \times \pqty{\pdv{x}{v}}}.
\]
Подставим это в выражение $\mu(S)$:
\begin{multline}
    \mu(S) = \int_{H} \sqrt{\det\pqty{\pqty{\pdv{x}{u}}^{T} \times \pqty{\pdv{x}{u}}}} \cdot \abs{\det\pqty{\pdv{u}{v}}} \dd v = \\ = \int_{H} \sqrt{\det\pqty{\pqty{\pdv{x}{v}}^{T} \times \pqty{\pdv{x}{v}}}} \cdot \underbrace{\abs{\det\pqty{\pdv{v}{u}}} \cdot \abs{\det\pqty{\pdv{u}{v}}}}_{= 1} \dd v = \\ = \int_{H} \sqrt{\det\pqty{\pqty{\pdv{x}{v}}^{T} \times \pqty{\pdv{x}{v}}}} \dd v.
\end{multline}
Произведение определителей равно 1, поскольку это произведение определителей двух взаимно обратных матриц.
Таким образом, утверждение доказано.
