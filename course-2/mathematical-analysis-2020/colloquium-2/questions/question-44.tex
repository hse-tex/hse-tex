% Здесь НЕ НУЖНО делать begin document, включать какие-то пакеты..
% Все уже подрубается в головном файле
% Хедер обыкновенный хсе-теха, все его команды будут здесь работать
% Пожалуйста, проверяйте корректность теха перед пушем

% Здесь формулировка билета
\subsection{Дайте определение полярных координат (формулы, область задания, координатные линии, матрица Якоби перехода, якобиан)}
Полярные координаты (r, $\phi$) в пространстве $\mathbb{R}^2$. Точка соединяется с началом координат(длина этого отрезка -- r).

$\phi$~-- угол, который отсчитывается от положительной оси против часовой стрелки до линии.\\

Для точки (0, 0) неоднозначно определено $\phi$, поэтому мы ее выкалываем.\\


X = $\mathbb{R}^2\backslash\{(0, 0)\}$


U = $(0, +\infty)\times[0, 2\pi)$
\begin{flalign*}
\begin{cases}
x = r\cos\phi\\
y = r\sin\phi
\end{cases}
\end{flalign*}
Матрица Якоби перехода:
\begin{align*}
J = 
\left|
\begin{array}{cc}
\cos\phi& -r\sin\phi\\
\sin\phi& r\cos\phi
\end{array}
\right|
= r
\end{align*}
Координатные линии r лучи, выходящие из начала координат. Координатные линии $\phi$ окружности с центром в начале координат.

