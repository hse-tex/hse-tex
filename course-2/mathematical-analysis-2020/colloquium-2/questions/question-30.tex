% Здесь НЕ НУЖНО делать begin document, включать какие-то пакеты..
% Все уже подрубается в головном файле
% Хедер обыкновенный хсе-теха, все его команды будут здесь работать
% Пожалуйста, проверяйте корректность теха перед пушем

% Здесь формулировка билета
\subsection{Сформулируйте и докажите теорему о среднем значении.}
\underline{Теор.:} 
	Пусть жорданово множество $D$ замкнуто и связно, $f$ непрерывна на $D$,
	а $g$ ограничена, неотрицательна и интегрируема. Покажите, что
	найдётся такая точка $a \in D$, что
	\[ \int\limits_D f(x)g(x)dx = f(a)\int\limits_D g(x)dx.\]

\underline{Док-во:} 
	Сначала докажем, что при данных условиях на $f$ и $g$ верно, что
	\[ m \int\limits_{D}g(x)dx \leq \int\limits_{D}f(x)g(x)dx \leq M \int\limits_{D}g(x)dx,\]
    где $m = \inf\limits_{x\in D}f(x)$ и $M = \sup\limits_{x \in D}f(x)$.

    Действительно, произведение ограниченных интегрируемых функций -- интегрируемая функция.
	Остается воспользоваться монотонностью интеграла.

	Если $\int\limits_D g(x)dx = 0$, то, исходя из предыдушего,
	$\int\limits_D f(x)g(x)dx = 0$ и требуемое равенство выполняется
	при любом выборе точки $a$. Пусть теперь $\int\limits_D g(x)dx > 0$.
	Непрерывная на ограниченном замкнутом множестве $D$ функция $f$ достигает
	на $D$ своих точных граней $m, M$. Соединим точки, в которых достигаются
	эти значения, непрерывной кривой, целиком лежащей в $D$. На этой
	кривой функция $f$ принимает все значения из промежутка $[m; M]$.
	В частности, принимает значение $\frac{\int\limits_D f(x)g(x)dx}{\int\limits_D g(x)dx}$
	в некоторой точке $a \in D$.
    \begin{flushright}
    $\blacksquare$
    \end{flushright}

