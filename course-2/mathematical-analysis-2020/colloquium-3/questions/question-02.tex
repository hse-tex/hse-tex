\subsection{Равномерная сходимость семейства функций. Определение. Критерий Коши равномерной сходимости.}

\subsubsection{Равномерная сходимость семейства функций}

Мы будем рассматривать фактически функции от двух переменных $f(x, y)$, где $x \in D$, $y \in H$.
Будем говорить про $f(x, y)$ как про {\it семейство функций}, понимая, что это либо функция от $x$ при фиксированном $y$, либо функция от $y$ при фиксированном $x$.

\begin{definition}
    Пусть точка $a$ принадлежит замыканию множества $D$ (обозн. $a \in \overline{D}$), то есть лежит либо во множестве $D$, либо на его границе.
    Семейство функций $f(x, y)$ {\it сходится} при $x \to a$ к функции $g(y)$, если для любого $y \in H$
    \[
        f(x, y) \underset{x \to a}{\to} g(y).
    \]
\end{definition}

\begin{definition}
    Семейство $f(x, y)$ сходится при $x \to a$ {\it равномерно} по $y \in H$, если
    \[
         \forall \varepsilon > 0~ \exists \delta(\varepsilon) > 0 \colon \forall x \in D ~ 0 < \abs{x - a} < \delta \implies \forall y \in H ~ \abs{f(x, y) - g(y)} < \varepsilon.
    \]
    Равномерность здесь состоит в том, что при данном $\varepsilon$ мы можем найти $\delta > 0$ одинаковое для всех $y$.
    Альтернативно, определение можно переписать через супремумы,
    \[
        \sup\limits_{y \in H} \abs{f(x, y) - g(y)} \underset{x \to a}{\to} 0.
    \]
\end{definition}

\paragraph*{Пример}
Рассмотрим функцию $f(x, y) = y^{x} - y^{2x}$, и будем считать, что $x \in [1, +\infty)$, $y \in [0, 1]$, $x \to +\infty$.
Мы можем сказать, что параметр $x$ стремится к бесконечности, а переменной является $y$, либо правильнее, наверное, будет сказать, что то, что стремится --- переменная, а то, что не стремится --- параметр.
Поэтому такие обозначения субъективны и мы считаем $f$ просто функцией двух переменных.
Понятно, что
\[
    y^{x} \to \begin{cases}
        0, & y \in [0, 1), \\
        1, & y = 1.
    \end{cases} \implies y^{x} - y^{2x} \to 0.
\]
Таким образом, мы имеем предельную функцию $g(y) = 0$ для любого $y \in [0, 1]$.
Для ответа на вопрос о равномерности сходимости семейства $f$ к функции $g$, нужно вычислить супремум
\[
    \sup\limits_{y \in [0, 1]}\abs{f(x, y) - g(y)} = \sup\limits_{y \in [0, 1]}\abs{y^{x} - y^{2x} - 0} = [t = y^{x}] = \sup\limits_{t \in [0, 1]}\abs{t - t^{2}} = \frac{1}{4} \underset{x \to +\infty}{\centernot\to} 0.
\]
Супремум отделен от нуля, поэтому равномерной сходимости нет.
Получаем, что $f$ сходится поточечно к $g$ (при $x \to +\infty$), но не сходится равномерно.

\subsubsection{Критерий Коши}

\begin{theorem}
    Семейство функций $f(x, y)$ сходится равномерно по $y \in H$, при $x \to a$, к функции $g(y)$ тогда и только тогда, когда
    \[
        \forall \varepsilon > 0~ \exists \delta(\varepsilon) > 0 \colon \forall x_1, x_2 \in D ~ \begin{cases}
            0 < \abs{x_{1} - a} < \delta, \\
            0 < \abs{x_{2} - a} < \delta;
        \end{cases} \implies \forall y \in H ~ \abs{f(x_{1}, y) - f(x_{2}, y)} < \varepsilon.
    \]
\end{theorem}

Поскольку <<свойство Коши>> (выражение выше в кванторах) не зависит от функции $g$, мы можем что-то утверждать про равномерную сходимость семейства функций $f$ без вычисления самой предельной функции $g$.

\begin{proof}
    \begin{description}
        \item[$\implies$]
        \[
            \abs{f(x_{1}, y) - f(x_{2}, y)} = \abs{f(x_{1}, y) - g(y) + g(y) - f(x_{2}, y)} \leqslant \abs{f(x_{1}, y) - g(y)} + \abs{g(y) - f(x_{2}, y)}.
        \]
        Поскольку имеет место равномерная сходимость, то
        \begin{align}
            \exists \delta \colon 0 < \abs{x_{1} - a} < \delta &\implies \abs{f(x_{1}, y) - g(y)} < \frac{\varepsilon}{2}, \\
            \exists \delta \colon 0 < \abs{x_{2} - a} < \delta &\implies \abs{f(x_{2}, y) - g(y)} < \frac{\varepsilon}{2}.
        \end{align}
        Тогда,
        \[
            \abs{f(x_{1}, y) - f(x_{2}, y)} \leqslant \abs{f(x_{1}, y) - g(y)} + \abs{g(y) - f(x_{2}, y)} < \varepsilon.
        \]
        \item[$\impliedby$] Давайте зафиксируем $y \in H$, то есть будем рассматривать $\abs{f(x_{1}, y) - f(x_{2}, y)}$ как функцию от $x_{1}$ и $x_{2}$.
        Положим $F(x) = f(x, y)$ (мы можем так сделать, потому что $y$ фиксировано).
        Тогда, $F(x)$ при $x \to a$ удовлетворяет критерию Коши, то есть существует {\bf константа} $g$ такая, что $F(x) \underset{x \to a}{\to} g$.
        Если мы будем менять $y$, то получится, что существует {\bf функция} $g(y)$ такая, что $f(x, y) \underset{x \to a}{\to} g(y)$.
        Докажем равномерность этой сходимости.
        Для этого рассмотрим неравенство
        \[
            \abs{f(x_{1}, y) - f(x_{2}, y)} < \varepsilon,
        \]
        и устремим $x_{2} \to a$, тогда, поскольку предельный переход может повлиять разве что на строгость неравенства, имеем
        \[
            \abs{f(x_{1}, y) - g(y)} \leqslant \varepsilon,
        \]
        что равносильно определению равномерной сходимости.
    \end{description}
\end{proof}










