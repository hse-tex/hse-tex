\subsection{Эйлеровы B- и $\Gamma$- функции. Определение B-функции, ее область определения и свойства: симметричность, формула понижения, случайно натурально-значных аргументов. Формула Эйлера -- Гаусса. Формула дополнения (с использованием разложения $\sin$ в бесконечное произведение без доказательства). Связь между B- и $\Gamma$- функциями.}

\subsubsection{Бета-функция Эйлера}
\(
B(p, q) = \int_0^1 x^{p-1}(1 - x)^{q-1}dx
\)

Чтобы понять, что это за интеграл, нам нужно внимательно присмотреться
к нашей функции, зависящей от двух парметров $p$ и $q$. Мы понимаем, что потенциально
здесь возможна неприятность в 0 и 1 из-за того, что степень может оказаться отрицательной.
Нам никто не запрещает рассматривать степени $p < 1$ или $q < 1$, поэтому степень
$x$ или $(1 - x)$ может оказаться отрицательной. Это означает, что нам нужно
посмотреть как себя ведёт подинтегральная функция при $x \to 0$ и при $x \to 1$.

\(
x \to 0: x^{p-1}(1-x)^{q - 1} = x^{p-1} \cdot (1 + o(1)) \implies
\text{инт. сходится при } p-1 > -1 \implies p > 0
\)

\(
x \to 1: x^{p-1}(1-x)^{q - 1} = (1 - x)^{q-1} \cdot (1 + o(1)) \implies
\text{инт. сходится при } q-1 > -1 \implies q > 0
\)

Из этого следует, что область определения Бета функции --- первая четверть на плоскости $pq$
(положительные $p$ и $q$)

Отметим ещё одну важную формулу Бета-функции, сделав замену
\(
x = \frac{t}{1 + t}, t \in [0; +\infty)
\)

\(
B(p, q) = \int_0^{+\infty}\frac{t^{p-1}}{(1 + t)^{p + q}}dt
\)

Свойства:
\begin{enumerate}
    \item Симметричность
          \(
          x = 1 - y, ~ y \in [0; 1]
          \)

          \(
          B(p, q) = \int_0^1(1 - y)^{p - 1} \cdot y^{q - 1}dy =
          \int_0^1 x^{q - 1} \cdot (1 - x)^{p - 1}dx = B(q, p)
          \)
    \item Формула понижения
          $p > 0, ~ q > 0$

          здесь юзнём интегрирование по частям: $\int udv = uv - \int vdu$

          $
              B(p + 1, q) = \int_0^1 x^p \cdot (1 - x)^{q - 1} dx =
              -\frac{1}{q} \int_0^1 x^p \cdot d((1 - x)^q) =
              \left[
                  \begin{array}{cc}
                      u = x^p         & v = (1-x)^q          \\
                      du = px^{p-1}dx & dv = -q(1-x)^{q-1}dx \\
                  \end{array}
                  \right] =
              \\
              = -\frac{1}{q}x^p \cdot (1 - x)^q \Bigg|_0^1 + \frac{1}{q}\cdot
              \int_0^1(1-x)^{q} \cdot px^{p - 1} dx =
              \left[ (1 - x)^q = (1 - x) \cdot (1 - x)^{q - 1} = (1 - x)^{q - 1} -
                  x \cdot (1 - x)^{q - 1} \right] =
              \\
              = \frac{p}{q} \cdot \int_0^1 x^{p - 1} \cdot ((1 - x)^{q - 1} -
              x \cdot (1 - x)^{q - 1}) dx = \frac{p}{q} \cdot B(p, q) -
              \frac{p}{q}B(p + 1, q)
              \\
              (1 + \frac{p}{q})B(p + 1, q) = \frac{p}{q}B(p, q) \implies
              B(p + 1, q) = \frac{p}{p + q} \cdot B(p, q)
          $

          Воспользовавшись тем, что функция симметрична можно легко
          вывести аналогичную формулу для $B(p, q+1)$

          Итог:

          $B(p + 1, q) = \frac{p}{p + q} \cdot B(p, q)$

          $B(p, q + 1) = \frac{q}{p + q} \cdot B(p, q)$
    \item Поведение при натуральных аргументах

          \begin{itemize}
              \item $q = n \in \NN$

                    $
                        B(p, 1) = \int_0^1 x^{p - 1}dx = \frac{1}{p}
                        \\
                        B(p, n) = \frac{n - 1}{p + n - 1} \cdot B(p, n - 1) = \dots =
                        \frac{n-1}{p + n - 1} \cdot \dots \cdot \frac{1}{p + 1} \cdot B(p, 1) =
                        \frac{(n - 1)!}{p(p+1)\dots(p + n - 1)}
                    $
              \item $p = n \in \NN, q = m \in \NN$

                    $
                        B(n, m) = \frac{(n - 1)!(m - 1)!}{(n + m - 1)!}
                    $
          \end{itemize}

\end{enumerate}

\subsubsection{Гамма-функция Эйлера}

$\Gamma(p) = \int_0^{+\infty}x^{p - 1} \cdot e^{-x} dx$

Аналогично с Бета-функцией можно вывести, что область определения Гамма-функции ---
$p > 0$:

В бесконечности отсутствие проблем очевидно, проверим 0.

$x \to 0 \colon x^{p - 1} \cdot e^{-x} = x^{p - 1} \cdot (1 + \mathrm{o}(1))
    \implies$ инт. сходится при $p-1 > -1 \implies p > 0$

Свойства:

\begin{enumerate}
    \item Формула понижения

          здесь юзнём интегрирование по частям: $\int udv = uv - \int vdu$

          $\Gamma(p + 1) = \int_0^{+\infty}x^p \cdot e^{-x}dx =
              -\int_0^{+\infty}x^p \cdot d(e^{-x}) =
              \left[
                  \begin{array}{cc}
                      u = x^p         & v = e^{-x}     \\
                      du = px^{p-1}dx & dv = -e^{-x}dx \\
                  \end{array}
                  \right] =
              \\
              = -x^p \cdot e^{-x} \Bigg|_0^{+\infty} + \int_0^{+\infty}
              e^{-x} \cdot px^{p-1}dx = p \cdot \Gamma(p)
          $

          Итог: $\Gamma(p + 1) = p \cdot \Gamma(p)$

    \item Случай натурального аргумента, $p = n \in \NN$

          $\Gamma(1) = \int_0^{+\infty}e^{-x}dx = 1$

          $
              \Gamma(n + 1) = n \cdot \Gamma(n) = n \cdot (n - 1) \Gamma(n - 1) =
              \dots = n!
          $

    \item Формула Эйлера-Гаусса

          $
              \Gamma(p) = \int_0^{+\infty}x^{p - 1} e^{-x} dx = \left[x =
                  -\ln t = \ln\frac{1}{t}, ~ t \in [0; 1] \right] = \int_0^1
              \ln^{p - 1}\left(\frac{1}{t}\right) dt
          $

          \begin{minipage}{.50\textwidth}
              Пусть $\varphi(z) = t^z, t \in (0; 1)$ --- выпуклая $\implies$ при
              стремлении $z \to +0$ наклонный коэффициент секущей, проведённой
              через $(0, \varphi(0))$ и $(z, \varphi(z))$ будет убывать в силу выпуклости
              функции, а в пределе даст нам производную:

              $\frac{t^z - t^0}{z - 0} \underset{z \to +0}{\searrow} \varphi'(0) = \ln t$

              $z = \frac{1}{n} \colon n \cdot(t^\frac{1}{n} - 1)
                  \underset{n \to +\infty}{\searrow} \ln t$, а если мы поменяем знак,
              то получится $n \cdot(1 - t^\frac{1}{n}) \underset{n \to +\infty}{\nearrow}
                  \ln \frac{1}{t}$. И сама функция, и её предельная функция непрерывны,
              сходимость монотонная, по теореме Дини получаем равномерную сходимость
              по $t$ на любом отрезке $[a, b] \subset (0; 1)$
          \end{minipage}\hfill
          \begin{minipage}{.40\textwidth}
              \definecolor{uuuuuu}{rgb}{0.26666666666666666,0.26666666666666666,0.26666666666666666}
              \definecolor{qqqqff}{rgb}{0,0,1}
              \begin{tikzpicture}[line cap=round,line join=round,x=1cm,y=1cm]
                  \begin{axis}[
                          x=1.5cm,y=1.5cm,
                          axis lines=middle,
                          ymajorgrids=true,
                          xmajorgrids=true,
                          xmin=-2,
                          xmax=2,
                          ymin=-1,
                          ymax=3,
                          xtick={-2,-1,...,2},
                          ytick={-1,0,...,3},
                          xlabel=$z$,
                          ylabel=$\varphi(z)$,
                          x label style={at={(axis description cs: 1.06, 0.28)}, anchor=north}]
                      %\clip(-4.973001254650355,-2.8738695450002516) rectangle (2.0559109591260905,4.437045164815357);
                      %\draw[line width=2pt,color=qqqqff,smooth,samples=100,domain=-4.973001254650355:2.0559109591260905] plot(\x,{0.4});
                      \addplot[line width=1pt,color=qqqqff,smooth,samples=100,domain=-2:2]{0.4^x};
                      \addplot[line width=1pt,color=uuuuuu,smooth,samples=100,domain=-2:2]{(0.4^1.5 - 1)* x/1.5 + 1};
                      \node[circle,fill,inner sep=1.5pt] at (axis cs:1.5,0.4^1.5) {};
                      \node[circle,fill,inner sep=1.5pt] at (axis cs:0,1) {};
                      \node at (axis cs:1.5, -0.15) {z};
                      \node at (axis cs:0.15, -0.15) {0};
                      \addplot[line width=1pt,color=uuuuuu] coordinates {(1.5, 0) (1.5, 0.4^1.5)};
                      \addplot[line width=1pt,color=uuuuuu] coordinates {(0, 0) (0, 1)};
                      %\draw [line width=2pt,domain=-4.973001254650355:2.0559109591260905] plot(\x,{(--1.4848559046467604-0.74348283681205=triangle 45,05*\x)/1.4848559046467604});
                      %\draw [line width=2pt] (1.4848559046467604,-2.8738695450002516) -- (1.4848559046467604,4.437045164815357);
                      %\draw (1.5201062166516575,-0.01859427260360577) node[anchor=north west] {z};
                      %\begin{scriptsize}
                      %    \draw[color=qqqqff] (-1.5255207405714362,4.40531988401095) node {$f$};
                      %    \draw [fill=black] (0,1) circle (2pt);
                      %    \draw [fill=black] (1.4848559046467604,0.2565171631879494) circle (2.5pt);
                      %    \draw[color=black] (-4.909550693041541,3.4253612102748177) node {$g$};
                      %    \draw[color=black] (1.4002551558350078,4.40531988401095) node {$h$};
                      %    \draw [fill=uuuuuu] (1.4848559046467604,0) circle (0.5pt);
                      %\end{scriptsize}
                  \end{axis}
              \end{tikzpicture}
          \end{minipage}

          Получаем $\Gamma(p) = \int_0^1 \ln^{p - 1}\left(\frac{1}{t}\right) dt,
              \ln \frac{1}{t} = \lim_{n \to \infty}
              \underbrace{n \cdot (1 - t^\frac{1}{n})}_{f_n(t)}$

          $
              \begin{rcases*}
                  f_n(t) \geq 0 ~ \text{и непрерывны при } t \in (0; 1), \forall n \\
                  n \to \infty ~ f_n(t) \searrow \ln \frac{1}{t}                   \\
                  \ln \frac{1}{t} \text{ --- непрерывна}                           \\
                  \int_0^1 \ln^{p - 1}\left(\frac{1}{t}\right)dt ~ \text{сходися}
              \end{rcases*} \implies
          $
          можем внести предел под знак интеграла
          (следует из теоремы Дини по пункту 5.2)

          $
              \lim_{n \to \infty} \int_0^1(n\cdot(1 - t^\frac{1}{n}))^{p-1}dt =
              \int_0^1 \ln^{p-1}\left(\frac{1}{t}\right)dt
              \\
              \implies \Gamma(p) = \lim_{n \to \infty}\int_0^1n^{p-1} \cdot (1 -
              t^\frac{1}{n})^{p-1}dt = \left[t = z^n \right] = \lim_{n \to \infty}
              n^p \cdot \underbrace{\int_0^1 z^{n-1} \cdot (1 - z)^{p-1}dz}_{B(n, p)} =
              \lim_{n \to \infty} n^p \cdot \mathrm{B}(n, p) =
              \\
              = \lim_{n \to \infty}\frac{n^p \cdot (n - 1)!}{p(p+1)\dots(p+n-1)}
          $

          Итог:

          $\Gamma(p) = \lim_{n \to \infty} \frac{n^p \cdot (n - 1)!}{p(p+1)\dots(p+n-1)}$
          --- формула Эйлера-Гаусса

    \item Формула дополнения, $p \in (0; 1)$

          $\Gamma(p) \cdot \Gamma(1 - p) = \left[\text{формула Эйлера-Гаусса}
                  \right] = \lim_{n \to \infty} \left(
              \frac{n^p \cdot (n-1)!}{p(p+1)\dots(p+n-1)} \cdot
              \frac{n^{1-p}\cdot(n-1)!}{(1-p)(2-p)\dots(n-p)}
              \right) =
              \\
              = \left[\text{поделили в обеих дробях числитель и знаменатель
                      на } (n-1)!\right] =
              \\
              = \lim_{n \to \infty} \frac{n}{p(n-p)} \cdot
              \frac{1}{(\frac{p}{1} + 1)(\frac{p}{2} + 1)\dots(\frac{p}{n-1}+1)}
              \cdot \frac{1}{(1 - \frac{p}{1})(1 - \frac{p}{2})\dots(1 - \frac{p}{n - 1})} =
              \\
              = \frac{1}{p} \cdot \prod_{k=1}^\infty \frac{1}{\left(1 -
                  \left(\frac{p}{k}\right)^2\right)}
          $

          Далее нам придётся воспользоваться формулой бесконечного произведения
          для синуса

          $
              \sin(\pi p) = \pi p \cdot \prod_{k=1}^\infty(1 - \frac{p^2}{k^2})
          $

          $
              \implies \Gamma(p) \cdot \Gamma(1 - p) = \frac{\pi}{\sin(\pi p)}
          $ --- формула дополнения

          Если мы возьмём $p = \frac{1}{2}$, то получим $\Gamma\left(\frac{1}{2}\right)^2 =
              \frac{\pi}{\sin \frac{\pi}{2}} \implies \Gamma\left(\frac{1}{2}\right) = \sqrt{\pi}$

          Это даёт нам ещё один способ вычисления интеграла Эйлера-Пуассона

          $\Gamma\left(\frac{1}{2}\right) = \int_0^{+\infty}x^{-\frac{1}{2}}e^{-x}dx =
              \left[x = t^2\right] = 2 \cdot \int_0^{+\infty}e^{-t^2}dt \implies
              \int_0^{+\infty}e^{-t^2}dt = \frac{\sqrt{\pi}}{2}$
\end{enumerate}

\subsubsection{Связь между $\mathrm{B}$ и $\Gamma$}

$
    \Gamma(p) = \int_0^{+\infty}t^{p-1} e^{-t}dt = [t = x(y + 1), y > 0, y = \mathrm{const}] =
    (1 + y)^p\int_0^{+\infty}x^{p-1}e^{-x(1+y)}dx
$

$
    \frac{\Gamma(p + q) \cdot y^{p-1}}{(1+y)^{p+q}} = y^{p-1} \cdot \int_0^{+\infty}
    x^{p+q-1} \cdot e^{-x(1+y)}dx
$

$
    \Gamma(p + q) \cdot \mathrm{B}(p, q) = \Gamma(p + q) \cdot \int_0^{+\infty}
    \frac{y^{p-1}}{(1+y)^{p+q}}dy = \int_0^{+\infty}dy \int_0^{+\infty}
    y^{p-1} \cdot x^{p + q - 1} \cdot e^{-x(1 + y)}dx = \left[\text{потом обоснуем}\right] =
    \\
    = \int_0^{+\infty}dx \int_0^{+\infty}
    y^{p-1} \cdot x^{p + q - 1} \cdot e^{-x(1 + y)}dy = \int_0^{+\infty}dx \cdot
    x^{q - 1} \cdot e^{-x} \cdot \int_0^{+\infty}(xy)^{p-1}\cdot e^{-xy}\cdot x \cdot dy =
    \Gamma(p) \cdot \int_0^{+\infty}dx \cdot x^{q - 1} \cdot e^{-x} = \Gamma(p)
    \cdot \Gamma(q)
    \\
    \implies \mathrm{B}(p, q) = \frac{\Gamma(p) \cdot \Gamma(q)}{\Gamma(p + q)}
$
\\
\\
"Потом обоснуем": нам надо обосновать внесение одного несобственного интеграла
внутрь другого несобственного интеграла

$f(x, y) = x^{p+q-1} \cdot y^{p - 1} \cdot e^{-(1 + y)x} \geq 0$ и непрерывна
на $(0; +\infty)\times(0; +\infty)$

$\int_0^{+\infty}dx \int_0^{+\infty}f(x, y)dy$ --- сходится в силу того что равен
произведению Гамма функций и там всё хорошо

Оба интеграла $\int_0^{+\infty}f(x, y)dx$ и $\int_0^{+\infty}f(x, y)dy$ сходятся равномерно
на любом вложенном $[a; b]$, потому что любую из степеней икса и игрека мы можем
мажорировать через экспоненту $\implies \int_0^{+\infty}dy$ можно внести
внутрь $\int_0^{+\infty}dx$ (по 6.3. Теорема о несобственном интегрировании по параметру под знаком несобственного интеграла).
