\section{Лекция 5 - 29.09.2020 - Исследование сходимости функциональных рядов}
\subsection{Свойства равномерно сходящейся последовательности}
\begin{enumerate}
\item $-\infty \leq a < b \leq +\infty$, рассмотрим $D= (a; b)$, $D = [a;b]$

      Пусть $f_n \to f$, $x \in D$, $y_n = \lim_{x\to x_0} f_n(x)$, $\{y_n\}$ -- сход., $y_n \to y$

      Тогда $\lim_{x \to x_0} f(x) = y$, т.е. $\lim_{x\to x_0}(\lim_{n \to \infty} f_n(x)) = \lim_{n\to \infty} (\lim_{x \to x_0} f_n(x))$

      \begin{proof}

        $|y -f(x)| \leq |y - y_n| + |y_n - f_n(x)| + |f_n(x) - f(x)|$


        Пусть $n$ такое, что $|y - y_n| < \dfrac{\varepsilon}{3}, ||f_n - f|| < \dfrac{\varepsilon}{3}$, $|x - x_0| < \delta, |f(x) - y| < \dfrac{\varepsilon}{3}$


        Тогда $|y -f(x)| \leq |y - y_n| + |y_n - f_n(x)| + |f_n(x) - f(x)| < \varepsilon$
      \end{proof}
\item $-\infty \leq a < b \leq +\infty$, рассмотрим $D= (a; b)$, $D = [a;b]$

      Пусть $f_n$ дифференцируемы на $D$, $f'_n \overset{D}{\rightrightarrows} g, \exists c \in D: \{f_n(c)\}$ сход

      Тогда $\exists f: f_n \to f$ (причем, если $D$ огр., то сходимость равномерная)

      $f$ -- дифференцируема, $f' = g$.

      $(\lim_{n \to \infty} f_n(x))' = \lim_{n \to \infty} f'_n(x)$
\item $-\infty < a < b < +\infty$, $D=[a;b]$
      
      Пусть $f_n$ непрерывны на $D$, $f_n \overset{D}{\rightrightarrows} f$ $(\implies f \text{непрерывна на} D)$

      Тогда $\int_{a}^{x} f_n(t)dt \to^{D} \int_{a}^{x} f(t)dt$
\end{enumerate}

\subsection{Равномерная сходимость функционального ряда}
$D \subseteq \RR$, $a_n: D \to R$

Функциональный ряд: $\sum_{n=1}^{\infty} a_n(x)$

Частичные суммы: $S_N(x) = \sum_{n=1}^{N} a_n(x)$

Множество абсолютной сходимости -- множество всех тех значений $x$, при которых ряд сходится абсолютно.

\subsection{Необходимое условие равномерной сходимости}

Если $\sum_{n=1}^{\infty} a_n(x)$ равномерно сходится к сумме $S(x)$, то $a_n \overset{D}{\rightrightarrows} 0$

\begin{proof}
$S_n(x) = a_1(x) + \dots + a_n(x)$, $a_n(x) = S_n(x) - S_{n-1}(x)$

$S_n \overset{D}{\rightrightarrows} S \implies a_n \overset{D}{\rightrightarrows} (S - S) = 0$
\end{proof}

\begin{example}
$\sum_{n=0}^{\infty} \dfrac{x^n}{n!}, D = \RR$ -- не является сходящейся равномерно, т.к. $\dfrac{x^n}{n!} ! \to^\RR 0$
\end{example}

\subsection{Критерий Коши равномерной сходимости}
\begin{theorem}
Функциональный ряд $\sum_{n=1}^{\infty} a_n(x)$ сходится равномерно на $D \iff$ $\forall \varepsilon > 0$ $\exists N(\varepsilon)$, $\forall n \geq N$, $\forall m$: 
$$||a_n + a_{n + 1} + \dots + a_{n + m}|| < \varepsilon$$

Т.е. $|a_n(x) + a_{n + 1}(x) + \dots + a_{n + m}(x)| < \varepsilon$ $\forall x \in D$.
\end{theorem}

Отрицание: если $\exists \{x_n\} \subset D$, $\exists \{m_n\} \in \NN$, $\exists \varepsilon_0$:
$$|a_n(x_n) + a_{n + 1}(x_n) + \dots + a_{m_n}(x_n)| > \varepsilon_0$$

То ряд не является сходящимся равномерно.

\begin{example}
$\sum_{n = 1}^{\infty} \dfrac{x}{x^2 + n^2}, D = \RR$ -- сходится, т.к. $\approx \sum \dfrac{1}{n^2}$
      Докажем, что сходится неравномерно. Возьмём $x_n = n$, $m_n = 2n$:
      $$\dfrac{n}{n^2 + n^2} + \dfrac{n}{n^2 + (n + 1)^2} + \dots + \dfrac{n}{n^2 + (2n)^2} > \dfrac{n}{5n^2} \cdot n = \dfrac{1}{5}$$
\end{example}
\subsection{Признаки Вейерштрасса и Даламбера}
\begin{theorem}
(Признак Вейерштрасса) Если $|a_n(x)| \leq b_n$ при $\forall n \geq n_0$, $\forall x \in D$, а ряд $\sum b_n$ сходится, то $\sum a_n(x)$ сходится на $D$ абсолютно и равномерно.
\end{theorem}

\begin{proof}
$|a_n(x) + a_{n + 1}(x) + \dots + a_{n + m}(x)| \leq b_n + b_{n+1} + \dots + b_{n + m} < \varepsilon$
\end{proof}

\begin{theorem}
(Признак Даламбера) Если $\exists q < 1$: $|a_{n+1}(x)| \leq q \cdot |a_n(x)|$ при $\forall n \geq n_0$, $\forall x \in D$, причём $a_{n_0}(x)$ -- ограничена на $D$, то $\sum a_n(x)$ сходится на $D$ абсолютно и равномерно.
\end{theorem}

\begin{example}
$\sum_{n = 0}^{\infty} \dfrac{x^n}{n!}$, $D=[-r; r]$, $r > 0$

$\left|\dfrac{x^{n + 1}}{(n + 1)!}\right| \leq q \cdot \left|\dfrac{x^n}{n!}\right|$

$\left|\dfrac{x}{n + 1}\right| \leq q$. Пусть $n_0: \dfrac{r}{n_0 + 1} < 1$, берём $q = \dfrac{r}{n_0 + 1}$. Значит, ряд абсолютно и равномерно сходится.
\end{example}

\subsection{Признак Лейбница}

Знакочередующийся функциональный ряд: $\sum_{n=1}^{\infty} (-1)^n \cdot u_n(x)$, $u_n(x) \geq 0$ на $D$.

\begin{theorem}
(Признак Лейбница) Если $u_n(x) \downarrow_{(n)}$ и $u_n \overset{D}{\rightrightarrows} 0$, то ряд сходится равномерно.
\end{theorem}

\begin{example}
$\sum \dfrac{(-1)^n}{(n + x)^p} \downarrow_{(n)}$, $|u_n(x)| \leq \dfrac{1}{n^p} \to 0 \implies u_n \to^0 0$ 
\end{example}

\subsection{Признаки Дирихле и Абеля}

Рассмотрим ряд $\sum_{n=1}^{\infty} a_n(x) \cdot b_n{x}$

\begin{theorem}
(Признак Дирихле) Если $a_n(x) \downarrow_{(n)}$ и $a_n \overset{D}{\rightrightarrows} 0$, а $||b_1 + \dots + b_n|| \leq C$ $\forall n$, то ряд равномерно сходится на $D$.
\end{theorem}

\begin{theorem}
(Признак Абеля) Если $a_n(x)$ монотонна по $n$ (при $\forall x \in D$), и $||a_n|| \leq C$ при всех $n$, а ряд $\sum b_n(x)$ сходится равномерно, то ряд $\sum_{n=1}^{\infty} a_n(x) \cdot b_n(x)$ сходится равномерно.
\end{theorem}

\subsection{Свойства равномерно сходящегося ряда}

\begin{enumerate}
    \item $-\infty \leq a < b \leq +\infty$, $D= (a; b)$, $D = [a; b]$
          
          Пусть функциональный ряд $\sum_{n=1}^{\infty} c_n(x)$ сходится равномерно на $D$, $x_0 \in D$, $\exists \lim_{x \to x_0} c_n(x) = y_n$ и $\exists \sum_{n=1}^{\infty} y_n = y$.

          Тогда $\lim_{x \to x_0} \sum_{n = 1}^{\infty} c_n(x) = \sum_{n=1}^{\infty} \lim_{x \to x_0} c_n(x) = \sum_{n=1}^{\infty} y_n = y$
    \item $-\infty \leq a < b \leq +\infty$, $D= (a; b)$, $D = [a; b]$
          
          Пусть $c_n(x)$ дифференцируемы на $D$ и $\sum_{n=1}^{\infty} c'_n(x)$ сходится равномерно на $D$.

          Тогда ряд $\sum_{n=1}^{\infty} c_n(x)$ сходится на $D$ (а если $D$ огр, то сходится равномерно), а его сумма будет дифференцируемой функцией на $D$ и $\left(\sum_{n=1}^{\infty} c_n(x)\right)' = \sum_{n=1}^{\infty} c'_n(x)$
    \item $-\infty < a < b < +\infty$, $D= (a; b)$, $D = [a; b]$
    
          $\int_{a}^{x}\left(\sum_{n=1}^{\infty} c_n(t)\right) dt = \sum_{n=1}^{\infty} \int_{a}^{x} c_n(t) dt$ -- сходится равномерно на $D$.
\end{enumerate}