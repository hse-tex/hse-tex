\section{Лекция 6 - 6.10.2020 - Степенные ряды}
\subsection{Основные понятия}
$\sum_{n=0}^{\infty} c_n \cdot(x - x_0)^n$

$\{c_n\}$ -- числовая последовательность (коэффициенты), $x_0 \in \RR$, $x \in \RR$

$S_N(x) = \sum_{n=0}^N c_n \cdot (x - x_0)^n$ -- многочлен.
\subsection{Теорема Абеля, радиус и интервал последовательности}
\begin{theorem}
(Абеля)
\begin{enumerate}
\item Если степенной ряд сходится в точке $x_1 \neq x_0$, то он сходится при всех $x : |x - x_0| < |x_1 - x_0|$
\item Если степенной ряд расходится в точке $x_2 \neq x_0$, то он расходится при всех $x : |x - x_0| > |x_2 - x_0|$
\end{enumerate}
\end{theorem}

\begin{proof}
\begin{enumerate}
\item $\left|\sum_{n = m}^{N} c_n (x - x_0)^n\right| = \left|\sum_{n = m}^{N} c_n \cdot (x_1 - x_0)^n \cdot \left(\frac{x-x_0}{x_1 - x_0}\right)^n\right| \leq 
\sum_{n = m}^{N} \left|c_n \cdot (x_1 - x_0)^n \right| \cdot \left|\left(\frac{x-x_0}{x_1 - x_0}\right)^n\right| \leq \varepsilon (q^m + \dots + q^N) \leq \varepsilon \cdot q^m \cdot \frac{1}{1-q} \to 0$
\end{enumerate}
\end{proof}

Пусть:

$R_{cv} = \sup \{|x - x_0|: \text{ряд сходится}\}$

$R_{dv} = \inf \{|x - x_0|: \text{ряд расходится}\} \text{ или} +\infty \text{, если ряд сходится всюду}$

$\exists R = R_{cv} = R{dv}$ -- радиус сходимости.

$\sum_{n=0}^{\infty} c_n \cdot(x - x_0)^n$

Применим радикальный признак Коши:

$\sqrt[n]{|a_n(x)|} = \sqrt[n]{|c_n|} \cdot |x - x_0|$

$\overline{\lim}\sqrt[n]{|a_n(x)|} = |x - x_0| \cdot \overline{\lim}\sqrt[n]{|c_n|}$

Если $|x - x_0| \cdot \overline{\lim}\sqrt[n]{|c_n|} < 1$, то ряд сходится

Если $|x - x_0| \cdot \overline{\lim}\sqrt[n]{|c_n|} > 1$, то ряд расходится

$R = \frac{1}{\overline{\lim}\sqrt[n]{|c_n|}}$ -- формула Коши-Адамара

Pro tip: если $\exists \lim{\left|\frac{c_{n}}{c_{n+1}}\right|}$, то $\lim\sqrt[n]{|c_n|} = \lim\left|\frac{c_{n}}{c_{n+1}}\right|$

\subsection{Равномерная сходимость степенного ряда}

Если $R > 0$, то степенной ряд сходится равномерно при $|x - x_0| \leq r$, где $r < R$ (доказательство через признак Вейерштрасса).

\subsection{Сходимость ряда в граничных точках интервала сходимости}

Пусть $\sum c_n R^n$ сходится. Тогда степенной ряд $\sum c_n (x - x_0)^n$ сходится равномерно на $[x_0; x_0 + R]$.

\begin{proof}
$\sum_{n=0}^{\infty} c_n(x - x_0)^n$ =  $\sum_{n=0}^{\infty} (c_n \cdot R^n) \cdot \left(\frac{x - x_0}{R}\right)^n$

$b_n = c_n \cdot R^n$, $a_n = \left(\frac{x - x_0}{R}\right)^n$

$\sum_{n=0}^{\infty} b_n$ сходится $\implies$ сходится равномерно.

$a_n(x) \downarrow_{(n)}$

Значит, ряд сходится равномерно по признаку Абеля.
\end{proof}

\subsection{Дифференцирование и интегрирование степенного ряда}

$\sum c_n (x - x_0)^n$, $R > 0$ -- его радиус сходимости.

\begin{enumerate}
      \item Дифференцирование
      
      При почленном дифференцировании получаем $\sum_{n=0}^{\infty} c_n \cdot n \cdot (x - x_0)^{n - 1}$
      Его радиус сходимости равен радиусу исходного ряда $\implies$ он сходится равномерно при $|x - x_0| \leq r < R$
      Значит по теореме о почленном дифференцировании функционального ряда:
      $\left(\sum_{n=0}^{\infty}c_n\cdot (x - x_0)^n\right)' = \sum_{n=0}^{\infty} c_n \cdot n \cdot (x - x_0)^{n - 1} = \sum_{n=0}^{\infty} c_{n+1}(n+1)(x - x_0)^n$
      \item Интегрирование
      
      $\int_{x_0}^{x}\left(\sum_{n=0}^{\infty}c_n(t - x_0)^n\right) dt = \sum_{n=0}^{\infty} \frac{c_n}{n+1} (x - x_0)^{n + 1}$
\end{enumerate}

\subsection{Бесконечное дифференцирование}

Функция называется бесконечно дифференцируемой в точке $a$, если $\forall n$ она $n$ раз дифференцируема в точке $a$.

Сумма степенного ряда с $R > 0$ является бесконечно дифференцируемой функцией.

\subsection{Ряд Тейлора}

Если функция $f(x)$ бесконечно дифференцируема в точке $x_0$, то функции $f(x)$ можно сопоставить её ряд Тейлора:

$\sum_{n=0}^{\infty} \frac{f^{(n)}(x_0)}{n!}(x - x_0)^n$

При этом $f(x) = \sum_{n=0}^{N} \frac{f^{(n)}(x_0)}{n!}(x - x_0)^n + r_N(x)$

$r_N(x) = \frac{f^{(N+1)}(x_0+\theta)(x - x_0)}{(N+1)!}(x - x_0)^{N+1}$, $\theta \in (0; 1)$ -- формула Лагранжа

$r_N(x) = \frac{f^{(N+1)}(x_0+\theta)(x - x_0)}{N!}(1-\theta)^N(x - x_0)^{N+1}$, $\theta \in (0; 1)$ -- формула Коши

\begin{definition}
Функция называется аналитической в т.$x_0$, если она представима в окрестности этой точки в виде степенного ряда.
\end{definition}

Не всякая бесконечно дифференцируемая функция будет аналитической:

\begin{example}
$f(x) = \begin{cases}
      e^{-\frac{1}{x^2}}, x \neq 0 \\
      0, x = 0
\end{cases}$

$f(0) = f'(0) = f''(0) = \dots = 0$, ряд Тейлора при $x_0 = 0$ равен 0
\end{example}

\subsubsection{Ряды Тейлора основных элементарных функций}

\begin{enumerate}
\item $e^x$ --- $\sum_{n=0}^{\infty} \dfrac{x^n}{n!}, R = \infty$
\item $(1+x)^p$ --- $\sum_{n=0}^{\infty} \dfrac{(p)_n}{n!}x^n$, где $(p)_n = p(p-1)\dots(p - n + 1), R = 1$
\item $\ln(1 + x)$ --- $\sum_{n=0}^{\infty} \dfrac{(-1)^{(n+1)} x^n}{n!}$
\end{enumerate}