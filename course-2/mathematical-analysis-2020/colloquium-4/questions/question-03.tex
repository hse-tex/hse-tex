\subsection{Дифференциальная 1-форма в области пространства. Перенесение дифференциальной 1-формы на гладкую кривую. Ориентация кривой. Криволинейный интеграл II-го рода. Выражение криволинейного интеграла II-го рода через криволинейный интеграл I-го рода.}

\subsubsection{Дифференциальная 1-форма в области пространства.}
\begin{definition*}
    Пусть в области $D \subset \RR^n$ определены функции: $a_1,\, \ldots,\, a_n : D \to \RR$. Тогда \textbf{линейной дифференциальной формой}, или \textbf{дифференциальной 1-формой} называется следующая линейная комбинация дифференциалов:
    \begin{align*}
        \omega (\overline{x},\, d\overline{x}) = a_1(\overline{x})dx_1 + \ldots + a_n(\overline{x})dx_n.
    \end{align*} 
    ($\omega(\overline{x}, d\overline{x})$~--- функция от двух векторов, $d\overline{x} = \begin{pmatrix}
        dx_1 \\
        \vdots \\
        dx_n
    \end{pmatrix}$~--- вектор, составленный из дифференциалов)
\end{definition*}

Если ввести векторное поле $\overline{a}: D \to \RR^n$, то соответствующую линейную дифференциальную форму можно записать короче:

\begin{align*}
    \omega (\overline{x}, d\overline{x}) = \langle \overline{a}(\overline{x}),\, d\overline{x} \rangle
\end{align*}

Подчеркнём, что линейная дифференциальная форма может быть дифференциалом некоторой функции, а может и не быть. Понятно, что далеко не всякая линейная комбинация дифференциалов окажется дифференциалом какой-то функции.

Физический смысл и примеры: \href{https://youtu.be/WS-N2Dka3xU?list=PLEwK9wdS5g0qV-430pfXzTawd6pI_VUgq&t=912}{бац}.

\subsubsection{Перенесение дифференциальной 1-формы на гладкую кривую.}
Пусть $G \subset \RR$~--- произвольный ограниченный промежуток, $\varphi : G \to \RR^n$~--- непрерывно дифференцируемое, инъективное и локально инъективное отображение, то есть задаёт гладкую кривую $L = \varphi(G)$.

Пусть в $D \subset \RR^n$определена линейная дифференциальная форма $\omega (\overline{x}, d\overline{x})$ и пусть $L \subset D$.

Введём понятие перенесения функции и формы с помощью отображения $\varphi$. Функция $f : D \to \RR$ и линейная дифференциальная форма $\omega$ посредством отображения $\varphi$ переносятся на кривую $L$ следующим образом:
\begin{align*}
    f(\overline{x}) = f(\varphi(u))
\end{align*}
Тем самым, мы перешли от функции, заданной в некоторой области пространства, к функции, заданной на кривой.
\begin{designation}
    $(\varphi^* f)(u) := f(\varphi(u)).$ 
\end{designation}
Аналогично перенос посредством отображения $\varphi$ действует на линейную дифференциальную форму $\omega$:
\begin{align*}
    \omega (\overline{x},\, d\overline{x}) = \omega(\varphi(u), \frac{d\varphi}{du} \cdot du)
\end{align*}
\begin{designation}
    $(\varphi^* \omega)(u,\, du) := \omega(\varphi(u), \frac{d\varphi}{du} \cdot du).$ 
\end{designation}

\subsubsection{Ориентация кривой.}
Пусть $\psi: H \to \RR^n$~--- другая параметризация кривой $L$. Тогда
\begin{align*}
    (\varphi^* \omega)(u,\, du) := \omega(\varphi(u),\, \frac{d\varphi}{du} \cdot du) =
\end{align*}
Заметим, что $\overline{x} \in L$ может представляться как $\varphi(u)$ и как $\psi(v)$.
\begin{align*}
    = \omega(\psi(v), \underbracket{\frac{d\overline{x}}{dv}}_{\frac{d\psi}{dv}} \underbracket{\cdot \frac{dv}{du} \cdot du}_{dv}) = \omega(\psi(v), \frac{d\psi}{dv} \cdot dv) = (\psi^* \omega)(v,\,dv)
\end{align*}
При этом обратим внимание на то, что когда мы делаем замену переменной в дифференциале, якобиан берётся по модулю, поэтому здесь важно, будут ли параметризация функцией $\varphi$ и параметризация функцией $\psi$ задавать одинаковую ориентацию или разную ориентацию. 

Если эти параметризации задают одинаковую ориентацию на кривой $L$, то $\frac{dv}{du} > 0$, значит, $\frac{dv}{du} = \left\lvert \frac{dv}{du} \right\rvert$~--- одномерный якобиан. При этом, в случае одинаковой ориентации, как видно из выкладок (а выкладки сделаны для случая с одинаковой ориентацией), дифференциальная форма не зависит от параметризации.

В противном случае (в случае смены ориентации кривой) знак дифференциальной формы меняется.

Что вообще является ориентацией?

\begin{definition*}
    \textbf{Ориентацией} является направление движения по кривой.
\end{definition*}
(Маевский это явно на доске не записывал, но там идёт пример чё то про интегралы. Ссылка на всякий случай: \href{https://youtu.be/WS-N2Dka3xU?list=PLEwK9wdS5g0qV-430pfXzTawd6pI_VUgq&t=2905}{буп}.)

\subsubsection{Криволинейный интеграл II-го рода. (КРИ-2)}
Пусть $\overline{a}(\overline{x}) : D \to \RR^n$~--- непрерывное векторное поле, $L$~--- гладкая кривая в $D$ с фиксированной ориентацией, тогда:
\begin{definition*}
    \textbf{КРИ-2} от дифференциальной формы $\omega = \langle \overline{a}, d\overline{x} \rangle$ называется:
    \begin{align*}
        \displaystyle
        \int_L \omega = \int_L \langle \overline{a}, d\overline{x} \rangle = \int_L a_1 dx_1 + \ldots + a_n dx_n = \int_G (\varphi^* \omega) = \int_G \langle \overline{a}, \frac{d\varphi}{du} \rangle du = \int_G (a_1 \frac{d\varphi_1}{du} + \ldots + a_n \frac{d\varphi_n}{du})du
    \end{align*}
    ($G$~--- всё ещё произвольный одномерный ограниченный промежуток)
\end{definition*}
В предыдущем пункте мы показали, что дифференциальная форма в случае фиксированной ориентации не зависит от параметризации, поэтому определение корректно.

\subsubsection{Выражение криволинейного интеграла II-го рода через криволинейный интеграл I-го рода.}
Напоминание про КРИ-1:
\begin{align*}
    \int_L f(x) dl = \int_G f(\varphi(u)) \left\lvert \frac{d\varphi}{du} \right\rvert du
\end{align*}
(в силу локальной инъективности $\left\lvert \frac{d\varphi}{du} \right\rvert \neq 0$)
Теперь выражение:
\begin{align*}
    \underbracket{\int_L \omega}_{\text{КРИ-2}} = \int_G \langle \overline{a}, \frac{d\varphi}{du} \rangle du = \int_G \langle \overline{a}, \frac{d\varphi/du}{\left\lvert d\varphi/du \right\rvert} \left\lvert \frac{d\varphi}{du} \right\rvert du = \underbracket{\int_L \langle \overline{a}, \overline{l} \rangle dl}_{\text{КРИ-1}},
\end{align*}
где $\overline{l} = \frac{d\varphi/du}{\left\lvert d\varphi/du \right\rvert}$~--- единичный вектор касательной к кривой в данной точке. 

Заметим, что ориентация задаётся вектором $\overline{l}$ (ну то есть там либо $\overline{l}$, либо $-\overline{l}$).
