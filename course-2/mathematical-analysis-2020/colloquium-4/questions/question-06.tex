\subsection{Формула Остроградского-Гаусса и её приложение к вычислению объема тела. Внешний дифференциал 3-мерной 2-формы, дивергенция векторного поля и краткая запись формы Остроградского-Гаусса.}

\begin{theorem*}
    (Формула Остроградского-Гаусса)

    Пусть $D$ -- замкнутая жорданова область, ограниченная кусочно гладкой поверхностью $S = \partial D$, также
    пусть $\omega = P dy \wedge dz + Q dz \wedge dx + R dx \wedge dy$ -- непрерывно дифференцируемая 2-форма в $D$.

    Тогда $\iint_{\partial D} \omega = \iiint_{D} d\omega$.

\end{theorem*}

\begin{proof}
    См. \href{https://drive.google.com/file/d/1FxWvJSWedEf2lZ2jY8S7vkcWpLBmSr9d/view}{сюда}, страница 228.
\end{proof}

\begin{definition*}
    Внешним дифференциалом назвается выражение $d\omega = \left(\dfrac{\partial P}{\partial x} + \dfrac{\partial Q}{\partial y} + \dfrac{\partial R}{\partial z}\right) dx \wedge dy \wedge dz$.
\end{definition*}

\begin{definition*}
    Выражение $\dfrac{\partial P}{\partial x} + \dfrac{\partial Q}{\partial y} + \dfrac{\partial R}{\partial z}$ называется \textit{дивергенцией} векторного поля. Обозначение: div$(P, Q, R)$
\end{definition*}

Более подробная запись формулы Остроградского-Гаусса: $$\iint_{S} P dy \wedge dz + Q dz \wedge dx + R dx \wedge dy = \iiint_{D}\left(\dfrac{\partial P}{\partial x} + \dfrac{\partial Q}{\partial y} + \dfrac{\partial R}{\partial z}\right) dx \wedge dy \wedge dz$$

Рассмотрим случай $P = x,\ Q = y,\ R = z$, затим, что div$(x, y, z) = 3$, тогда мы можем вычислить обьем $D$:
$$\mu(D) = \dfrac{1}{3} \iiint_{D} 3 dx \wedge dy \wedge dz = \dfrac{1}{3}\oiint_{\partial D} \left(x dy \wedge dz + y dz \wedge dx + z dx \wedge dy\right) = \dfrac{1}{3}\oiint_{\partial D} \langle \overline x, \overline n\rangle ds$$

Где $\overline x = \begin{pmatrix}
	x & y & z
\end{pmatrix}^\top$.
