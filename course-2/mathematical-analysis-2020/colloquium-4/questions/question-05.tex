\subsection{Дифференциальная 2-форма в области пространства. Перенесение дифференциальной 2-формы на гладкую поверхность. Ориентация поверхности и вектор нормали. Поверхностный интеграл II-го рода. Выражение поверхностного интеграла II-го рода через поверхностный интеграл I-го рода.}


Пункты переставлены местами для лучшей понимабельности.

\subsubsection{Ориентация поверхности и вектор нормали}

Пусть~ $G$~ - область в ~$\mathbb{R}^k$~ с фиксированным базисом ~$u = \{ u_1, \ldots, u_k \}$. Пусть ~$\phi: G \rightarrow R^m$~ - непр. дифф. инъективное и локально инъективное отображение. 

\begin{definition*} 
	Ориентацией поверхности $S = \phi(G)$ называется ориентация исходного базиса ~$u = \{ u_1, \ldots, u_k \}$.
\end{definition*}

\vspace{1em}

В каждой точке поверхности $S$ задан базис ~$\dfrac{\partial \phi}{\partial u_1}, \ldots, \dfrac{\partial \phi}{\partial u_k}$.

Рассмотрим случай ~$k = 2 , m = 3$.

\begin{definition*} 		
	Вектором нормали к поверхности $S$ называется векторное произведение $\dfrac{\partial \phi}{\partial u_1} \times \dfrac{\partial \phi}{\partial u_2}$.
\end{definition*}


\subsubsection{Дифференциальная 2-форма в области пространства}

	Пусть ~$D$ - область в ~$\mathbb{R}^n$.

	\begin{definition*} 
		Дифференциальной 2-формой называется линейная комбинация базисных 2-форм вида $dx_i \wedge dx_j$ :
		
		\[
		\omega(\bar{x}, d\bar{x}) = \sum_{i<j} a_{ij}(x) \cdot dx_i \wedge dx_j		
		\]
		
		где ~$a_{ij}: D \rightarrow \mathbb{R}$
		
	\end{definition*}
	
	\vspace{1em}
	
	Например, при $n = 3$:
	\[
		\omega = a_{12} dx_1 \wedge dx_2 + a_{13} dx_1 \wedge dx_3 + a_{23} dx_2 \wedge dx_3
	\] 
	
\subsubsection{Перенесение дифференциальной 2-формы на гладкую поверхность}
	
	Пусть ~$G$ - Жорданова область в ~$\mathbb{R}^2$. Пусть ~$\phi: G \rightarrow \mathbb{R}^n$~ есть непр. дифф. инъективное и локально инъективное отображение. ~$S = \phi(G)$ -- гладкая поверхность. 
	
	Пусть ~$S \subset D$ , где $D$ -- область с заданной дифференциальной 2-формой :
	
	\[
	\omega(\bar{x}, d\bar{x}) = \sum_{i<j} a_{ij}(x) \cdot dx_i \wedge dx_j		
	\]
	
	\begin{definition*} 
		Переносом $\omega$ на поверхность $S$ называется применение операции $\phi^*$ , действующей следующим образом :
		
		\[
		(\phi^* \omega)(u, du) = \omega(x, dx) \Biggl|_{
				x = \phi(u), ~~
				dx = \frac{\partial \phi}{\partial u_1}du_1 + \frac{\partial \phi}{\partial u_2}du_2
		}
		\]
		
	\end{definition*}
	
	\vspace{1em}
	
	Выражение выше \textit{можно} раскрыть следующим образом. Так как ~$\phi$ - векторная функция и ~$x_i = \phi_i(u_1, ~u_2)$
	\[
	dx_i = \frac{\partial \phi_i}{\partial u_1}du_1 + \frac{\partial \phi}{\partial u_2}du_2
	\]
	
	\[
	dx_i \wedge dx_j = 
	\begin{vmatrix} 
	\frac{\partial \phi_i}{\partial u_1} & \frac{\partial \phi_i}{\partial u_2} \\
	\frac{\partial \phi_j}{\partial u_1} & \frac{\partial \phi_j}{\partial u_2}
	\end{vmatrix} du_1 \wedge du_2
	\]
	
	Подставим вышенаписанное в определение 2-формы
	
	\[
	(\phi^* \omega)(u, du) = \sum_{i<j} a_{ij}(\phi(u)) \cdot 
	det \left( \frac{\partial (x_i, x_j)}{\partial (u_1, u_2)}	\right) du_1 \wedge du_2
	\]


\subsubsection{Поверхностный интеграл II-го рода}
	
	\textit{В терминах обозначений из предыдущего пункта}.
	
	\begin{definition*} 	
			
		Поверхностным интегралом II-го рода называется выражение
		\[
			\iint_S \omega(x, dx) = \iint_S  \sum_{i<j} a_{ij}(x) \cdot dx_i \wedge dx_j = \left[ \begin{aligned} \text{подставляем $\phi$, которая} \\ \text{параметризует поверхность $S$} \end{aligned} \right] = \iint_G (\phi^* \omega) (u, du) 
		\]
		
		В развернутой записи
		
		\[
		\iint_G (\phi^* \omega) (u, du) =  \iint_G \sum_{i<j} a_{ij}(\phi(u)) \cdot 
		det \left( \frac{\partial (x_i, x_j)}{\partial (u_1, u_2)}	\right) du_1 du_2
		\]
	\end{definition*}
	

\subsubsection{Выражение поверхностного интеграла II-го рода через поверхностный интеграл I-го рода}

	\textit{В терминах обозначений из предыдущего пункта}.
	
	Сведем поверхностный интеграл II-го рода к интегралу I-го рода на следующем примере для ~$n  = 3$ (как на лекции)
	
	\[
		\iint_S  \sum_{i<j} a_{ij}(x) \cdot dx_i \wedge dx_j = \iint_S v_1 dy \wedge dz + v_2 dz \wedge dx + v_3 dx \wedge dy
	\]
	
	\begin{proposition*}
	Сведение интеграла выше к интегралу I-го рода.
	\[
	\iint_S v_1 dy \wedge dz + \ldots = \left[~ \text{параметризация}~ \bar{\phi}(u_1, u_2) ~\right] 
	= \iint_G (~v_1 \left( \frac{\partial y}{\partial u_1}  \frac{\partial z}{\partial u_2} -  \frac{\partial y}{\partial u_2} \frac{\partial z}{\partial u_1} \right) + ~\ldots )\cdot du_1 du_2 = 
	\]
	
	\begin{equation}
	= \iint_G 	\begin{vmatrix} 
	v_1 & \frac{\partial x}{\partial u_1} & \frac{\partial x}{\partial u_2} \\
	v_2 & \frac{\partial y}{\partial u_1} & \frac{\partial y}{\partial u_2} \\
	v_3 & \frac{\partial z}{\partial u_1} & \frac{\partial z}{\partial u_2}
	\end{vmatrix} du_1 du_2 
	= \iint_G \left\langle \bar{v}, \left[  \frac{\partial x}{\partial u_1},  \frac{\partial x}{\partial u_2} \right] \right\rangle du_1 du_2 
	\label{eqn:wtf}
	\end{equation}
	
	Пусть теперь $\bar{n}$ - вектор нормали
	
	\[
		\bar{n} = \dfrac{  \left[  \frac{\partial x}{\partial u_1},  \frac{\partial x}{\partial u_2} \right] }{ \left|  \left[  \frac{\partial x}{\partial u_1},  \frac{\partial x}{\partial u_2} \right] \right| } 
	\]
	
	Тогда страшную штуку ~(\ref{eqn:wtf}) выше можно переписать так:
	\[
	= \iint_G \left\langle \bar{v}, \bar{n}  \right\rangle \underbrace{ \left|  \left[  \frac{\partial x}{\partial u_1},  \frac{\partial x}{\partial u_2} \right] \right| \cdot du_1 du_2 }_{ds}
	= \underbrace{ \iint_S  \left\langle \bar{v}, \bar{n}  \right\rangle \cdot ds }_{\text{пов. инт. I рода}}
	\]
	
	где $\left[ x, y \right]$ - векторное произведение, $ds$ -- элемент площади поверхности
	
	
	\end{proposition*}


