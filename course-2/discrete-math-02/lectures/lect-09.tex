\ProvidesFile{lect-09.tex}[Лекция 09]

\section{Лекция 9}

В прошлой лекции мы говорили о структурах, об изоморфизме структур, поняли, что выразимые отношения сохраняются всеми автоморфизмами.
Вообще, мы показали как себя ведет значение формулы при изоморфизме.
А теперь мы хотим пойти немножко в другом направлении, и мы хотим установить общие законы нашей логики предикатов.
Неформально выражаясь, логика предикатов --- это свойства формул первого порядка\footnote{так называются те формулы, которые были нами определены.}, которые имеют место для всех структур (интерпретаций).
То есть, это такие свойства, которые не зависят от интерпертации (являются законами логики).

Будем считать, что у нас какая-нибудь сигнатура $\sigma$ фиксирована и мы рассматриваем формулы этой сигнатуры.

\begin{definition}
    Формула $\varphi$ {\it общезначима}, если для любой интерпретации $\MM$, для любой оценки переменных $\pi \colon \Var \to M$
    $$
        [\varphi]_{\MM}(\pi) = 1.
    $$
\end{definition}

Что такое общезначимая формула?
Это такая формула, которая всегда, как ее не интерпретируй, что с ней не делай, она будет принимать значение 1, которое всегда истинно.

\paragraph{Пример}
Формула
$$
    Px \implies (\neg Px \implies Qyz)
$$
является общезначимой.
Почему?
Ну давайте возьмем какую-нибудь интерпретацию, какую-то оценку, то, в конечном счете, значение нашей формулы сведется к
$$
    [\varphi]_{\MM}(\pi) = \boolimplies(\sigma, \boolimplies(\boolneg(\sigma), \tau)),
$$
где $\sigma, \tau \in \{0, 1\}$.
Какие бы значения $\sigma, \tau$ не были, значенеи формулы будет 1 (можно проверить по таблице истинности).
Поэтому эта формула является общезначной.

Следует заметить, что не всякая общезначная формула общезначна в силу импликации и отрицания.

\paragraph{Пример}
Рассмотрим формулу
$$
    \varphi = \forall x Px \implies \exists x Px.
$$
Понятно, что в ней ничего не зависит от того, что мы понимаем под свойством $P$.
Это будет всегда так, если $M \neq \varnothing$ (по определению $M$, это правда).

\paragraph{Контр-пример}
Рассмотрим формулу
$$
    \varphi \eqcirc Px \lor \neg Qy.
$$
Придумаем интерпретацию и оценку переменных такие, что $\varphi$ ложна.
Положим $M = \N$, $P^{\MM} = Q^{\MM} =$ \enquote{равно 0}, $\pi(x) = 2020$, $\pi(y) = 0$.
Тогда значение $\varphi$ ложно.

И так, общезначимые формулы --- это законы логики.
Наша задача --- понять, как эти законы устроены.
Двойственным к общезначности является следующее понятие:

\begin{definition}
    Формула $\varphi$ {\it выполнима}, если существует интерпретация $\MM$ и оценка $\pi$ такая, что
    $$
        [\varphi]_{\MM}(\pi) = 1.
    $$
\end{definition}

То есть, общезначимая формула истинна всегда, а выполнимая истинна в каком-то случае.

\begin{lemma}
    Формула $\varphi$ общезначима $\iff$ $\neg \varphi$ не выполнима.
\end{lemma}

\begin{lemma}
    Формула $\varphi$ выполнима $\iff$ $\neg \varphi$ не общезначима.
\end{lemma}

\begin{definition}
    Формулы $\varphi$ и $\psi$ (логически) эквивалентны, если для любой структуры $\MM$, для любой оценки $\pi$
    $$
        [\varphi]_{\MM}(\pi) = [\psi]_{\MM}(\pi).
    $$
\end{definition}

\begin{lemma}
    Верны следующие утверждения:
    \begin{enumerate}
        \item $\varphi \equiv \varphi$;
        \item $\varphi \equiv \psi \implies \psi \equiv \varphi$;
        \item $\varphi \equiv \psi \land \psi \equiv \theta \implies \psi \equiv \theta$.
    \end{enumerate}
\end{lemma}

Таким образом, наша эквивалентность является отношением эквивалентности на множестве формул.

\begin{lemma}[о конгруэнции отношения эквивалентности]
    Если $\varphi \equiv \varphi^{\prime}$, то
    \begin{itemize}
        \item $\neg \varphi \equiv \neg \varphi^{\prime}$;
        \item $\varphi \land \psi \equiv \varphi^{\prime} \land \psi$.
        \item $\varphi \lor \psi \equiv \varphi^{\prime} \lor \psi$.
        \item $\varphi \implies \psi \equiv \varphi^{\prime} \implies \psi$.
        \item $\psi \implies \varphi \equiv \psi \implies \varphi^{\prime}$.
        \item $\forall x \varphi \equiv \forall x \varphi^{\prime}$.
        \item $\exists x \varphi \equiv \exists x \varphi^{\prime}$.
    \end{itemize}
\end{lemma}

Это связано с тем, что значение сложной формулы определяется через значения ее подформул.
И если значение $\varphi$ всегда можно заменить на значение $\varphi^{\prime}$ в любой интерпретации и при любой оценке, то тогда, конечно же, значение исходной формулы не изменится.
Проверим один пункт с квантором.

\begin{proof}
    $[\forall x \varphi]_{\MM}(\pi) = 1 \iff \forall m \in M~[\varphi]\left( \pi_{x}^{m} \right) = 1$, но $[\varphi]\left( \pi_{x}^{m} \right) = [\varphi^{\prime}]\left( \pi_{x}^{m} \right)$, тогда $[\varphi^{\prime}]\left( \pi_{x}^{m} \right) = 1$.
\end{proof}

\paragraph{Упражнение}
Существуют $\varphi, \varphi^{\prime}$ такие, что
$$
    \forall x \varphi = \forall x \varphi^{\prime},
$$
но $\varphi \centernot\equiv \varphi^{\prime}$, то есть снимать квантор таким образом нельзя.

Можно ли через эквивалентность выразить общезначимость (и наоборот)?
Ну конечно можно!

\begin{lemma}
    $\varphi \equiv \psi \iff $ общезначима формула
    $$
        (\varphi \implies \psi) \land (\psi \implies \varphi).
    $$
\end{lemma}

\begin{proof}
    Значение импликации один тогда и только тогда, когда значение посылки не больше значения заключения, а в лемме сказано, что они одинаковы.
    Тогда формулы $\varphi$ и $\psi$ эквивалентны.
\end{proof}

\begin{lemma}
    Формула $\varphi$ общезначима тогда и только тогда, когда
    $$
        \varphi \equiv \text{\enquote{заведомая истина}}.
    $$
\end{lemma}

Что это за заведомая истина?
На самом деле, обычно, логики считают, что есть связки
\begin{itemize}
     \item $\top$ --- тождественная истина.
     \item $\bot$ --- тождественная ложь.
 \end{itemize}
Если же их в сигнатуре нет, то приходится как-то выкручиваться.
Например, если в сигнатуре есть предикатный символ $P$, то в качестве тождественной истины можно взять формулу $Px \lor \neg Px$.

\begin{corollary}
    Все общезначимые формулы эквивалентны друг другу.
\end{corollary}

Теперь давайте посмотрим некоторые свойства, которые характерны именно для кванторов.
Одно из этих свойств очень простое, но в тоже время весьма полезное --- лемма о фиктивном кванторе.

\begin{lemma}[о фиктивном кванторе]
    Если $x \notin \FreeVar(\varphi)$, то $\forall x \varphi \equiv \varphi$.
    Аналогично, $\exists x \varphi \equiv \varphi$.
\end{lemma}

\begin{proof}
    Нам надо доказать эквивалентность, то есть то, что происходит в любой интерпретации $\MM$ и при любой оценке $\pi$.
    Рассмотрим формулу
    $$
        [\forall x \varphi]_{\MM}(\pi) = 1 \iff \forall m \in M~ [\varphi](\pi_{x}^{m}) = 1.
    $$
    Мы знаем, что если для любого $z \in \FreeVar(\psi)$ две оценки $\pi_{1}(z)$ и $\pi_{2}(z)$ совпадают, то $[\psi]_{\MM}(\pi_{1}) = [\psi]_{\MM}(\pi_{2})$. % TODO: вставить ссылку на утверждение
    Мы знаем, что $x \notin \FreeVar(\varphi)$, посмотрим тогда на следующее выражение
    $$
        \pi_{x}^{m}(z) = \pi(z),
    $$
    где
    $$
        \pi_{x}^{m}(z) = \begin{cases}
            m, & y = x,  \\
            \pi(y), & y \neq x.
        \end{cases}
    $$
    Заметим, что значение $\pi_{x}^{m}$ совпадает cо значением $\pi$ везде, кроме $x$, тогда $[\psi]_{\MM}(\pi_{1}) = [\psi]_{\MM}(\pi_{2}) \implies [\varphi]_{\MM}\left(\pi_{x}^{m}\right) = [\varphi]_{\MM}(\pi)$.
    Получается, что $[\varphi]\left(\pi_{x}^{m}\right) = [\varphi](\pi)$.
    Дальше мы снимаем внешний фиктивный квантор: мы получили, что $\forall m \in M~[\varphi](\pi) = 1$, что не зависит от $m$, тогда это равносильно $[\varphi](\pi) = 1$.
\end{proof}

\begin{corollary}
    $\forall x \exists x \varphi \equiv \exists x \varphi$.
\end{corollary}

\begin{statement}
    Формула $\varphi$ общезначима $\iff$ $\forall x \varphi$ общезначима.
\end{statement}

\begin{proof}~
    \begin{description}
        \item[$\implies$] Дано: $\forall \MM,~\forall \pi \colon [\varphi]_{\MM}(\pi) = 1$.
        Хотим: $\forall \MM,~\forall \rho \colon [\forall x \varphi]_{\MM}(\rho) = 1$.
        Зафиксируем какие-то $\MM$ и $\rho$, тогда
        $$
            [\forall x \varphi](\rho) = 1 \iff \forall m \in M~ [\varphi]\left(\rho_{x}^{m}\right) = 1.
        $$
        Давайте положим $\pi = \rho_{x}^{m}$, тогда $[\varphi](\pi) = 1$.
        Поскольку $\varphi$ общезначима, то можно навесить квантор всеобщности, поскольку нам нужно, чтобы $\varphi$ была истинна для некоторых специальных оценок, а она истинна всегда.
        \item[$\impliedby$] Дано: $\forall \MM,~\forall \pi \colon [\forall\varphi]_{\MM}(\pi) = 1$.
        Хотим: $\forall \MM,~\forall \rho \colon [\varphi]_{\MM}(\rho) = 1$.
        Мы знаем, что $\forall m \in M \colon [\varphi]_{\MM}\left(\pi_{x}^{m}\right) = 1$.
        Положим $\pi = \rho$, $m = \rho(x)$, тогда $\pi_{x}^{m} = \rho_{x}^{\rho(x)}$.
        Эта оценка ведет себя как $\rho$ везде, кроме $x$, где она также ведет себя как $\rho$, то есть $\pi_{x}^{m} = \rho$.
        Имеем $[\varphi]_{\MM}(\rho) = 1$. \qedhere
    \end{description}
\end{proof}

\paragraph{Упражнение}
$\varphi$ выполнима $\iff$ $\exists x \varphi$ выполнима.

\paragraph{Эквивалентности алгебры логики}
\begin{align}
    \varphi \implies \psi &\equiv \neg \varphi \lor \psi, \\
    \neg \varphi \implies \neg \psi &\equiv \psi \implies \varphi, \\
    \neg (\varphi \implies \psi) &\equiv \varphi \land \neg \psi, \\
    \neg \neg \varphi &\equiv \varphi, \\
    \neg (\varphi \land \psi) &\equiv \neg \varphi \lor \neg \psi, \\
    \neg (\varphi \lor \psi) &\equiv \neg \varphi \land \neg \psi, \\
    \varphi \land (\psi \lor \theta) &\equiv  (\varphi \land \psi) \lor (\varphi \land \theta), \\
    \varphi \lor (\psi \land \theta) &\equiv (\varphi \lor \psi) \land (\varphi \lor \theta), \\
    \varphi \land (\varphi \lor \psi) &\equiv \varphi, \\
    \varphi \lor (\varphi \land \psi) &\equiv \varphi, \\
    \varphi \land \psi &\equiv \psi \land \varphi, \\
    \varphi \land \varphi &\equiv \varphi, \\
    (\varphi \land \psi) \land \theta &\equiv \varphi \land (\psi \land \theta).
 \end{align}
Последние три эквивалентности также верны, если $\land$ заменить на $\lor$.
Все эквивалентности доказываются по таблице истинности.

\begin{theorem}[основные эквивалентности с кванторами]~
    \begin{enumerate}
        \item \begin{align}
            \forall x (\varphi \land \psi) &\equiv \forall x \varphi \land \forall x \psi, \\
            \exists x (\varphi \lor \psi) &\equiv \exists x \varphi \lor \exists x \psi.
        \end{align}
        \item Пусть $x \notin \FreeVar(\psi)$, тогда \begin{align}
            \forall x (\varphi \land \psi) &\equiv \forall x \varphi \land \psi, \\
            \exists x (\varphi \land \psi) &\equiv \exists x \varphi \land \psi, \\
            \forall x (\varphi \lor \psi) &\equiv \forall x \varphi \lor \psi, \\
            \exists x (\varphi \lor \psi) &\equiv \exists x \varphi \lor \psi.
        \end{align}
        \item \begin{align}
            \neg \forall x \varphi &\equiv \exists x \neg \varphi, \\
            \neg \exists x \varphi &\equiv \forall x \neg \varphi.
        \end{align}
    \end{enumerate}
\end{theorem}

Доказательство проводится за счет нашего определения: мы постепенно превращаем это утверждение в утверждение про наши неформальные кванторы, никакого другого инструментария у нас нет.
Проверим некоторые из них.

\begin{proof}
    Зафиксируем $\MM$ и $\pi$.

    \begin{description}
        \item[2.2] $[\exists x (\varphi \land \psi)](\pi) = 1 \iff \exists m \in M~ [\varphi \land \psi]\left(\pi_{x}^{m}\right) = 1$.
        Заметим, что $\varphi \land \psi = \text{И}\left([\varphi]\left(\pi_{x}^{m}\right), [\psi]\left(\pi_{x}^{m}\right)\right)$.
        Мы знаем, что $x$ не входит свободно в $\psi$, то есть $x \notin \FreeVar(\psi)$.
        Что это означает?
        Давайте посмотрим на оценки $\pi_{x}^{m}$ и $\pi$.
        Они ведь совпадают всюду, кроме $x$.
        В частности, $\forall z \in \FreeVar(\psi)$ $\pi_{x}^{m}(z) = \pi(z)$.
        Отсюда, по теореме (хз какой, надо найти), $[\psi]\left(\pi_{x}^{m}\right) = [\psi](\pi)$.
        У нас получается следующая вещь: найдется элемент $m \in M$ такой, что $\text{И}\left([\varphi]\left(\pi_{x}^{m}\right), [\psi]\left(\pi\right)\right) = 1$.
        Это равносильно тому, что
        $$
            \exists m \in M \quad ([\varphi]\left(\pi_{x}^{m}\right) = 1 \text{ и } [\psi](\pi) = 1).
        $$
        Теперь заметим, что значение $[\psi](\pi)$ не зависит от $m$, поэтому это можно переписать следующим образом:
        \begin{equation*}
            (\exists m \in M [\varphi]\left(\pi_{x}^{m}\right) = 1) \text{ и } [\psi](\pi) = 1 \iff [\exists x \varphi](\pi) = 1 \text{ и } [\psi](\pi) = 1 \iff [\exists x \varphi](\pi) = 1.
            \qedhere
        \end{equation*}
    \end{description}
\end{proof}

\begin{corollary}[Выразимость одного квантора через другой]
    $\forall x \varphi \equiv \forall x \neg \neg \varphi \equiv \neg \exists x \neg \varphi$.
    $\exists x \varphi \equiv \neg \forall x \neg \varphi$.
\end{corollary}

\begin{corollary}
    Пусть $x \notin \FreeVar(\psi)$, тогда
    \begin{align}
        \forall x (\varphi \implies \psi) &\equiv \exists x\varphi \implies \psi, \\
        \exists x (\varphi \implies \psi) &\equiv \forall x\varphi \implies \psi, \\
        \forall x (\psi \implies \varphi) &\equiv \psi \implies \forall \varphi, \\
        \exists x (\psi \implies \varphi) &\equiv \psi \implies \exists \varphi.
    \end{align}
\end{corollary}

Как это запомнить?
При вынесении квантора из посылки, он меняется на противоположный.
При вынесении квантора из влечения, он остается тем же.

\begin{proof}
    $\forall x (\varphi \implies \psi) \equiv \forall x (\neg \varphi \lor \psi)$.
    Мы помним, что $x \notin \FreeVar(\psi)$, тогда
    \begin{equation*}
        \forall x (\neg \varphi \lor \psi) \equiv \forall x \neg \varphi \lor \psi \equiv \neg \exists \varphi \lor \psi \equiv \exists x \varphi \implies \psi.
        \qedhere
    \end{equation*}
\end{proof}

\subsection{Дизъюнктивные и конъюнктивные нормальные формы}

Будем работать с ДНФ и КНФ для формул первого порядка.
Давайте рассмотрим формулу
$$
    (\forall x Px \lor \neg Qy) \implies \exists x Px.
$$
Обратим внимание, что здесь есть кванторы, здесь есть атомы (отношения, в которые подставлены термы).
Давайте выделим блоки $\forall x Px$, $Qy$ и $\exists x P x$, и посмотрим на структуру нашей формулы
$$
    (~\square~ \lor \neg~\square~) \implies \square.
$$
Здесь наши квадраты --- либо атомарные формулы, либо формулы, начинающиеся с квантора\footnote{такие вещи называются {\it квазиатомами}}.
Мы хотим проанализировать формулу с точки зрения логических связок.

Пусть фиксирован набор попарно различных формул (атомарных или вида $\forall x \varphi$, $\exists x \varphi$) $(F_{1}, \ldots, F_{n})$ в сигнатуре $\sigma$.
\begin{definition}
    Определим множество $\BB(F_{1}, \ldots, F_{n})$ {\it булевых комбинаций} формул из $(F_{1}, \ldots, F_{n})$ следующим образом:
    \begin{enumerate}
        \item $\forall i~F_{i} \in \BB(F_{1}, \ldots, F_{n})$;
        \item $\varphi, \psi \in \BB(F_{1}, \ldots, F_{n}) \implies \neg \varphi, (\varphi \land \psi), (\varphi \lor \psi), (\varphi \implies \psi), (\varphi \iff \psi) \in \BB(F_{1}, \ldots, F_{n})$.
    \end{enumerate}
\end{definition}

Мы хотим понять, как определяется истинность булевой комбинации.
Это очень просто.
Каждой $\varphi \in \BB(F_{1}, \ldots, F_{n})$ поставим в соответствие булеву функцию $f_{\varphi} \colon \{0, 1\}^{n} \to \{0, 1\}$ таким образом, что (дальше рекурсией по построению)
\begin{enumerate}
    \item $\varphi \eqcirc F_{i} \implies f_{\varphi}(x_{1}, \ldots, x_{n}) = x_{i}$;
    \item \begin{enumerate}
        \item $\varphi \eqcirc \neg \psi \implies f_{\varphi}(\vec{x}) = \text{НЕ}(f_{\psi}(\vec{x}))$;
        \item $\varphi \eqcirc \psi \lor \theta \implies f_{\varphi}(\vec{x}) = \text{ИЛИ}(f_{\psi}(\vec{x}), f_{\theta}(\vec{x}))$;
        \item Аналогично для других связок.
    \end{enumerate}
\end{enumerate}

\begin{lemma} \label{lemma::09::01}
    Для любой формулы $\varphi \in \BB(F_{1}, \ldots, F_{n})$, $\forall M$, $\forall \pi$
    $$
        [\varphi]_{\MM}(\pi) = f_{\varphi}\left([F_{1}]_{\MM}(\pi), \ldots, [F_{n}]_{\MM}(\pi)\right).
    $$
\end{lemma}

\begin{proof}
    Индукцией по построению $\varphi$.
\end{proof}

\begin{definition}
    Формула $\varphi \in \BB(F_{1}, \ldots, F_{n})$ называется {\it тавтологией}, если $\forall \vec{x}$ $f_{\varphi}(\vec{x}) = 1$.
\end{definition}

\begin{statement}
    Если $\varphi$ тавтология, то $\varphi$ общезначима.
\end{statement}

\begin{proof}
    Следует из леммы \ref{lemma::09::01}.
\end{proof}

Не всякая общезначимая формула является тавтологией.
Например, формула $\varphi = \forall x Px \implies \exists x Px$ не является тавтологией, потому что, если представить ее в виде булевой комбинации, она будет иметь вид $F_{1} \implies F_{2}$, и ей будет соответствовать функция $\text{ИМП}(x_{1}, x_{2})$, которая не всегда истинна.

\begin{definition}[ДНФ над набором формул]
    Пусть фиксирован набор формул $(F_{1}, \ldots, F_{n})$ (как прежде).
    Определим дизъюнктивную нормальную форму следующим образом:
    \begin{enumerate}
        \item $F_{i}$ или $\neg F_{i}$ --- {\it литералы};
        \item Если $l_{1}, \ldots, l_{k}$ --- литералы, то $l_{1} \land \ldots \land  l_{k}$ --- {\it элементарная конъюнкция};
        \item Если $c_{1}, \ldots, c_{m}$ --- элементарные конъюнкции, то $c_{1} \lor \ldots \lor c_{m}$ --- ДНФ над набором $(F_{1}, \ldots, F_{n})$.
    \end{enumerate}
\end{definition}

\begin{definition}[КНФ над набором формул]
    Пусть фиксирован набор формул $(F_{1}, \ldots, F_{n})$ (как прежде).
    Определим конъюнктивную нормальную форму следующим образом:
    \begin{enumerate}
        \item $F_{i}$ или $\neg F_{i}$ --- {\it литералы};
        \item Если $l_{1}, \ldots, l_{k}$ --- литералы, то $l_{1} \lor \ldots \lor  l_{k}$ --- {\it элементарная дизъюнкция};
        \item Если $d_{1}, \ldots, d_{m}$ --- элементарные дизъюнкции, то $d_{1} \land \ldots \land d_{m}$ --- КНФ над набором $(F_{1}, \ldots, F_{n})$.
    \end{enumerate}
\end{definition}

ДНФ и КНФ, вообще говоря, являются разными объектами.
Например,
$$
    (F_{1} \land \neg F_{2} \land F_{1}) \lor (\neg F_{3}) \lor (F_{1} \land F_{4})
$$
является ДНФ, но не является КНФ, потому что в дереве разбора формулы ниже дизъюнкции стоит конъюнкции.

Однако, существуют формулы, которые являются одновременно КНФ и ДНФ, например
$$
    F_{1} \land \neg F_{2}; \quad F_{3} \land \neg F_{4} \land F_{5}.
$$
На самом деле, одновременно КНФ и ДНФ являются литералы и элементарные конъюнкции (дизъюнкции).

\begin{corollary}
    $\CNF\left(\vec{F}\right) \cup \DNF\left(\vec{F}\right) \subseteq \BB\left(\vec{F}\right)$.
\end{corollary}
