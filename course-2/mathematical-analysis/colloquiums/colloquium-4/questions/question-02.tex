\subsection{Кусочно-гладкая поверхность и её площадь. Элемент площади для параметрически заданной поверхности. Поверхностный интеграл I-го рода.}
\begin{center}
        \textbf{Кусочно-гладкая поверхность}
    \end{center}
    \textbf{Область} - открытое связное множество в $\mathbb{R}^k$\\
    Замкнутая область - это замыкание некоторой области\\
    Жорданова область - ограниченная область, измеримая по Жордану\\
    Пусть $G \subset \mathbb{R}^k, k<m$ - замкнутая жорданова область и $\varphi: G \rightarrow \mathbb{R}^m$ - непрерывно дифференцируемая инъективная функция.\\
    $x= \varphi(u), x_i = \phi_i(u_1, ..., u_k)$\\
    $x \in \mathbb{R}^m, u \in G$, причем \\
    $$\frac{\partial x}{\partial u} = 
    \begin{pmatrix}
    \frac{\partial x_1}{\partial u_1} & ... & \frac{\partial x_1}{\partial }u_k \\
    \vdots & ... & \vdots \\
    \frac{\partial x_m}{\partial u_1} & ... & \frac{\partial x_m}{\partial u_k} \\
    \end{pmatrix}$$\\
    имеет в любом $u \in G$ максимальный ранг $k$.\\
    Тогда образ $\varphi(G)$ называется гладкой $k$-мерной поверхностью $S$ (при $k \geq 2$).\\
    $S = \bigsqcup_{i=1}^{n} S_i$ - кусочно гладкая поверхность\\
    
    \begin{center}
        \textbf{Элемент площади для параметрически заданной поверхности}
    \end{center}
    $G \subset \mathbb{R}^2$ - замкнутая жорданова область\\
    $\varphi: G \rightarrow \mathbb{R}^m$ - параметризующее отображение\\
    $(u, v) \in G$ - параметры поверхности\\
    $$\mu(S) = \iint_G \sqrt{
    \begin{vmatrix}
    |\frac{\partial x}{\partial u}|^2 & <\frac{\partial x}{\partial u}, \frac{\partial x}{\partial v}>\\
    <\frac{\partial x}{\partial u}, \frac{\partial x}{\partial v}> & |\frac{\partial x}{\partial v}|^2\\
    \end{vmatrix}
    }\ dudv = \iint_G \sqrt{|\frac{\partial x}{\partial u}|^2 \cdot |\frac{\partial x}{\partial v}|^2 - <\frac{\partial x}{\partial u}, \frac{\partial x}{\partial v}>^2}\ dudv$$
    
    \begin{center}
        \textbf{Поверхностный интеграл I-го рода}
    \end{center}
    $G \subset \mathbb{R}^2$\\
    $\varphi: G \rightarrow \mathbb{R}^m$\\
    $S = \varphi(G)$\\
    $f: S \rightarrow \mathbb{R}$\\
    $f(x) = f(\varphi)$\\
    $f(x_1, x_2, x_3) = f(\varphi_1(u, v), \varphi_2(u, v), \varphi_3(u, v))$\\
    $ds = \sqrt{|\frac{\partial x}{\partial u}|^2 \cdot |\frac{\partial x}{\partial v}|^2 - <\frac{\partial x}{\partial u}, \frac{\partial x}{\partial v}>^2}\ dudv$
    $$\int_S f(x) ds = \iint_G f(\phi(u, v)) \cdot \sqrt{|\frac{\partial x}{\partial u}|^2 \cdot |\frac{\partial x}{\partial v}|^2 - <\frac{\partial x}{\partial u}, \frac{\partial x}{\partial v}>^2}\ dudv$$
