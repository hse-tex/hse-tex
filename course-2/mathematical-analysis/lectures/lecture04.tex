
\section{Компакты в $\mathbb{R}^n$}
\subsection{Замкнутый брус — компакт}
\theorem Пусть $I\subset\mathbb{R}^n$ — замкнутый брус $\Longrightarrow I$ — компакт

\proof Пойдем от противного

Пусть $I=[a_1;b_1]\times\ldots\times[a_n;b_n]$
\begin{enumerate}
    \item Положим, что $I$ — не компакт. Значит, существует его покрытие $\{A_{\alpha}\}$ — открытые множества, такие что $I\subset \{A_{\alpha}\}$, не допускающее выделения конечного подкпорытия
    \item Поделим каждую сторону пополам. Тогда, $\exists I_1$, такой что не допускает конечного подпокрытия. Иначе, $I$ — компакт
    \item Аналогично, повторим процесс и получим систему вложенных брусов: $$I\supset I_1\supset I_2\supset \ldots$$
    То есть на каждой стороне возникает последовательность вложенных отрезков, которые стягиваются в точку $a=(a_1,\ldots,a_n)$

    \begin{center}
        \documentclass[a4paper]{article}
\usepackage[english,russian]{babel}         


\usepackage{tikz}
\usetikzlibrary{arrows.meta}
\begin{document}

\begin{tikzpicture}[scale=2, every node/.style={inner sep=0,outer sep=0}]

% Первый квадрат (деление на 4, выделен верхний правый)
\begin{scope}[xshift=-3.0cm]
  % внешний квадрат
  \draw[thick] (0,0) rectangle (1,1);
  % сетка (деление пополам)
  \draw (0.5,0) -- (0.5,1);
  \draw (0,0.5) -- (1,0.5);
  % выделенный кусок (например, верхний правый)
  \fill[red!60] (0.5,0.5) rectangle (1,1);
  \node[below] at (0.5,-0.12) {Шаг 0};
\end{scope}

% Второй квадрат (берём выделенный кусок и делим его на 4; выделяем его верхний правый)
\begin{scope}[xshift=-1.4cm]
  \draw[thick] (0,0) rectangle (1,1);
  \draw (0.5,0) -- (0.5,1);
  \draw (0,0.5) -- (1,0.5);
  \draw[thin] (0.5,0.5) -- (1,0.5);
  \draw[thin] (0.75,0.5) -- (0.75,1); % вертикальная внутри (точно по середине от 0.5 до 1)
  \draw[thin] (0.5,0.75) -- (1,0.75); % горизонтальная
  % выделяем в этом блоке верхний правый (т.е. от (0.75,0.75) до (1,1))
  \fill[red!60] (0.5,0.5) rectangle (0.75,0.75);
  \node[below] at (0.5,-0.12) {Шаг 1};
\end{scope}

% Третий квадрат (ещё одно деление внутри выделенного квадрата)
\begin{scope}[xshift=0.2cm]
  \draw[thick] (0,0) rectangle (1,1);
  \draw (0.5,0) -- (0.5,1);
  \draw (0,0.5) -- (1,0.5);
  % внутри верхнего правого квадрата делим ещё раз
  % координаты верхнего правого квадрата: от (0.5,0.5) до (1,1)
  % его середина в (0.75,0.75)
  \draw[thin] (0.5,0.5) -- (1,0.5); % горизонтальная внутри
  \draw[thin] (0.75,0.5) -- (0.75,1); % вертикальная внутри (точно по середине от 0.5 до 1)
  \draw[thin] (0.5,0.75) -- (1,0.75); % горизонтальная
  % выделяем в этом блоке верхний правый (т.е. от (0.75,0.75) до (1,1))
  \draw[thin] (0.5, 0.625) -- (0.75, 0.625);
  \draw[thin] (0.625, 0.5) -- (0.625, 0.75);
  \fill[red!60] (0.625,0.5) rectangle (0.75,0.625);

  \node[below] at (0.5,-0.12) {Шаг 2};
\end{scope}

% Многоточие
\node at (2.0,0.5) {\Large $\dots$};

% Финальный предельный квадрат с точкой (символически)
\begin{scope}[xshift=2.2cm]
  %\draw[thick] (0.3,0.3) rectangle (0.7,0.7);
  \fill[red!60] (0.5,0.5) circle (0.75pt);
  \node[right] at (0.55,0.45) {$a$};
  \node[below] at (0.5,-0.12) {Предел};
\end{scope}

% Подсказка / пояснение (русский)
\node[align=center] at (0,-0.7) {
Последовательность вложенных брусов в $\R^2$: на каждом шаге выбираем \\ квадрат, что по предположению нельзя покрыть(выделен цветом) \\ и делим его на 4 части. В итоге стягиваются в точку.};

\end{tikzpicture}

\end{document}

    \end{center}

    При этом, $\exists a = \displaystyle\bigcap_{i=1}^{\infty}I_i$

    \item $a\in I\Longrightarrow a\in \bigcup A_{\alpha}\Longrightarrow\exists \alpha_0:a\in \underbrace{A_{\alpha_0}}_{\text{открытое}}\Longrightarrow\exists \ve>0: B_{\ve}(a)\subset A_{\alpha_0}$

    \begin{center}
        

\begin{tikzpicture}[
    set/.style={dashed, thick},
    arrow/.style={-{Stealth[scale=1.2]}, thick}
]

% Открытое множество (клякса)
\draw[set] (0,0) to[out=30,in=150] (3,0.5)
           to[out=-30,in=60] (4,-1)
           to[out=-120,in=-30] (2,-2)
           to[out=150,in=-120] (0,-1)
           to[out=60,in=180] cycle;

% Квадрат внутри множества
\draw[thin] (1,-1) rectangle (2,0);
\fill[red!] (1.5, -0.5) circle (0.85pt);
\node at (1.7, -0.3) {$a$};

\draw[set] (1.5, -0.5) circle (25pt);

% Стрелка с подписью
\draw[arrow] (2.75, -0.4) -- (2.4, -0.2);
\node at (3.2, -0.5) {$B_{e} (a)$};

% Стрелка с подписью
\draw[arrow] (3.8,0.5) -- (3.4,0.3);
\node at (4.2, 0.6) {$A_{\alpha_0}$};

\node[align=center] at (2.2,-2.5) {Покрытие $A_{\alpha_0}$};

\end{tikzpicture}


    \end{center}


    \item Из построения получили, что $I \supset I_1 \supset \ldots \supset a \Longrightarrow \exists N:\forall n > N\ I_n\subset B_{\ve}(a)\subset A_{\alpha_0}$

    Получается, что $\forall n>N\ I_n$ покрывается одним лишь $A_{\alpha_0}$ из системы $\{A_{\alpha}\}$

    Получаем противоречие тому, что любое $I_n$ не допускает конечного подпокрытия, а у нас получилось, что $I_n\in A_{\alpha_0}\forall n>N \Longrightarrow I - \text{компакт}$
    \qed
\end{enumerate}

\comment Любое ограниченное множество можно вписать в замкнутый брус. Потому что можно вокруг него описать шарик, который точно можно вписать в брус

\subsection{Критерий компактности}
\theorem $K\subset \mathbb{R}^n$. $K$ — компакт $\Longleftrightarrow$ $K$ замкнуто и ограниченно

\proof Докажем необходимость $(\Longrightarrow)$
\begin{itemize}
    \item \textit{Ограниченность.} $K$ — компакт $\Longrightarrow \forall \{A_\alpha\}_{\alpha\in\N}$ — можно выделить конечное подпокрытие $\Longrightarrow$

    $\Longrightarrow$ Пусть $\{A_\alpha\}=\{B_n(0)\}_{n=1}^\infty$ $\Longrightarrow \exists N \in \N : \forall n > N ~ K \subset \displaystyle\bigcup_{n=1}^N B_n(0)$ и так как $B_n(0)$ — вложены шары $\Longrightarrow$

    $\Longrightarrow K \subset B_N(0) \Longrightarrow$ по определению $K$ — ограничено

    \begin{center}
       
\begin{tikzpicture}[
    set/.style={dashed, thick},
    arrow/.style={-{Stealth[scale=1.2]}, thick}
]

% Открытое множество (клякса)
\draw[thin] (0,0) to[out=30,in=150] (2,1.0)
           to[out=-30,in=60] (3.5,-1.5)
           to[out=-120,in=-30] (1,-2)
           to[out=120,in=-120] (0.5,-1)
           to[out=60,in=180] cycle;

\fill (-2, -0.5) circle (0.75pt);
\node[right] at (-2.0,-0.3) {$0$};

\begin{scope}
    \clip (-2.4, -2.8) rectangle(4.9, 1.6);

    \draw[set] (-2, -0.5) circle (25pt);
    \draw[set] (-2, -0.5) circle (65pt);
    \draw[set] (-2, -0.5) circle (125pt);
    \draw[set] (-2, -0.5) circle (175pt);

    \node[right] at (-1.4,-1.2) {$B_{1}(0)$};
    \node[right] at (-0.2,-2.0) {$B_{2}(0)$};
    \node[right] at (2.0,-2.6) {$B_{3}(0)$};
    \node[right] at (3.9,-2.2) {$B_{4}(0)$};
\end{scope}

% Стрелка с подписью
\draw[arrow] (3.5,0.8) -- (3.2,0.1);
\node[right] at (3.3,1) {$K$};

\node[align=center] at (1.5,-3.3) {Пример покрытия $K$ вокруг точки $0$ с помощью шаров};

\end{tikzpicture}

    \end{center}


    \item \textit{Замкнутость.} Пойдем от противного. $K$ — компакт, тогда возьмем $\{B_{\frac{\delta(x)}{2}}(0)\}_{x\in K}$ — покрытие открытыми шарами, где $\delta(x)=\rho(x,x_0)$. $x_0$ — предельная точка, которая $\notin K$ (или же $\in \mathbb{R}^n\setminus K$)

    Так как $K$ — компакт, $\exists x_1,\ldots, x_s:K\subset\displaystyle\bigcup_{i=1}^{s} B_{\frac{\delta(x_i)}{2}}(x_i)$

    Пусть $\delta=\min\limits_{1\leqslant i\leqslant s}{\delta(x_i)}$, тогда
    \begin{equation*}
        \begin{aligned}
            B_{\frac{\delta}{2}}(x_0)\cap\bigcup_{i=1}^{s}B_{\frac{\delta(x_i)}{2}}(x_i)=\varnothing&\Longrightarrow B_{\frac{\delta}{2}}(x_0)\subset\mathbb{R}^n\setminus K\\
            &\Longrightarrow\stackrel{\circ}{B}_{\frac{\delta}{2}}(x_0)\cap K=\varnothing
        \end{aligned}
    \end{equation*}

    Значит, $x_0$ \textit{не является предельной точкой} $K$, что противоречит нашему предположению

    \begin{center}
        %\documentclass[tikz,border=3.14mm]{standalone}
%\usepackage[english,russian]{babel} % локализация и переносы
%\usetikzlibrary{arrows.meta}

% \begin{document}
\begin{tikzpicture}[
    set/.style={dashed, thick},
    arrow/.style={-{Stealth[scale=1.2]}, thick}
]

% Открытое множество (клякса)
\draw[thin] (0,0) to[out=30,in=150] (2,1.0)
           to[out=-30,in=60] (3.5,-1.5)
           to[out=-120,in=-30] (1,-2)
           to[out=120,in=-120] (0.5,-1)
           to[out=60,in=180] cycle;

\path (-2, -0.5) node (point0) {};

\fill (point0) circle (1.5pt);
\node[right] at (-2.2,-0.2) {$x_0$};

\draw[set][red!] (point0) circle (35pt);
\node[right] at (-2.8,-2.1) {$B_{\frac{\delta}{2}}(x_0)$};


% Стрелка с подписью
\draw[arrow] (3.5,0.8) -- (3.2,0.1);
\node[right] at (3.3,1) {$K$};

\path (1, 0.5) node (point_x1) {};
\draw[set] (point0) to (point_x1);
\draw[set](point_x1) circle (44pt);
\fill (point_x1) circle (1.5pt);
\node[right] at (point_x1) {$x_1$};
\node[right] at (2,1.8) {$B_{\frac{\delta_1}{2}}(x_1)$};
\fill[blue!] (-0.5, -0) circle (2pt);
\draw[blue!] (-0.65, 0.47) to (-0.34, -0.47);


\path (1.2, -1.8) node (point_x2) {};
\draw[set] (point0) to (point_x2);
\draw[set](point_x2) circle (48pt);
\fill (point_x2) circle (1.5pt);
\node[right] at (point_x2) {$x_2$};
\node[right] at (2.5,-2.8) {$B_{\frac{\delta_2}{2}}(x_2)$};
\fill[blue!] (-0.4, -1.15) circle (2pt);
\draw[blue!] (-0.21, -0.68) to (-0.58, -1.61);

\node[align=center] at (0.4,-4.1) {Пример как мы строим $B_{\frac{\delta}{2}}$ вокруг точки $x_0$. \\ Синие точки - середины отрезков на которых они лежат};

\end{tikzpicture}
%\end{document}

    \end{center}


\end{itemize}

\proof Докажем достаточность

$K$ — замкнуто и ограничено $\Longrightarrow \exists r>0:B_r(0)\supset K\Longrightarrow\exists I$ — замкнутый брус, такой что
$$K\subset I\text{ и }I=[-r;r]^n$$
% тут позже будет картинка

Пусть $\{A_{\alpha}\}_{\alpha\in\N}$ — произвольное покрытие открытыми множествами для $K$. Тогда, $I\subset \{A_{\alpha}\}\cup\underbrace{\{\mathbb{R}^n\setminus K\}}_{\text{открыто}}$. Так как $I$ — компакт, то $\exists $ конечное подпокрытие
$$\{A_{\alpha_i}\}_{i=1}^m\cup\{\mathbb{R}^n\setminus K\}\supset I\supset K\text{ — покрытие для $I$}$$

Значит, $K\subset\{A_{\alpha_i}\}_{i=1}^{m}$ — конечное и $\{A_{\alpha}\}$ — произвольное, тогда $K$ — компакт по определению\qed

\begin{center}
    

\begin{tikzpicture}[
    set/.style={dashed, thick},
    arrow/.style={-{Stealth[scale=1.2]}, thick}
]

% Открытое множество (клякса)
\draw[thin] (0,0) to[out=30,in=150] (0.5,0.3)
           to[out=-30,in=60] (1.,-1.)
           to[out=-120,in=-30] (0.3,-1.2)
           to[out=120,in=-120] (-0.3,-0.2)
           to[out=60,in=180] cycle;

\fill (-0.75, -1.5) circle (0.75pt);
\node[right] at (-0.75,-1.5) {$0$};

\draw[set] (-0.75, -1.5) circle (68pt);

\node at (1.5,0.3) {$B_{1}(0)$};

\draw[thin] (-3.5, -4) rectangle (2, 1);
\draw[arrow] (-0.72,-4.2) -- (1.95,-4.2);
\draw[arrow] (-0.78,-4.2) -- (-3.45,-4.2);
\node at (-2, -4.5) {$r$};
\node at (0.7, -4.5) {$r$};

\draw[arrow] (-3.7,-1.55) -- (-3.7,0.95);
\draw[arrow] (-3.7,-1.65) -- (-3.7,-4.);
\node at (-4, -2.75) {$r$};
\node at (-4, -0.4) {$r$};

\draw[set][red!] (-0.4, -0.4) circle (14pt);
\draw[set][red!] (0.75, -0.6) circle (14pt);
\draw[set][red!] (0.1, 0) circle (14pt);
\draw[set][red!] (0.3, -1.0) circle (14pt);
\draw[set][red!] (1, -1.2) circle (14pt);
\node[red!] at (-1.0, 0.5) {$\{A_{\alpha}\}$};

\node at (0.5,-0.5) {$K$};

\node[align=center] at (-0.5,-5) {Строим замкнутый брус вокруг точки 0, пользуясь \\ существованием конечного покрытия покрываем наш компакт $K$};

\end{tikzpicture}

\end{center}

