\section{Лекция 4 - 22.09.2020}

\subsection{Умножение рядов}

$\sum_{k=1}^{\infty} a_k$, $\sum_{m=1}^{\infty} b_m$

$\left(\sum_{k=1}^{K} a_k\right) \cdot \left(\sum_{m=1}^{M} b_m \right) = \sum_{1 \leq k \leq K, 1 \leq m \leq M} a_k \cdot b_m$

Если эта сумма имеет предел при $K, M \to \infty$, не зависящий от порядка суммирования, то говорят, что определено произведение рядов.

\begin{theorem}
(Коши) Если $\sum a_k$, $\sum b_m$ сходятся абсолютно, то определено их произведение.

$\left(\sum_{k=1}^{\infty} a_k\right) \cdot \left(\sum_{m=1}^{\infty} b_m \right) = \sum_{n=1}^{\infty} a_{k_n} \cdot b_{m_n}$
\end{theorem}

Произведение рядов по Коши:

$c_2 = a_1 \cdot b_1$

$c_3 = a_2 \cdot b_1 + a_1 \cdot b_2$

$c_4 = a_3 \cdot b_1 + a_2 \cdot b_2 + a_1 \cdot b_3$

$\dots$

$$\left(\sum_{k=1}^{\infty} a_k\right) \cdot \left(\sum_{m=1}^{\infty} b_m \right) = \sum_{n=2}^{\infty} c_n$$

\subsection{Бесконечное произведение}

\subsubsection{Основные понятия}

$\prod_{n=1}^{N} a_n = a_1 \cdot a_2 \cdot \dots \cdot a_N$ -- частичное произведение.

Бесконечным произведением называют формальную запись $\prod_{n=1}^{\infty} a_n$

Значением бесконечного произведения является предел частичного произведения:

$\prod_{n=1}^{\infty} a_n = \lim_{N \to \infty} \prod_{n=1}^{N} a_n$

\subsubsection{Сходимость бесконечного произведения}

Необходимое условие сходимости:

Если $P_N = \prod_{n=1}^N a_n$ сходится, то $a_n = \frac{P_n}{P_{n - 1}} \to 1$

$\prod_{n=1}^{N} a_n = e^{\ln \prod_{n=1}^{N} a_n} = e^{\sum_{n=1}^{N} \ln a_n}$

$\prod_{n=1}^{\infty} a_n = P \iff \sum_{n=1}^{\infty} \ln a_n = \ln P$ $(P \neq 0, a_n \to 1)$

Пусть $a_n \geq 1$. Тогда $\sum_{n=1}^{\infty} \ln a_n$ -- положительный ряд

$\ln a_n =  (a_n - 1) + o(1)$, т. к. $a_n \to 1$

$\sum_{n=1}^{\infty} \ln a_n$ сходится $\iff \sum_{n=1}^{\infty} (a_n - 1)$ сходится.

\subsubsection{Абсолютная сходимость бесконечного произведения}

$\prod_{n=1}^{\infty} a_n$ называется абсолютно сходящимся, если абсолютно сходится соответствующий ему ряд $\sum_{n=1}^{\infty} \ln a_n$

\begin{comment}
    $\prod_{n=1}^{\infty} a_n$ сходится абсолютно $\iff \sum_{n=1}^{\infty} (a_n - 1)$ сходится абсолютно.
\end{comment}

\begin{example}
(Произведение Валлиса)
$\prod_{n=1}^{\infty} \frac{4n^2}{4n^2 - 1} = \frac{\pi}{2}$ -- получается из анализа интегралов $\int_{0}^{\frac{\pi}{2}} \sin^n x dx$

Прим. ред.: есть отличное \href{https://www.youtube.com/watch?v=8GPy_UMV-08}{видео} с интуитивно понятным доказательством.
\end{example}

\begin{example}
    (Дзета-функция Римана) $\zeta(s) = \sum_{n=1}^{\infty} \frac{1}{n^s}, s > 1$

    Тождество Эйлера:

    $\zeta(s) = \dfrac{1}{\prod_{n=1}^{\infty}(1 - \frac{1}{p_n^s})}$, где $p_1 = 2, p_2 = 3, p_3 = 5, \dots$
\end{example}

\subsection{Функциональные последовательности}

\subsubsection{Поточечная и равномерная сходимость}

Пусть при всех $n \in \NN$ функции $f_n: D \to \RR$, $D \subseteq \RR$

Говорят, что $a \in D$ -- точка сходимости $\{f_n(x)\}$, если последовательность $\{f_n(a)\}$ сходится.

Множество всех точек сходимости называется множеством сходимости.

Говорят, что последовательность сходится на $D$ поточечно, если $D$ -- множество сходимости.

Говорят, что $f_n(x)$ сходится к $f(x)$ равномерно на $D$, если $\sup|f_n(x) - f(x)| \to 0$

\textbf{Свойства}

\begin{enumerate}
    \item $f_n \overset{D}{\rightrightarrows} f \implies f_n \overset{D}{\rightarrow} f$
    \item Если $D = D_1 \cup D_2$, то:

          $f_n \overset{D}{\rightrightarrows} f \iff (f_n \overset{D_1}{\rightrightarrows} f $ и $f_n \overset{D_2}{\rightrightarrows} f)$
\end{enumerate}

\subsubsection{Равномерная норма. Критерий Коши}

Рассмотрим множество всех функций $D \to \RR$

$||f|| = \sup_{x \in D} |f(x)|$

Таким образом, $f_n \overset{D}{\rightrightarrows} f \iff ||f_n - f|| \to 0$

\subsubsection{Теорема Дини}

Пусть $f_n: [a, b] \to \RR, f_n(x)$ монотонна по $n$ при каждом $x \in [a, b]$, $f_n \to f$ на $[a, b]$

Тогда $f_n \overset{D}{\rightrightarrows} f$