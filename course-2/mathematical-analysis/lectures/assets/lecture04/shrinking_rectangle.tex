

\begin{tikzpicture}[scale=2, every node/.style={inner sep=0,outer sep=0}]

% Первый квадрат (деление на 4, выделен верхний правый)
\begin{scope}[xshift=-3.0cm]
  % внешний квадрат
  \draw[thick] (0,0) rectangle (1,1);
  % сетка (деление пополам)
  \draw (0.5,0) -- (0.5,1);
  \draw (0,0.5) -- (1,0.5);
  % выделенный кусок (например, верхний правый)
  \fill[red!60] (0.5,0.5) rectangle (1,1);
  \node[below] at (0.5,-0.12) {Шаг 0};
\end{scope}

% Второй квадрат (берём выделенный кусок и делим его на 4; выделяем его верхний правый)
\begin{scope}[xshift=-1.4cm]
  \draw[thick] (0,0) rectangle (1,1);
  \draw (0.5,0) -- (0.5,1);
  \draw (0,0.5) -- (1,0.5);
  \draw[thin] (0.5,0.5) -- (1,0.5);
  \draw[thin] (0.75,0.5) -- (0.75,1); % вертикальная внутри (точно по середине от 0.5 до 1)
  \draw[thin] (0.5,0.75) -- (1,0.75); % горизонтальная
  % выделяем в этом блоке верхний правый (т.е. от (0.75,0.75) до (1,1))
  \fill[red!60] (0.5,0.5) rectangle (0.75,0.75);
  \node[below] at (0.5,-0.12) {Шаг 1};
\end{scope}

% Третий квадрат (ещё одно деление внутри выделенного квадрата)
\begin{scope}[xshift=0.2cm]
  \draw[thick] (0,0) rectangle (1,1);
  \draw (0.5,0) -- (0.5,1);
  \draw (0,0.5) -- (1,0.5);
  % внутри верхнего правого квадрата делим ещё раз
  % координаты верхнего правого квадрата: от (0.5,0.5) до (1,1)
  % его середина в (0.75,0.75)
  \draw[thin] (0.5,0.5) -- (1,0.5); % горизонтальная внутри
  \draw[thin] (0.75,0.5) -- (0.75,1); % вертикальная внутри (точно по середине от 0.5 до 1)
  \draw[thin] (0.5,0.75) -- (1,0.75); % горизонтальная
  % выделяем в этом блоке верхний правый (т.е. от (0.75,0.75) до (1,1))
  \draw[thin] (0.5, 0.625) -- (0.75, 0.625);
  \draw[thin] (0.625, 0.5) -- (0.625, 0.75);
  \fill[red!60] (0.625,0.5) rectangle (0.75,0.625);

  \node[below] at (0.5,-0.12) {Шаг 2};
\end{scope}

% Многоточие
\node at (2.0,0.5) {\Large $\dots$};

% Финальный предельный квадрат с точкой (символически)
\begin{scope}[xshift=2.2cm]
  %\draw[thick] (0.3,0.3) rectangle (0.7,0.7);
  \fill[red!60] (0.5,0.5) circle (0.75pt);
  \node[right] at (0.55,0.45) {$a$};
  \node[below] at (0.5,-0.12) {Предел};
\end{scope}

% Подсказка / пояснение (русский)
\node[align=center] at (0,-0.7) {
Последовательность вложенных брусов в $\R^2$: на каждом шаге выбираем \\ квадрат, что по предположению нельзя покрыть(выделен цветом) \\ и делим его на 4 части. В итоге стягиваются в точку.};

\end{tikzpicture}
