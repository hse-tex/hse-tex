% \documentclass[a4paper,10pt]{article}
% %\documentclass[a4paper,10pt]{scrartcl}
%
% \usepackage[utf8]{inputenc}         % кодировка исходного текста
% \usepackage[english,russian]{babel} % локализация и переносы
% \usepackage{tikz}
% \usetikzlibrary{arrows.meta, patterns, patterns.meta}
%
% \begin{document}


\begin{tikzpicture}[scale=2, every node/.style={inner sep=0,outer sep=0},
    set/.style={dashed, thick},
    arrow/.style={-{Stealth[scale=1.2]}, thick}]
    \begin{scope}
        \draw[thick] (0, 0) rectangle (1, 1);
        \draw[thick][blue] (0.5, 0) -- (0.5, 1);
        \draw[thick][blue] (0, 0.5) -- (1, 0.5);
        \node at (0.53, -0.3) {разбиение $T_1$};
    \end{scope}




    \begin{scope}[xshift=2cm]
        \draw[thick] (0, 0) rectangle (1, 1);
        \draw[thick] (0.5, 0) -- (0.5, 1);
        \draw[thick] (0, 0.5) -- (1, 0.5);

        \fill[pattern = {Lines[angle = -45, line width = 1pt, distance = 5pt]},
    pattern color = blue,
    even odd rule]
            (0, 0) rectangle (0.5, 0.5)
            (0.05, 0.05) rectangle (0.45, 0.45)

            (0.5, 0) rectangle (1, 0.5)
            (0.55, 0.05) rectangle (0.95, 0.45)

            (0.5, 0.5) rectangle (1, 1)
            (0.55, 0.55) rectangle (0.95, 0.95)

            (0, 0.5) rectangle (0.5, 1)
            (0.05, 0.55) rectangle (0.45, 0.95);
        \draw[set][blue]
            (0.05, 0.05) rectangle (0.45, 0.45)
            (0.55, 0.55) rectangle (0.95, 0.95)
            (0.05, 0.55) rectangle (0.45, 0.95)
            (0.55, 0.05) rectangle (0.95, 0.45);

        \node at (0.53, -0.3) {Граница $G$ бруса $T_1$};
    \end{scope}



    \begin{scope}[xshift=4cm]
        \draw[thick] (0, 0) rectangle (1, 1);
        \draw[thick][red] (0.1, 0) -- (0.1, 1);
        \draw[thick][red] (0.25, 0) -- (0.25, 1);
        \draw[thick][red] (0.4, 0) -- (0.4, 1);
        \draw[thick][red] (0.6, 0) -- (0.6, 1);
        \draw[thick][red] (0.9, 0) -- (0.9, 1);
        \draw[thick][red] (0, 0.4) -- (1, 0.4);
        \draw[thick][red] (0, 0.6) -- (1, 0.6);
        \draw[thick][red] (0, 0.1) -- (1, 0.1);
        \draw[thick][red] (0, 0.9) -- (1, 0.9);
        \node at (0.53, -0.3) {Какое-то разбиение $T_2$};
    \end{scope}

    \begin{scope}[xshift=0.2cm, yshift=-2cm]
        \draw[thick] (0, 0) rectangle (1, 1);
        \draw[thick] (0.5, 0) -- (0.5, 1);
        \draw[thick] (0, 0.5) -- (1, 0.5);

        \fill[pattern = {Lines[angle = -45, line width = 1pt, distance = 5pt]},
    pattern color = blue,
    opacity=0.3,
    even odd rule]
            (0, 0) rectangle (0.5, 0.5)
            (0.05, 0.05) rectangle (0.45, 0.45)

            (0.5, 0) rectangle (1, 0.5)
            (0.55, 0.05) rectangle (0.95, 0.45)

            (0.5, 0.5) rectangle (1, 1)
            (0.55, 0.55) rectangle (0.95, 0.95)

            (0, 0.5) rectangle (0.5, 1)
            (0.05, 0.55) rectangle (0.45, 0.95);
        \draw[set][blue][opacity=0.3]
            (0.05, 0.05) rectangle (0.45, 0.45)
            (0.55, 0.55) rectangle (0.95, 0.95)
            (0.05, 0.55) rectangle (0.45, 0.95)
            (0.55, 0.05) rectangle (0.95, 0.45);



        \draw[thick] (0, 0) rectangle (1, 1);
        \draw[thick][red][opacity=0.5]
            (0.1, 0) -- (0.1, 1)
            (0.25, 0) -- (0.25, 1)
            (0.4, 0) -- (0.4, 1)
            (0.6, 0) -- (0.6, 1)
            (0.9, 0) -- (0.9, 1)
            (0, 0.4) -- (1, 0.4)
            (0, 0.6) -- (1, 0.6)
            (0, 0.1) -- (1, 0.1)
            (0, 0.9) -- (1, 0.9);
        \node[align=center] at (0.53, -0.3) {Как прошлые разбиения и граница $G$ \\ выглядят на одном рисунке};
    \end{scope}



    \begin{scope}[xshift=3.8cm, yshift=-2cm]
        \draw[thick] (0, 0) rectangle (1, 1);

        \fill[pattern = {Lines[angle = -45, line width = 1pt, distance = 5pt]},
    pattern color = red,
    opacity=0.3,
    even odd rule]
            (0, 0) rectangle (0.5, 0.5)
            (0.05, 0.05) rectangle (0.45, 0.45)

            (0.5, 0) rectangle (1, 0.5)
            (0.55, 0.05) rectangle (0.95, 0.45)

            (0.5, 0.5) rectangle (1, 1)
            (0.55, 0.55) rectangle (0.95, 0.95)

            (0, 0.5) rectangle (0.5, 1)
            (0.05, 0.55) rectangle (0.45, 0.95);
        \draw[set][red][opacity=0.7]
            (0.05, 0.05) rectangle (0.45, 0.45)
            (0.55, 0.55) rectangle (0.95, 0.95)
            (0.05, 0.55) rectangle (0.45, 0.95)
            (0.55, 0.05) rectangle (0.95, 0.45);

        \draw[arrow][red] (-0.3, 0.5) -- (0.05, 0.51);
        \node at (-0.4, 0.5) {$A$};



        \fill[pattern = {Lines[angle = 55, line width = 1pt, distance = 5pt]},
    opacity=0.5]
            (0.05, 0.05) rectangle (0.45, 0.45)

%             (0.5, 0) rectangle (1, 0.5)
            (0.55, 0.05) rectangle (0.95, 0.45)

%             (0.5, 0.5) rectangle (1, 1)
            (0.55, 0.55) rectangle (0.95, 0.95)

%             (0, 0.5) rectangle (0.5, 1)
            (0.05, 0.55) rectangle (0.45, 0.95);


        \draw[arrow] (1.3, 0.5) -- (0.8, 0.7);
        \node at (1.4, 0.5) {$B$};

        \node[align=center] at (0.53, -0.3) {Как выглядят множества $A$ и $B$};
    \end{scope}

\end{tikzpicture}
%
%
% \end{document}
