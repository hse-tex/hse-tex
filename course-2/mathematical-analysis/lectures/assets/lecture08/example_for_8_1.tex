
% \documentclass[tikz,border=10pt]{standalone}
% \usepackage[utf8]{inputenc}         % кодировка исходного текста
% \usepackage[english,russian]{babel}
% \usetikzlibrary{decorations.pathmorphing}
% \usetikzlibrary{arrows.meta, patterns, patterns.meta}

% \begin{document}
\begin{tikzpicture}[scale=2]

    \draw (0, 0) rectangle (3, 2);
    \node at (0.4, 0.4) { $I_1$ };

    \draw (1.5, 1) rectangle (5, 2.5);
    \node at (4.5, 2.1) { $I_2$ };


    \draw[thin] (2, 1.8) to[out=0,in=110] (2.8,1.5)
           to[out=-50, in=-30] (2.4,1.2)
           to[out=110, in=-70] (1.8,1.1)
           to[out=110, in=-180] cycle;
    \node at (1.67, 1.8) { $I$ };
    \node at (2.3, 1.5) { $D$ };

    \node[align=center] at (2.3, -0.3) {Как выглядят наши множества $I_1, I_2, I, D$};

\end{tikzpicture}
% \end{document}
%

