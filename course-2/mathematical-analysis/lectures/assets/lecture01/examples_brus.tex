% \documentclass[a4paper]{article}
% \usepackage[english,russian]{babel}
%
%
% \usepackage{tikz}
\usetikzlibrary{arrows.meta, 3d, perspective}
% \begin{document}

\begin{tikzpicture}[
    set/.style={dashed, thick},
    arrow/.style={-{Stealth[scale=1.2]}, thick}
]


\def\a{2} % длина
\def\b{2} % ширина
\def\c{2} % высота

    % Координаты вершин куба
\coordinate (A) at (0,0,0);
\coordinate (B) at (\a,0,0);
\coordinate (C) at (\a,\b,0);
\coordinate (D) at (0,\b,0);
\coordinate (E) at (0,0,\c);
\coordinate (F) at (\a,0,\c);
\coordinate (G) at (\a,\b,\c);
\coordinate (H) at (0,\b,\c);

    % Рисуем невидимые ребра (пунктиром)
\draw[dashed] (A) -- (D);
\draw[dashed] (A) -- (B);
\draw[dashed] (A) -- (E);

    % Рисуем видимые ребра
\draw (B) -- (C) -- (G) -- (F) -- (B);
\draw (D) -- (H);
\draw (C) -- (D);
\draw (F) -- (G) -- (H) -- (E) -- (F);

\draw[<->,thick] (0.2, 0, \c + 0.5) -- node[below] {$[a_1 ; b_1]$} (\a + 0.1, 0, \c + 0.5);

\draw[<->,thick] (\a + 0.2, 0.1, 0) -- node[below,rotate=90] {$[a_3 ; b_3]$} (\a + 0.2, \b, 0);

\draw[<->,thick] (\a + 0.2, 0, \c) -- node[below,sloped] {$[a_2 ; b_2]$} (\a + 0.2, 0, \c - 1.8);


\draw (-5, -0.8) rectangle (-3, 1.2);
\draw[<->,thick] (-5, -1) to (-3, -1);
\node at (-4, -1.3) {$[a_1 ; b_1]$};
\draw[<->,thick] (-2.8, 1.2) to (-2.8, -0.8);
\node[rotate=90] at (-2.4, 0.2) {$[a_2 ; b_2]$};

\draw[|-|] (-9, 0.2) to (-7, 0.2);
\draw[<->,thick] (-9, -0.1) to (-7, -0.1);
\node at (-8, -0.4) {$[a_1 ; b_1]$};



\node[align=center] at (-4,-2) {Пример брусов размерности с 1 по 3};

\end{tikzpicture}


% \end{document}
