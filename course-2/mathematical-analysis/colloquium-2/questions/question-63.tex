% Здесь НЕ НУЖНО делать begin document, включать какие-то пакеты..
% Все уже подрубается в головном файле
% Хедер обыкновенный хсе-теха, все его команды будут здесь работать
% Пожалуйста, проверяйте корректность теха перед пушем

% Здесь формулировка билета
\subsection{Рассмотрите частные случаи определения площади гладкой $k$-мерной поверхности при $k = 1$ и $k = 2$}

\begin{description}
    \item[$k = 1 \colon$] В этом случае у нас есть только один параметр $u$, тогда
    \[
        \pdv{x}{u} = \begin{pmatrix}
            \pdv{x_{1}}{u} \\
            \vdots \\
            \pdv{x_{m}}{u}
        \end{pmatrix}.
    \]
    Матрица Грама записывается следующим образом:
    \[
        \pqty{\pdv{x}{u}}^{T} \times \pqty{\pdv{x}{u}} = \langle \pdv{x}{u}, \pdv{x}{u} \rangle = \norm{\pdv{x}{u}}^{2}.
    \]
    Ее определитель
    \[
        \det G = \norm{\pdv{x}{u}}^{2}.
    \]
    А корень из определителя
    \[
        \sqrt{\det G} = \norm{\pdv{x}{u}}.
    \]
    Таким образом, мы можем найти длину кривой $S$:
    \[
        \mu(S) = \int_{a}^{b} \norm{\pdv{x}{u}} \dd u = \int_{a}^{b} \sqrt{\pqty{\pdv{x_{1}}{u}}^{2} + \ldots + \pqty{\pdv{x_{m}}{u}}^{2}} \dd u.
    \]
    \item[$k = 2 \colon$] У нас есть некоторые параметры $(u, v) \in G \subset \mathbb{R}^{2}$, тогда матрица Грама будет выглядеть так:
    \[
        \begin{pmatrix}
            \langle \pdv{x}{u}, \pdv{x}{u} \rangle & \langle \pdv{x}{u}, \pdv{x}{v} \rangle \\
            \langle \pdv{x}{v}, \pdv{x}{u} \rangle & \langle \pdv{x}{v}, \pdv{x}{v} \rangle
        \end{pmatrix}.
    \]
    Ее определитель
    \[
        \det(\ldots) = \norm{\pdv{x}{u}}^{2} \cdot \norm{\pdv{x}{v}} ^{2} - \left\langle \pdv{x}{u}, \pdv{x}{v} \right\rangle^{2}.
    \]
    Ну и тогда площадь поверхности
    \[
        \mu(S) = \iint_{G} \sqrt{\norm{\pdv{x}{u}}^{2} \cdot \norm{\pdv{x}{v}} ^{2} - \left\langle \pdv{x}{u}, \pdv{x}{v} \right\rangle^{2}} \dd u \dd v.
    \]
\end{description}

