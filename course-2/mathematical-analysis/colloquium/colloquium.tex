\documentclass[a4paper]{article}
\usepackage{../header}
\begin{document}
\title{\LARGE{Математический анализ—2}\\ Коллоквиум}
\author{Лектор: Зароднюк Алёна Владимировна\\
\href{https://github.com/vinerdanya/EDS-Study/blob/main/2nd\%20course/Calculus-2/Colloquim/Colloquim.pdf}{оригинальный \LaTeX\ }by Винер Даниил \href{https://t.me/danya_vin}{@danya\_vin}
\\
\\
Авторы \href{https://hse-tex.me/course-2/mathematical-analysis.pdf}{текущего документа}:
\\
Жуков Андрей | \href{https://github.com/shiro-eden}{github} \qquad
Мелисов Тимур | \href{https://github.com/coffecat46}{github}}
\date{Версия от \today}
\maketitle

\tableofcontents
\newpage
\setlength{\parindent}{15pt}
\setlength{\parskip}{2mm}
\setlist[itemize]{left=1cm}
% \setlist[enumerate]{left=1cm}
\section{Основные определения}
\subsection{Определение (замкнутого) бруса (координатного промежутка, параллепипеда)}
\definition Замкнутый брус (координатный промежуток) в $\mathbb{R}^n$ — множество, описываемое как
\begin{equation*}
\begin{aligned}
    I&=\{x\in\mathbb{R}^n\ |\ a_i\leq x_i\leq b_i,\ i\in\{1,n\}\}\\
    &=\left[a_1,b_1\right]\times\ldots\times\left[a_n,b_n\right]
\end{aligned}
\end{equation*}
\comment $I=\{a_1,b_1\}\times\ldots\times\{a_n,b_n\}$, где $\{a_i, b_i\}$ может быть отрезком, интервалом и т.д.
\begin{center}
    % \documentclass[a4paper]{article}
% \usepackage[english,russian]{babel}
%
%
% \usepackage{tikz}
\usetikzlibrary{arrows.meta, 3d, perspective}
% \begin{document}

\begin{tikzpicture}[
    set/.style={dashed, thick},
    arrow/.style={-{Stealth[scale=1.2]}, thick}
]


\def\a{2} % длина
\def\b{2} % ширина
\def\c{2} % высота

    % Координаты вершин куба
\coordinate (A) at (0,0,0);
\coordinate (B) at (\a,0,0);
\coordinate (C) at (\a,\b,0);
\coordinate (D) at (0,\b,0);
\coordinate (E) at (0,0,\c);
\coordinate (F) at (\a,0,\c);
\coordinate (G) at (\a,\b,\c);
\coordinate (H) at (0,\b,\c);

    % Рисуем невидимые ребра (пунктиром)
\draw[dashed] (A) -- (D);
\draw[dashed] (A) -- (B);
\draw[dashed] (A) -- (E);

    % Рисуем видимые ребра
\draw (B) -- (C) -- (G) -- (F) -- (B);
\draw (D) -- (H);
\draw (C) -- (D);
\draw (F) -- (G) -- (H) -- (E) -- (F);

\draw[<->,thick] (0.2, 0, \c + 0.5) -- node[below] {$[a_1 ; b_1]$} (\a + 0.1, 0, \c + 0.5);

\draw[<->,thick] (\a + 0.2, 0.1, 0) -- node[below,rotate=90] {$[a_3 ; b_3]$} (\a + 0.2, \b, 0);

\draw[<->,thick] (\a + 0.2, 0, \c) -- node[below,sloped] {$[a_2 ; b_2]$} (\a + 0.2, 0, \c - 1.8);


\draw (-5, -0.8) rectangle (-3, 1.2);
\draw[<->,thick] (-5, -1) to (-3, -1);
\node at (-4, -1.3) {$[a_1 ; b_1]$};
\draw[<->,thick] (-2.8, 1.2) to (-2.8, -0.8);
\node[rotate=90] at (-2.4, 0.2) {$[a_2 ; b_2]$};

\draw[|-|] (-9, 0.2) to (-7, 0.2);
\draw[<->,thick] (-9, -0.1) to (-7, -0.1);
\node at (-8, -0.4) {$[a_1 ; b_1]$};



\node[align=center] at (-4,-2) {Пример брусов размерности с 1 по 3};

\end{tikzpicture}


% \end{document}

\end{center}

\subsection{Определение меры (объема) бруса}
\definition Мера бруса — его объём:
\begin{equation*}
    \begin{aligned}
        \mu(I)&=|I|
        =\prod_{i=1}^{n} (b_i-a_i)
    \end{aligned}
\end{equation*}


\subsection{Определение разбиения бруса}
\definition Пусть $I$ — замкнутый, невырожденный брус и $\displaystyle\bigcup_{i=1}^kI_i = I$, где $I_i$ попарно не имеют общих внутренних точек. Тогда набор $\T = \{I_i\}_{i=1}^k$ называется разбиением бруса $I$

\subsection{Определение диаметра множества в $\mathbb{R}^n$}
\definition Диаметр произвольного ограниченного множества $M\subset\R^n$ будем называть 
\begin{equation*}
\begin{aligned}
    d(M) = \displaystyle\sup_{x,y\in M}\|x-y\|,\text{ где}\\
    \|x-y\|=\sqrt{\sum_{i=1}^{n}\left(x_i-y_i\right)^2}
\end{aligned}
\end{equation*}

\begin{center}
    % \documentclass[a4paper]{article}
% \usepackage[english,russian]{babel}
%
%
% \usepackage{tikz}
% \usetikzlibrary{arrows.meta}
% \begin{document}

\begin{tikzpicture}[
    set/.style={dashed, thick},
    arrow/.style={-{Stealth[scale=1.2]}, thick}
]

\draw (-4, 0) circle (40pt);
\draw[latex-latex, line width=1pt] (-3, -1) -- (-5, 1);
\node[rotate=-45] at (0.2, 0.2) {d};

\draw[set] (0, 0) circle (40pt);
\draw[latex-latex, line width=1pt] (1, -1) -- (-1, 1);
\node[rotate=-45] at (-3.8, 0.2) {d};

\fill (3, 1.3) circle (1.75pt);
\fill (4, 0) circle (1.75pt);
\fill (5, -1.3) circle (1.75pt);
\draw[latex-latex, line width=1pt] (5.15, -1.2) -- (3.15, 1.4);
\draw[Bar-Bar, line width=0.7pt] (5.15, -1.2) -- (3.15, 1.4);
\node[rotate=-55] at (4.4, 0.2) {d};

\node[align=center] at (0,-2) {Пример диаметра для разных ограниченных множеств(Для всех трёх он равен $d$)};

\end{tikzpicture}


% \end{document}

\end{center}

\subsection{Определение ограниченного множества в $\mathbb{R}^n$}
\definition Множество $M\subset \mathbb{R}^n$ называется \textit{ограниченным}, если $$\exists x_0\in\mathbb{R}^n\text{ и }\exists r>0\text{, такой что }M\subset B_{r}(x_0)$$

\subsection{Определение масштаба (диаметра) разбиения}
\definition Масштаб разбиения $\T=\{I_i\}_{i=1}^k$ — число $\lambda(\T) = \Delta_{\T} = \displaystyle\max_{1\leqslant i\leqslant k} d(I_i)$

\subsection{Определения отмеченных точек и размеченного разбиения}
\definition Пусть $\forall\ I_i$ выбрана точка $\xi_i\in I_i$. Тогда, набор $\xi = \{\xi_i\}_{i=1}^k$ будем называть \textbf{отмеченными точками}

\definition Размеченное разбиение — пара $(\T, \xi)$

\subsection{Определение интегральной суммы Римана}
Пусть $I$ — невырожденный, замкнутый брус, функция $f: I\rightarrow \R$ определена на $I$

\definition \label{1.8} Интегральная сумма Римана функции $f$ на $(\T, \xi)$ — величина
$$\sigma(f, \T, \xi) := \sum_{i=1}^kf(\xi_i)\cdot|I_i|$$

\begin{center}
    % \documentclass[tikz,border=5pt]{standalone}
% \usepackage{pgfplots}
\pgfplotsset{compat=newest}
% \usepackage[english,russian]{babel}

% \begin{document}
\begin{tikzpicture}
  \begin{axis}[
    view={120}{30}, % угол обзора
    axis lines=center,
    xlabel={$x$}, ylabel={$y$}, zlabel={$z$},
    ticks=none,
    domain=0:1, y domain=0:1,
    samples=31, samples y=31,
    colormap/viridis,
    zmax=0.8, zmin=0,
    xmin=0, xmax=1.2,
    ymin=0, ymax=1.2
  ]

  % поверхность f(x,y) = 0.4 + 0.2*sin(2*pi*x)*cos(2*pi*y)
  \addplot3[surf, opacity=0.7]
    {0.6 + 0.05*sin(deg(2*pi*x))*cos(deg(2*pi*y))};

  % сетка внизу
  \foreach \i in {0,0.2,...,1} {
    \addplot3[black, thin] coordinates {(\i,0,0) (\i,1,0)};
    \addplot3[black, thin] coordinates {(0,\i,0) (1,\i,0)};
  }

  % координаты выбранного квадратика
  \pgfmathsetmacro{\xA}{0.4}
  \pgfmathsetmacro{\xB}{0.6}
  \pgfmathsetmacro{\yA}{0.4}
  \pgfmathsetmacro{\yB}{0.6}

  % значения функции в углах
  \pgfmathsetmacro{\zA}{0.6 + 0.05*sin(2*pi*\xA r)*cos(2*pi*\yA r)}
  \pgfmathsetmacro{\zB}{0.6 + 0.05*sin(2*pi*\xB r)*cos(2*pi*\yA r)}
  \pgfmathsetmacro{\zC}{0.6 + 0.05*sin(2*pi*\xB r)*cos(2*pi*\yB r)}
  \pgfmathsetmacro{\zD}{0.6 + 0.05*sin(2*pi*\xA r)*cos(2*pi*\yB r)}

  % нижний квадратик (основание)
  \addplot3[fill=red, opacity=0.2]
    coordinates {(\xA,\yA,0) (\xB,\yA,0) (\xB,\yB,0) (\xA,\yB,0) (\xA,\yA,0)};

  % верхний квадратик (на поверхности)
  \addplot3[fill=red, opacity=0.3]
    coordinates {(\xA,\yA,\zA) (\xB,\yA,\zB) (\xB,\yB,\zC) (\xA,\yB,\zD) (\xA,\yA,\zA)};

  % вертикальные рёбра
  \addplot3[red, thick, opacity=0.3] coordinates {(\xA,\yA,0) (\xA,\yA,\zA)};
  \addplot3[red, thick, opacity=0.3] coordinates {(\xB,\yA,0) (\xB,\yA,\zB)};
  \addplot3[red, thick, opacity=0.3] coordinates {(\xB,\yB,0) (\xB,\yB,\zC)};
  \addplot3[red, thick, opacity=0.3] coordinates {(\xA,\yB,0) (\xA,\yB,\zD)};

  % точка внутри квадратика внизу
  \addplot3[only marks, mark=*, red] coordinates {(0.5,0.5,0)};

  % точка на поверхности
  \pgfmathsetmacro{\zP}{0.6 + 0.05*sin(2*pi*0.5 r)*cos(2*pi*0.5 r)}
  \addplot3[only marks, mark=*, red] coordinates {(0.5,0.5,\zP)};

  % соединение точек
  \addplot3[dashed, red] coordinates {(0.5,0.5,0) (0.5,0.5,\zP)};

  \end{axis}

  \node[align=center] at (4.2,0) {Пример интегрирования в $\R^2$ по определению};

\end{tikzpicture}
% \end{document}

\end{center}

\subsection{Определение интегрируемой по Риману функции на замкнутом брусе в $\mathbb{R}^n$}
\definition Функция $f$ интегрируема по Риману на замкнутом брусе $I$ ($f:I\rightarrow\R$), если 
\begin{equation*}
    \exists A\in\R: \forall \varepsilon > 0\ \exists \delta > 0: \forall(\T, \xi): \Delta_{\T} < \delta:\text{ верно }|\sigma(f, \T, \xi) - A| < \varepsilon
\end{equation*}
Тогда $A$ называется \textit{кратным интегралом Римана} и 
$$A = \int\limits_If(x)\d{x} = \underset{I}{\int\ldots\int}f(x_1, \ldots, x_n)\d{x_1}\ldots \d{x_n}$$
Обозначение: $f\in\mathcal{R}(I)$

\subsection{Определение множества меры нуль по Лебегу}
\definition Множество $M\subset\R^n$ будем называть \textbf{множеством меры 0 по Лебегу}, если $\forall\ve>0$ существует не более чем счетный набор (замкнутых) брусов $\{I_i\}$ и выполняются:
\begin{enumerate}[label=\textbullet]
    \item $M\subset \displaystyle\bigcup_iI_i$
    \item $\displaystyle\sum_i|I_i| < \ve\,\, \quad \forall \ve > 0$
\end{enumerate}

% \ex $x_0\in\R^n$ — множество меры нуль по Лебегу в $\R^n$

% \begin{minipage}{0.5\textwidth}
%     \proof Пусть $x_0 = (x_{01}, \ldots, x_{0n})$.
%     Покроем точку замкнутым брусом, причем $$I = [x_{01}-d, x_{01}+d] \times\ldots\times[x_{0n}-d, x_{0n}+d]$$
%     $$\forall \ve > 0\,\,\exists I: |I| = (2d)^n<\ve \implies d < \frac{\sqrt[n]{\ve}}{2}$$
% Значит, точка является множеством меры нуль по Лебегу
% \end{minipage}
% % \proof Пусть $x_0 = (x_{01}, \ldots, x_{0n})$.
% % Покроем точку замкнутым брусом, причем $$I = [x_{01}-d, x_{01}+d] \times\ldots\times[x_{0n}-d, x_{0n}+d]$$
% \begin{minipage}{0.5\textwidth}
%     $$
% \begin{tikzpicture}[scale=2]
%     % Рисуем квадрат
%     \draw[thick] (0,0) rectangle (2,2);
%     \draw[black, thin] (0,1)node[black, left] {$d$} -- (1, 1);
%     \draw[black, thin] (1, 0) node[black, below] {$d$} -- (1, 1);

%     % % \node[black, above] (0.5,1) {$d$};
%     % \node[black] (1,0.5) {$d$};
%     % Рисуем точку в центре
%     \fill (1,1) node[black, below right] {$x_0$} circle (1pt); % размер точки можно изменить, подбирая значение радиуса
% \end{tikzpicture}
% $$
% \end{minipage}

\subsection{Определение внутренней точки множества}
\definition Пусть имеется $M\subset\mathbb{R}^n$. Точку $x_0\in M$ будем называть \textit{внутренней} точкой $M$, если $$\exists\ve>0:B_{\ve}(x_0)\subset M$$

\subsection{Определение внешней точки множества}
\definition Точку $x_0\in \mathbb{R}^n$ будем называть \textit{внешней} точкой $M$, если $$\exists\ve>0:B_{\ve}(x_0)\subset (\mathbb{R}^n\setminus M)$$

\subsection{Определение граничной точки множества}
\definition Точку $x_0\in\mathbb{R}^n$ будем называть \textit{граничной} точкой $M$, если $$\forall \ve>0:\ \left(B_{\ve}(x_0)\cap M\right)\ne\varnothing\wedge B_{\ve}(x_0)\cap(\mathbb{R}^n\setminus M)\ne\varnothing$$ 

\subsection{Определение изолированной точки множества}
\definition Точку $x_0\in M$ будем называть \textit{изолированной} точкой $M$, если $$\exists \ve>0:\ \stackrel{\circ}{B_{\ve}}(x_0)\cap M=\varnothing$$ 

\subsection{Определение предельной точки множества}
\definition Точку $x_0\in\mathbb{R}^n$ будем называть \textit{предельной} точкой $M$, если $$\forall \ve>0:\ \stackrel{\circ}{B_{\ve}}(x_0)\cap M\ne\varnothing$$ 

\subsection{Определение точки прикосновения множества}
\definition Точку $x_0\in\mathbb{R}^n$ будем называть \textit{точкой прикосновения} $M$, если $$\forall \ve>0:\ B_{\ve}(x_0)\cap M\ne\varnothing$$ 

\subsection{Определение открытого множества}
\definition Множество $M\subset\mathbb{R}^n$ называется \textit{открытым}, если все его точки внутренние

\subsection{Определение замкнутого множества}
\definition Множество $M\subset R^n$ называется замкнутым, если $\mathbb{R}^n\setminus M$ — открыто

\subsection{Определение компакта}
\definition Множество $K\subset\R^n$ называется \textit{компактом}, если из $\forall$ его покрытия открытыми множествами можно выделить конечное подпокрытие

\subsection{Определение колебания функции на множестве}
\definition Колебанием функции $f$ на множестве $M\subset \R^n$ будем называть число $\omega(f, M)$:
\begin{equation*}
    \omega(f, M) = \sup_{x, y\in M}|f(x) - f(y)| = \sup_{x\in M} f(x) - \inf_{y\in M} f(y)
\end{equation*}

\subsection{Определение колебания функции в точке}
\definition Колебанием функции $f$ в точке $x_0 \in M \subset\R^n$ будем называть число
\begin{equation*}
    \omega(f, x_0):= \lim_{r \to 0+} \omega(f, B_r^M(x_0)), \text{ где } B_r^M = B_r(x_0)\cap M
\end{equation*}

\subsection{Определение непрерывности функции в точке и на множестве}
\definition Функция $f:M \subset \R^n; \ f \colon M \rightarrow\mathbb{R}$ \textit{непрерывна в точке} $x_0$, если
\begin{enumerate}
    \item \textbf{Через колебание:}
    \begin{equation*}
        f \text{ — непрерывна в точке } x_0 \iff \omega(f, x_0) = 0
    \end{equation*}
    \item \textbf{Классическое определение:}
    \begin{equation*}
        \forall \ve >0\ \exists\delta>0:\ \forall x\in M\text{ такого что } |x-x_0|<\delta\text{ верно } |f(x)-f(x_0)|<\ve
    \end{equation*}
\end{enumerate}
\textbf{На множестве:}
\begin{enumerate}
    \item Непрервына на множестве всюду --- непрерывность в каждой точке.
    \item Непрерывно на множестве почти всюду --- непрерывность выполняется везде кроме множества меры нуль.
\end{enumerate}



\subsection{Определение выполненения свойства почти всюду}
\definition Если какое-то свойство не выполняется лишь на множестве меры нуль, то говорят, что это свойство выполняется почти всюду

\subsection{Определение пересечения двух разбиений}
\definition Пусть $\T_1 = \{I^1_k\}$ и $\T_2 = \{I^2_m\}$ — два разбиения бруса $I \subset \R^n$. 

Пересечением разбиений $(\T_1 \cap \T_2)$ будем называть множество всех брусов $\{I_{ij}\}: \forall I_{ij}$ выполняется
\begin{itemize}
    \item $\exists k: I_{ij} \in \{I^1_k\}$
    \item $\exists m: I_{ij} \in \{I^2_m\}$
    \item $\{I_{ij}\}$ — разбиение бруса $I$
\end{itemize}

% ПОФИКСИТЬ ФОТО
\subsection{Определение измельчения разбиения}
\definition Разбиение $\T_1 = \{I^1_k\}$ будем называть измельчением разбиения $\T_2 = \{I^2_m\}$, если $\forall k\ \exists m: I_k^1 \in I_m^2 \implies \T = \T_1\cap \T_2$ является измельчением $\T_1$ и $\T_2$
\begin{figure}[ht]
\centering

\begin{tikzpicture}[scale=1.1]
  \def\W{5} \def\H{3}

  % --------- левый прямоугольник: I_i ----------
  \begin{scope}
    \draw[thick,rounded corners=2pt] (0,0) rectangle (\W,\H);
    \foreach \x in {1.25,2.5,3.75} {\draw[black!60] (\x,0)--(\x,\H);}
    \foreach \y in {1,2}           {\draw[black!60] (0,\y)--(\W,\y);}
    \path[fill=black!12,draw=black!60] (1.25,1) rectangle (2.5,2);
    \node at (1.875,1.5) {$I_i$};
  \end{scope}

  % --------- правый прямоугольник: I_j ----------
  \begin{scope}[xshift=7cm]
    \draw[thick,rounded corners=2pt] (0,0) rectangle (\W,\H);
    \foreach \x in {1,2,3,4}       {\draw[red!70] (\x,0)--(\x,\H);}
    \foreach \y in {0.75,1.5,2.25} {\draw[red!70] (0,\y)--(\W,\y);}
    \path[fill=red!20,draw=red!70] (2,0.75) rectangle (3,1.5);
    \node at (2.5,1.125) {$I_j$};
  \end{scope}

  % --------- стрелки (диагонально к нижней картинке) ----------
  % центр ячейки пересечения в нижнем прямоугольнике имеет глобальные координаты (5.75, -3.75)
  \draw[-{Latex[length=3mm]}] (2.5,0) -- (4.5,-1.75);
  \draw[-{Latex[length=3mm]}] (9.5,0) -- (7.5,-1.75);

  % --------- нижний прямоугольник: пересечение I_{ij} ----------
  \begin{scope}[yshift=-5cm,xshift=3.5cm]
    \draw[thick,rounded corners=2pt] (0,0) rectangle (\W,\H);
    \foreach \x in {1.25,2.5,3.75} {\draw[black!60] (\x,0)--(\x,\H);}
    \foreach \y in {1,2}           {\draw[black!60] (0,\y)--(\W,\y);}
    \foreach \x in {1,2,3,4}       {\draw[red!70] (\x,0)--(\x,\H);}
    \foreach \y in {0.75,1.5,2.25} {\draw[red!70] (0,\y)--(\W,\y);}
    \path[fill=red!40,draw=red!70] (2,1) rectangle (2.5,1.5);
    \node at (2.25,1.25) {$I_{ij}$};
  \end{scope}
\end{tikzpicture}

\caption{Пересечение разбиений $\T_1$ и $ \T_2$}
\end{figure}
% ПОФИКСИТЬ ФОТО


\subsection{Определение верхней и нижней суммы Дарбу}
\definition Пусть $I$ — замкнутый брус, $f: I\mapsto \R$, $\T = \{I_i\}_{i=1}^{K}$ — разбиение бруса $I$, $m_i = \displaystyle\inf_{I_i} (f)$, и $M_i = \displaystyle\sup_{I_i} (f)$

Тогда числа $\underline{S}(f, \T) = \sum_{i=1}^{K}m_i|I_i|$ и $\overline{S}(f, \T) = \sum_{i=1}^KM_i|I_i|$ будем называть \textit{нижней и верхней суммой Дарбу} соответственно

\subsection{Определение верхнего и нижнего интеграла Дарбу}
\definition \textbf{Верхним и нижним интегралом} Дарбу будем называть числа соответственно
\begin{equation*}
    \oi := \inf_{\T}\os(f, \T) \qquad \ui := \sup_{\T}\us(f, \T)
\end{equation*}

\subsection{Определение допустимого множества}
\definition Множество $D\subset\mathbb{R}^n$ называется \textit{допустимым}, если
\begin{itemize}
    \item $D$ — ограниченно
    \item $\partial D$ — множество меры нуль по Лебегу
\end{itemize}

\subsection{Определение интеграла Римана по допустимому множеству}
\definition Пусть $D\subset\mathbb{R}^n \text{ --- допустимое множество}, f:D\rightarrow\mathbb{R}$. Тогда, интегралом Римана $f$ по $D$ называется число $\mathcal{I}$:
\begin{equation*}
    \mathcal{I}=\int\limits_{D}f(\overline{x})\d{\overline{x}}=\int\limits_{I\supset D}f\cdot \chi_{D}(\overline{x})\d{\overline{x}}\text{, где }\chi_{D}=\begin{cases}
        1,\overline{x} \in D\\
        0,\overline{x} \not\in D
    \end{cases}
\end{equation*}
Если $\mathcal{I} < \infty$, то $f \in\riman{D}$
\begin{center}
    
\begin{tikzpicture}
    % Рисуем полосатый прямоугольник
    \fill[pattern=north east lines, opacity=0.4] (1,0) rectangle (5,3);

    % Определяем гладкую кляксу (неправильной формы)
    \begin{scope}
        \clip (2,1.5) .. controls (2.5,2.5) and (3.5,2.8) .. (4,2)
              .. controls (4.3,1.5) and (4,0.8) .. (3.5,0.5)
              .. controls (2.8,0.2) and (1.8,0.7) .. (2,1.5) -- cycle;
        \fill[white] (0,0) rectangle (6,4); % Закрашиваем кляксу белым
    \end{scope}
    \node at (1.7, 2.5) {$I$};

    % Контуры для ясности
    \draw (1,0) rectangle (5,3);
    \draw (2,1.5) .. controls (2.5,2.5) and (3.5,2.8) .. (4,2)
          .. controls (4.3,1.5) and (4,0.8) .. (3.5,0.5)
          .. controls (2.8,0.2) and (1.8,0.7) .. (2,1.5) -- cycle;
    \node at (3, 1.5) {$D$};

    \node[align=center] at (2.9, -1) {Закрашенная область не вносит вклад в объем \\ так как $f(x)\cdot\chi_D=0$};
\end{tikzpicture}

\end{center}


\subsection{Определение сходимости функциональной последовательности в точке}
Пусть $X\subset\mathbb{R}$ и $f_n:X\rightarrow\mathbb{R}\ \forall n\in\mathbb{N}$.

\definition Последовательность функций $\{f_n(x)\}_{n=1}^{\infty}$ \textit{сходится в точке} $x_0\in X$, если сходится соответствующая числовая последовательность $\{f_n(x_0)\}_{n=1}^{\infty}$:
\begin{equation*}
    x_0\in X,\ \forall\ve >0\ \exists N:\forall n>N\hookrightarrow|f_n(x_0)-a_{x_0}|<\ve\Longrightarrow a_{x_0}=\lim\limits_{n\to\infty} f_n{x_0}
\end{equation*}

\subsection{Определение множества сходимости функциональной последовательности}
\definition Множество $D\subset X$ точек, в которых последовательность функций $\{f_n(x)\}_{n=1}^{\infty}$ сходится называется \textit{множеством сходимости}

\subsection{Определение предельной функции функциональной последовательности}
\definition Пусть $D\subset X$ — множество сходимости $\{f_n(x)\}_{n=1}^{\infty}$ и $\forall x\in D$ $f_n(x)\rightarrow f(x)$. Тогда, $f(x)=\lim\limits_{n\to\infty} f_n(x)$ будем называть \textit{предельной функцией} $\{f_n(x)\}$

\subsection{Определение поточечной сходимости функциональной последовательности на множестве}
\definition $D\subset \mathbb{R}, f,f_n:D\rightarrow\mathbb{R}$. Будем говорить, что $\{f_n(x)\}$ \textit{сходится поточечно} к $f(x)$ на $D$, если 
\begin{equation*}
    \forall x\in D,\ \forall\ve >0\ \exists N:\ \forall n>N\hookrightarrow|f_n(x)-f(x)|<\ve
\end{equation*}

Обозначение: $f_n(x)\overset{D}{\longrightarrow}f(x)$

\subsection{Определение равномерной сходимости функциональной последовательности на множестве}
\definition Пусть $D\subset\mathbb{R};\ f_n,f:D\longrightarrow\mathbb{R}$. Будем говорить, что $\{f_n(x)\}$ \textit{сходится равномерно} к $f(x)$ на $D$, Если
\begin{equation*}
    \forall\ve>0\ \exists N:\ \forall n>N,\ \forall x\in D\text{ такое, что }|f_n(x)-f(x)|<\ve
\end{equation*}

Обозначение: $f_n\overset{D}{\rightrightarrows} f$

\subsection{Определение поточечной сходимости функционального ряда на множестве}
Пусть $D\subset\mathbb{R}$, $f_n, S:D\to\mathbb{R}\ (\forall n\in\mathbb{N})$, а также $S_k(x)=\sum_{n=1}^{k}f_n(x)$ — частичные суммы функционального ряда
\definition Если $\exists S(x): S_k\overset{D}{\longrightarrow}S$, то будем говорить, что функциональный ряд $\sum_{n=1}^{\infty}f_n(x)$ \textit{сходится поточечно} к $S(x)$ на $D$

\subsection{Определение равномерной сходимости функционального ряда на множестве}
Пусть $D\subset\mathbb{R}$, $f_n, S:D\to\mathbb{R}\ (\forall n\in\mathbb{N})$, а также $S_k(x)=\sum_{n=1}^{k}f_n(x)$ — частичные суммы функционального ряда
\definition Если $\exists S(x): S_k\overset{D}{\rightrightarrows}S$, то будем говорить, что функциональный ряд $\sum_{n=1}^{\infty}f_n(x)$ \textit{сходится равномерно} к $S(x)$

\subsection{Определение абсолютной сходимости сходимости функционального ряда на множестве}
\definition Ряд $\sum_{n=1}^{\infty}f_n(x)$ \textit{сходится абсолютно}, если
\begin{equation*}
    \forall x_0\in \sum_{n=1}^{\infty}f_n(x_0)\text{ — сходится абсолютно}
\end{equation*}


\newpage
\section{Основные формулировки}
\subsection{Свойства меры бруса в $\mathbb{R}^n$}
\begin{enumerate}
    \item \textbf{Однородность:} $\mu(I_{\lambda a,\lambda b})=\lambda^n\cdot\mu(I_{a,b})$, где $\lambda\geqslant
    0$
    \item \textbf{Аддитивность:} Пусть $I, I_1, \ldots, I_k$ — брусы
    
    Тогда, если $\forall i, j\, I_i, I_j$ не имеют общих внтренних точек, и $\displaystyle\bigcup_{i=1}^kI_i = I$, то
    \begin{equation*}
        |I| = \sum_{i=1}^k|I_i|
    \end{equation*}
    \item \textbf{Монотонность}: Пусть $I$ — брус, покрытый конечной системой брусов, то есть $I\subset \displaystyle\bigcup_{i=1}^kI_i$, тогда
    \begin{equation*}
        |I| < \sum_{i=1}^k|I_i|
    \end{equation*}
\end{enumerate}

\subsection{Необходимое условие интегрируемости функции по Риману}
\theorem Пусть $I$ — замкнутый брус
\begin{equation*}
    f\in \mathcal{R}(I) \implies f \text{ ограничена на } I
\end{equation*}

\subsection{Свойства интеграла Римана}
\begin{enumerate}
    \item \textbf{Линейность.}
    \begin{equation*}
        f, g \in \mathcal{R}(I) \implies (\alpha f + \beta g)\in \mathcal{R}(I)\ \forall \alpha, \beta \in \R
    \end{equation*}
    И верно, что:
    \begin{equation*}
            \int_I(\alpha f + \beta g)\d{x} = \alpha\int_I f\d{x} + \beta\int_Ig\d{x}
    \end{equation*}
    \item \textbf{Монотонность}
    \begin{equation*}
        f, g\in \mathcal{R}(I);\ f|_I\leqslant g|_I \implies \int_If\d{x} \leqslant \int_Ig\d{x}
    \end{equation*}
    \item \textbf{Оценка интеграла (сверху)}
    \begin{equation*}
        f\in \mathcal{R}(I) \implies \left|\int_If\d{x}\right| \leqslant\sup\limits_{I}|f||I|
    \end{equation*}
\end{enumerate}

\subsection{Свойства множества меры нуль по Лебегу}
\begin{enumerate}
    \item В определении множества меры нуль можно использовать \textit{открытые} брусы
    \item $M$ — множество меры нуль, $L\subset M\Longrightarrow L$ — множество меры нуль
    \item Не более чем счетное объединение множеств меры нуль является множеством меры нуль
\end{enumerate}

\subsection{Критерий замкнутости множества в $\mathbb{R}^n$}
\theorem $M$ — замкнуто $\Longleftrightarrow$ $M$ содержит \textbf{все} cвои предельные точки

\subsection{Теорема о компактности замкнутого бруса в $\mathbb{R}^n$}
\theorem Пусть $I\subset\mathbb{R}^n$ — замкнутый брус $\Longrightarrow I$ — компакт

\subsection{Критерий компактности в $\mathbb{R}^n$}
\theorem Пусть $K\subset \mathbb{R}^n$. $K$ — компакт $\Longleftrightarrow$ $K$ замкнуто и ограниченно

\subsection{Теорема Вейерштрасса о непрерывной функции на компакте}
\theorem Пусть $K\subset \R^n$ — компакт и функция $f: K\mapsto \R$ - непрерывная. Тогда $f$ на $K$ достигает наибольшего и наименьшего значений

\subsection{Теорема о связи непрерывности функции в точке с колебанием}
\theorem Пусть $x_0 \in M \subset \R^n; \,\, f: M\mapsto\R$. $f$ — непрерывна в точке $x_0 \iff \omega(f, x_0) = 0$

\subsection{Критерий Лебега интегрируемости функции по Риману}
\theorem Если $I\subset \mathbb{R}^n$ — замкнутый невырожденный брус, $f: I\to\R$, то $f\in R(I) \iff f$ ограничена и непрерывна почти всюду на $I$

\subsection{Свойства интегральных сумм Дарбу}
\begin{enumerate}
    \item \begin{equation*}
        \us(f, \T) = \inf_{\xi}\sigma(f, \T, \xi) \le \sup_{\xi}\sigma(f, \T, \xi) = \os(f, \T)
    \end{equation*}

    \item Пусть $\tilde{\T}$ — измельчение разбиения $\T$, тогда
    \begin{equation*}
        \us(f, \T) \le \us(f, \tilde{\T}) \le \os(f, \tilde{\T}) \le \os(f, \T)
    \end{equation*}

    \item $\forall \T_1, \T_2:\ \us(f, \T_1) \le \os(f, \T_2)$
\end{enumerate}

\subsection{Теорема об интегралах Дарбу как пределах интегральных сумм Дарбу}
\theorem Пусть $I\subset \R^n$ — замкнутый брус, а $f: I \mapsto \R$ — ограничена. Тогда:
\begin{equation*}
    \oi = \lim_{\Delta_{\T}\to0}\os(f, \T) \qquad \text{и} \qquad \ui = \lim_{\Delta_{\T} \to 0} \us(f, \T)
\end{equation*}

\subsection{Критерий Дарбу интегрируемости функции на замкнутом брусе}
\theorem $I\in\mathbb{R}^n\text{ — замкнутый брус, } f:I\mapsto \mathbb{R}, f\in \mathcal{R}(I)\Longleftrightarrow f$ — ограничена на $I$ и $\ui=\oi$

\subsection{Утверждение о независимости определения допустимого множества от выбора бруса}
Пусть $I_1\supset D, I_2\supset D$, тогда 
\begin{equation*}
    \int\limits_{I_1} f\cdot\chi_D\d{x}\text{ и }\int\limits_{I_2}f\cdot\chi_{D}\d{x}
\end{equation*}

либо существуют и равны, либо оба не существуют вообще

\subsection{Теорема Фубини о переходе к повторному интегралу}
Пусть имеются $I_x\subset\mathbb{R}^n, I_y\subset\mathbb{R}^m, I_x\times I_y\subset \mathbb{R}^{m+n}$ — замкнутые брусы, $f:I_x\times I_y\rightarrow \mathbb{R}$, $f\in\riman{I_x\times I_y}$ и $\forall$ фиксированного $x\in I_x \implies f(x,y)\in\riman{I_y}\Longrightarrow$
\begin{equation*}
    \int\limits_{I_x\times I_y} f(\overline{x}, \overline{y})\d{\overline{x}}\d{\overline{y}}=\int\limits_{I_x}\left(\int\limits_{I_y}f(\overline{x},\overline{y})\d{\overline{y}}\right)\d{\overline{x}}=\int\limits_{I_x}\d{\overline{x}}\int\limits_{I_y}f(\overline{x}, \overline{y})\d{\overline{y}}
\end{equation*}

\comment аналагочино, если взять для $\forall$ фиксированного $y\in I_y$

\subsection{Cупремальный критерий равномерной сходимости функциональной последовательности}
\theorem $f_n\overset{D}{\rightrightarrows} f\Longleftrightarrow \lim\limits_{n\to\infty}\left(\sup\limits_{D} \left|f_n(x)-f(x)\right|\right)=0$

\subsection{Критерий Коши равномерной сходимости функциональной последовательности}
\theorem $f_n(x)\overset{D}{\rightrightarrows} f(x)\Longleftrightarrow\forall\ve>0\ \exists N:\ \forall n,m>N,\ \forall x\in D\hookrightarrow |f_n(x)-f_m(x)|<\ve$

\comment Отрицание Критерия Коши: 
\begin{equation*}
    f_n(x)\not\overset{D}{\rightrightarrows} f(x)\Longleftrightarrow\exists\ve_0>0\ \forall N:\ \exists n,m>N,\ \exists x_0\in D\ |f_n(x)-f_m(x)|\geqslant\ve_0
\end{equation*}

\subsection{Теорема о почленном переходе к пределу для функциональной последовательности}
\theorem Пусть $f_n,f: D\longrightarrow\mathbb{R},\ x_0\text{ — предельная точка } D,\ f_n\overset{D}{\rightrightarrows} f,\ \forall n\in\mathbb{N}\ \exists\lim\limits_{x\to x_0} f_n(x)=c_n$

Тогда,
\begin{equation*}
    \begin{aligned}
        &\exists\lim\limits_{n\to\infty} c_n=\lim\limits_{x\to x_0} f(x)\\
        &\left(\text{или }\lim\limits_{n\to\infty} \left(\lim_{x\to x_0} f_n(x)\right)=\lim_{x\to x_0}\left(\lim_{n\to\infty} f_n(x)\right)\right)
    \end{aligned}
\end{equation*}

\subsection{Теорема о непрерывности предельной функции}
\theorem Пусть имеется $\left.\begin{aligned}
    &f_n,f: D\longrightarrow\mathbb{R},\\
    &f_n\overset{D}{\rightrightarrows} f,\\
    &\forall n\in\mathbb{N}\ f_n\in C(D)
\end{aligned}\right\}\Longrightarrow f\in C(D)$

\subsection{Утверждение о неравномерной сходимости фун. послед. при наличии разрыва}
\theorem Пусть имеется $\left.\begin{aligned}
    &f_n\in C\left([a;b)\right),\\
    &f\in C((a;b))+\text{разрыв в т.}a,\\
    &f_n\overset{[a;b)}{\longrightarrow} f
\end{aligned}\right\}\Longrightarrow f_n\overset{(a;b)}{\not\rightrightarrows} f$

То есть будет поточечная сходимость, но не будет равномерной:
\begin{equation*}
    f_n\overset{(a;b)}{\longrightarrow}f,\text{ но не }f_n\overset{(a;b)}{\rightrightarrows} f
\end{equation*}

\subsection{Утверждение о неравномерной сходимости фун. послед. при наличии расходимости в точке}
\theorem Пусть имеется $\left.\begin{aligned}
    &f_n\in C\left([a;b)\right)\\
    &f_n\overset{(a;b)}{\longrightarrow} f\\
    &\not\exists \lim\limits_{n\to\infty}f_n(a)
\end{aligned}\right\}\Longrightarrow f_n\overset{(a;b)}{\not\rightrightarrows} f$

\subsection{Теорема о почленном интегрировании функциональной последовательности}
\theorem Пусть имеется $\left.\begin{aligned}
    &f_n,f:[a;b]\to\mathbb{R}\\
    &f_n\overset{[a;b]}{\rightrightarrows}f\\
    &f_n\in\riman{[a;b]}\forall n\in\mathbb{N}
\end{aligned}\right\}\Longrightarrow f\in\riman{[a;b]}\text{ и }\lim\limits_{n\to\infty}\int\limits_{a}^b f_n(x)\d{x}=\int\limits_{a}^b f(x)\d{x}$

\subsection{Теорема о почленном дифференцировании функциональной последовательности}
\theorem Пусть имеется $\left.\begin{aligned}
    &f_n,f,g:[a;b]\to\mathbb{R}\\
    &f_n\in D([a;b])\\
    &\exists c\in[a;b]:\exists \lim\limits_{n\to\infty} f_n(c)\\
    &\exists g(x):\ f^{\prime}_n\overset{[a;b]}{\rightrightarrows}g(x)
\end{aligned}\right\}\Longrightarrow \begin{aligned}
    &\exists f:\ f_n\overset{[a;b]}{\rightrightarrows}f\\
    &\oplus f^{\prime}(x)=g(x)
\end{aligned}$

\subsection{Критерий Коши равномерной сходимости функционального ряда}
\theorem Пусть $f_n:D\to\mathbb{R}\ \forall n\in\mathbb{N}$, $\sum_{n=1}^{\infty}f_n(x)\overset{D}{\rightrightarrows}$ тогда и только тогда, когда
\begin{equation*}
    \forall \ve>0\ \exists N:\ \forall m>k>N\ \forall x\in D\hookrightarrow |S_m(x)-S_k(x)|=\left|\sum_{n=k+1}^{m}f_n(x)\right|<\ve
\end{equation*}

\subsection{Необходимое условие равномерной сходимости функционального ряда}
\corollary Пусть $\left.\begin{aligned}
   &f_n:D\to\mathbb{R}\ (\forall n\in\mathbb{N})\\
   &\sum_{n=1}^{\infty}f_n(x)\overset{D}{\rightrightarrows}
\end{aligned}\right\}\Longrightarrow f_n(x)\overset{D}{\rightrightarrows}0$

\subsection{Сравнительный признак равномерной сходимости функционального ряда}
\theorem Имеется $\quad \left.\begin{aligned}
    &\sum_{n=1}^{\infty}a_n(x)\text{ и }\sum_{n=1}^{\infty}b_n(x):\\
    &\exists N\ \forall n>N\ \forall x\in D\ |a_n(x)|\leqslant b_n(x)\\
    &\sum_{n=1}^{\infty}b_n(x)\overset{D}{\rightrightarrows}
\end{aligned}\right\}\Longrightarrow \sum_{n=1}^{\infty}a_n(x)\overset{D}{\rightrightarrows}\text{ и }\sum_{n=1}^{\infty}a_n(x)\text{ сходится абсолютно }D$

\subsection{Мажорантный признак Вейерштрасса о равномерной сходимости функционального ряда}
\corollary Из признака сравнения. $\quad \left.\begin{aligned}
    &\sum_{n=1}^{\infty}a_n(x):\\
    &\exists N\ \forall n>N\ \sup\limits_{D}|a_n(x)|\leqslant M_n\\
    &\sum_{n=1}^{\infty} M_n\text{ — сходится}
\end{aligned}\right\}\Longrightarrow \begin{aligned}
    &\sum_{n=1}^{\infty}a_n\overset{D}{\rightrightarrows}\\
    &\sum_{n=1}^{\infty}a_n\text{ сходится абсолютно на }D
\end{aligned}$





\newpage
\section{Вопросы на доказательство}

\subsection{Необходимое условие интегрирования.}
\theorem Пусть $I$ — замкнутый брус. 
\begin{equation*}
    f\in \mathcal{R}(I) \implies f \text{ ограничена на } I
\end{equation*}

\proof От противного.

\begin{enumerate}
    \item $f\in \mathcal{R}(I) \implies \exists {A} \in \R$, такая что $\forall \epsilon > 0$, а значит для $\epsilon = 1$ тоже:
    \begin{equation}
        \exists \delta > 0 \colon \forall (\T, \xi): \Delta_{\T} \leq \delta \text{ верно } \abs{\sigma(f, \T, \xi) - A} < 1
    \end{equation}

    Отсюда
    \begin{equation}
        A - 1 < \sigma < A + 1 \implies \sigma \text{ ограничена}
    \end{equation}

    \item С другой стороны, так как предположили, что $f$ --- неограничена на $I$
    \begin{equation}
        \forall \T = \{I_i\}^k_{i=1} \quad \exists i_0 \colon f \text{ неограничена на } I_{i_0}
    \end{equation}
    
    Тогда рассмотрим интегральную сумму
    \begin{equation}
        \sigma(f, \T, \xi) = \sum_{i \neq i_0} f(\xi_i) \cdot \abs{I_i} + f(\xi_{i_0}) \cdot \abs{I_{i_0}}
    \end{equation}

    Выбором подходящего $\xi_{i_0}$ можно сделать $f(\xi_{i_0})$ сколь угодно большой $\implies \sigma$ будет не ограничена - \mbox{противоречние}
\end{enumerate}

Из противоречния пунктов 1 и 2 следует, что
\begin{equation*}
    f\in \mathcal{R}(I) \implies f \text{ ограничена на } I
\end{equation*}
\qed


\subsection{Свойства интеграла Римана}

\begin{enumerate}
    \item {\textbf{Линейность.}
    \begin{equation*}
        f, g \in \mathcal{R}(I) \implies (\alpha f + \beta g)\in \mathcal{R}(I)\ \forall \alpha, \beta \in \R
    \end{equation*}
    И верно, что:
    \begin{equation*}
            \int_I(\alpha f + \beta g)\d{x} = \alpha\int_I f\d{x} + \beta\int_Ig\d{x}
    \end{equation*}

\proof 

\begin{equation*}
\begin{aligned}
    &f \in \mathcal{R}(I): \exists A_f, \text{что} \quad \forall \varepsilon > 0 \, \exists\delta_1>0\ \forall(\T,\xi)\colon \Delta_{\T} < \delta_1 &&\text{ верно }
    \abs{\sigma(f, \T, \xi)  - \int_If\d{x}} =: \abs{\sigma_f - A_f} < \frac{\epsilon}{|\alpha| + |\beta| + 1}
    \\
    &g \in \mathcal{R}(I): \exists A_g, \text{что} \quad \forall \varepsilon > 0 \, \exists\delta_2>0\ \forall(\T,\xi)\colon \Delta_{\T} < \delta_2 &&\text{ верно }
    \abs{\sigma(g, \T, \xi)  - \int_Ig\d{x}} =: \abs{\sigma_g - A_g} < \frac{\epsilon}{|\alpha| + |\beta| + 1}
    \end{aligned}
\end{equation*}
Тогда $\forall (\T, \xi) \colon \Delta_{\T} < min(\delta_f, \delta_g) = \delta:$
\begin{align}
    \abs{\sigma(\alpha f+\beta g, \T, \xi) - \alpha A_f+ \beta A_g} &= \abs{\sum(\alpha f(\xi_i) + \beta g(\xi_i)) \cdot \abs{I_i} - \alpha A_f - \beta A_g} \leq \\
    &\leq |\alpha|\cdot|\sigma_f - A_f| + |\beta|\cdot|\sigma_g-A_g|
    < \left(|\alpha| + |\beta|\right) \frac{\varepsilon}{|\alpha|+|\beta|+1} < \ve
\end{align}
\qed
    }
\item {\textbf{Монотонность}
\begin{equation*}
    f, g\in \mathcal{R}(I);\ f \leq g \text{ на } I \implies \int_If\d{x} \leqslant \int_Ig\d{x}
\end{equation*}
\proof
    \begin{equation*}
        f\in \mathcal{R}(I) \implies \exists A_f\in \R \colon \forall \ve > 0\ \exists\delta: \forall(\T, \xi): \Delta_{\T} < \delta, \text{ выполняется } |\sigma_f - A_f| < \ve\
    \end{equation*}
    Аналогично для $g\in \mathcal{R}(I)$, тогда:
    \begin{equation}
    \begin{cases}
        A_f - \epsilon < \sigma_1 < A_f + \epsilon \\
        A_g - \epsilon < \sigma_2 < A_g + \epsilon \\
        \sigma_f \leq \sigma_g
    \end{cases}
    \end{equation}
    
    Отсюда
    \begin{equation*}
    \begin{aligned}
        A_f - \ve < \sigma_f \leqslant \sigma_g < A_g + \ve \implies
        A_f - \epsilon < A_g + \epsilon \implies A_f < A_g + 2 \epsilon \qquad \forall \epsilon > 0
    \end{aligned}
    \end{equation*}
    % Что верно для $\forall \ve > 0$, даже при $\ve \to 0 \implies A_f \le A_g$
\qed
}
\item {\textbf{Оценка интеграла (сверху)}
\begin{equation*}
    f\in \mathcal{R}(I) \implies \left|\int_If\d{x}\right| \leqslant\sup\limits_{I}|f||I|
\end{equation*}
\proof
По необходимому условию для интегрируемости функции (см. ниже)
\begin{equation*}
    \begin{aligned}
        f\in \mathcal{R}(I) &\implies f \text{ Ограничена на } I\\
        &\implies -\sup_I|f| \leqslant f \leqslant \sup_I|f|
    \end{aligned}
\end{equation*}
Тогда,
\begin{equation*}
    \begin{aligned}
        -\int_I\sup|f|\d{x} &\leqslant \int_If\d{x} &\leqslant\int_I\sup|f|dx\\
        -\sup_I|f||I|&\leqslant \int_If\d{x}&\leqslant \sup_I|f||I|
    \end{aligned}
\end{equation*}
\qed
}
\end{enumerate}

\subsection{Свойства множества меры нуль по Лебегу}
\begin{enumerate}
    \item {Если в определении $\{I_i\}$ заменить на открытые брусы, то определение останется верным.

    \proof Пусть $\{I_i\}$ — открытые брусы, тогда $\forall \epsilon > 0 \ \exists$ не более чем счетный набор $\{I_i\}$:
    $M\subset \displaystyle\bigcup_iI_i \ $ и $ \ \sum |I_i| < \epsilon$
    
    Пусть $\{\bar I_i\}$ — открытые брусы + границы = замкнутые брусы $I_i$, причём объем ``добавленных'' плоскостей будет нулевой, так как объем бруса $n-1$ размерности, будет нулевым для объема бруса размерности $n$
    \begin{equation*}
        \begin{aligned}
            M\subset\bigcup_iI_i \subset\bigcup_i\bar I_i, \ \text{при этом} \ |I_i| = |\bar I_i|
        \end{aligned}
    \end{equation*}
    Если


    \begin{equation*}
        \forall\ve\, \exists\{I_i\}: M \subset \bigcup_iI_i: \sum_i|I_i|<\ve
    \end{equation*}
    то
    \begin{equation*}
        \forall\ve\, \exists\{\bar I_i\}: M \subset \bigcup_i\bar I_i: \sum_i|\bar I_i|<\ve
    \end{equation*}
    \textbf{Докажем в обратную сторону.} Мы хотим увеличить замкнутый брус в два раза и увеличенный брус взять открытым.
    
    Пусть $\{I_i\}$ — набор замкнутых брусов
    %сюда бы рисунок из конца 2 лекции но там в тикзе надо уметь рисовать...
    \begin{equation*}
        I_i = [a^1_i, b^1_i]\times\ldots\times[a^n_i, b^n_i], \quad V_i = \sum_i|I_i|<\frac{\ve}{2^n}
    \end{equation*}
    Так как $\left(\frac{a_i^k}{2}, \frac{b_i^k}{2}\right)$ --- центр $i$-го бруса в $k$-ом измерении, увеличить изначальный брус в два раза по этому измерению можно сдвинувшись от центра не на половину, а на целую сторону, то есть на $b_i^k - a_i^k$

    Таким образом: 

    \begin{equation*}
        \tilde{I_i} = \left(\frac{a_i^1+b_i^1}{2} - (b_i^1-a_i^1) ; \frac{a_i^1 + b_i^1}{2} + (b_i^1 - a_i^1)\right) \times \ldots\times \left(\frac{a_i^n+b_i^n}{2} - (b_i^n-a_i^n) ; \frac{a_i^n + b_i^n}{2} + (b_i^n - a_i^n)\right)
    \end{equation*}
    $\implies V_2 = \displaystyle\sum_i|\tilde{I_i} | = 2^n \cdot V_1 < \ve$
    \qed
    }
    \item {Если $M \subset \R^n$ - множество меры нуль по Лебегу, то из $L \subset M \implies L$ - множество меры нуль по Лебегу

    \proof Докажем по транзитивности
    \begin{equation*}
        \forall\ve\ > 0, \ \exists \ \text{не более чем счетный набор} \ \{I_i\}: L \subset M \subset \bigcup_iI_i \implies L \subset \bigcup_iI_i
    \end{equation*}

    По условию нам дано, что для $M \subset \bigcup_iI_i$ верно $\sum_i |I_i| < \ve$, и тоже самое выполнено и для $L \subset \bigcup_iI_i$, тогда $L$ по определнию является множеством меры нуль по Лебегу
    \qed
    }
    \item {Не более чем счетное объединение множеств меры нуль по Лебегу, тоже является множеством меры нуль по Лебегу

    \proof пусть $M = \bigcup_i^\infty M_k$ - объединение не более чем счетного числа множеств $\forall k \ M_k$ - множество меры нуль по Лебегу $\implies \forall k, \ \forall\ve>0 \ \exists\{I_i\}_{i=1}^{\infty}$ по определению множества меры нуль для них верно

    \begin{enumerate}[label=\textbullet]
        \item $M_k \subset \displaystyle\bigcup_i^\infty I_i^k$ \footnote{$I_i^k$ - это $i$-ый для $M_k$, а не степень}
        \item $\displaystyle\sum_i|I_i| < \ve_k\,\, \quad \forall \ve_k > 0$
    \end{enumerate}
    Отсюда получаем $M = \displaystyle\bigcup_i^\infty M_k \subset \displaystyle\bigcup_i^\infty I_i^k$ и $\sum_{k=1}^{\infty}\sum_{i=1}^{\infty} |I_i^k| < \sum_{k=1}^{\infty}\ve_k$ - если теперь взять $\ve_k = \frac{\ve}{2^k}$, то мы получим
    \begin{equation*}
     \sum_{k=1}^{\infty}\ve_k = \sum_{k=1}^{\infty} \frac{\ve}{2^k} < \ve
    \end{equation*}

    \qed
    }
\end{enumerate}


\subsection{Критерий замкнутости}
% Наличие социофобии

\theorem $M$ — замкнуто $\Longleftrightarrow$ $M$ содержит \textbf{все} cвои предельные точки

\proof Докажем необходимость и достаточность
\begin{enumerate}
    \item \textit{(Необходимость)} Докажем $\Longrightarrow$ от противного
    \begin{itemize}
        \item Пусть $x_0$ — предельная для $M$ и $x_0\notin M$. Тогда, $\forall\ve>0\ \stackrel{\circ}{B_{\ve}}(x_0)\cap M\ne\varnothing\text{ и }x_0\in\mathbb{R}^n$
        \item По условию $M$ — замкнуто, то есть $\mathbb{R}^n\setminus M$ — открыто $\Longrightarrow$ все его точки внутренние и $\exists r>0$:
        $$B_{r}(x_0)\subset\mathbb{R}^n\setminus M\Longrightarrow\stackrel{\circ}{B_r(x_0)}\subset\mathbb{R}^n\setminus M\text{ и }\stackrel{\circ}{B_r}(x_0)\cap M=\varnothing$$

        Пришли к противоречию $\Longrightarrow$ $M$ содержит все свои предельные точки\qed
    \end{itemize}
    \item \textit{(Достаточность)} Докажем $\Longleftarrow$

    Пусть $y_0$ — не является предельной для $M$, то есть $y_0\in\mathbb{R}^n\setminus M\Longrightarrow\exists r>0$:
    \begin{equation*}
        \begin{cases}
            \stackrel{\circ}{B_{r}}(y_0)\cap M=\varnothing\\
            y_0\in\mathbb{R}^n\setminus M
        \end{cases}\Longrightarrow B_r(y_0)\subset \mathbb{R}^n\setminus M
    \end{equation*}
    $\Longrightarrow\mathbb{R}^n\setminus M$ — открытое и состоит из всех точек, не являющихся предельными $\Longrightarrow$ $M$ — замкнуто по определению\qed
\end{enumerate}

\subsection{Теорема о компактности замкнутого бруса}
\theorem Пусть $I\subset\mathbb{R}^n$ — замкнутый брус $\Longrightarrow I$ — компакт

\proof Пойдем от противного

Пусть $I=[a_1;b_1]\times\ldots\times[a_n;b_n]$
\begin{enumerate}
    \item Положим, что $I$ — не компакт. Значит, существует его покрытие $\{A_{\alpha}\}$ — открытые множества, такие что $I\subset \{A_{\alpha}\}$, не допускающее выделения конечного подкпорытия
    \item Поделим каждую сторону пополам. Тогда, $\exists I_1$, такой что не допускает конечного подпокрытия. Иначе, $I$ — компакт
    \item Аналогично, повторим процесс и получим систему вложенных брусов: $$I\supset I_1\supset I_2\supset \ldots$$
    То есть на каждой стороне возникает последовательность вложенных отрезков, которые стягиваются в точку $a=(a_1,\ldots,a_n)$

    \begin{center}
        \documentclass[a4paper]{article}
\usepackage[english,russian]{babel}         


\usepackage{tikz}
\usetikzlibrary{arrows.meta}
\begin{document}

\begin{tikzpicture}[scale=2, every node/.style={inner sep=0,outer sep=0}]

% Первый квадрат (деление на 4, выделен верхний правый)
\begin{scope}[xshift=-3.0cm]
  % внешний квадрат
  \draw[thick] (0,0) rectangle (1,1);
  % сетка (деление пополам)
  \draw (0.5,0) -- (0.5,1);
  \draw (0,0.5) -- (1,0.5);
  % выделенный кусок (например, верхний правый)
  \fill[red!60] (0.5,0.5) rectangle (1,1);
  \node[below] at (0.5,-0.12) {Шаг 0};
\end{scope}

% Второй квадрат (берём выделенный кусок и делим его на 4; выделяем его верхний правый)
\begin{scope}[xshift=-1.4cm]
  \draw[thick] (0,0) rectangle (1,1);
  \draw (0.5,0) -- (0.5,1);
  \draw (0,0.5) -- (1,0.5);
  \draw[thin] (0.5,0.5) -- (1,0.5);
  \draw[thin] (0.75,0.5) -- (0.75,1); % вертикальная внутри (точно по середине от 0.5 до 1)
  \draw[thin] (0.5,0.75) -- (1,0.75); % горизонтальная
  % выделяем в этом блоке верхний правый (т.е. от (0.75,0.75) до (1,1))
  \fill[red!60] (0.5,0.5) rectangle (0.75,0.75);
  \node[below] at (0.5,-0.12) {Шаг 1};
\end{scope}

% Третий квадрат (ещё одно деление внутри выделенного квадрата)
\begin{scope}[xshift=0.2cm]
  \draw[thick] (0,0) rectangle (1,1);
  \draw (0.5,0) -- (0.5,1);
  \draw (0,0.5) -- (1,0.5);
  % внутри верхнего правого квадрата делим ещё раз
  % координаты верхнего правого квадрата: от (0.5,0.5) до (1,1)
  % его середина в (0.75,0.75)
  \draw[thin] (0.5,0.5) -- (1,0.5); % горизонтальная внутри
  \draw[thin] (0.75,0.5) -- (0.75,1); % вертикальная внутри (точно по середине от 0.5 до 1)
  \draw[thin] (0.5,0.75) -- (1,0.75); % горизонтальная
  % выделяем в этом блоке верхний правый (т.е. от (0.75,0.75) до (1,1))
  \draw[thin] (0.5, 0.625) -- (0.75, 0.625);
  \draw[thin] (0.625, 0.5) -- (0.625, 0.75);
  \fill[red!60] (0.625,0.5) rectangle (0.75,0.625);

  \node[below] at (0.5,-0.12) {Шаг 2};
\end{scope}

% Многоточие
\node at (2.0,0.5) {\Large $\dots$};

% Финальный предельный квадрат с точкой (символически)
\begin{scope}[xshift=2.2cm]
  %\draw[thick] (0.3,0.3) rectangle (0.7,0.7);
  \fill[red!60] (0.5,0.5) circle (0.75pt);
  \node[right] at (0.55,0.45) {$a$};
  \node[below] at (0.5,-0.12) {Предел};
\end{scope}

% Подсказка / пояснение (русский)
\node[align=center] at (0,-0.7) {
Последовательность вложенных брусов в $\R^2$: на каждом шаге выбираем \\ квадрат, что по предположению нельзя покрыть(выделен цветом) \\ и делим его на 4 части. В итоге стягиваются в точку.};

\end{tikzpicture}

\end{document}

    \end{center}

    При этом, $\exists a = \displaystyle\bigcap_{i=1}^{\infty}I_i$

    \item $a\in I\Longrightarrow a\in \bigcup A_{\alpha}\Longrightarrow\exists \alpha_0:a\in \underbrace{A_{\alpha_0}}_{\text{открытое}}\Longrightarrow\exists \ve>0: B_{\ve}(a)\subset A_{\alpha_0}$

    \begin{center}
        

\begin{tikzpicture}[
    set/.style={dashed, thick},
    arrow/.style={-{Stealth[scale=1.2]}, thick}
]

% Открытое множество (клякса)
\draw[set] (0,0) to[out=30,in=150] (3,0.5)
           to[out=-30,in=60] (4,-1)
           to[out=-120,in=-30] (2,-2)
           to[out=150,in=-120] (0,-1)
           to[out=60,in=180] cycle;

% Квадрат внутри множества
\draw[thin] (1,-1) rectangle (2,0);
\fill[red!] (1.5, -0.5) circle (0.85pt);
\node at (1.7, -0.3) {$a$};

\draw[set] (1.5, -0.5) circle (25pt);

% Стрелка с подписью
\draw[arrow] (2.75, -0.4) -- (2.4, -0.2);
\node at (3.2, -0.5) {$B_{e} (a)$};

% Стрелка с подписью
\draw[arrow] (3.8,0.5) -- (3.4,0.3);
\node at (4.2, 0.6) {$A_{\alpha_0}$};

\node[align=center] at (2.2,-2.5) {Покрытие $A_{\alpha_0}$};

\end{tikzpicture}


    \end{center}


    \item Из построения получили, что $I \supset I_1 \supset \ldots \supset a \Longrightarrow \exists N:\forall n > N\ I_n\subset B_{\ve}(a)\subset A_{\alpha_0}$

    Получается, что $\forall n>N\ I_n$ покрывается одним лишь $A_{\alpha_0}$ из системы $\{A_{\alpha}\}$

    Получаем противоречие тому, что любое $I_n$ не допускает конечного подпокрытия, а у нас получилось, что $I_n\in A_{\alpha_0}\forall n>N \Longrightarrow I - \text{компакт}$
    \qed
\end{enumerate}

\comment Любое ограниченное множество можно вписать в замкнутый брус. Потому что можно вокруг него описать шарик, который точно можно вписать в брус

\subsection{Критерий компактности в $\mathbb{R}^n$}
\theorem $K\subset \mathbb{R}^n$. $K$ — компакт $\Longleftrightarrow$ $K$ замкнуто и ограниченно

\proof Докажем необходимость $(\Longrightarrow)$
\begin{itemize}
    \item \textit{Ограниченность.} $K$ — компакт $\Longrightarrow \forall \{A_\alpha\}_{\alpha\in\N}$ — можно выделить конечное подпокрытие $\Longrightarrow$

    $\Longrightarrow$ Пусть $\{A_\alpha\}=\{B_n(0)\}_{n=1}^\infty$ $\Longrightarrow \exists N \in \N : \forall n > N ~ K \subset \displaystyle\bigcup_{n=1}^N B_n(0)$ и так как $B_n(0)$ — вложены шары $\Longrightarrow$

    $\Longrightarrow K \subset B_N(0) \Longrightarrow$ по определению $K$ — ограничено

    \begin{center}
       
\begin{tikzpicture}[
    set/.style={dashed, thick},
    arrow/.style={-{Stealth[scale=1.2]}, thick}
]

% Открытое множество (клякса)
\draw[thin] (0,0) to[out=30,in=150] (2,1.0)
           to[out=-30,in=60] (3.5,-1.5)
           to[out=-120,in=-30] (1,-2)
           to[out=120,in=-120] (0.5,-1)
           to[out=60,in=180] cycle;

\fill (-2, -0.5) circle (0.75pt);
\node[right] at (-2.0,-0.3) {$0$};

\begin{scope}
    \clip (-2.4, -2.8) rectangle(4.9, 1.6);

    \draw[set] (-2, -0.5) circle (25pt);
    \draw[set] (-2, -0.5) circle (65pt);
    \draw[set] (-2, -0.5) circle (125pt);
    \draw[set] (-2, -0.5) circle (175pt);

    \node[right] at (-1.4,-1.2) {$B_{1}(0)$};
    \node[right] at (-0.2,-2.0) {$B_{2}(0)$};
    \node[right] at (2.0,-2.6) {$B_{3}(0)$};
    \node[right] at (3.9,-2.2) {$B_{4}(0)$};
\end{scope}

% Стрелка с подписью
\draw[arrow] (3.5,0.8) -- (3.2,0.1);
\node[right] at (3.3,1) {$K$};

\node[align=center] at (1.5,-3.3) {Пример покрытия $K$ вокруг точки $0$ с помощью шаров};

\end{tikzpicture}

    \end{center}


    \item \textit{Замкнутость.} Пойдем от противного. $K$ — компакт, тогда возьмем $\{B_{\frac{\delta(x)}{2}}(0)\}_{x\in K}$ — покрытие открытыми шарами, где $\delta(x)=\rho(x,x_0)$. $x_0$ — предельная точка, которая $\notin K$ (или же $\in \mathbb{R}^n\setminus K$)

    Так как $K$ — компакт, $\exists x_1,\ldots, x_s:K\subset\displaystyle\bigcup_{i=1}^{s} B_{\frac{\delta(x_i)}{2}}(x_i)$

    Пусть $\delta=\min\limits_{1\leqslant i\leqslant s}{\delta(x_i)}$, тогда
    \begin{equation*}
        \begin{aligned}
            B_{\frac{\delta}{2}}(x_0)\cap\bigcup_{i=1}^{s}B_{\frac{\delta(x_i)}{2}}(x_i)=\varnothing&\Longrightarrow B_{\frac{\delta}{2}}(x_0)\subset\mathbb{R}^n\setminus K\\
            &\Longrightarrow\stackrel{\circ}{B}_{\frac{\delta}{2}}(x_0)\cap K=\varnothing
        \end{aligned}
    \end{equation*}

    Значит, $x_0$ \textit{не является предельной точкой} $K$, что противоречит нашему предположению

    \begin{center}
        %\documentclass[tikz,border=3.14mm]{standalone}
%\usepackage[english,russian]{babel} % локализация и переносы
%\usetikzlibrary{arrows.meta}

% \begin{document}
\begin{tikzpicture}[
    set/.style={dashed, thick},
    arrow/.style={-{Stealth[scale=1.2]}, thick}
]

% Открытое множество (клякса)
\draw[thin] (0,0) to[out=30,in=150] (2,1.0)
           to[out=-30,in=60] (3.5,-1.5)
           to[out=-120,in=-30] (1,-2)
           to[out=120,in=-120] (0.5,-1)
           to[out=60,in=180] cycle;

\path (-2, -0.5) node (point0) {};

\fill (point0) circle (1.5pt);
\node[right] at (-2.2,-0.2) {$x_0$};

\draw[set][red!] (point0) circle (35pt);
\node[right] at (-2.8,-2.1) {$B_{\frac{\delta}{2}}(x_0)$};


% Стрелка с подписью
\draw[arrow] (3.5,0.8) -- (3.2,0.1);
\node[right] at (3.3,1) {$K$};

\path (1, 0.5) node (point_x1) {};
\draw[set] (point0) to (point_x1);
\draw[set](point_x1) circle (44pt);
\fill (point_x1) circle (1.5pt);
\node[right] at (point_x1) {$x_1$};
\node[right] at (2,1.8) {$B_{\frac{\delta_1}{2}}(x_1)$};
\fill[blue!] (-0.5, -0) circle (2pt);
\draw[blue!] (-0.65, 0.47) to (-0.34, -0.47);


\path (1.2, -1.8) node (point_x2) {};
\draw[set] (point0) to (point_x2);
\draw[set](point_x2) circle (48pt);
\fill (point_x2) circle (1.5pt);
\node[right] at (point_x2) {$x_2$};
\node[right] at (2.5,-2.8) {$B_{\frac{\delta_2}{2}}(x_2)$};
\fill[blue!] (-0.4, -1.15) circle (2pt);
\draw[blue!] (-0.21, -0.68) to (-0.58, -1.61);

\node[align=center] at (0.4,-4.1) {Пример как мы строим $B_{\frac{\delta}{2}}$ вокруг точки $x_0$. \\ Синие точки - середины отрезков на которых они лежат};

\end{tikzpicture}
%\end{document}

    \end{center}


\end{itemize}

\proof Докажем достаточность

$K$ — замкнуто и ограничено $\Longrightarrow \exists r>0:B_r(0)\supset K\Longrightarrow\exists I$ — замкнутый брус, такой что
$$K\subset I\text{ и }I=[-r;r]^n$$
% тут позже будет картинка

Пусть $\{A_{\alpha}\}_{\alpha\in\N}$ — произвольное покрытие открытыми множествами для $K$. Тогда, $I\subset \{A_{\alpha}\}\cup\underbrace{\{\mathbb{R}^n\setminus K\}}_{\text{открыто}}$. Так как $I$ — компакт, то $\exists $ конечное подпокрытие
$$\{A_{\alpha_i}\}_{i=1}^m\cup\{\mathbb{R}^n\setminus K\}\supset I\supset K\text{ — покрытие для $I$}$$

Значит, $K\subset\{A_{\alpha_i}\}_{i=1}^{m}$ — конечное и $\{A_{\alpha}\}$ — произвольное, тогда $K$ — компакт по определению\qed

\begin{center}
    

\begin{tikzpicture}[
    set/.style={dashed, thick},
    arrow/.style={-{Stealth[scale=1.2]}, thick}
]

% Открытое множество (клякса)
\draw[thin] (0,0) to[out=30,in=150] (0.5,0.3)
           to[out=-30,in=60] (1.,-1.)
           to[out=-120,in=-30] (0.3,-1.2)
           to[out=120,in=-120] (-0.3,-0.2)
           to[out=60,in=180] cycle;

\fill (-0.75, -1.5) circle (0.75pt);
\node[right] at (-0.75,-1.5) {$0$};

\draw[set] (-0.75, -1.5) circle (68pt);

\node at (1.5,0.3) {$B_{1}(0)$};

\draw[thin] (-3.5, -4) rectangle (2, 1);
\draw[arrow] (-0.72,-4.2) -- (1.95,-4.2);
\draw[arrow] (-0.78,-4.2) -- (-3.45,-4.2);
\node at (-2, -4.5) {$r$};
\node at (0.7, -4.5) {$r$};

\draw[arrow] (-3.7,-1.55) -- (-3.7,0.95);
\draw[arrow] (-3.7,-1.65) -- (-3.7,-4.);
\node at (-4, -2.75) {$r$};
\node at (-4, -0.4) {$r$};

\draw[set][red!] (-0.4, -0.4) circle (14pt);
\draw[set][red!] (0.75, -0.6) circle (14pt);
\draw[set][red!] (0.1, 0) circle (14pt);
\draw[set][red!] (0.3, -1.0) circle (14pt);
\draw[set][red!] (1, -1.2) circle (14pt);
\node[red!] at (-1.0, 0.5) {$\{A_{\alpha}\}$};

\node at (0.5,-0.5) {$K$};

\node[align=center] at (-0.5,-5) {Строим замкнутый брус вокруг точки 0, пользуясь \\ существованием конечного покрытия покрываем наш компакт $K$};

\end{tikzpicture}

\end{center}


\subsection{Теорема Вейерштрасса о непрерывной функции на компакте}
\theorem Пусть $K\in \R^n$ — компакт и функция $f: K\mapsto \R$ - непрерывная. Тогда $f$ на $K$ достигает наибольшее и наименьшее значения.

\proof
\begin{itemize}
    \item \textit{Ограниченность.} От противного: пусть существует последовательность $\{x^k\} \subset K \,:\, |f(x^k)| > k$. Из ограниченности $K$ следует ограниченность последовательности $\{x^k\}$, и как следствие ограничены последовательности отдельных коордиант:
    \begin{equation*}
        |x_i^k| = \sqrt{|x_i^k|^2} \leqslant \sqrt{\sum_{i=1}^n|x_i^k|^2} = ||x^k|| \leqslant C \quad \text{для некоторого }C
    \end{equation*}

    По теореме Больцано-Вейерштрасса у $\{x_1^k\}$ существует сходящаяся подпоследовательность $x_1^{k_{j_1}} \to a_1, j_1 \to \infty$. Для последовательности $\{x_2^{k_{j_1}}\}$ существует сходящаяся последовательность $x_2^{k_{j_2}} \to a_2, j_2 \to \infty$. И т.д. Получаем сходящуюся подпоследовательность:
    \begin{equation*}
        x^{k_j} = (x_1^{k_j}, x_2^{k_J}, \ldots, x_n^{k_j})\to(a_1, a_2, \ldots, a_n) = a
    \end{equation*}

    Точка $a$ — предельная для $K$. В силу замкнутости $K$ т. $a\in K$. А из непрерывности функции $f$ получаем $f(x^{k_j}) \to f(a)$. А с другой стороны, $f(x^{k_j})\to\infty$ из выбора исходной последовательности. \textbf{противоречие}

    \item \textit{Достижение наибольшего (наименьшего) значения.} Итак, мы доказали, что $f$ — ограничена на $K$. Выберем последовательность $\{x^k\}$:
    \begin{equation*}
        \sup_K f - \frac{1}{k_j} \le f(x^{k_j}) \le \sup_K f
    \end{equation*}
    в силу непрерывности $f$:
    \begin{equation*}
        \sup_K f \le f(a) \le \sup_K f
    \end{equation*}
    Получаем $f(a) = \displaystyle\sup_K f$, т.е. максимальное значение достиигается в точке $x = a$. Для $\displaystyle\inf_K f$ доказательство аналогично
    \qed
\end{itemize}

\subsection{Теорема о связи непрерывности функции в точке с колебанием}
\theorem Пусть $x_0 \in M \subset \R^n; \,\, f: M\mapsto\R$. $f$ — непрерывна в точке $x_0 \iff \omega(f, x_0) = 0$

\proof
\begin{itemize}
    \item \textit{Необходимость}\\

        $f$ — непрерывна в т. $x_0 \in M \implies \forall \ve > 0 \,\, \exists\delta > 0:\,\, \forall x \in B_{\delta}(x_0)\cap M = B_{\delta}^M(x_0) \implies |f(x) - f(x_0)| < \frac{\ve}{3}$\\
        Рассмотрим $\omega(f, x_0) := \displaystyle\lim_{\delta\to0+} \omega(f, B_{\delta}^M(x_0))$:
        \begin{equation*}
            \begin{aligned}
                \omega(f, B_{\delta}^M(x_0)) = \sup_{x, y \in B_{\delta}(x_0)}|f(x) - f(y)| \le \sup_{x\in B_{\delta}(x_0)} |f(x) - f(x_0)| + \sup_{y\in B_{\delta}(x_0)} |f(y) - f(x_0)| \le \frac{2\ve}{3} < \ve
            \end{aligned}
        \end{equation*}
        При $\ve\to0 \implies \delta\to0$ и $\omega(f, B_{\delta}^M(x_0))\to0$, т.е. $\omega(f, x_0) = 0$

    \item \textit{Достаточность}

        Пусть $0 = \omega(f, x_0) := \displaystyle\lim_{\delta\to0+} \omega(f, B_{\delta}^M(x_0))$, т.е.
        \begin{equation*}
            \forall \ve > 0 \,\, \exists\delta > 0: \quad \forall x, y \in B_{\delta}^M(x_0) \quad \sup_{x, y\in B_{\delta}^M(x_0)} |f(x)-f(y)| < \ve
        \end{equation*}
        Получаем, что
        \begin{equation*}
            \forall \ve > 0 \,\, \exists \delta > 0: \forall x \in B_{\delta}^M(x_0) \implies |f(x)-f(x_0)| < \ve \implies
        \end{equation*}
\end{itemize}\qed

\subsection{Свойства интегральных сумм Дарбу}
\subsubsection{Нижняя сумма Дарбу не больше верхней}
\theorem \begin{equation*}
    \us(f, \T) = \int_{\xi}\sigma(f, \T, \xi) \le \sup_{\xi}\sigma(f, \T, \xi) = \os(f, \T)
\end{equation*}

\proof
\begin{equation*}
    \begin{aligned}
    &\us(f, \T) = \sum_{i=1}^{K}m_i|I_i| = \sum_{i}\inf_{\xi_i}(f(\xi_i))|I_i| = \inf_{\xi}\sum_{i}f(\xi_i)|I_i| = \inf_{\xi}\sigma(f, \T, \xi) \le\\
    &\sup_{\xi}\sigma(f, \T, \xi) = \sum_i (f(\xi_i))|I_i| = \sum_{i}M_i|I_i| = \os(f, \T)
    \end{aligned}
    \end{equation*}

\subsubsection{Монотонность сумм относительно измельчений разбиения}
\theorem Пусть $\tilde{\T}$ — измельчение разбиения $\T$, тогда
\begin{equation*}
    \us(f, \T) \le \us(f, \tilde{\T}) \le \os(f, \tilde{\T}) \le \os(f, \T)
\end{equation*}\qed

\proof Если $L \subset M$, то $\inf L \ge \inf M$ и $\sup L \le \sup M$, тогда:
\begin{equation*}
    \us(f, \T) \le \us(f, \tilde{\T}) \underset{\text{по 6.2}}{\le} \os (f, \tilde{\T}) \le \os (f, \T)
\end{equation*}\qed

\subsubsection{Никакая нижняя сумма Дарбу не больше какой-либо верхней суммы на том же брусе}
\theorem $\forall \T_1, \T_2 : \quad \us(f, \T_1) \le \os(f, \T_2)$

\proof $\forall \T_1, \T_2$ рассмотрим $\tilde{\T} = \T_1 \cap \T_2$, тогда по 6.3:
\begin{equation*}
    \us(f, \T_1) \le \us(f, \tilde{\T}) \le \os(f, \tilde{\T}) \le \os(f, \T_2)
\end{equation*} \qed


\subsection{Теорема об интегралах Дарбу как пределах интегральных сумм Дарбу}
\theorem Пусть $I\subset \R^n$ — замкнутый брус, а $f: I \mapsto \R$ — ограничена. Тогда:
\begin{equation*}
    \oi = \lim_{\Delta_{\T}\to0}\os(f, \T) \qquad \text{и} \qquad \ui = \lim_{\Delta_{\T} \to 0} \us(f, \T)
\end{equation*}

\proof Докажем, что $\ui = \displaystyle\lim_{\Delta_{\T} \to 0} \us(f, \T) \quad (= \displaystyle\sup_{\T} \us (f, \T))$
\begin{enumerate}
    \item $f$-ограничена на $I \implies \exists C > 0: \forall x \in I\quad |f(x)| \leqslant C$
    \item т.к. по определению $\underline{I} = \displaystyle\sup_{\T}\us(f, \T)$, то $\forall \ve > 0 \,\, \exists\T_1 = \{I_i^1\}_{i=1}^{m_1}:\ \ui-\ve < \us(f, \T_1) \leqslant \ui < \ui + \ve$
    \item Пусть $G = \displaystyle\bigcup_{i=1}^{m_1}\partial I_i^1$ - объединение границ брусов $I^1_i \in \T_1$ (без повторов). Тогда $G$ множество меры нуль по Лебегу (т.к. границы --- мн-ва меры нуль по Лебегу)
    \item Пусть $\T_2$ - произвольное разбиение $I: \,\, \T_2 = \{I_i^2\}_{i=1}^{m_2}$ \\
    Рассмотрим два множества брусов:\\
    \begin{equation*}
    \begin{aligned}
        A = \{I_i^2 \in \T_2: I_i^2 \cap G \ne \varnothing\} \qquad \text{и} \qquad B = \T_2\backslash A \implies\\
        \forall \ve > 0 \,\, \exists \delta(\ve) > 0 : \forall \T_2: \Delta_{\T_2} < \delta \text{ верно, что } \sum_{I_i^2 \in A} |I_i^2| < \epsilon
    \end{aligned}
    \end{equation*}
    т.к. наши брусочки $I^2_i$ по построению лежат в $G$, а по 3 пункту оно множество меры нуль.

    \begin{center}
        % \documentclass[a4paper,10pt]{article}
% %\documentclass[a4paper,10pt]{scrartcl}
%
% \usepackage[utf8]{inputenc}         % кодировка исходного текста
% \usepackage[english,russian]{babel} % локализация и переносы
% \usepackage{tikz}
% \usetikzlibrary{arrows.meta, patterns, patterns.meta}
%
% \begin{document}


\begin{tikzpicture}[scale=2, every node/.style={inner sep=0,outer sep=0},
    set/.style={dashed, thick},
    arrow/.style={-{Stealth[scale=1.2]}, thick}]
    \begin{scope}
        \draw[thick] (0, 0) rectangle (1, 1);
        \draw[thick][blue] (0.5, 0) -- (0.5, 1);
        \draw[thick][blue] (0, 0.5) -- (1, 0.5);
        \node at (0.53, -0.3) {разбиение $T_1$};
    \end{scope}




    \begin{scope}[xshift=2cm]
        \draw[thick] (0, 0) rectangle (1, 1);
        \draw[thick] (0.5, 0) -- (0.5, 1);
        \draw[thick] (0, 0.5) -- (1, 0.5);

        \fill[pattern = {Lines[angle = -45, line width = 1pt, distance = 5pt]},
    pattern color = blue,
    even odd rule]
            (0, 0) rectangle (0.5, 0.5)
            (0.05, 0.05) rectangle (0.45, 0.45)

            (0.5, 0) rectangle (1, 0.5)
            (0.55, 0.05) rectangle (0.95, 0.45)

            (0.5, 0.5) rectangle (1, 1)
            (0.55, 0.55) rectangle (0.95, 0.95)

            (0, 0.5) rectangle (0.5, 1)
            (0.05, 0.55) rectangle (0.45, 0.95);
        \draw[set][blue]
            (0.05, 0.05) rectangle (0.45, 0.45)
            (0.55, 0.55) rectangle (0.95, 0.95)
            (0.05, 0.55) rectangle (0.45, 0.95)
            (0.55, 0.05) rectangle (0.95, 0.45);

        \node at (0.53, -0.3) {Граница $G$ бруса $T_1$};
    \end{scope}



    \begin{scope}[xshift=4cm]
        \draw[thick] (0, 0) rectangle (1, 1);
        \draw[thick][red] (0.1, 0) -- (0.1, 1);
        \draw[thick][red] (0.25, 0) -- (0.25, 1);
        \draw[thick][red] (0.4, 0) -- (0.4, 1);
        \draw[thick][red] (0.6, 0) -- (0.6, 1);
        \draw[thick][red] (0.9, 0) -- (0.9, 1);
        \draw[thick][red] (0, 0.4) -- (1, 0.4);
        \draw[thick][red] (0, 0.6) -- (1, 0.6);
        \draw[thick][red] (0, 0.1) -- (1, 0.1);
        \draw[thick][red] (0, 0.9) -- (1, 0.9);
        \node at (0.53, -0.3) {Какое-то разбиение $T_2$};
    \end{scope}

    \begin{scope}[xshift=0.2cm, yshift=-2cm]
        \draw[thick] (0, 0) rectangle (1, 1);
        \draw[thick] (0.5, 0) -- (0.5, 1);
        \draw[thick] (0, 0.5) -- (1, 0.5);

        \fill[pattern = {Lines[angle = -45, line width = 1pt, distance = 5pt]},
    pattern color = blue,
    opacity=0.3,
    even odd rule]
            (0, 0) rectangle (0.5, 0.5)
            (0.05, 0.05) rectangle (0.45, 0.45)

            (0.5, 0) rectangle (1, 0.5)
            (0.55, 0.05) rectangle (0.95, 0.45)

            (0.5, 0.5) rectangle (1, 1)
            (0.55, 0.55) rectangle (0.95, 0.95)

            (0, 0.5) rectangle (0.5, 1)
            (0.05, 0.55) rectangle (0.45, 0.95);
        \draw[set][blue][opacity=0.3]
            (0.05, 0.05) rectangle (0.45, 0.45)
            (0.55, 0.55) rectangle (0.95, 0.95)
            (0.05, 0.55) rectangle (0.45, 0.95)
            (0.55, 0.05) rectangle (0.95, 0.45);



        \draw[thick] (0, 0) rectangle (1, 1);
        \draw[thick][red][opacity=0.5]
            (0.1, 0) -- (0.1, 1)
            (0.25, 0) -- (0.25, 1)
            (0.4, 0) -- (0.4, 1)
            (0.6, 0) -- (0.6, 1)
            (0.9, 0) -- (0.9, 1)
            (0, 0.4) -- (1, 0.4)
            (0, 0.6) -- (1, 0.6)
            (0, 0.1) -- (1, 0.1)
            (0, 0.9) -- (1, 0.9);
        \node[align=center] at (0.53, -0.3) {Как прошлые разбиения и граница $G$ \\ выглядят на одном рисунке};
    \end{scope}



    \begin{scope}[xshift=3.8cm, yshift=-2cm]
        \draw[thick] (0, 0) rectangle (1, 1);

        \fill[pattern = {Lines[angle = -45, line width = 1pt, distance = 5pt]},
    pattern color = red,
    opacity=0.3,
    even odd rule]
            (0, 0) rectangle (0.5, 0.5)
            (0.05, 0.05) rectangle (0.45, 0.45)

            (0.5, 0) rectangle (1, 0.5)
            (0.55, 0.05) rectangle (0.95, 0.45)

            (0.5, 0.5) rectangle (1, 1)
            (0.55, 0.55) rectangle (0.95, 0.95)

            (0, 0.5) rectangle (0.5, 1)
            (0.05, 0.55) rectangle (0.45, 0.95);
        \draw[set][red][opacity=0.7]
            (0.05, 0.05) rectangle (0.45, 0.45)
            (0.55, 0.55) rectangle (0.95, 0.95)
            (0.05, 0.55) rectangle (0.45, 0.95)
            (0.55, 0.05) rectangle (0.95, 0.45);

        \draw[arrow][red] (-0.3, 0.5) -- (0.05, 0.51);
        \node at (-0.4, 0.5) {$A$};



        \fill[pattern = {Lines[angle = 55, line width = 1pt, distance = 5pt]},
    opacity=0.5]
            (0.05, 0.05) rectangle (0.45, 0.45)

%             (0.5, 0) rectangle (1, 0.5)
            (0.55, 0.05) rectangle (0.95, 0.45)

%             (0.5, 0.5) rectangle (1, 1)
            (0.55, 0.55) rectangle (0.95, 0.95)

%             (0, 0.5) rectangle (0.5, 1)
            (0.05, 0.55) rectangle (0.45, 0.95);


        \draw[arrow] (1.3, 0.5) -- (0.8, 0.7);
        \node at (1.4, 0.5) {$B$};

        \node[align=center] at (0.53, -0.3) {Как выглядят множества $A$ и $B$};
    \end{scope}

\end{tikzpicture}
%
%
% \end{document}

    \end{center}


    \item С другой стороны $\forall I_i^2 \in B$ верно, что $I_i^2 \in \T_1 \cap \T_2$
\end{enumerate}

Хотим рассмотреть
\begin{equation*}
\begin{aligned}
    |\ui - \us(f, \T_2)| = |I-\us(f, \T_1\cap \T_2) + \us(f, \T_1\cap \T_2) -\us(f, \T_2)| &\leqslant \underbrace{|I-\us(f, \T_1\cap \T_2)|}_* + \underbrace{|\us(f, \T_1\cap \T_2) -\us(f, \T_2)|}_{**} \\
    &< \ve + 2 C \ve = \ve(1 + 2C)
\end{aligned}
\end{equation*}\qed

\begin{itemize}
    \item[*] из пункта 2: $\ui - \ve < \us(f, \T_1) \leqslant \us(f, \T_1\cap \T_2) \leqslant \ui < \ui + \ve \implies |\ui - \us(f, \T_1\cap\T_2)| < \ve$\\
    \item[**] Пояснение ниже \begin{equation*}
        \begin{aligned}
            \left|\us(f, \T_1\cap\T_2) - \us(f, \T_2)\right| &= \left|\sum_{I_i^2\in B}m_i|I_i^2| + \sum_{I_i\in \T_1\cap A}m_i|I_i^2| - \sum_{I_i^2\in B}m_i|I_i^2| - \sum_{I_i^2\in A}m_i|I_i^2|\right| ~~~ \textit{\text{Переход с равном по пункту 5}} \\
            &\leqslant \left|\sum_{I_i\in \T_1\cap A}m_i|I_i^2|\right| + \left|\sum_{I_i^2\in A}m_i|I_i^2|\right|\\
            &\leqslant 2\left|\sum_{I_i^2\in A}m_i|I_i^2|\right| ~~~ \textit{\text{Следующий переход по пункту 1}} \\
            &\leqslant 2C\left|\sum_{I_i^2\in A}|I_i^2|\right| ~~~ \textit{\text{Следующий переход по пункту 4}} \\
            & < 2C\ve
        \end{aligned}
    \end{equation*}
\end{itemize}


\subsection{Критерий Дарбу интегрируемости функции на замкнутом брусе}
$I\in\mathbb{R}^n\text{ — замкнутый брус, } f:I\mapsto \mathbb{R}, f\in \mathcal{R}(I)\Longleftrightarrow f$ — ограничена на $I$ и $\ui=\oi$

\proof Необходимость
\begin{itemize}
    \item $f\in\riman{I}\Longrightarrow$ по необходимому условию интегрируемости функции по Риману на замкнутом брусе, $f$ — ограничена на $I$
    \item Покажем, что $\underline{\mathcal{I}}=\mathcal{I},\overline{\mathcal{I}}=\mathcal{I}\Longrightarrow\underline{\mathcal{I}}=\overline{\mathcal{I}}$

    \begin{enumerate}
        \item $f\in\riman{I}\Longrightarrow\forall \ve >0\ \exists \delta >0\ \forall(\mathbb{T},\xi):\Delta_{\mathbb{T}}<\delta \ |\sigma(f,\mathbb{T},\xi)-\mathcal{I}|<\ve$
        \item $\ui=\sup\limits_{\mathbb{T}}\us(f,\T)=\lim\limits_{\Delta\rightarrow 0}\us(f,\T) \Longrightarrow |\ui-\us|<\ve$

        $\forall \ve>0\ \exists\delta\ \exists\mathbb{T}:\Delta_{\mathbb{T}}<\delta: |\ui-\us|<\ve$
        \item $\us(\mathbb{T},\xi)=\inf\limits_{\xi}\sigma(f,\mathbb{T},\xi)$

        $\forall\mathbb{T},\ \forall \ve>0\ \exists \xi:|\us-\sigma|<\ve$
    \end{enumerate}
\end{itemize}

$|\mathcal{I}-\ui|\leqslant|\mathcal{I}-{\ui}-\sigma+\sigma+{\us}-\us|\leqslant|\mathcal{I}-\sigma|+|\ui-\us|+|\sigma-\us|<3\ve$\qed

\proof Достаточность

$f$ — ограничена и $\ui=\oi$. Имеем
\begin{equation*}
    \us(f,\mathcal{T})=\inf\limits_{\xi}\leqslant\sigma(f,\mathbb{T},\xi)\leqslant \sup\limits_{\xi}(f,\mathbb{T},\xi)=\os(f,\mathbb{T})
\end{equation*}

Тогда, при $\lim\limits_{\Delta_{\T}\rightarrow 0} \us=\ui,\ \lim\limits_{\Delta_{\T}\rightarrow0}\os=\oi$ получаем $\ui=\oi$(Условие ограниченнсоти $f$ даёт нам возможность применять неравенство выше)\qed


\subsection{Утверждение о независимости определения допустимого множества от выбора бруса}
 Пусть $D \subset I_1 \subset \R^n, D \subset I_2 \subset \R^n$ - замкнутые брусы, тогда
\begin{equation*}
    \int\limits_{I_1} f\cdot\chi_D\d{x}\text{ и }\int\limits_{I_2}f\cdot\chi_{D}\d{x}
\end{equation*}

либо существуют и равны, либо оба не существуют вообще


\begin{center}
    
\begin{tikzpicture}[scale=1.75]

    \draw (0, 0) rectangle (3, 2);
    \node at (0.4, 0.4) { $I_1$ };

    \draw (1.5, 1) rectangle (5, 2.5);
    \node at (4.5, 2.1) { $I_2$ };


    \draw[thin] (2, 1.8) to[out=0,in=110] (2.8,1.5)
           to[out=-50, in=-30] (2.4,1.2)
           to[out=110, in=-70] (1.8,1.1)
           to[out=110, in=-180] cycle;
    \node at (1.67, 1.8) { $I$ };
    \node at (2.3, 1.5) { $D$ };

    \node[align=center] at (2.3, -0.3) {Как выглядят наши множества $I_1, I_2, I, D$};

\end{tikzpicture}

\end{center}



\proof Введем $I = I_1 \cap I_2 \supset D$, $I$ не пустое по построению. Покажем существование
\begin{itemize}
    \item $f\cdot\chi_D\in\riman{I_1}\Longrightarrow$ по критерию Лебега $f\cdot\chi_D$ ограничена на $I_1\Longrightarrow$ $f\cdot\chi_D$ ограничена на $D\Longrightarrow f$ ограничена на $D\Longrightarrow f\cdot\chi_D$ ограничена на $I_2$
    \item $f\cdot\chi_D\in\riman{I_1}\Longrightarrow$ по критерию Лебега $f\cdot\chi_D$ непрерывна почти всюду на $I_1\Longrightarrow f\cdot\chi_D$ непрерынва почти всюду на $D\Longrightarrow $ в худшем случае для $f\cdot\chi_D$ на $I_2$ добавятся разрывы на $\partial D\Longrightarrow f\cdot\chi_D$ непрерынва почти всюду на $I_2$
    \item Тогда, $f\cdot\chi_D\in\riman{I_1}\Longleftrightarrow f\cdot\chi_D\in\riman{I_2}$
\end{itemize}

Покажем равенство
\begin{itemize}
    \item Пусть $\mathbb{T}_i$ — разбиение на $I_i:\mathbb{T}_1$ и $\mathbb{T}_2$ совпадают на $I$
    \item Пусть $\xi^i$ — отмеченные точки для $\T_i$
    \item $\sigma(f\chi_D,\mathbb{T}_1,\xi^1)=\sum_{j}f\chi_D(\xi^1_j)|I_j^1|=\sum_j f(\xi^1_j)|I^1_j|=\sum_j f(\xi^2_j)|I^2_j|=\sum_j f\chi_D(\xi_j^2)|I_j^2|=\sigma(f\chi_D, \mathbb{T}_2, \xi^2)$
\end{itemize}\qed

\comment Все свойства интеграла Римана и критерия Лебега для бруса справедливы и для других допустимых множеств
 
\subsection{Теорема Фубини о переходе к повторному интегралу}
Пусть имеются $I_x\subset\mathbb{R}^n, I_y\subset\mathbb{R}^m, I_x\times I_y\subset \mathbb{R}^{m+n}$ — замкнутые брусы, $f:I_x\times I_y\rightarrow \mathbb{R}$, $f\in\riman{I_x\times I_y}$ и $\forall$ фиксированного $x\in I_x \implies f(x,y)\in\riman{I_y}\Longrightarrow$
\begin{equation*}
    \int\limits_{I_x\times I_y} f(\overline{x}, \overline{y})\d{\overline{x}}\d{\overline{y}}=\int\limits_{I_x}\left(\int\limits_{I_y}f(\overline{x},\overline{y})\d{\overline{y}}\right)\d{\overline{x}}=\int\limits_{I_x}\d{\overline{x}}\int\limits_{I_y}f(\overline{x}, \overline{y})\d{\overline{y}}
\end{equation*}

\comment аналагочино, если взять для $\forall$ фиксированного $y\in I_y$

\proof Воспользуемся тем, что $f\in\riman{I_x\times I_y}, \ f\in\riman{I_y}$, а также Критерием Дарбу
\begin{itemize}
    \item $\mathbb{T}_x=\{I_i^x\}$ — разбиение на $I_x$, $\mathbb{T}_y=\{I_j^y\}$ — разбиение на $I_y$, $\mathbb{T}_{x,y}=\{I_i^x\times I^y_j\}=\{I_{ij}\}$ — разбиение на $I_x\times I_y$, и при этом верно $|I_i^x| \cdot |I_j^y| = |I_{ij}|$
    \item \begin{equation*}
        \begin{aligned}
            \us(f,\mathbb{T}_{x,y})=\sum_{i,j} \inf\limits_{(x,y)\in I_{ij}} f(x,y)|I_{ij}|& \underset{\text{рис. ниже}}\leqslant \sum_{i,j} \inf\limits_{x\in I_i^x} \left(\inf\limits_{y\in I_j^y} f(x,y) \cdot |I_j^y|\right)|I_i^x|=\sum_i \inf\limits_{I^x_i} \underbrace{\left(\sum_j \inf\limits_{I^y_j} f(x,y)|I_j^y|\right)}_{\us(f(y), \mathbb{T}_y)}|I_i^x|\\
            &\leqslant \sum_i \inf\limits_{I^x_i}\underbrace{\left(\int\limits_{I_y} f(x,y)\d{y}\right)}_{g(x)}|I_i^x| \leqslant \us(g(x),\mathbb{T}_x)\\
            % &\leqslant \os(g(x),\mathbb{T}_x)\leqslant\ldots\leqslant \os(f,\mathbb{T}_{x,y})
            &\leqslant \os(g(x),\mathbb{T}_x)
        \end{aligned}
    \end{equation*}

    $\us(f,\mathbb{T}_{x,y})\leqslant \us(g(x),\mathbb{T}_x)\leqslant\os(g(x), \mathbb{T}_x)\leqslant\os(f,\mathbb{T}_{x,y})\Longrightarrow\exists \oi=\lim\limits_{\delta\rightarrow0} \us(g(x),\mathbb{T}_x) = \int\limits_{I_x\times I_y} f(\overline{x}, \overline{y})\d{\overline{x}}\d{\overline{y}}$

    \comment Последний знак неравенства, получен аналогичными действиями для длинного неравенства выше, просто развернув в обратную сторону знаки неравенства для $\sup$
\end{itemize}\qed



\subsection{Cупремальный критерий равномерной сходимости функциональной последовательности}
\theorem $f_n\overset{D}{\rightrightarrows} f\Longleftrightarrow \lim\limits_{n\to\infty}\left(\sup\limits_{D} \left|f_n(x)-f(x)\right|\right)=0$

\proof Докажем необходимость $(\Longrightarrow)$

Заметим, что $\sup\limits_{D} \left|f_n(x)-f(x)\right|\geqslant 0$. Тогда,
\begin{equation*}
    \forall\ve >0\ \exists N:\ \forall n>N,\forall x\in D\hookrightarrow \sup\limits_D |f_n(x)-f(x)|<\ve
\end{equation*}

$f_n\overset{D}{\rightrightarrows} f\Longrightarrow \forall\ve>0\ \exists N:\forall n>N,\ \forall x\in D\hookrightarrow|f_n(x)-f(x)|<\frac{\ve}{2}$

В худшем случае, $\sup\limits_{D} |f_n(x)-f(x)|\leqslant \frac{\ve}{2}<\ve$

\proof Докажем достаточность $(\Longleftarrow)$

$\forall \ve>0\ \exists N:\forall n>N\hookrightarrow\sup\limits_{D}|f_n(x)-f(x)|<\ve$, тем более $\forall x\in D\ \sup \geqslant |f_n(x)-f(x)|$ 

Тогда, $f_n\overset{D}{\rightrightarrows} f$\qed 

\comment $f\rightrightarrows f\Longrightarrow f_n\longrightarrow f$, но в обратную сторону это не работает


\subsection{Критерий Коши равномерной сходимости функциональной последовательности}
\theorem $f_n(x)\overset{D}{\rightrightarrows} f(x)\Longleftrightarrow\forall\ve>0\ \exists N:\ \forall n,m>N,\ \forall x\in D\hookrightarrow |f_n(x)-f_m(x)|<\ve$

\proof $\Longrightarrow$ Докажем необходимость

Так как $f_n(x)\overset{D}{\rightrightarrows} f(x)$, то
\begin{equation*}
    \forall\ve>0\ \exists N:\forall n>N\ \forall x\in D\hookrightarrow |f_n(x)-f(x)|<\frac{\ve}{2}
\end{equation*}

Рассмотрим $|f_n(x)-f_m(x)|\leqslant |f_n(x)-f(x)|+|f(x)-f_m(x)|<\frac{\ve}{2}+\frac{\ve}{2}=\ve$

Таким образом, мы показали, что $\forall\ve>0\ \exists N:\ \forall n,m>N,\ \forall x\in D\hookrightarrow |f_n(x)-f_m(x)|<\ve$\qed

\proof $\Longleftarrow$ Докажем достаточность

Распишем определение равномерной сходимости:
\begin{equation*}
    \forall\ve>0\ \exists N:\ \forall n,m>N,\ \exists x\in D:\ |f_n(x)-f_m(x)|<\frac{\ve}{2}
\end{equation*}

Зафиксируем $x_0\in D\Longrightarrow\exists\lim\limits_{n\to\infty} f_n(x_0)=f(x_0)$\footnote[1]{по критерию Коши для числовой последовательности $f_n(x_0)$}
\begin{equation*}
    x_0\in D:\forall\ve>0\exists N:\forall n,m>N: |f_n(x_0)-f_m(x_0)|<\frac{\ve}{2}
\end{equation*}

В худшем случае, $\forall x\in D:\text{при $m\to\infty$ } |f_n(x)-f(x)|\leqslant\frac{\ve}{2}<\ve$

Тогда,
\begin{equation*}
    \forall\ve>0\ \exists N:\ \forall n>N\ \forall x\in D\hookrightarrow|f_n(x)-f(x)|<\ve
\end{equation*}\qed

\comment Отрицание Критерия Коши: 
\begin{equation*}
    f_n(x)\not\overset{D}{\rightrightarrows} f(x)\Longleftrightarrow\exists\ve_0>0\ \forall N:\ \exists n,m>N,\ \exists x_0\in D\ |f_n(x)-f_m(x)|\geqslant\ve_0
\end{equation*}


\subsection{Теорема о почленном переходе к пределу для функциональной последовательности}
\theorem Пусть $f_n,f: D\longrightarrow\mathbb{R},\ x_0\text{ — предельная точка } D,\ f_n\overset{D}{\rightrightarrows} f,\ \forall n\in\mathbb{N}\ \exists\lim\limits_{x\to x_0} f_n(x)=c_n$

Тогда,
\begin{equation*}
    \begin{aligned}
        &\exists\lim\limits_{n\to\infty} c_n=\lim\limits_{x\to x_0} f(x)\\
        &\left(\text{или }\lim\limits_{n\to\infty} \left(\lim_{x\to x_0} f_n(x)\right)=\lim_{x\to x_0}\left(\lim_{n\to\infty} f_n(x)\right)\right)
    \end{aligned}
\end{equation*}

\proof Сначала покажем, что $\exists\lim\limits_{n\to\infty }c_n=c$, а потом что $\exists c=\lim\limits_{n\to\infty }c_n$
\begin{enumerate}
    \item Рассмотрим $|c_n-c|\leqslant\underbrace{|c_n-f_n|}_{(a)}+\underbrace{|f_n-f_m|}_{(b)}+|\underbrace{f_m-c_m|}_{(c)}<\displaystyle\frac{\ve}{3}+\frac{\ve}{3}+\frac{\ve}{3}=\ve$
    \begin{enumerate}
        \item[(a), (c)] По условию, $\forall n\in\mathbb{N}\ \exists \lim\limits_{x\to x_0} f_n(x)=c_n$ получим 
        \begin{equation*}
            \forall\ve>0\ \exists\delta>0:\forall x\in \overset{\circ}{B_{\delta}}(x_0)\cap D\hookrightarrow|f_n(x)-c_n|<\frac{\ve}{3}
        \end{equation*}
        \item[(b)] $f_n\overset{D}{\rightrightarrows} f\Longrightarrow$ по Критерию Коши 
        \begin{equation*}
            \forall\ve>0\ \exists N:\forall n,m>N\ \forall x\in D\hookrightarrow|f_n(x)-f_m(x)|<\frac{\ve}{3}
        \end{equation*}
        Получаем, что $\forall x\in \overset{\circ}{B_{\delta}}(x_0)$
    \end{enumerate}
    Собираем: $\forall\ve>0\ \exists N:\forall n,m>N:\forall x\in \overset{\circ}{B_{\delta}}(x_0):|c_n-c_m|<\ve\Longrightarrow\exists c=\lim\limits_{n\to\infty} c_n$\qed

    \item Теперь покажем, что $\exists\lim\limits_{x\to x_0}f(x)=c$, то есть $\forall\ve>0\exists\delta:\forall x\in \overset{\circ}{B_{\delta}}(x_0): |f(x)-c|<\ve$

    Рассмотрим $|f(x)-c|\leqslant\underbrace{|f(x)-f_n(x)|}_{(a)}+\underbrace{|f_n(x)-c_n|}_{(b)}+\underbrace{|c_n-c|}_{(c)}$
    \begin{enumerate}
        \item $f_n\overset{D}{\rightrightarrows} f(x)\Longrightarrow\forall\ve>0\exists N_1:\forall n>N_1\forall x\in D:|f_n(x)-f(x)|<\frac{\ve}{3}$
        \item $\forall n\in\mathbb{N}\ \exists\lim\limits_{x\to x_0}f_n(x)=c_n\Longrightarrow \forall \ve>0\ \exists\delta:\forall x\in \overset{\circ}{B_{\delta}}(x_0)\hookrightarrow|f_n(x)-c_n|<\frac{\ve}{3}$
        \item По доказанному в п. 1 следует, что
        \begin{equation*}
            \exists\lim_{n\to\infty} c_n=c\Longrightarrow \forall\ve>0\ \exists N_2\ \forall n>N_2\hookrightarrow|c_n-c|<\frac{\ve}{3}
        \end{equation*}
    \end{enumerate}

    Собираем: $\forall\ve>0\ (\exists N=\max(N_1,N_2))\ \exists\delta>0:\ \forall x\in\overset{\circ}{B_{\delta}}(x_0):|f(x)-c|<\ve$\qed
\end{enumerate}


\subsection{Теорема о непрерывности предельной функции}
\theorem Пусть имеется $\left.\begin{aligned}
    &f_n,f: D\longrightarrow\mathbb{R},\\
    &f_n\overset{D}{\rightrightarrows} f,\\
    &\forall n\in\mathbb{N}\ f_n\in C(D)
\end{aligned}\right\}\Longrightarrow f\in C(D)$

\proof Нужно доказать, что $f\in C(D)$. Значит, надо показать, что 
\begin{equation*}
    \forall x_0\in D: \forall\ve>0\ \exists\delta>0:\forall x\in B_{\delta}(x_0)\cap D\hookrightarrow|f(x)-f(x_0)|
\end{equation*}

Рассмотрим $|f(x)-f(x_0)|\leqslant\underbrace{|f(x)-f_n(x)|}_{(1)}+\underbrace{|f_n(x)-f_n(x_0)|}_{(2)}+\underbrace{|f_n(x_0)-f(x_0)|}_{(3)}<\frac{\ve}{3}+\frac{\ve}{3}+\frac{\ve}{3}=\ve$
\begin{enumerate}
    \item $f_n\overset{D}{\rightrightarrows} f:\forall\ve>0\ \exists N:\forall n>N,\ \forall x\in D\hookrightarrow|f_n(x)-f(x)|<\frac{\ve}{3}$
    \item Так как $\forall n\in\mathbb{N}\ f_n\in C(D)\Longrightarrow\forall x_0\in D,\ \forall\ve>0\ \exists\delta>0:\forall x\in B_{\delta}(x_0)\cap D\hookrightarrow|f_n(x)-f_n(x_0)|<\frac{\ve}{3}$
    \item $f_n\overset{D}{\rightrightarrows} f:\forall\ve>0\ \exists N:\forall n>N,\ \forall x_0\in D\hookrightarrow|f_n(x_0)-f(x_0)|<\frac{\ve}{3}$
\end{enumerate}

Тогда, собрав три части, получим, что $\forall x_0\in D$
\begin{equation*}
    \begin{aligned}
        \forall\ve>0\ \exists \delta>0:(\exists N:\forall n>N)\ \forall x\in B_{\delta}(x_0)\cap D\hookrightarrow|f(x)-f(x_0)|<\ve&\Longrightarrow f(x)\in C(x_0)\ \forall x_0\in D\\
        &\Longrightarrow f(x)\in C(D)
    \end{aligned}
\end{equation*}\qed


\subsection{Утверждение о неравномерной сходимости фун. послед. наличии разрыва}
\theorem Пусть имеется $\left.\begin{aligned}
    &f_n\in C\left([a;b)\right),\\
    &f\in C((a;b))+\text{разрыв в т.}a,\\
    &f_n\overset{[a;b)}{\longrightarrow} f
\end{aligned}\right\}\Longrightarrow f_n\overset{(a;b)}{\not\rightrightarrows} f$

То есть будет поточечная сходимость, но не будет равномерной:
\begin{equation*}
    f_n\overset{(a;b)}{\longrightarrow}f,\text{ но не }f_n\overset{(a;b)}{\rightrightarrows} f
\end{equation*}

\proof От противного

\begin{enumerate}
    \item Пусть $f_n\overset{(a;b)}{\rightrightarrows} f\Longrightarrow \forall\ve>0\ \exists N:\forall n>N\ \forall x\in [a;b)\hookrightarrow|f_n(x)-f(x)|<\ve$
    \item $f_n\overset{[a;b)}{\longrightarrow} f \Longrightarrow f_n(a)\longrightarrow f(a)\Longrightarrow\forall\ve>0\ \exists N_2:\forall n>N_2\hookrightarrow|f_n(a)-f(a)|<\ve$
    \item $f_n\overset{[a;b)}{\rightrightarrows} f$, так как $\forall\ve>0\ \exists N=\max(N_1,N_2)\ \forall n>N,\ \forall x\in [a;b) \hookrightarrow|f_n(x)-f(x)|<\ve$
    \item Получаем, что 
    \begin{equation*}
        \begin{cases}
            f_n\overset{[a;b)}{\rightrightarrows} f\\
            f_n\in C([a;b))
        \end{cases}
    \end{equation*}
    Тогда, по теореме о непрерывности предельной функции следует, что $f\in C([a;b))$, но $f$ имеет разрыв в точке $a$. Противоречие
\end{enumerate}\qed

\subsection{Утверждение о неравномерной сходимости фун. послед. при наличии расходимости в точке}
\theorem Пусть имеется $\left.\begin{aligned}
    &f_n\in C\left([a;b)\right)\\
    &f_n\overset{(a;b)}{\longrightarrow} f\\
    &\not\exists \lim\limits_{n\to\infty}f_n(a)
\end{aligned}\right\}\Longrightarrow f_n\overset{(a;b)}{\not\rightrightarrows} f$

\proof От противного

\begin{enumerate}
    \item Пусть $f_n\overset{(a;b)}{\rightrightarrows} f\Longrightarrow \forall\ve>0\ \exists N:\forall n,m>N\ \forall x\in (a;b)\hookrightarrow|f_n(x)-f_m(x)|<\frac{\ve}{3}$
    \item $f_n\in C([a;b))$, тогда
    \begin{equation*}
        \forall x_0\in[a;b):\ \forall\ve>0\ \exists\delta>0:\ \forall x\in B_{\delta}(x_0)\cap[a;b)\hookrightarrow|f_n(x)-f_n(x_0)|<\frac{\ve}{3}
    \end{equation*}
    В частности, это верно для $x_0=a$:
    \begin{equation*}
        \forall\ve>0\ \exists \delta>0:\forall x\in \overset{\circ}{B_{\delta}}(a)\cap(a;b)\footnote[2]{верно $\forall x\in B_{\delta}(a)\cap[a;b)$, а потому $a$ выколота}\hookrightarrow|f_n(x)-f_n(a)|<\frac{\ve}{3}
    \end{equation*}
    \item Рассмотрим
    \begin{equation*}
        |f_n(a)-f_m(a)|\leqslant \underbrace{|f_n(a)-f_n(x)|}_{\text{по п.2}}+\underbrace{|f_n(x)-f_m(x)|}_{\text{по п.1}}+\underbrace{|f_m(x)-f_m(a)|}_{\text{по п.2}}<\frac{\ve}{3}+\frac{\ve}{3}+\frac{\ve}{3}
    \end{equation*}
    получаем, что
    \begin{equation*}
        \forall\ve>0\ \exists N(\exists\delta>0):\ \forall n,m>N(\forall x\in \overset{\circ}{B_{\delta}}(a)\cap(a;b))\hookrightarrow|f_n(a)-f_m(a)|<\ve
    \end{equation*}
    то есть, по Критерию Коши для числовой последовательности $\exists\lim\limits_{n\to\infty}f_n(a)$, что противоречит условию, а значит $f_n\overset{(a;b)}{\not\rightrightarrows} f$\qed
\end{enumerate}

\subsection{Теорема о почленном интегрировании функциональной последовательности}
\theorem Пусть имеется $\left.\begin{aligned}
    &f_n,f:[a;b]\to\mathbb{R}\\
    &f_n\overset{[a;b]}{\rightrightarrows}f\\
    &f_n\in\riman{[a;b]}\forall n\in\mathbb{N}
\end{aligned}\right\}\Longrightarrow f\in\riman{[a;b]}\text{ и }\lim\limits_{n\to\infty}\int\limits_{a}^b f_n(x)\d{x}=\int\limits_{a}^b f(x)\d{x}$

\proof По Критерию Дарбу $f\in\riman{[a;b]}\Longleftrightarrow f$ — ограничена на $[a;b]$ и $\ui=\oi$

\begin{itemize}
    \item Покажем \textit{ограниченность}
    \begin{enumerate}
        \item $\forall n\in\mathbb{N}:\ f_n\in\riman{[a;b]}\Longrightarrow f_n$ ограничена на $[a;b]$ и 
        \begin{equation*}
            \forall n\in\mathbb{N}\ \exists M_n\geqslant 0\ \forall x\in[a;b]\hookrightarrow|f_n(x)|\leqslant M_n
        \end{equation*}
        \item $f_n\overset{[a;b]}{\rightrightarrows}f$, тогда $\forall\ve>0\ \exists N:\ \forall n>N\ \forall x\in[a;b]\hookrightarrow|f_n(x)-f(x)|<\ve$
        
        Рассмотрим $\ve=1$, тогда $\exists N_1=N:\ \forall x\in[a;b]\hookrightarrow|f_{N_1+1}(x)-f(x)|<1$

        Тогда, для $f(x)$ верно $\forall x\in[a;b]$
        \begin{equation*}
            |f(x)|\leqslant|f(x)-f_{N_1+1}(x)|+|f_{N_1+1}(x)|<1+M_{N_1+1},
        \end{equation*}
        то есть $f(x)$ — ограничена 
    \end{enumerate}

    \item Покажем \textit{интегрируемость} 
    
    Напомним, что $\oi = \lim\limits_{\Delta_{\T}\to0}\os(f, \T) \text{ и } \ui = \lim\limits_{\Delta_{\T} \to 0} \us(f, \T)$

    Рассмотрим $\T$ — разбиение $[a;b]$
    \begin{equation*}
        |\us(f,\T)-\os(f,\T)|\leqslant\underbrace{|\us(f,\T)-\us(f_n,\T)|}_{(1)}+\underbrace{|\us(f_n,\T)-\os(f_n,\T)|}_{(2)}+\underbrace{|\os(f_n,\T)-\os(f,\T)|}_{(3)}
    \end{equation*}
    \begin{enumerate}
        \item[(1)] Распишем в виде неравенств
        \begin{equation*}
            |\us(f,\T)-\us(f_n,\T)|\leqslant\sum_{i}|\inf\limits_{I_i}(f)-\inf\limits_{I_i}(f_n)||I_i|\leqslant \sum_{i}\sup\limits_{I_i}|f-f_n|\cdot|I_i|\leqslant\sup\limits_{[a;b]}|f-f_n|\cdot|b-a|<\frac{\ve}{3}
        \end{equation*}
        Так как $f_n\overset{[a;b]}{\rightrightarrows}f$, то по супремальному критерию:
        \begin{equation*}
            \forall\ve>0\ \exists N:\ \forall n>N\hookrightarrow\sup\limits_{[a;b]}|f-f_n|<\frac{\ve}{3|b-a|}
        \end{equation*}
        \item[(2)] $f_n\in\riman{[a;b]}\Longrightarrow$
        \begin{equation*}
            \forall\ve>0\ \exists\delta>0:\ \forall\T:\ \Delta_{\T}<\delta\ |\us(f_n,\T)-\os(f_n,\T)|<\frac{\ve}{3}
        \end{equation*}
        \item[(3)] Аналогично (1): $|\os(f_n,\T)-\os(f,\T)|\leqslant\sup\limits_{[a;b]}|f-f_n|<\frac{\ve}{3}$
    \end{enumerate}
    Получаем, что
    \begin{equation*}
        \forall\ve>0\ \exists\delta>0\ (\exists N)\ \forall \T:\ \Delta_{\T}<\delta\ (\forall n>N)\hookrightarrow|\us(f,\T)-\os(f,\T)|<\frac{\ve}{3}+\frac{\ve}{3}+\frac{\ve}{3}=\ve
    \end{equation*}
    $\Longrightarrow f(x)\in\riman{[a;b]}$

    \item Покажем, что $\lim\limits_{n\to\infty}\displaystyle\int\limits_{a}^b f_n(x)\d{x}=\int\limits_{a}^b f(x)\d{x}$
    
    Рассмотрим
    \begin{equation*}
        \left|\int\limits_{a}^b f_n(x)\d{x}-\int\limits_{a}^b f(x)\d{x}\right|\leqslant\int\limits_a^b|f_n(x)-f(x)|\d{x}\leqslant\sup\limits_{[a;b]}|f_n(x)-f(x)|\cdot|b-a|<\ve
    \end{equation*}

    Так как $f_n\overset{[a;b]}{\rightrightarrows}f$, то $\forall\ve>0\ \exists N:\ \forall n>N\ \sup\limits_{[a;b]}|f_n(x)-f(x)|<\displaystyle\frac{\ve}{|b-a|}$ и получаем, что 
    \begin{equation*}
        \forall \ve>0\ \exists N:\ \forall n>N\hookrightarrow \left|\int\limits_{a}^b f_n(x)\d{x}-\int\limits_{a}^b f(x)\d{x}\right|<\ve
    \end{equation*}
\end{itemize}\qed

\subsection{Теорема о почленном дифференцировании функциональной последовательности}
\theorem Пусть имеется $\left.\begin{aligned}
    &f_n,f,g:[a;b]\to\mathbb{R}\\
    &f_n\in D([a;b])\\
    &\exists c\in[a;b]:\exists\lim\limits_{n\to\infty} f_n(c)\\
    &\exists g(x):\ f^{\prime}_n\overset{[a;b]}{\rightrightarrows}g(x)
\end{aligned}\right\}\Longrightarrow \begin{aligned}
    &\exists f:\ f_n\overset{[a;b]}{\rightrightarrows}f\\
    &\oplus f^{\prime}(x)=g(x)
\end{aligned}$

\proof Покажем \textit{существование}
    
\theorem (Лагранжа) $f\in C([a,b]),\ f\in D((a, b))\Longrightarrow\exists c\in(a,b):\ f(b)-f(a)=f'(c)(b-a)$
\begin{enumerate}
    \item Рассмотрим $\varphi(x)=f_n(x)-f_m(x)$
    \item $\forall n\in\mathbb{N}\ f_n\in D([a;b])\Longrightarrow f_n\in C([a;b])\Longrightarrow \varphi(x)\in D([a;b])$ и $\varphi(x)\in C([a;b])$
    \item Рассмотрим: для $c$ из условия теоремы Лагранжа $$\varphi(x)-\varphi(c)=\varphi^{\prime}(\xi)\cdot(x-c),\text{ где }\xi\in[c;x]\ ([x;c])$$
    
    Тогда, $\varphi(x)=\varphi^{\prime}(\xi)(c-x)+\varphi(x)$ %уточнить не надо ли тут \varphi^{\prime}(\xi)(x-c)
    \item Оценим $|\varphi(x)|\leqslant |\varphi^{\prime}(\xi)|\cdot|c-x|+|\varphi(c)|=\underbrace{|f^{\prime}_n(\xi)-f^{\prime}_m(\xi)|}_{\star}\cdot|c-x|+\underbrace{|f_n(c)-f_m(c)|}_{\star\star}$
    
    $\star\ f^{\prime}_n\overset{[a;b]}{\rightrightarrows}g(x)\Longrightarrow\forall\ve>0\ \exists N_1:\ \forall n,m>N_1\ \forall x\in[a;b]\hookrightarrow|f^{\prime}_n(\xi)-f^{\prime}_m(\xi)|<\displaystyle\frac{\ve}{2|b-a|}$\\
    $\star\star\ \exists\lim\limits_{n\to\infty}f_n(c)\Longrightarrow\forall\ve>0\ \exists N_2:\ \forall n,m>N_2\hookrightarrow|f_n(c)-f_m(c)|<\displaystyle\frac{\ve}{2}$

    Тогда,
    \begin{equation*}
        |\varphi(x)|\leqslant |\varphi^{\prime}(\xi)|\cdot|c-x|+|\varphi(c)|=\underbrace{|f^{\prime}_n(\xi)-f^{\prime}_m(\xi)|}_{\star}\cdot|c-x|+\underbrace{|f_n(c)-f_m(c)|}_{\star\star}<\frac{\ve}{2|b-a|}\cdot|c-x|+\frac{\ve}{2}<\ve
    \end{equation*}
    то есть
    \begin{equation*}
        \forall\ve>0\ \exists N=\max\{N_1,N_2\}:\ \forall n,m >N\ \forall x\in[a;b]\hookrightarrow|\varphi(x)|=|f_n(x)-f_m(x)|<\ve\Longrightarrow\exists f: f_n\overset{[a;b]}{\rightrightarrows}f
    \end{equation*}
\end{enumerate}

\proof Покажем, что $f^{\prime}(x)=g(x)$

Пусть имеется $x_0\in[a;b]$, но он произвольный
\begin{enumerate}
    \item Рассмотрим $\psi_n(x)=\displaystyle\frac{f_n(x)-f_n(x_0)}{x-x_0}$
    
    Покажем по Критерию Коши, что $\psi_n(x)\overset{[a;b]}{\rightrightarrows}$
    \begin{equation*}
        \begin{aligned}
            |\psi_n(x)-\psi_m(x)|&=\left|\frac{f_n(x)-f_n(x_0)-f_m(x)+f_m(x_0)}{x-x_0}\right|\\
            &=\left|\frac{(f_n(x)-f_m(x))-(f_n(x_0)-f_m(x_0))}{x-x_0}\right|\\
            &=\left|\frac{\varphi(x)-\varphi(x_0)}{x-x_0}\right|\\
            &\exists\xi\in[x_0, x]\\
            &=\frac{|\varphi'(\xi)||x-x_0|}{|x-x_0|}\\
            &=|\varphi'(\xi)|\\
            &=|f_n'(\xi)-f'_m(\xi)|<\ve
        \end{aligned}
    \end{equation*}
    так как $f_n\overset{[a;b]}{\rightrightarrows}$, то есть 
    \begin{equation*}
        \forall\ve>0,\exists N,\forall n,m>N,\forall x\in[a,b]\hookrightarrow|f_n'(x)-f'_m(x)|<\ve
    \end{equation*}
    то $\psi\overset{[a;b]}{\rightrightarrows}$

    \item $\forall n\in\mathbb{N},\exists\lim\limits_{x\to x_0}\psi_n(x)=\lim\limits_{x\to x_0}\displaystyle\frac{f_n(x)-f_n(x_0)}{x-x_0}=f_n'(x_0)$, так как $f_n\in D([a,b])$
    
    Получаем, что $\psi_n(x)\overset{[a, b]}{\rightrightarrows}$ и $\forall n\in\mathbb{N},\exists\lim\limits_{x\to x_0}\psi_n(x)=f'_n(x_0)$, тогда по теореме о почленном переходе к пределу
    \begin{equation*}
        \begin{aligned}
            g(x_0)&=\lim_{n\to\infty}f'_n(x_0)\\
            &=\lim_{n\to\infty}\lim_{x\to x_0}\psi_n(x)\\
            &=\lim_{n\to\infty}\lim_{x\to x_0}\left(\frac{f_n(x)-f_n(x_0)}{x-x_0}\right)\\
            &=\lim_{x\to x_0}\lim_{n\to\infty}\left(\frac{f_n(x)-f_n(x_0)}{x-x_0}\right)\\
            &=\lim_{x\to x_0}\frac{f(x)-f(x_0)}{x-x_0}\\
            &=f'(x_0)
        \end{aligned}
    \end{equation*}
\end{enumerate}\qed


\subsection{Сравнительный признак равномерной сходимости функционального ряда}
\theorem Имеется $\left.\begin{aligned}
    &\sum_{n=1}^{\infty}a_n(x)\text{ и }\sum_{n=1}^{\infty}b_n(x):\\
    &\exists N\ \forall n>N\ \forall x\in D\ |a_n(x)|\leqslant b_n(x)\\
    &\sum_{n=1}^{\infty}b_n(x)\overset{D}{\rightrightarrows}
\end{aligned}\right\}\Longrightarrow \sum_{n=1}^{\infty}a_n(x)\overset{D}{\rightrightarrows}\text{ и }\sum_{n=1}^{\infty}a_n(x)\text{ сходится абсолютно }D$

\proof Докажем по критерию Коши
\begin{equation*}
    (\star)\quad |a_{m+1}(x)+\ldots+a_k(x)|\leqslant |a_{m+1}(x)|+\ldots+|a_k(x)|\underset{\forall m,k>N\forall x\in D}{\leqslant}b_{m+1}(x)+\ldots+b_{k}(x)\underset{\forall m, k>N_1 \forall x\in D}{<}\ve 
\end{equation*}
Получаем, что 
\begin{equation*}
    \forall\ve>0\ \exists \overset{\sim}{N}=\max\{N,N_1\}: \forall k>m>\overset{\sim}{N}\ \forall x\in D\hookrightarrow|a_{m+1}(x)+\ldots+a_k(x)|<\ve\Longrightarrow\sum_{n=1}^{\infty}a_n(x)\overset{D}{\rightrightarrows}
\end{equation*}

Так как $(\star)$ выполняется для любого $x\in D$, то $\forall x_0\in D$ выполняется 
\begin{equation*}
    \forall\ve>0\ \exists \overset{\sim}{N}=\max\{N,N_1\}: \forall k>m>\overset{\sim}{N}\hookrightarrow |a_{m+1}(x)|+\ldots+|a_k(x)|<\ve,
\end{equation*}
то есть $\sum_{n=1}^{\infty}|a_n(x_0)|$ — сходится, а значит сходится абсолютно $\forall x_0\in D$\qed

\subsection{Мажорантный признак Вейерштрасса о равномерной сходимости функционального ряда}
\corollary $\left.\begin{aligned}
    &\sum_{n=1}^{\infty}a_n(x):\\
    &\exists N\ \forall n>N\ \sup\limits_{D}|a_n(x)|\leqslant M_n\\
    &\sum_{n=1}^{\infty} M_n\text{ — сходится}
\end{aligned}\right\}\Longrightarrow \begin{aligned}
    &\sum_{n=1}^{\infty}a_n\overset{D}{\rightrightarrows}\\
    &\sum_{n=1}^{\infty}a_n\text{ сходится абсолютно на }D
\end{aligned}$

\proof Если в признаке сравнения принять, что $\forall n\in \mathbb{N}\ b_n(x)=M_n=\text{const}(n)$, то условие теоремы выполняется\qed 

% светлая памяти части спешл фор оли
% \subsection{[FOR KIARA] Критерий Лебега интегрируемости функции по Риману}
% \theorem Если $I\subset \R^n$ — замкнутый невырожденный брус, $f: I\to\R$, то $f\in R(I) \iff f$ ограничена и непрерывна почти всюду на $I$

% \proof 
% \begin{itemize}
%     \item \textit{Необходимость}
    
%      Если $f$ интегрируема, то она ограничена по необходимому условию интегрируемости. Осталось показать, что множества разрыва меры нуль. От противного: пусть это не так.

%     Обозначим множество всех точек разрыва ф-ии $f$ на $I$ за $T$ и заметим, что $T = \displaystyle\bigcup_{k\in\mathbb{N}}T_k$, где\\
%     $T_k = \{x\in I | \omega(f, x) \ge \frac{1}{k}\}$. Если $T$ не меры нуль, то существует $T_{k_0}$ не меры нуль (если они все меры нуль, то по свойству множеств меры нуль счетное объединение таких множеств тоже было бы меры нуль).

%     Для произвольного разбиения $\T = \{I_i\}_{i=1}^m$ бруска $I$ разобъем эти бруски на две кучи: первая $A = \{I_i | I_i\cap T_{k_0} \ne \varnothing, \omega(f, I_i) \ge \frac{1}{2k_0}\}$ и вторая $B = \T\backslash A$. Покажем что $A$ является покрытием множества $T_{k_0}$, т.е. $T_{k_0} \subset \displaystyle\bigcup_{i: I_i\in A} I_i$ любая точка $x\in T_{k_0}$ является либо
%     \begin{itemize}
%         \item[a)] внутренней для некоторого бруска $I_i$. В этом случае $\omega(f, I_i) \ge \omega(f, x) \ge \frac{1}{k_0} > \frac{1}{2k_0}$, т.е. $I_i \in A$, либо
%         \item[b)] точка $x$ лежит на границе некоторого количества брусков (не более чем $2^n$ штук). Тогда хотя бы на одном из них колебание $\omega(f, I_i) \ge \frac{1}{2k_0}$ (т.е. $I_i \in A$): если бы такого не нашлось, то в любой малой окрестности $B_{\ve}(x)$ выполняется следующее:
%         \begin{equation*}
%             \omega(f, x) \le \sup_{x', x''\in B_{\ve}(x)} |f(x')-f(x'')| \le \sup_{x'\in B_{\ve}(x)}|f(x')-f(x)| + \sup_{x''\in B_{\ve}(x)}|f(x)-f(x'')| < \frac{1}{2k_0} + \frac{1}{2k_0} = \frac{1}{k_0}
%         \end{equation*}
%         т.е. $x\not\in T_{k_0}$ --- \textbf{противоречие}.
%     \end{itemize}

%     Таким образом, каждая точка $x\in T_{k_0}$ покрывается некоторым бруском $I_i \in A$, т.е. $A$ - покрытие $T_{k_0}$. Тогда существует $c: \displaystyle\sum_{i:I_i\in A}|I_i| \ge c > 0$ для всех разбиений $\T$ (если бы меняя разбиения мы могли получить сумму объемов этих брусков сколь угодно маленькую, то получилось бы, что $T_{k_0}$ меры нуль)

%     Возьмем два набора отмеченных точек $\xi^1$ и $\xi^2$. На брусках из кучки $B$ будем их брать одинаковыми, т.е. для $I_i\in B \,\, \xi_i^1 = \xi_i^2$. А на брусках из кучки $A$ будем брать такие, чтобы 
%     \begin{equation*}
%         f(\xi_i^1) - f(\xi_i^2) \ge \frac{1}{3k_0} \text{ (у нас там колебания} \ge 1/2k_0, \text{ так что такие найдутся)}
%     \end{equation*}

%     Получаем:
%     \begin{equation*}
%         \begin{aligned}
%             |\sigma(f, \T, \xi^1) - \sigma(f, \T, \xi^2) &= \left|\sum_i(f(\xi_i^1) - f(\xi_i^2))|I_i|\right|\\
%             &= \left|\sum_{i: I_i\in A}(f(\xi_i^1) - f(\xi_i^2))|I_i| + \sum_{i:I_i\in B}(f(\xi_i^1) - f(\xi_i^2))|I_i|\right|\\
%             &= \left|\sum_{i: I_i\in A} (f(\xi_i^1) - f(\xi_i^2))|I_i|\right| \ge \frac{1}{3k_0} \sum_{i:I_i\in A}|I_i| \ge \frac{c}{3k_0} > 0
%         \end{aligned}
%     \end{equation*}
%     т.е. интегральные суммы не могут стремиться к одному и тому же числу, значит $f$ не интегрируема --- \textbf{противоречие}.

%     \item \textit{Достаточность}

%     Для любого $\ve > 0$ рассмотрим $T_{\ve} = \{x\in I| \omega(f, x) \ge \ve\}$. Покажем, что это множество - компакт. Ограниченность очевидна (подмножества бруска), а замкнутость проверим от противного. Пусть $a$ - предельная точка $T_{\ve}: \,\, a\not\in T_{\ve}$. Т.к. она предельная, то существует $\{x^k\}: x^k \in B_{\frac{1}{k}}(a)$. Т.к. $B_{\frac{1}{k}}$ - открытые шары, то наши точки лежат в них с окрестностями, т.е. сущесвтуют $\delta_k : B_{\delta_k}(x_K) \subset B_{\frac{1}{k}}(a)$. Тогда
%     \begin{equation*}
%         \omega(f, B_{\frac{1}{k}}(a)) \ge \omega(f, B_{\delta_k}(x_K)) \ge \omega(f, x_k) \ge \ve
%     \end{equation*}
%     Переходя к пределу $k\to\infty : \omega(f, a) \ge \ve$, т.е. $a\in T_{\ve}$ - противоречие. Значит $T_{\ve}$ - замкнуто, и, следовательно, компактно.

%     Множество $T_{\ve}$ - множество меры нуль (как подмножество множества меры нуль). Значит, его можно покрыть не более чем счетным объединением открытых брусков $I_i: \displaystyle\sum_i|I_i| < \ve$. Т.к. это открытое покрытие, а $T_{\ve}$ - компакт, то существует конечное подпокрытие: $T_{\ve} \subset \displaystyle\bigcup_{i=1}^m I_i$, при этом $\displaystyle\sum_{i=1}^m |I_i| < \ve$.

%     Обозначим три множества: $C_1 = \displaystyle\bigcup_{i=1}^mI_i, \quad C_2 = \displaystyle\bigcup_{i=1}^mI_i', C_3 = \displaystyle\bigcup_{i=1}^mI_i''$, где $I_i', I_i''$ - бруски, полученные гомотетией с центром в центре $I_i$ с коэффициентом 2 и 3 соответственно.

%     Заметим, что
%     \begin{itemize}
%         \item[a)] $|C_3| \le \displaystyle\sum_{i=1}^m|I_i''|| = 3^n \displaystyle\sum_{i=1}^m|I_i| < 3^n \ve$
%         \item[b)] расстояние $\rho(\partial C_2, \partial C_3) = \delta_1 > 0$ (теорема про расстояние между компактами)
%         \item[c)] Множество $K = I\backslash(C_2\backslash \partial C_2)$ - компакт. Кстати, любое множество с диаметром меньше $\delta_1$ либо польностью лежит в $C_3$, либо полностью в $K$.
%         \item[d)] $T_{\ve} \cap K = \varnothing$, т.к. $T_{\ve} \subset C_1 \subset C_2$. Следовательно, $\forall x\in K \,\, \omega(f, x) < \ve$. Тогда по теореме Кантора-Гейне $\exists \delta_2 > 0: \,\, \forall x\in K \,\, \omega(f, B_{\delta_2}(x)) < \ve + \ve = 2\ve$
%     \end{itemize}

%     Выберем $\delta = \min\{\delta_1, \delta_2\}$. Тогда для любых разбиений $\T_1 = \{I_k^1\}, \T_2 = \{I_i^2\}: \lambda{\T_1} < \delta, \lambda(\T_2) < \delta$

%     Рассмотрим пересечение этих разбиений $\T = \T_1 \cap \T_2$, т.е. такое разбиение $\T = \{I_{ik}\}$, что $I_k^1 = I_{i_1k} \bigsqcup\ldots\bigsqcup I_{i_mk}$ и $I_i^2 = I_{ik_1} \bigsqcup \ldots\bigsqcup I_{ik_l}$. Очевидно $\lambda(\T) < \delta$.

%     Для произвольных наборов отмеченных точек:
%     \begin{equation*}
%         |\sigma(f, \T_1, \xi^1) - \sigma(f, \T_2, \xi^2)| \le |\sigma(f, \T_1, \xi^1) - \sigma(f, \T, \xi)| + |\sigma(f, \T_2, \xi^2) - \sigma(f, \T, \xi)|
%     \end{equation*}

%     Рассмотрим отдельное слагаемое:
%     \begin{equation*}
%         \begin{aligned}
%             |\sigma(f, \T_1, \xi^1) - \sigma(f, \T, \xi)| = \left|\sum_{i, j}(f(\xi_i^1) - f(\xi_{ij}))|I_{ij}\right|\
%             \le \sum_{I_{ij}\in C_3}|f(\xi_i^1) - f(\xi_{ij})||I_{ij}| + \sum_{I_{ij\in K}}|f(\xi_i^1)-f(\xi_{ij})||I_{ij}|\le 2M\cdot e^n\ve + 2\ve |I| = \epsilon(2M\cdot 3^n + 2|I|)
%         \end{aligned}
%     \end{equation*}
%     т.к. $f$ ограничена некоторой константой $M$ и см пункты $a), d)$, то

%     Т.к. для $(\T_2, \xi^2)$ все выкладки аналогичные, то получаем:
    
%     \begin{equation*}
%          |\sigma(f, \T_1, \xi^1) - \sigma(f, \T, \xi)| \le \epsilon(2M\cdot 3^n + 2|I|)
%     \end{equation*}

%         Следовательно, существует предел $\displaystyle\lim_{\lambda(\T)\to0}\sigma(f, \T, \xi)$ (Критерий Коши для функций)
% \end{itemize} \qed













\end{document}