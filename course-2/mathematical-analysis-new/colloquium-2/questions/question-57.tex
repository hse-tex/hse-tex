% Здесь НЕ НУЖНО делать begin document, включать какие-то пакеты..
% Все уже подрубается в головном файле
% Хедер обыкновенный хсе-теха, все его команды будут здесь работать
% Пожалуйста, проверяйте корректность теха перед пушем

% Здесь формулировка билета
\subsection{Что представляет собой гладкая $k$-мерная поверхность при $k = 1$, $k = 2$, $k = m - 1$}

\begin{description}
    \item[$k = 1 \implies$] поверхность является кривой;
    \item[$k = 2 \implies$] на лекции не было, но это просто поверхность второго порядка (параболоиды, эллипсоиды, etc);
    \item[$k = m - 1 \implies $] такая поверхность называется гиперповерхностью.
\end{description}

