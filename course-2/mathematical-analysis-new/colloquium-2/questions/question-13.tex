% Здесь НЕ НУЖНО делать begin document, включать какие-то пакеты..
% Все уже подрубается в головном файле
% Хедер обыкновенный хсе-теха, все его команды будут здесь работать
% Пожалуйста, проверяйте корректность теха перед пушем

% Здесь формулировка билета
\subsection{Что такое диаметр разбиения? Покажите, что при измельчении разбиения диаметр не увеличивается.} 

\textbf{\underline{Опр.:} } Пусть $\tau$ - некоторое разбиение, тогда \textit{диаметром} разбиения называют
\[\Delta(\tau) = \max_i{\sup_{x,y\in D_i}{|x - y|}}\]
\textbf{\underline{Утв.:} } При измельчении разбиения его диаметр не увеличивается.\\
\textbf{\underline{Док-во:} } Пусть $\tau$ и $\tau'$ - некоторые разбиения, причем $\tau \leq \tau'$ \\
Тогда пусть $D_i' \in \tau'$ - некоторое множество. По определению измельчения 
\[D_i' = D_{i_1}\sqcup ... \sqcup D_{i_k}\]
где $D_{i_1}, ..., D_{i_k} \in \tau$. Очевидно, что так как $\forall j \ D_{i_j} \in D_i'$, то диаметр $D_{i_j}$ не превосходит диаметр $D_i'$. Данное утверждение верно для любого $i$, а значит, что при измельчении разбиения диаметр не увеличивается.
\begin{flushright}
$\blacksquare$
\end{flushright}

