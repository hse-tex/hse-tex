\ProvidesFile{quest-01.tex}[Билет 1]

\section{Билет 1}

\begin{center}
    \it Теорема Пуассона.
    Задача про булочки с изюмом.
\end{center}

\sectionbreak
\subsection{Теорема Пуассона}

Рассмотрим серию испытаний по схеме Бернулли, причем пусть $N$-я серия состоит из $N$ испытаний и вероятность успеха в этой серии равна $p_N$.
Потребуем, чтобы произведение $N\cdot p_N=\lambda$ не зависело от $N$.
Нас интересует вероятность $P(S_N=k)$ наступления ровно $k$ успехов в $N$-ой серии.

\begin{theorem*}
    Пусть $N \cdot p_N = \lambda$ --- не зависит от $N$.
    Тогда
    \[
        P(S_N = k) = C_{N}^k p_N^k (1 - p_N)^{N - k} \to \frac{\lambda^k}{k!} e^{-\lambda}, \quad N \to +\infty.
    \]
\end{theorem*}

\begin{proof}
    Пусть $N\cdot p_N=\lambda$ --- не зависит от $N$. Тогда
    \[
        P(S_N = k) = C_{N}^k p_N^k (1 - p_N)^{N - k}.
    \]
    Распишем вероятность $P(S_N = k)$ в следующем виде:
    \[
        P(S_N = k) = \frac{N!}{k!(N - k)!} \cdot \frac{\lambda^{k}}{N^{k}} \cdot \left( 1 - \frac{\lambda}{N} \right)^{N - k} = \frac{\lambda^k}{k!} \cdot \frac{N \cdot (N - 1) \cdot \ldots \cdot (N - k + 1)}{N^{k}} \cdot \left(1 - \frac{\lambda}{N}\right)^{-k} \cdot \left(1 - \frac{\lambda}{N}\right)^{N}.
    \]
    Заметим следующие вещи:
    \begin{itemize}
        \item $\frac{N \cdot (N - 1) \cdot \ldots \cdot (N - k + 1)}{N^{k}} = \frac{N^{k} + o(N^{k})}{N^{k}} \underset{N \to +\infty}{\to} 1$;
        \item $\left(1 - \frac{\lambda}{N}\right)^{-k} \underset{N \to +\infty}{\to} 1$;
        \item $\left(1 - \frac{\lambda}{N}\right)^{N} \underset{N \to +\infty}{\to} e^{-\lambda}$.
    \end{itemize}
    Учитывая, что $\lambda$ и $k$ не меняются, устремляем $N \to \infty$ и получаем
    \[
        P(S_N = k) \underset{N \to +\infty}{\to} e^{-\lambda} \frac{\lambda^{k}}{k!}.
    \]
\end{proof}

\sectionbreak
\subsection{Задача про булочки с изюмом}

\paragraph*{Формулировка}
Какое в среднем количество изюма должны содержать булочки, для того чтобы
вероятность иметь хотя бы одну изюминку в булочке была не меньше $0.99$?

\paragraph*{Решение}
Предположим, что уже изготовлено тесто на некоторое количество булочек.
В это тесто добавлено $N$ изюминок так, что отношение числа изюминок к количеству булочек равно $\lambda$.
Значит количество булочек $b = \frac{N}{\lambda}$.

Выделим в тесте кусок, из которого будет изготовлена данная булочка.
Вероятность попадания одной изюминки в эту
булочку равна $\frac{1}{b} = \frac{\lambda}{N}$, а вероятность того, что хотя бы одна изюминка попала в булку, равна $1 - P(\text{булочка без изюма})$ и равна
\[
    1 - \Bigl( 1 - \frac{\lambda}{N} \Bigr)^{N}.
\]
Поскольку мы рассматриваем серийное производство булочек, то можно предполагать, что $N \to +\infty$, т. е. растет объем теста и количество изюма, но не меняется плотность $\lambda$.
Как и выше, получаем $\Bigl( 1 - \frac{\lambda}{N} \Bigr)^{N} \to e^{-\lambda}$.
Для решения задачи надо найти $\lambda$ такое, что $e^{-\lambda} < 0.01$.
Подходит $\lambda = 5$, т. е. плотность изюма должна быть не менее пяти изюминок на булочку.