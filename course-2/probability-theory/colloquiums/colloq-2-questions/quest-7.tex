\ProvidesFile{quest-07.tex}[Билет 7]

\section{Билет 7}

\begin{center}
    \it Дискретные и абсолютно непрерывные распределения случайных величин.
    Функция распределения дискретной случайной величины.
    Определение абсолютно непрерывного распределения случайной величины и определение плотности распределения.
    Основные свойства плотности и связь с функцией распределения.
    Примеры абсолютно непрерывных случайных величин.

    \textcolor{purple}{Еще будут пояснения для таких как Аня Смирнова чтобы осознать.}
\end{center}

\sectionbreak
\subsection{Функция распределения дискретной случайной величины.}

\begin{definition}
    Случайная величина $\xi$ называется {\it дискретной}, если множество ее значений конечно или счетно.
    Если $x_1, \ldots, x_N, \ldots$ --- различные значения $\xi$, то множества $A_i = \xi^{-1} \{ x_i \}$ попарно не пересекаются.
    Пусть $p_i = P(A_i)$.
    Тогда распределение $\mu_{\xi}$ имеет вид
    \[
        \mu_{\xi} = p_1 \delta_{x_i} + \ldots + p_N \delta_{x_N} + \ldots
    \]
    и полностью определяется значениями $x_i$ и $p_i$.
    В этой формуле $\delta_{x_i}(A) := 1$, если $x_i \in A$ и $\delta_{x_i}(A) = 0$, если $x_i \notin A$ для каждого $A\in \mathcal{B}(\mathbb{R})$.
\end{definition}

\textcolor{purple}{\it Вот это (мю кси) $\mu_{\xi}$ --- вероятностная мера, а $\mathcal{B}(\mathbb{R})$ --- борелевская сигма-алгебра.}

\sectionbreak
\subsection{Определение абсолютно непрерывного распределения случайной величины и определение плотности распределения}

\begin{definition}
    Говорят, что случайная величина $X$ имеет {\it абсолютно непрерывное} распределение (или является абсолютно непрерывной), если существует такая неотрицательная (и интегрируемая) функция $\rho_X$, что
    \[
        F_X(t) = \int_{-\infty}^t \rho_X(x) \dd x,
    \]
    Функция $\rho_X$ называется {\it плотностью} случайной величины $X$.
\end{definition}

\textcolor{purple}{\it $F_X(t)$ --- это функция распределения случ. величины, и она абсолютно непрерывна, если ее можно задать какой-то функцией $\rho_X$ и бахнуть интеграл, а так обычно определение другое.}

\paragraph{Факты}
Отметим, что в данном случае
\[
    \mu_X((a, b]) = F_X(b) - F_X(a) = \int_a^b \rho_X(x) \dd x,
\]
кроме того
\[
    P(X = a) = \lim\limits_{n \to \infty}[F_X(a) - F_X(a - 1/n)] = 0 \text{ (непрерывность интеграла с переменным пределом).}
\]
На самом деле можно доказать, что
\[
    \mu_X(A) = \int_A \rho_X(x) \dd x
\]
для всякого множества $A$, для которого имеет смысл интеграл в правой части, т. е. функция $I_A \rho_X$ интегрируема по Риману, где $I_A(x) = 1$ при $x \in A$ и $I_A(x) = 0$ при $x \notin A$.

\sectionbreak
\subsection{Основные свойства плотности и связь с функцией распределения.}

\begin{proposal*}
    Отметим несколько свойств плотности распределения:
    \begin{enumerate}
        \item $\rho_X \geqslant 0$;
        \item $\displaystyle\int\limits_{-\infty}^{+\infty} \rho_X(x) \dd x = 1$;
        \item $F_X^{\prime}(x) = \rho_X(x)$ для любой точки непрерывности функции $\rho_X$.
    \end{enumerate}
\end{proposal*}

Последнее свойство следует из теоремы о дифференцируемости интеграла с переменным верхнем пределом.

\textcolor{purple}{\it Конечно же мы ее не помним: Пусть функция  интегрируема на $[a, b]$ и непрерывна в точке $x_{0} \in [a, b].$
Тогда функция $F$ дифференцируема в точке $x_0$ и $F^{\prime}(x_{0}) = f(x_{0})$.}

\sectionbreak
\subsection{Примеры абсолютно непрерывных случайных величин.}

\subsubsection{Равномерное распределение}

Случайная величина имеет {\it равномерное распределение на отрезке $[a, b]$}, если
ее распределение задано плотностью
\[
    \rho(x) = \begin{cases}
        \frac{1}{b - a} & x \in (a, b], \\
        0 & x \notin (a, b].
    \end{cases}
\]
Такая случайная величина описывает случайное бросание точки в отрезок $[a, b]$.
Вероятность того, что точка попадёт в отрезок $[c, d] \subset [a, b]$ равна $\frac{d - c}{b - a}$.

\subsubsection{Нормальное распределение}

Случайная величина имеет {\it нормальное распределение} с параметрами $a$ и $\sigma^2$, если ее распределение задано плотностью
\[
    \rho(x) = \frac{1}{\sqrt{2 \pi \sigma^2}} e^{-\frac{(x - a)^2}{2 \sigma^2}}.
\]
В случае $a = 0$ и $\sigma = 1$ эта плотность появлялась в теореме Муавра-Лапласа.
\textcolor{purple}{\it Помним этот гроб билет первого коллока.}

\subsubsection{Экспоненциальное (показательное) распределение}

Случайная величина имеет {\it экспоненциальное распределение} (которое еще иногда называется показательным) с параметром $\lambda > 0$, если ее распределение задано плотностью
\[
    \rho(x) = \begin{cases}
        \lambda e^{-\lambda x}, & x \geqslant 0, \\
        0, & x < 0.
    \end{cases}
\]
Функция распределения такой случайной величины имеет вид $F(t) = 1 - e^{-\lambda x}$.
