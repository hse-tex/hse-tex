\section{Лекция 12.04.2018}

\subsection{Метрические задачи в $\mathbb{R}^3$}

\textbf{Расстояние от точки $v$ до прямой $l = \{v_0 + at\}$} 

$\rho(v, l) = |ort_{<a>} (v - v_0)| = \frac{|[v-v_0, a]|}{|a|}$

\vspace{\baselineskip}
\textbf{Расстояние от точки $v$ до плоскости $P$ c нормалью $n$ и точкой $v_0$, $S$ -- направляющее пространство для $P$}

$\rho(v, P) = |ort_S (v-v_0)| = |pr_{<n>} (v-v_0)| = |\frac{(v - v_0, n)}{(n, n)} n| = \frac{|(v - v_0, n)|}{|n|}$

\vspace{\baselineskip}
\textbf{Расстояние между двумя скрещивающимися прямыми $l_1 = \{v_1 + a_1 t\}$ и $l_2 = \{v_2 + a_2 t\}$}

$P_1 = v_1 + <a_1, a_2>$

$P_2 = v_2 + <a_1, a_2>$

$P_1 || P_2$

$\rho (l_1, l_2) = \rho (P_1, P_2) = \frac{|(a_1, a_2, v_2 - v_1)|}{|[a_1, a_2]|}$

\vspace{\baselineskip}
\textbf{Угол между двумя прямыми $l_1 = \{v_1 + a_1 t\}$ и $l_2 = \{v_2 + a_2 t\}$}

$\angle (l_1, l_2):= min( \angle (a_1, a_2), \angle (a_1, -a_2))$

\vspace{\baselineskip}
\textbf{Угол между прямой $l = \{v_1 + a_1 t\}$ и плоскостью  $P$ с нормалью $n$}

$\angle (l, P) := \frac{\pi}{2} - \angle (l, <n>)$

\vspace{\baselineskip}
\textbf{Угол между двумя плоскостями $P_1$ и $P_2$ с нормалями $n_1, n_2$}

$\angle (P_1, P_2) := \angle (<n_1>, <n_2>)$

\subsection{Линейные операторы}

$V$ -- векторное пространство над полем $F$

\textbf{Определение.} \textit{Линейным оператором (или линейным преобразованием)} в(на) $V$ называется всякое линейное отображение $\varphi: V \rightarrow V$ (т.е. из $V$ в себя).

\vspace{\baselineskip}
$L(H) := Hom(V, V) = $ {все линейные операторы $V \rightarrow V$}

$e = (e_1, \dots, e_n)$ -- базис $V \Rightarrow (\varphi(e_1), \dots, \varphi(e_n)) = (e_1, \dots, e_n) A$

$A$ называется матрицей линейного оператора $\varphi$ в базисе $e$

Столбец $A^{(i)}$ состоит из координат вектора $\varphi(e_i)$ в базисе $e$.

\vspace{\baselineskip}
Примеры.

1) $\lambda \in F, \ \varphi: V \rightarrow V, \varphi(v) = \lambda v$ (скалярный оператор)

В любом базисе имеет матрицу $\lambda E$

2) $\varphi: \mathbb{R}^2 \rightarrow \mathbb{R}^2$ -- поворот на угол $\alpha$

$e$ -- положительно ориентированный ортогональный базис $\Rightarrow A(\varphi, e) = \begin{pmatrix} cos \alpha & -sin \alpha \\ sin \alpha & cos \alpha \end{pmatrix}$

3) $V = \mathbb{R}[x]_{\leq n}, \varphi: V \rightarrow V, f \rightarrow f', e = (1, x, x^2, \dots, x^n) \Rightarrow A(\varphi, e) = \begin{pmatrix} 0 & 1 & 0 & \dots & 0 \\ 0 & 0 & 2 & \dots & 0 \\ \vdots & \vdots & \vdots & \vdots & \vdots \\ 0 & 0 & 0 & \dots & n \\ 0 & 0 & 0 & \dots & 0 \end{pmatrix}$

\vspace{\baselineskip}
\textbf{Следствия общих фактов о линейных отображениях}

1) $\varphi \in L(V), e$ -- базис $V \Rightarrow$ отображение $L(V) \rightarrow M_n(F), \varphi \rightarrow A(\varphi, e)$ является изоморфизмом векторных пространств

1а) Всякий линейный оператор $\varphi \in L(V)$ однозначно определяется своей матрицей в любом фиксированном базисе

1б) $e$ -- произвольный базис, $A \in M_n(F) \Rightarrow ! \exists$ линейный оператор $\varphi \in L(V)$, такой что $A(\varphi, e) = A$

2) $e = (e_1, \dots, e_n)$ -- базис, $x = x_1 e_1 + \dots + x_n e_n, \varphi (x) = y_1 e_1 + \dots + y_n e_n \Rightarrow \begin{pmatrix} y_1 \\ \vdots \\ y_n \end{pmatrix} = A \begin{pmatrix} x_1 \\ \vdots \\ x_n
\end{pmatrix}$

3) $e' = (e'_1, \dots, e'_n)$ -- другой базис в $V$

$(e'_1, \dots, e'_n) = (e_1, \dots, e_n) C$ -- матрица перехода

$A = A(\varphi, e)$

$A' = A(\varphi, e')$

$A' = C^{-1}AC$

\vspace{\baselineskip}
\textbf{Следствия из 3)} а) $det A$ не зависит от выбора базиса

б) $tr A$ не зависит от выбора базиса

\vspace{\baselineskip}
\textbf{\textit{Доказательство.}} $\rhd$ а) $det (C^{-1} A C) = det C^{-1} det A det C = det A$

б) $tr(C^{-1} AC) = tr (AC C^{-1}) = tr A \ \lhd$

\vspace{\baselineskip}
\textbf{Замечание.} $det$ и $tr$ являются инвариантами самого линейного оператора $\varphi$, а не его матрицы

Обозначение: $det \varphi, tr \varphi$

\vspace{\baselineskip}
\textbf{Определение.} Две матрицы $A, A' \in M_n(F)$ называются \textit{подобными}, если $\exists C \in M_n(F), detC \neq 0$, такая что $A' = C^{-1} A C$. Отношение подобия является отношением эквивалентности на $M_n(F) \Rightarrow M_n(F)$ разбивается на классы подобных матриц

\vspace{\baselineskip}
\textbf{Предложение.} $\varphi \in L(V) \Rightarrow$ следующие условия эквивалентны

1) $Ker \varphi = {0}$

2) $Im \varphi = V$

3) $\varphi$ обратим (т.е. $\varphi$ -- изоморфизм пространства $V$ на себя)

4) $det \varphi \neq 0$

\vspace{\baselineskip}
\textbf{\textit{Доказательство.}} $\rhd$ 1) $\Leftrightarrow$ 2), т.к. $dim V = dim Ker \varphi + dim Im \varphi$

1) $\&$ 2) $\Leftrightarrow$ 3) было для линейного отображения

2) $\Leftrightarrow$ 4) $Im \varphi = V \Leftrightarrow rk \varphi = dim V \Leftrightarrow det \varphi \neq 0 \ \lhd$

\vspace{\baselineskip}
\textbf{Определение.} Линейный оператор $\varphi \in L(V)$ называется \textit{невырожденным}, если $det \varphi \neq 0$, и \textit{вырожденным} иначе.

\vspace{\baselineskip}
$\varphi \in L(V)$

\textbf{Определение.} Подпространство $U \subseteq V$ называется \textit{инвариантным относительно $\varphi$ (или $\varphi$-инвариантным)}, если $\varphi(U) \subseteq U$ (т.е. $\forall \ u \in U: \varphi(u) \in U$).

В этой ситуации корректно определен оператор $\varphi |_u : U \rightarrow U, u \rightarrow \varphi(u)$, он называется \textit{ограничением} линейного оператора $\varphi$ на подпространство $U$.

\vspace{\baselineskip}
Примеры.

1) $\{0\}$ и $V \varphi$-инвариантны $\forall \ \varphi \in L(V)$

2) $Ker \varphi$ $\varphi$-инвариантно, т.к. $\varphi(Ker \varphi) = \{0\} \subseteq Ker \varphi$

3) $Im \varphi$ $\varphi$-инвариантно, т.к. $\varphi (Im \varphi) \subseteq \varphi (V) = Im \varphi$

\subsection{Наблюдения}

1) $U \subseteq V$ -- $\varphi$-инвариантное подпространство, $(e_1, \dots, e_k)$ -- базис $U$, дополним его до базиса $(e_1, \dots, e_n)$ всего $V$, тогда $A(\varphi, e) = \left(
\begin{array}{c|c}
  A & B  \\
  \hline
  0 & C  \\
\end{array}
\right), A \in M_k, B \in Mat_{k \times (n - k)}, C \in M_{n-k} (*)$

При этом:

если $U = Ker \varphi$, то $A = 0$

если $U = Im \varphi$, то $C = 0$

Обратно, если в некотром базисе $(e_1, \dots, e_n) \ A(\varphi, e)$ имеет вид (*), то $<e_1, \dots, e_k>$ -- это $\varphi$-инвариантное подпространство

\vspace{\baselineskip}
2) Последние $k$ векторов базиса $e$ порождают $\varphi$-инвариантное подпространство $\Leftrightarrow A(\varphi, e) = \left(
\begin{array}{c|c}
  A & 0  \\
  \hline
  B & C  \\
\end{array}
\right), A \in M_{n-k}, B \in Mat_{k \times (n - k)}, C \in M_k$

\vspace{\baselineskip}
3) $V = U_1 \oplus U_2, U_1 = <e_1, \dots, e_k>, U_2 = <e_{k+1}, \dots, e_n>$

$U_1, U_2$ $\varphi$-инвариантны $\Leftrightarrow A(\varphi, e) = \left(
\begin{array}{c|c}
  A & 0  \\
  \hline
  0 & B  \\
\end{array}
\right), A \in M_k, B \in M_{n-k}$

